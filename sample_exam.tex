% Options for packages loaded elsewhere
\PassOptionsToPackage{unicode}{hyperref}
\PassOptionsToPackage{hyphens}{url}
%
\documentclass[
  11pt,
]{article}
\usepackage{amsmath,amssymb}
\usepackage{lmodern}
\usepackage{ifxetex,ifluatex}
\ifnum 0\ifxetex 1\fi\ifluatex 1\fi=0 % if pdftex
  \usepackage[T1]{fontenc}
  \usepackage[utf8]{inputenc}
  \usepackage{textcomp} % provide euro and other symbols
\else % if luatex or xetex
  \usepackage{unicode-math}
  \defaultfontfeatures{Scale=MatchLowercase}
  \defaultfontfeatures[\rmfamily]{Ligatures=TeX,Scale=1}
  \setmainfont[BoldFont = SF Pro Rounded Semibold, Scale =
MatchLowercase]{SF Pro Rounded}
  \setmathfont[]{SF Pro Rounded}
\fi
% Use upquote if available, for straight quotes in verbatim environments
\IfFileExists{upquote.sty}{\usepackage{upquote}}{}
\IfFileExists{microtype.sty}{% use microtype if available
  \usepackage[]{microtype}
  \UseMicrotypeSet[protrusion]{basicmath} % disable protrusion for tt fonts
}{}
\makeatletter
\@ifundefined{KOMAClassName}{% if non-KOMA class
  \IfFileExists{parskip.sty}{%
    \usepackage{parskip}
  }{% else
    \setlength{\parindent}{0pt}
    \setlength{\parskip}{6pt plus 2pt minus 1pt}}
}{% if KOMA class
  \KOMAoptions{parskip=half}}
\makeatother
\usepackage{xcolor}
\IfFileExists{xurl.sty}{\usepackage{xurl}}{} % add URL line breaks if available
\IfFileExists{bookmark.sty}{\usepackage{bookmark}}{\usepackage{hyperref}}
\hypersetup{
  pdftitle={Sample Exam},
  pdfauthor={Philosophy 305},
  hidelinks,
  pdfcreator={LaTeX via pandoc}}
\urlstyle{same} % disable monospaced font for URLs
\usepackage[margin=1.5in]{geometry}
\usepackage{graphicx}
\makeatletter
\def\maxwidth{\ifdim\Gin@nat@width>\linewidth\linewidth\else\Gin@nat@width\fi}
\def\maxheight{\ifdim\Gin@nat@height>\textheight\textheight\else\Gin@nat@height\fi}
\makeatother
% Scale images if necessary, so that they will not overflow the page
% margins by default, and it is still possible to overwrite the defaults
% using explicit options in \includegraphics[width, height, ...]{}
\setkeys{Gin}{width=\maxwidth,height=\maxheight,keepaspectratio}
% Set default figure placement to htbp
\makeatletter
\def\fps@figure{htbp}
\makeatother
\setlength{\emergencystretch}{3em} % prevent overfull lines
\providecommand{\tightlist}{%
  \setlength{\itemsep}{0pt}\setlength{\parskip}{0pt}}
\setcounter{secnumdepth}{-\maxdimen} % remove section numbering
\DeclareSymbolFont{symbolsC}{U}{txsyc}{m}{n}
\DeclareMathSymbol{\boxright}{\mathrel}{symbolsC}{128}
\usepackage{gensymb}
\usepackage{nicefrac}
\usepackage{caption}
\usepackage{istgame}
\usepackage{mathastext}
\ifluatex
  \usepackage{selnolig}  % disable illegal ligatures
\fi

\title{Sample Exam}
\author{Philosophy 305}
\date{TBC}

\begin{document}
\maketitle

\hypertarget{instructions}{%
\subsection{Instructions}\label{instructions}}

\begin{itemize}
\tightlist
\item
  You have \textbf{3 hours} for the exam.
\item
  Type up any answers you can.
\item
  But for things you can't type - \emph{especially trees} - write them
  out on paper, take a photo of them, and upload the photo.
\item
  Note that there will be \emph{fewer} questions than this on the final,
  but the structure will be similar. The point of this is to give you a
  sense of the kind of questions that there will be.
\end{itemize}

\hypertarget{truth-tables}{%
\subsection{Truth Tables}\label{truth-tables}}

For each of these sequents, do a truth table to test whether they are
valid. In each case, say whether they are valid.

\begin{enumerate}
\def\labelenumi{\arabic{enumi}.}
\tightlist
\item
  \(A \vee B, B \rightarrow A \vDash A\)
\item
  \(\neg (A \wedge B), \neg (B \rightarrow A) \vDash A\)
\item
  \(A \rightarrow B \vDash B \rightarrow A\)
\item
  \(A \rightarrow (B \vee C), C \rightarrow (A \vee B) \vDash B\)
\end{enumerate}

\hypertarget{truth-trees}{%
\subsection{Truth Trees}\label{truth-trees}}

For each of these sequents, do a truth table to test whether they are
valid. In each case, say whether they are valid.

\begin{enumerate}
\def\labelenumi{\arabic{enumi}.}
\setcounter{enumi}{4}
\tightlist
\item
  \(A \vee B, B \rightarrow A \vDash A\)
\item
  \(\neg (A \wedge B), \neg (B \rightarrow A) \vDash A\)
\item
  \(A \rightarrow B \vDash B \rightarrow A\)
\item
  \(A \rightarrow (B \vee C), C \rightarrow (A \vee B) \vDash B\)
\end{enumerate}

\newpage

\hypertarget{proofs}{%
\subsection{Proofs}\label{proofs}}

Construct a proof for each of the following

\begin{enumerate}
\def\labelenumi{\arabic{enumi}.}
\setcounter{enumi}{8}
\tightlist
\item
  \(P \rightarrow (Q \wedge R), S \wedge P \vdash R \wedge (S \vee T)\)
\item
  \((P \wedge Q) \rightarrow R \vdash P \rightarrow (Q \rightarrow R)\)
\item
  \(P \rightarrow R, Q \rightarrow R \vdash (P \vee Q) \rightarrow R\)
\item
  \(P \rightarrow (Q \wedge R), P \rightarrow (R \rightarrow \neg Q) \vdash \neg P\)
\end{enumerate}

\hypertarget{probability}{%
\subsection{Probability}\label{probability}}

\begin{enumerate}
\def\labelenumi{\arabic{enumi}.}
\setcounter{enumi}{12}
\tightlist
\item
  A fair coin (with equal chance of landing heads and landing tails) is
  about to be flipped. Ankita is offered the following bet - if it lands
  heads she wins \$200, and if it lands tails she loses \$100. Do we
  know enough to advise Ankita whether or not she should take the bet?
  Why or why not?
\item
  Explain why the following decision rule is not generally reasonable:
  Identity the most likely state; then choose an act which maximizes
  utility in that state. (Hint: Describe a situation where this would
  lead to doing something unreasonable.)
\end{enumerate}

\hypertarget{modal-logic}{%
\subsection{Modal Logic}\label{modal-logic}}

For each of the following sentences, do \textbf{three} truth trees: one
to check whether it is a logical truth in K, one to check whether it is
a logical truth in S4, and one to check whether it is a logical truth in
KT4B (i.e., S5). You can use the simplified rules for S5.

\begin{enumerate}
\def\labelenumi{\arabic{enumi}.}
\setcounter{enumi}{14}
\tightlist
\item
  \(\Box(\Box A \rightarrow B) \vee \Box A\)
\item
  \(\Diamond(A \rightarrow \Diamond \Box A)\)
\end{enumerate}

\hypertarget{conditionals}{%
\subsection{Conditionals}\label{conditionals}}

\begin{enumerate}
\def\labelenumi{\arabic{enumi}.}
\setcounter{enumi}{16}
\tightlist
\item
  Show that
  \(\Box (A \rightarrow B) \rightarrow \Box ((A \wedge C) \rightarrow B)\)
  is a theorem of S5.
\item
  Describe a sphere model (from the minimal change semantics chapter of
  \emph{Boxes And Diamonds}) that shows
  \(((A \boxright B) \wedge (B \boxright C)) \rightarrow (A \boxright C)\)
  is not a logical truth in the minimal change semantics.
\end{enumerate}

\end{document}
