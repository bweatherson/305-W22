\documentclass[11pt,]{article}
\usepackage{lmodern}
\usepackage{amssymb,amsmath}
\usepackage{ifxetex,ifluatex}
\usepackage{fixltx2e} % provides \textsubscript
\ifnum 0\ifxetex 1\fi\ifluatex 1\fi=0 % if pdftex
  \usepackage[T1]{fontenc}
  \usepackage[utf8]{inputenc}
\else % if luatex or xelatex
  \ifxetex
    \usepackage{mathspec}
  \else
    \usepackage{fontspec}
  \fi
  \defaultfontfeatures{Ligatures=TeX,Scale=MatchLowercase}
    \setmainfont[]{SF Pro Text Light}
\fi
% use upquote if available, for straight quotes in verbatim environments
\IfFileExists{upquote.sty}{\usepackage{upquote}}{}
% use microtype if available
\IfFileExists{microtype.sty}{%
\usepackage{microtype}
\UseMicrotypeSet[protrusion]{basicmath} % disable protrusion for tt fonts
}{}
\usepackage[margin=1.2in]{geometry}
\usepackage{hyperref}
\hypersetup{unicode=true,
            pdftitle={Assignment 7},
            pdfauthor={Philosophy 305},
            pdfborder={0 0 0},
            breaklinks=true}
\urlstyle{same}  % don't use monospace font for urls
\usepackage{graphicx,grffile}
\makeatletter
\def\maxwidth{\ifdim\Gin@nat@width>\linewidth\linewidth\else\Gin@nat@width\fi}
\def\maxheight{\ifdim\Gin@nat@height>\textheight\textheight\else\Gin@nat@height\fi}
\makeatother
% Scale images if necessary, so that they will not overflow the page
% margins by default, and it is still possible to overwrite the defaults
% using explicit options in \includegraphics[width, height, ...]{}
\setkeys{Gin}{width=\maxwidth,height=\maxheight,keepaspectratio}
\IfFileExists{parskip.sty}{%
\usepackage{parskip}
}{% else
\setlength{\parindent}{0pt}
\setlength{\parskip}{6pt plus 2pt minus 1pt}
}
\setlength{\emergencystretch}{3em}  % prevent overfull lines
\providecommand{\tightlist}{%
  \setlength{\itemsep}{0pt}\setlength{\parskip}{0pt}}
\setcounter{secnumdepth}{0}
% Redefines (sub)paragraphs to behave more like sections
\ifx\paragraph\undefined\else
\let\oldparagraph\paragraph
\renewcommand{\paragraph}[1]{\oldparagraph{#1}\mbox{}}
\fi
\ifx\subparagraph\undefined\else
\let\oldsubparagraph\subparagraph
\renewcommand{\subparagraph}[1]{\oldsubparagraph{#1}\mbox{}}
\fi

%%% Use protect on footnotes to avoid problems with footnotes in titles
\let\rmarkdownfootnote\footnote%
\def\footnote{\protect\rmarkdownfootnote}

%%% Change title format to be more compact
\usepackage{titling}

% Create subtitle command for use in maketitle
\providecommand{\subtitle}[1]{
  \posttitle{
    \begin{center}\large#1\end{center}
    }
}

\setlength{\droptitle}{-2em}

  \title{Assignment 7}
    \pretitle{\vspace{\droptitle}\centering\huge}
  \posttitle{\par}
    \author{Philosophy 305}
    \preauthor{\centering\large\emph}
  \postauthor{\par}
      \predate{\centering\large\emph}
  \postdate{\par}
    \date{Due 1 November, 2019}

\usepackage{gensymb}
\usepackage{nicefrac}
\usepackage{caption}
\usepackage{mathastext}

\begin{document}
\maketitle

Note that all these questions are from \emph{Odds and Ends},
occasionally with minor changes to the wording.

\hypertarget{expected-value-questions}{%
\section{Expected Value Questions}\label{expected-value-questions}}

\hypertarget{questions-1-to-3}{%
\subsection{Questions 1 to 3}\label{questions-1-to-3}}

\begin{enumerate}
\def\labelenumi{\arabic{enumi}.}
\tightlist
\item
  What is the expected monetary value of playing a slot machine that
  costs \(\$100\) to play, and has a \(1/25\) chance of paying out
  \(\$500\)? (The rest of the time it pays nothing.)
\item
  Suppose a slot machine pays off \(\$25\) a fiftieth of the time and
  costs a \(\$1\) to play, and a video poker machine pays off \(\$10\) a
  twentieth of the time and costs \(\$2\) to play. Which machine is the
  better bet in terms of expected monetary value?
\item
  You're considering downloading a new game for your phone. The game
  costs \(\$0.99\). But as a promotion, the first \(50,000\) downloaders
  are being entered in a fair lottery with a \(\$10,000\) cash prize. If
  you know you'll be one of the first \(50,000\) downloaders, what is
  the expected monetary value of downloading the game?
\end{enumerate}

\hypertarget{question-4}{%
\subsection{Question 4}\label{question-4}}

A local casino offers a game which costs \(\$2\) to play. A fair coin is
flipped up to three times, and the payouts work as follows:

\begin{itemize}
\tightlist
\item
  If the coin lands heads on the first toss, you win \(\$2\) and the
  game is over.
\item
  If the coin lands heads on the second toss, you win \(\$4\) and the
  game is over.
\item
  If the coin lands heads on the third toss, you win \(\$8\) and the
  game is over.
\item
  If the coin lands tails all three times, you win \(\$0\).
\end{itemize}

\begin{enumerate}
\def\labelenumi{\arabic{enumi}.}
\setcounter{enumi}{3}
\tightlist
\item
  What is the expected monetary value of this game?
\end{enumerate}

\hypertarget{question-5}{%
\subsection{Question 5}\label{question-5}}

\begin{enumerate}
\def\labelenumi{\arabic{enumi}.}
\setcounter{enumi}{4}
\tightlist
\item
  Suppose you can bet on either of two dogs: Santa's Little Helper or
  She's the Fastest. If you bet on Santa's Little Helper and he wins,
  you get \(\$5\). If he loses you pay \(\$2\). If you bet on She's the
  Fastest and she loses, you pay \(\$10\). The two dogs have the same
  chance of winning. How much would a winning bet on She's the Fastest
  have to pay for the bets to have the same value?
\end{enumerate}

\hypertarget{questions-6-to-9}{%
\subsection{Questions 6 to 9}\label{questions-6-to-9}}

Suppose Michigan is deciding whether to enact a new tax. If the tax is
enacted, it will bring in \(\$700\) million in revenue. But it could
also hurt the economy. The chance of harm to the economy is small, just
\(1/5\). But it would cost the country \(\$1,200\) million in lost
earnings. (The \(\$700\) million in revenue would still be gained,
partially offsetting this loss.) Treat gains as positive and losses as
negative.

\begin{enumerate}
\def\labelenumi{\arabic{enumi}.}
\setcounter{enumi}{5}
\tightlist
\item
  What is the expected monetary value of enacting the new tax?
\end{enumerate}

The government also has the option of conducting a study before deciding
whether to enact the new tax. If the study's findings are bad news, that
means the chance of harm to the economy is actually double what they
thought. If its findings are good news, then the chance of harm to the
economy is actually half of what they thought.

\begin{enumerate}
\def\labelenumi{\arabic{enumi}.}
\setcounter{enumi}{6}
\tightlist
\item
  Suppose the government conducts the study and its findings are good
  news. What will the expected monetary value of enacting the tax be
  then?
\item
  Suppose the government conducts the study and its findings are bad
  news. What will the expected monetary value of enacting the tax be
  then?
\item
  If the expected monetary value of the tax is what you said in question
  6, and the study is bound to deliver good news or bad news, what is
  the probability of it delivering good news?
\end{enumerate}

\hypertarget{question-10}{%
\subsection{Question 10}\label{question-10}}

Consider the following game: I'm going to flip a fair coin up to three
times. If it comes up heads on the first toss, the game is over and you
win \(\$2\). If it comes up heads for the first time on the second toss,
you win \(\$40\) and the game is over. If the first heads comes up on
the third toss, you win \(\$800\) and the game is over. If it comes up
tails every time, you have to pay me \(\$x\).

\begin{enumerate}
\def\labelenumi{\arabic{enumi}.}
\setcounter{enumi}{9}
\tightlist
\item
  What does \(x\) have to be to make the game fair?
\end{enumerate}

\hypertarget{question-11}{%
\subsection{Question 11}\label{question-11}}

Some workplaces hold a weekly lottery. Suppose there are 30 people in
your workplace lottery, and each person pays in \$5 every Monday. A
finalist is chosen at random every Friday, for three weeks. Then, on the
fourth Friday, one of the three finalists from the previous three weeks
is chosen at random. That person gets all the prize money.

\begin{enumerate}
\def\labelenumi{\arabic{enumi}.}
\setcounter{enumi}{10}
\tightlist
\item
  What is the expected value of being in this lottery?
\end{enumerate}

\hypertarget{utility-questions}{%
\section{Utility Questions}\label{utility-questions}}

\hypertarget{question-12}{%
\subsection{Question 12}\label{question-12}}

Sonia has tickets to see The Weeknd tomorrow night. Her friend has
tickets to see Beyoncé, and also tickets to Katy Perry. Beyoncé is
Sonia's favourite performer, in fact she would rather see Beyoncé than
The Weeknd. Sonia's friend offers a gamble in exchange for her tickets
to The Weeknd. The gamble has a \(9/10\) chance of winning, in which
case Sonia gets the Beyoncé tickets (utility \(1\)). Otherwise she gets
the Katy Perry tickets (utility \(0\)).

\begin{enumerate}
\def\labelenumi{\arabic{enumi}.}
\setcounter{enumi}{11}
\item
  If Sonia declines the gamble, what can we conclude?

  \begin{enumerate}
  \def\labelenumii{\alph{enumii}.}
  \tightlist
  \item
    For Sonia, the utility of seeing The Weeknd is \(9/10\).
  \item
    For Sonia, the utility of seeing The Weeknd is greater than
    \(9/10\).
  \item
    For Sonia, the utility of seeing The Weeknd is less than \(9/10\).
  \item
    For Sonia, the utility of seeing The Weeknd is \(1/10\).
  \end{enumerate}
\end{enumerate}

\hypertarget{questions-13-to-14}{%
\subsection{Questions 13 to 14}\label{questions-13-to-14}}

After giving her calculus midterm, Professor X always offers her
students a chance to improve their grade by trying to solve an optional
``challenge'' problem. If they get it right, their grade is increased by
one letter grade: F changes to D, D changes to C, etc. But if they get
it wrong, their grade goes down by one letter-grade: A changes to B, B
changes to C, etc.

Hui got a C on his midterm. He asks the professor how often students get
the challenge problem right and she says they get it right half the
time. Hui decides to stick with his C. But he would be willing to try
the challenge problem if the chances of getting it right were higher:
\(2/3\) or more. Suppose getting a D has utility \(5/10\) for Hui, while
a B has utility \(8/10\).

\begin{enumerate}
\def\labelenumi{\arabic{enumi}.}
\setcounter{enumi}{12}
\tightlist
\item
  What is the expected utility for Hui of trying the challenge problem?
\item
  How much utility does a C have for Hui?
\end{enumerate}

\hypertarget{questions-15-to-17-2-points-each}{%
\subsection{Questions 15 to 17 (2 points
each)}\label{questions-15-to-17-2-points-each}}

Eleanor wants to get a job at Google so she's going to university to
study computer science. She has to decide between Wayne State and
Michigan State. Suppose \(1/100\) of Wayne State's computer science
students get jobs at Google and the rest get jobs at Facebook. For
Eleanor, a job at Google has utility \(200\) while a job at Facebook has
utility \(50\).

\begin{enumerate}
\def\labelenumi{\arabic{enumi}.}
\setcounter{enumi}{14}
\tightlist
\item
  What is the expected utility of going to Wayne State for Eleanor?
\end{enumerate}

Suppose Michigan State students have better odds of getting a job at
Google: \(5/400\). And \(360/400\) students go to work at Amazon, which
Eleanor would prefer to Facebook. On the other hand, the remaining
\(35/400\) of them don't get a job at all, which has utility zero for
Eleanor. After thinking about it, she can't decide: Wayne State and
Michigan State seem like equally good options to her.

\begin{enumerate}
\def\labelenumi{\arabic{enumi}.}
\setcounter{enumi}{15}
\tightlist
\item
  How much utility does working at Amazon have for Eleanor?
\end{enumerate}

Suppose Eleanor ends up going to Wayne State, and now she's about to
graduate. Unfortunately, Google isn't hiring any more. The only jobs
available are at Amazon and Facebook. She would have to take a special
summer training program to qualify for a job at Amazon, though. And that
would mean she can't get a job at Facebook. Facebook is offering her a
job, but she has to take it now or never. So, she has to either take the
guaranteed job at Facebook right now, or gamble on the summer program.
The summer program could get her a job at Amazon, or it could leave her
unemployed.

\begin{enumerate}
\def\labelenumi{\arabic{enumi}.}
\setcounter{enumi}{16}
\tightlist
\item
  How high would the probability of getting a job at Amazon have to be
  for Eleanor to be indifferent between taking and not taking the
  special summer program?
\end{enumerate}


\end{document}
