% Options for packages loaded elsewhere
\PassOptionsToPackage{unicode}{hyperref}
\PassOptionsToPackage{hyphens}{url}
%
\documentclass[
  11pt,
]{article}
\usepackage{amsmath,amssymb}
\usepackage{lmodern}
\usepackage{setspace}
\usepackage{ifxetex,ifluatex}
\ifnum 0\ifxetex 1\fi\ifluatex 1\fi=0 % if pdftex
  \usepackage[T1]{fontenc}
  \usepackage[utf8]{inputenc}
  \usepackage{textcomp} % provide euro and other symbols
\else % if luatex or xetex
  \usepackage{unicode-math}
  \defaultfontfeatures{Scale=MatchLowercase}
  \defaultfontfeatures[\rmfamily]{Ligatures=TeX,Scale=1}
  \setmainfont[Scale=MatchLowercase]{SF Pro Text Light}
\fi
% Use upquote if available, for straight quotes in verbatim environments
\IfFileExists{upquote.sty}{\usepackage{upquote}}{}
\IfFileExists{microtype.sty}{% use microtype if available
  \usepackage[]{microtype}
  \UseMicrotypeSet[protrusion]{basicmath} % disable protrusion for tt fonts
}{}
\makeatletter
\@ifundefined{KOMAClassName}{% if non-KOMA class
  \IfFileExists{parskip.sty}{%
    \usepackage{parskip}
  }{% else
    \setlength{\parindent}{0pt}
    \setlength{\parskip}{6pt plus 2pt minus 1pt}}
}{% if KOMA class
  \KOMAoptions{parskip=half}}
\makeatother
\usepackage{xcolor}
\IfFileExists{xurl.sty}{\usepackage{xurl}}{} % add URL line breaks if available
\IfFileExists{bookmark.sty}{\usepackage{bookmark}}{\usepackage{hyperref}}
\hypersetup{
  pdftitle={Assignment Week 4},
  pdfauthor={Philosophy 305},
  hidelinks,
  pdfcreator={LaTeX via pandoc}}
\urlstyle{same} % disable monospaced font for URLs
\usepackage[margin=1.2in]{geometry}
\usepackage{graphicx}
\makeatletter
\def\maxwidth{\ifdim\Gin@nat@width>\linewidth\linewidth\else\Gin@nat@width\fi}
\def\maxheight{\ifdim\Gin@nat@height>\textheight\textheight\else\Gin@nat@height\fi}
\makeatother
% Scale images if necessary, so that they will not overflow the page
% margins by default, and it is still possible to overwrite the defaults
% using explicit options in \includegraphics[width, height, ...]{}
\setkeys{Gin}{width=\maxwidth,height=\maxheight,keepaspectratio}
% Set default figure placement to htbp
\makeatletter
\def\fps@figure{htbp}
\makeatother
\setlength{\emergencystretch}{3em} % prevent overfull lines
\providecommand{\tightlist}{%
  \setlength{\itemsep}{0pt}\setlength{\parskip}{0pt}}
\setcounter{secnumdepth}{-\maxdimen} % remove section numbering
\usepackage{gensymb}
\usepackage{nicefrac}
\usepackage{caption}
\usepackage{mathastext}
\ifluatex
  \usepackage{selnolig}  % disable illegal ligatures
\fi

\title{Assignment Week 4}
\author{Philosophy 305}
\date{Due January 28, 2022, 5pm}

\begin{document}
\maketitle

\setstretch{1.4}
\hypertarget{trees-or-tables}{%
\section{Trees or Tables}\label{trees-or-tables}}

Using either Truth Trees or Truth Tables, work out which of the
following sentences are logical truths. You don't have to show your
working, just input the answers into Canvas. (But if you do want to show
workings, there is a place for uploading files, such as scans,
containing workings.)

\begin{enumerate}
\def\labelenumi{\arabic{enumi}.}
\tightlist
\item
  \(((A \rightarrow (B \vee D)) \rightarrow C) \rightarrow (\neg C \rightarrow A)\)
\item
  \(((A \vee D) \rightarrow (B \rightarrow C)) \rightarrow (B \rightarrow (A \rightarrow C))\)
\item
  \((A \rightarrow (B \rightarrow C)) \rightarrow ((B \vee D) \rightarrow (A \rightarrow C))\)
\item
  \((A \rightarrow \neg (B \wedge D)) \rightarrow (((\neg B \vee \neg D) \rightarrow \neg C) \rightarrow (C \rightarrow \neg A))\)
\item
  \((A \rightarrow \neg (B \vee D)) \rightarrow (((\neg B \wedge \neg D) \rightarrow \neg C) \rightarrow (C \rightarrow \neg A))\)
\item
  \((A \rightarrow B) \rightarrow C, C \rightarrow A, \vdash D \rightarrow A\)
\item
  \(A \rightarrow B \vdash (C \rightarrow \neg A) \vee (\neg B \rightarrow D)\)
\item
  \(A \rightarrow (B \wedge C), (A \wedge B) \rightarrow (C \rightarrow (D \vee E)) \vdash \neg D \rightarrow (A \rightarrow E)\)
\item
  \(A \rightarrow (\neg B \vee C), A \rightarrow ((\neg D \vee E) \rightarrow (B \rightarrow C)) \vdash (A \wedge D) \rightarrow E\)
\item
  \((A \rightarrow B) \rightarrow B, (\neg B \rightarrow C) \rightarrow C \vdash A \vee C\)
\end{enumerate}

\end{document}
