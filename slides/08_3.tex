% Options for packages loaded elsewhere
\PassOptionsToPackage{unicode}{hyperref}
\PassOptionsToPackage{hyphens}{url}
%
\documentclass[
  ignorenonframetext,
]{beamer}
\usepackage{pgfpages}
\setbeamertemplate{caption}[numbered]
\setbeamertemplate{caption label separator}{: }
\setbeamercolor{caption name}{fg=normal text.fg}
\beamertemplatenavigationsymbolsempty
% Prevent slide breaks in the middle of a paragraph
\widowpenalties 1 10000
\raggedbottom
\setbeamertemplate{part page}{
  \centering
  \begin{beamercolorbox}[sep=16pt,center]{part title}
    \usebeamerfont{part title}\insertpart\par
  \end{beamercolorbox}
}
\setbeamertemplate{section page}{
  \centering
  \begin{beamercolorbox}[sep=12pt,center]{part title}
    \usebeamerfont{section title}\insertsection\par
  \end{beamercolorbox}
}
\setbeamertemplate{subsection page}{
  \centering
  \begin{beamercolorbox}[sep=8pt,center]{part title}
    \usebeamerfont{subsection title}\insertsubsection\par
  \end{beamercolorbox}
}
\AtBeginPart{
  \frame{\partpage}
}
\AtBeginSection{
  \ifbibliography
  \else
    \frame{\sectionpage}
  \fi
}
\AtBeginSubsection{
  \frame{\subsectionpage}
}
\usepackage{amsmath,amssymb}
\usepackage{lmodern}
\usepackage{ifxetex,ifluatex}
\ifnum 0\ifxetex 1\fi\ifluatex 1\fi=0 % if pdftex
  \usepackage[T1]{fontenc}
  \usepackage[utf8]{inputenc}
  \usepackage{textcomp} % provide euro and other symbols
\else % if luatex or xetex
  \usepackage{unicode-math}
  \defaultfontfeatures{Scale=MatchLowercase}
  \defaultfontfeatures[\rmfamily]{Ligatures=TeX,Scale=1}
  \setmainfont[BoldFont = SF Pro Rounded Semibold]{SF Pro Rounded}
  \setmathfont[]{STIX Two Math}
\fi
\usefonttheme{serif} % use mainfont rather than sansfont for slide text
% Use upquote if available, for straight quotes in verbatim environments
\IfFileExists{upquote.sty}{\usepackage{upquote}}{}
\IfFileExists{microtype.sty}{% use microtype if available
  \usepackage[]{microtype}
  \UseMicrotypeSet[protrusion]{basicmath} % disable protrusion for tt fonts
}{}
\makeatletter
\@ifundefined{KOMAClassName}{% if non-KOMA class
  \IfFileExists{parskip.sty}{%
    \usepackage{parskip}
  }{% else
    \setlength{\parindent}{0pt}
    \setlength{\parskip}{6pt plus 2pt minus 1pt}}
}{% if KOMA class
  \KOMAoptions{parskip=half}}
\makeatother
\usepackage{xcolor}
\IfFileExists{xurl.sty}{\usepackage{xurl}}{} % add URL line breaks if available
\IfFileExists{bookmark.sty}{\usepackage{bookmark}}{\usepackage{hyperref}}
\hypersetup{
  pdftitle={305 Lecture 8.3 - The Crashing Websites},
  pdfauthor={Brian Weatherson},
  hidelinks,
  pdfcreator={LaTeX via pandoc}}
\urlstyle{same} % disable monospaced font for URLs
\newif\ifbibliography
\usepackage{longtable,booktabs,array}
\usepackage{calc} % for calculating minipage widths
\usepackage{caption}
% Make caption package work with longtable
\makeatletter
\def\fnum@table{\tablename~\thetable}
\makeatother
\setlength{\emergencystretch}{3em} % prevent overfull lines
\providecommand{\tightlist}{%
  \setlength{\itemsep}{0pt}\setlength{\parskip}{0pt}}
\setcounter{secnumdepth}{-\maxdimen} % remove section numbering
\let\Tiny=\tiny

 \setbeamertemplate{navigation symbols}{} 

% \usetheme{Madrid}
 \usetheme[numbering=none, progressbar=foot]{metropolis}
 \usecolortheme{wolverine}
 \usepackage{color}
 \usepackage{MnSymbol}
% \usepackage{movie15}

\usepackage{amssymb}% http://ctan.org/pkg/amssymb
\usepackage{pifont}% http://ctan.org/pkg/pifont
\newcommand{\cmark}{\ding{51}}%
\newcommand{\xmark}{\ding{55}}%

\DeclareSymbolFont{symbolsC}{U}{txsyc}{m}{n}
\DeclareMathSymbol{\boxright}{\mathrel}{symbolsC}{128}
\DeclareMathAlphabet{\mathpzc}{OT1}{pzc}{m}{it}

 \usepackage{tikz-qtree}
% \usepackage{markdown}
%\RequirePackage{bussproofs}
\RequirePackage[tableaux]{prooftrees}
\usetikzlibrary{arrows.meta}
 \forestset{line numbering, close with = x}
% Allow for easy commas inside trees
\renewcommand{\,}{\text{, }}


\usepackage{tabulary}

\usepackage{open-logic-config}

\setlength{\parskip}{1ex plus 0.5ex minus 0.2ex}

\AtBeginSection[]
{
\begin{frame}
	\Huge{\color{darkblue} \insertsection}
\end{frame}
}

\renewenvironment*{quote}	
	{\list{}{\rightmargin   \leftmargin} \item } 	
	{\endlist }

\definecolor{darkgreen}{rgb}{0,0.7,0}
\definecolor{darkblue}{rgb}{0,0,0.8}

\newcommand{\starttab}{\begin{center}
\vspace{6pt}
\begin{tabular}}

\newcommand{\stoptab}{\end{tabular}
\vspace{6pt}
\end{center}
\noindent}


\newcommand{\sif}{\rightarrow}
\newcommand{\siff}{\leftrightarrow}
\newcommand{\EF}{\end{frame}}


\newcommand{\TreeStart}[1]{
%\end{frame}
\begin{frame}
\begin{center}
\begin{tikzpicture}[scale=#1]
\tikzset{every tree node/.style={align=center,anchor=north}}
%\Tree
}

\newcommand{\TreeEnd}{
\end{tikzpicture}
%\end{center}
}

\newcommand{\DisplayArg}[2]{
\begin{enumerate}
{#1}
\end{enumerate}
\vspace{-6pt}
\hrulefill

%\hspace{14pt} #2
%{\addtolength{\leftskip}{14pt} #2}
\begin{quote}
{\normalfont #2}
\end{quote}
\vspace{12pt}
}

\newenvironment{ProofTree}[1][1]{
\begin{center}
\begin{tikzpicture}[scale=#1]
\tikzset{every tree node/.style={align=center,anchor=south}}
}
{
\end{tikzpicture}
\end{center}
}

\newcommand{\TreeFrame}[2]{
\begin{columns}[c]
\column{0.5\textwidth}
\begin{center}
\begin{prooftree}{}
#1
\end{prooftree}
\end{center}
\column{0.45\textwidth}
%\begin{markdown}
#2
%\end{markdown}
\end{columns}
}

\newcommand{\ScaledTreeFrame}[3]{
\begin{columns}[c]
\column{0.5\textwidth}
\begin{center}
\scalebox{#1}{
\begin{prooftree}{}
#2
\end{prooftree}
}
\end{center}
\column{0.45\textwidth}
%\begin{markdown}
#3
%\end{markdown}
\end{columns}
}

\usepackage[bb=boondox]{mathalfa}
\DeclareMathAlphabet{\mathbx}{U}{BOONDOX-ds}{m}{n}
\SetMathAlphabet{\mathbx}{bold}{U}{BOONDOX-ds}{b}{n}
\DeclareMathAlphabet{\mathbbx} {U}{BOONDOX-ds}{b}{n}


\newenvironment{oltableau}{\center\tableau{}} %wff format={anchor = base west}}}
       {\endtableau\endcenter}
       
\newcommand{\formula}[1]{$#1$}

\usepackage{tabulary}
\usepackage{booktabs}

\def\begincols{\begin{columns}}
\def\begincol{\begin{column}}
\def\endcol{\end{column}}
\def\endcols{\end{columns}}

\usepackage[italic]{mathastext}
\usepackage{nicefrac}

\definecolor{mygreen}{RGB}{0, 100, 0}
\definecolor{mypink2}{RGB}{219, 48, 122}
\definecolor{dodgerblue}{RGB}{30,144,255}

%\def\True{\textcolor{dodgerblue}{\text{T}}}
%\def\False{\textcolor{red}{\text{F}}}

\def\True{\mathbb{T}}
\def\False{\mathbb{F}}

% This is because arguments didn't have enough space after them
\usepackage{etoolbox}
\AfterEndEnvironment{description}{\vspace{9pt}}
\AfterEndEnvironment{oltableau}{\vspace{9pt}}
\BeforeBeginEnvironment{oltableau}{\vspace{9pt}}
\AfterEndEnvironment{center}{\vspace{12pt}}
\BeforeBeginEnvironment{tabular}{\vspace{9pt}}

\setlength\heavyrulewidth{0pt}
\setlength\lightrulewidth{0pt}

%\def\toprule{}
%\def\bottomrule{}
%\def\midrule{}

\setbeamertemplate{caption}{\raggedright\insertcaption}

\setlength\lightrulewidth{0.3pt}
\ifluatex
  \usepackage{selnolig}  % disable illegal ligatures
\fi

\title{305 Lecture 8.3 - The Crashing Websites}
\author{Brian Weatherson}
\date{}

\begin{document}
\frame{\titlepage}

\begin{frame}{Plan}
\protect\hypertarget{plan}{}
\begin{itemize}
\tightlist
\item
  This lecture will go over exercise 8.8 in the book, as an illustration
  of how to invert conditional probabilities in a slightly more
  complicated case.
\end{itemize}
\end{frame}

\begin{frame}{Associated Reading}
\protect\hypertarget{associated-reading}{}
Odds and Ends, Chapter 8
\end{frame}

\begin{frame}{Odds and Ends 8.8}
\protect\hypertarget{odds-and-ends-8.8}{}
A company makes websites, always powered by one of three server
platforms: Bulldozer, Kumquat, or Penguin. Bulldozer crashes 1 out of
every 10 visits, Kumquat crashes 1 in 50 visits, and Penguin only
crashes 1 out of every 200 visits.

Half of the websites are run on Bulldozer, 30\% are run on Kumquat, and
20\% are run on Penguin.

You visit one of their sites for the first time and it crashes. What is
the probability it was run on Penguin?
\end{frame}

\begin{frame}{Start with a Table}
\protect\hypertarget{start-with-a-table}{}
\begin{longtable}[]{@{}lcc@{}}
\toprule
& Crash & No Crash \\ \addlinespace
\midrule
\endhead
Bulldozer & & \\ \addlinespace
Kumquat & & \\ \addlinespace
Penguin & & \\ \addlinespace
\bottomrule
\end{longtable}

We will start by filling in the table - though really it is the left
hand column that matters here
\end{frame}

\begin{frame}{The Table}
\protect\hypertarget{the-table}{}
\begin{longtable}[]{@{}lcc@{}}
\toprule
& Crash & No Crash \\ \addlinespace
\midrule
\endhead
Bulldozer & \(0.5 \times 0.1 = 0.05\) & \\ \addlinespace
Kumquat & & \\ \addlinespace
Penguin & & \\ \addlinespace
\bottomrule
\end{longtable}

The formular for Bulldozer-and-Crash is

\[
\Pr(\text{Bulldozer}) \times \Pr(\text{Crash}|\text{Bulldozer}) = 0.5 \times 0.1 = 0.05
\]
\end{frame}

\begin{frame}{The Table}
\protect\hypertarget{the-table-1}{}
\begin{longtable}[]{@{}lcc@{}}
\toprule
& Crash & No Crash \\ \addlinespace
\midrule
\endhead
Bulldozer & \(0.5 \times 0.1 = 0.05\) & \\ \addlinespace
Kumquat & \(0.3 \times 0.02 = 0.006\) & \\ \addlinespace
Penguin & & \\ \addlinespace
\bottomrule
\end{longtable}

The formula for Kumquat-and-Crash is

\[
\Pr(\text{Kumquat}) \times \Pr(\text{Crash}|\text{Kumquat}) = 0.3 \times 0.02 = 0.006
\]
\end{frame}

\begin{frame}{The Table}
\protect\hypertarget{the-table-2}{}
\begin{longtable}[]{@{}lcc@{}}
\toprule
& Crash & No Crash \\ \addlinespace
\midrule
\endhead
Bulldozer & \(0.5 \times 0.1 = 0.05\) & \\ \addlinespace
Kumquat & \(0.3 \times 0.02 = 0.006\) & \\ \addlinespace
Penguin & \(0.2 \times 0.005 = 0.001\) & \\ \addlinespace
\bottomrule
\end{longtable}

The formula for Penguin-and-Crash is

\[
\Pr(\text{Penguin}) \times \Pr(\text{Crash}|\text{Penguin}) = 0.2 \times 0.005 = 0.001
\]
\end{frame}

\begin{frame}{The Table}
\protect\hypertarget{the-table-3}{}
\begin{longtable}[]{@{}lcc@{}}
\toprule
& Crash & No Crash \\ \addlinespace
\midrule
\endhead
Bulldozer & \(0.05\) & \\ \addlinespace
Kumquat & \(0.006\) & \\ \addlinespace
Penguin & \(0.001\) & \\ \addlinespace
\bottomrule
\end{longtable}

Let's rewrite it without the workings.
\end{frame}

\begin{frame}{The Table}
\protect\hypertarget{the-table-4}{}
\begin{longtable}[]{@{}lcc@{}}
\toprule
& Crash & No Crash \\ \addlinespace
\midrule
\endhead
Bulldozer & \(0.05\) & \(0.45\) \\ \addlinespace
Kumquat & \(0.006\) & \(0.294\) \\ \addlinespace
Penguin & \(0.001\) & \(0.199\) \\ \addlinespace
\bottomrule
\end{longtable}

We can fill in the right-hand column by noting that the rows have to add
up to 0.5, 0.3 and 0.2 respectively; since those are the probabilities
of the three server types.
\end{frame}

\begin{frame}{Crash!}
\protect\hypertarget{crash}{}
\begin{longtable}[]{@{}lcc@{}}
\toprule
& Crash & No Crash \\ \addlinespace
\midrule
\endhead
Bulldozer & \(0.05\) & \(0.45\) \\ \addlinespace
Kumquat & \(0.006\) & \(0.294\) \\ \addlinespace
Penguin & \(0.001\) & \(0.199\) \\ \addlinespace
\bottomrule
\end{longtable}

So the probability of a crash is

\[
0.05 + 0.006 + 0.001 = 0.057
\]
\end{frame}

\begin{frame}{Penguin given Crash}
\protect\hypertarget{penguin-given-crash}{}
So the probability of Penguin given Crash is

\[
\frac{\Pr(\text{Penguin-and-Crash})}{\Pr(\text{Crash})} = \frac{0.001}{0.057} \approx 0.0175
\]

\pause

That's really low, because Penguin sites don't crash.
\end{frame}

\begin{frame}{For Next Time}
\protect\hypertarget{for-next-time}{}
\begin{itemize}
\tightlist
\item
  We will look at a formula that some people find helpful for solving
  these kinds of problems.
\end{itemize}
\end{frame}

\end{document}
