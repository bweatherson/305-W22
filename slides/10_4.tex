% Options for packages loaded elsewhere
\PassOptionsToPackage{unicode}{hyperref}
\PassOptionsToPackage{hyphens}{url}
%
\documentclass[
  ignorenonframetext,
]{beamer}
\usepackage{pgfpages}
\setbeamertemplate{caption}[numbered]
\setbeamertemplate{caption label separator}{: }
\setbeamercolor{caption name}{fg=normal text.fg}
\beamertemplatenavigationsymbolsempty
% Prevent slide breaks in the middle of a paragraph
\widowpenalties 1 10000
\raggedbottom
\setbeamertemplate{part page}{
  \centering
  \begin{beamercolorbox}[sep=16pt,center]{part title}
    \usebeamerfont{part title}\insertpart\par
  \end{beamercolorbox}
}
\setbeamertemplate{section page}{
  \centering
  \begin{beamercolorbox}[sep=12pt,center]{part title}
    \usebeamerfont{section title}\insertsection\par
  \end{beamercolorbox}
}
\setbeamertemplate{subsection page}{
  \centering
  \begin{beamercolorbox}[sep=8pt,center]{part title}
    \usebeamerfont{subsection title}\insertsubsection\par
  \end{beamercolorbox}
}
\AtBeginPart{
  \frame{\partpage}
}
\AtBeginSection{
  \ifbibliography
  \else
    \frame{\sectionpage}
  \fi
}
\AtBeginSubsection{
  \frame{\subsectionpage}
}
\usepackage{amsmath,amssymb}
\usepackage{lmodern}
\usepackage{ifxetex,ifluatex}
\ifnum 0\ifxetex 1\fi\ifluatex 1\fi=0 % if pdftex
  \usepackage[T1]{fontenc}
  \usepackage[utf8]{inputenc}
  \usepackage{textcomp} % provide euro and other symbols
\else % if luatex or xetex
  \usepackage{unicode-math}
  \defaultfontfeatures{Scale=MatchLowercase}
  \defaultfontfeatures[\rmfamily]{Ligatures=TeX,Scale=1}
  \setmainfont[BoldFont = SF Pro Rounded Semibold]{SF Pro Rounded}
  \setmathfont[]{STIX Two Math}
\fi
\usefonttheme{serif} % use mainfont rather than sansfont for slide text
% Use upquote if available, for straight quotes in verbatim environments
\IfFileExists{upquote.sty}{\usepackage{upquote}}{}
\IfFileExists{microtype.sty}{% use microtype if available
  \usepackage[]{microtype}
  \UseMicrotypeSet[protrusion]{basicmath} % disable protrusion for tt fonts
}{}
\makeatletter
\@ifundefined{KOMAClassName}{% if non-KOMA class
  \IfFileExists{parskip.sty}{%
    \usepackage{parskip}
  }{% else
    \setlength{\parindent}{0pt}
    \setlength{\parskip}{6pt plus 2pt minus 1pt}}
}{% if KOMA class
  \KOMAoptions{parskip=half}}
\makeatother
\usepackage{xcolor}
\IfFileExists{xurl.sty}{\usepackage{xurl}}{} % add URL line breaks if available
\IfFileExists{bookmark.sty}{\usepackage{bookmark}}{\usepackage{hyperref}}
\hypersetup{
  pdftitle={305 Lecture 10.4 - Subjective Theories of Probability},
  pdfauthor={Brian Weatherson},
  hidelinks,
  pdfcreator={LaTeX via pandoc}}
\urlstyle{same} % disable monospaced font for URLs
\newif\ifbibliography
\setlength{\emergencystretch}{3em} % prevent overfull lines
\providecommand{\tightlist}{%
  \setlength{\itemsep}{0pt}\setlength{\parskip}{0pt}}
\setcounter{secnumdepth}{-\maxdimen} % remove section numbering
\let\Tiny=\tiny

 \setbeamertemplate{navigation symbols}{} 

% \usetheme{Madrid}
 \usetheme[numbering=none, progressbar=foot]{metropolis}
 \usecolortheme{wolverine}
 \usepackage{color}
 \usepackage{MnSymbol}
% \usepackage{movie15}

\usepackage{amssymb}% http://ctan.org/pkg/amssymb
\usepackage{pifont}% http://ctan.org/pkg/pifont
\newcommand{\cmark}{\ding{51}}%
\newcommand{\xmark}{\ding{55}}%

\DeclareSymbolFont{symbolsC}{U}{txsyc}{m}{n}
\DeclareMathSymbol{\boxright}{\mathrel}{symbolsC}{128}
\DeclareMathAlphabet{\mathpzc}{OT1}{pzc}{m}{it}

 \usepackage{tikz-qtree}
% \usepackage{markdown}
%\RequirePackage{bussproofs}
\RequirePackage[tableaux]{prooftrees}
\usetikzlibrary{arrows.meta}
 \forestset{line numbering, close with = x}
% Allow for easy commas inside trees
\renewcommand{\,}{\text{, }}


\usepackage{tabulary}

\usepackage{open-logic-config}

\setlength{\parskip}{1ex plus 0.5ex minus 0.2ex}

\AtBeginSection[]
{
\begin{frame}
	\Huge{\color{darkblue} \insertsection}
\end{frame}
}

\renewenvironment*{quote}	
	{\list{}{\rightmargin   \leftmargin} \item } 	
	{\endlist }

\definecolor{darkgreen}{rgb}{0,0.7,0}
\definecolor{darkblue}{rgb}{0,0,0.8}

\newcommand{\starttab}{\begin{center}
\vspace{6pt}
\begin{tabular}}

\newcommand{\stoptab}{\end{tabular}
\vspace{6pt}
\end{center}
\noindent}


\newcommand{\sif}{\rightarrow}
\newcommand{\siff}{\leftrightarrow}
\newcommand{\EF}{\end{frame}}


\newcommand{\TreeStart}[1]{
%\end{frame}
\begin{frame}
\begin{center}
\begin{tikzpicture}[scale=#1]
\tikzset{every tree node/.style={align=center,anchor=north}}
%\Tree
}

\newcommand{\TreeEnd}{
\end{tikzpicture}
%\end{center}
}

\newcommand{\DisplayArg}[2]{
\begin{enumerate}
{#1}
\end{enumerate}
\vspace{-6pt}
\hrulefill

%\hspace{14pt} #2
%{\addtolength{\leftskip}{14pt} #2}
\begin{quote}
{\normalfont #2}
\end{quote}
\vspace{12pt}
}

\newenvironment{ProofTree}[1][1]{
\begin{center}
\begin{tikzpicture}[scale=#1]
\tikzset{every tree node/.style={align=center,anchor=south}}
}
{
\end{tikzpicture}
\end{center}
}

\newcommand{\TreeFrame}[2]{
\begin{columns}[c]
\column{0.5\textwidth}
\begin{center}
\begin{prooftree}{}
#1
\end{prooftree}
\end{center}
\column{0.45\textwidth}
%\begin{markdown}
#2
%\end{markdown}
\end{columns}
}

\newcommand{\ScaledTreeFrame}[3]{
\begin{columns}[c]
\column{0.5\textwidth}
\begin{center}
\scalebox{#1}{
\begin{prooftree}{}
#2
\end{prooftree}
}
\end{center}
\column{0.45\textwidth}
%\begin{markdown}
#3
%\end{markdown}
\end{columns}
}

\usepackage[bb=boondox]{mathalfa}
\DeclareMathAlphabet{\mathbx}{U}{BOONDOX-ds}{m}{n}
\SetMathAlphabet{\mathbx}{bold}{U}{BOONDOX-ds}{b}{n}
\DeclareMathAlphabet{\mathbbx} {U}{BOONDOX-ds}{b}{n}


\newenvironment{oltableau}{\center\tableau{}} %wff format={anchor = base west}}}
       {\endtableau\endcenter}
       
\newcommand{\formula}[1]{$#1$}

\usepackage{tabulary}
\usepackage{booktabs}

\def\begincols{\begin{columns}}
\def\begincol{\begin{column}}
\def\endcol{\end{column}}
\def\endcols{\end{columns}}

\usepackage[italic]{mathastext}
\usepackage{nicefrac}

\definecolor{mygreen}{RGB}{0, 100, 0}
\definecolor{mypink2}{RGB}{219, 48, 122}
\definecolor{dodgerblue}{RGB}{30,144,255}

%\def\True{\textcolor{dodgerblue}{\text{T}}}
%\def\False{\textcolor{red}{\text{F}}}

\def\True{\mathbb{T}}
\def\False{\mathbb{F}}

% This is because arguments didn't have enough space after them
\usepackage{etoolbox}
\AfterEndEnvironment{description}{\vspace{9pt}}
\AfterEndEnvironment{oltableau}{\vspace{9pt}}
\BeforeBeginEnvironment{oltableau}{\vspace{9pt}}
\AfterEndEnvironment{center}{\vspace{12pt}}
\BeforeBeginEnvironment{tabular}{\vspace{9pt}}

\setlength\heavyrulewidth{0pt}
\setlength\lightrulewidth{0pt}

%\def\toprule{}
%\def\bottomrule{}
%\def\midrule{}

\setbeamertemplate{caption}{\raggedright\insertcaption}

\ifluatex
  \usepackage{selnolig}  % disable illegal ligatures
\fi

\title{305 Lecture 10.4 - Subjective Theories of Probability}
\author{Brian Weatherson}
\date{}

\begin{document}
\frame{\titlepage}

\begin{frame}{Plan}
\protect\hypertarget{plan}{}
\begin{itemize}
\tightlist
\item
  We're going to look at versions of the subjective theory of
  probability
\end{itemize}
\end{frame}

\begin{frame}{Associated Reading}
\protect\hypertarget{associated-reading}{}
Odds and Ends, chapter 16.
\end{frame}

\begin{frame}{Subjective Theory}
\protect\hypertarget{subjective-theory}{}
Probability is degree of confidence.
\end{frame}

\begin{frame}{Two Questions}
\protect\hypertarget{two-questions}{}
\begin{enumerate}
\tightlist
\item
  Whose confidence? \pause
\item
  Actual confidence or idealised confidence?
\end{enumerate}
\end{frame}

\begin{frame}{Actual}
\protect\hypertarget{actual}{}
Question 2 seems easiest to answer.

\begin{itemize}
\tightlist
\item
  No matter how confident you or I am that the moon is made of green
  cheese, it is not probable that the moon is made of green cheese.
\item
  If we're really confident that it is, then we're getting something
  very badly wrong.
\item
  But if probability just was how confident we actually are in
  something, then it would be probable that the moon is made of green
  cheese.
\end{itemize}
\end{frame}

\begin{frame}{Idealised}
\protect\hypertarget{idealised}{}
A better view is, as the textbook says, that

\begin{quote}
Probability is ultimately about belief. It's about how certain you
should be that something is true.
\end{quote}

So not what you actually believe, but what you should believe. This gets
out of the green cheese problem.
\end{frame}

\begin{frame}{Several Challenges}
\protect\hypertarget{several-challenges}{}
\begin{enumerate}
\tightlist
\item
  Why think that `how certain you should be that something is true'
  obeys these rules? \pause
\item
  Who is the `you' in how certain you should be that something is true?
  \pause
\item
  What happens if rationality is \textbf{permissive}; there are several
  attitudes you can rationally have about how likely something is?
  \pause
\item
  Especially if that last one is true, how does probability have a role
  to play in objective science? \pause
\item
  How do we even measure how confident people, real or idealised, are in
  propositions? \pause
\end{enumerate}

We could do weeks on every one of these questions, but I'll focus on
3/4.
\end{frame}

\begin{frame}{Bayesianism}
\protect\hypertarget{bayesianism}{}
The main subjective theory is known as Bayesianism.

\begin{itemize}
\tightlist
\item
  The name comes from the Rev.~Thomas Bayes, an 18th century
  mathematician.
\item
  Just what is and is not a form of Bayesianism is a slightly contested,
  though mostly terminological question.
\end{itemize}
\end{frame}

\begin{frame}{Conditionalisation}
\protect\hypertarget{conditionalisation}{}
The core principle behind Bayesianism is that updating is by
conditionalisation.

\begin{itemize}
\tightlist
\item
  That is, the new \(\Pr(H)\) after getting evidence \(E\) is the old
  \(\Pr(H | E)\).
\item
  And \(\Pr(H | E)\) is given by the formula.
\end{itemize}

\[
\Pr(H | E) = \frac{\Pr(E | H)\Pr(H)}{\Pr(E)}
\]
\end{frame}

\begin{frame}{Problem of the Priors}
\protect\hypertarget{problem-of-the-priors}{}
\begin{itemize}
\tightlist
\item
  That tells you how to convert an old probability into a new
  probability given some evidence \(E\).
\item
  That is, it tells you how to generate a \textbf{posterior}
  probability. \pause
\item
  But it does not tell you where the \textbf{prior} probability comes
  from.
\item
  And this question is one that Bayesians have never quite had a good
  answer to.
\end{itemize}
\end{frame}

\begin{frame}{Green Cheese}
\protect\hypertarget{green-cheese}{}
To make the problem vivid, imagine that I start out with the following
probabilities.

\begin{itemize}
\tightlist
\item
  The probability that the moon is made of green cheese is 0.99.
\item
  The things that happen around here are probabilistically independent
  of hypotheses about the material composition of the moon.
\end{itemize}

The first of these is absurd, but the second isn't ridiculous I suppose.
\end{frame}

\begin{frame}{Updating}
\protect\hypertarget{updating}{}
Now some stuff happens, I get evidence \(E\).

\begin{itemize}
\tightlist
\item
  Let \(H\) be that the moon is made of green cheese.
\item
  Since \(\Pr(H) = 0.99\), and \(H\) is independent of \(E\), it follows
  that \(\Pr(H | E) = 0.99\). \pause
\item
  So the Bayesians' favourite updating rule says that the new
  probability for \(H\) should be 0.99. \pause
\item
  But that's fairly absurd - what could it mean to say that I
  \textbf{should} have probability 0.99 in \(H\)?
\end{itemize}
\end{frame}

\begin{frame}{Two Available Moves Here}
\protect\hypertarget{two-available-moves-here}{}
\begin{enumerate}
\tightlist
\item
  Say what the one true prior is - and say that there are two rules: use
  the one true prior, and update it by conditionalisation. \pause
\item
  Say that there are a large class of acceptable priors, and argue that
  conditionalising any of them with enough evidence will produce a
  rational outcome. \pause
\end{enumerate}

The very rough history of this is that 19th century folks were
sympathetic to option 1, but in the 20th century, most of the focus was
on option 2.
\end{frame}

\begin{frame}{For Next Time}
\protect\hypertarget{for-next-time}{}
\begin{itemize}
\tightlist
\item
  We will take a short look at the `one true prior' option.
\end{itemize}
\end{frame}

\end{document}
