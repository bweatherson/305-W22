% Options for packages loaded elsewhere
\PassOptionsToPackage{unicode}{hyperref}
\PassOptionsToPackage{hyphens}{url}
%
\documentclass[
  ignorenonframetext,
]{beamer}
\usepackage{pgfpages}
\setbeamertemplate{caption}[numbered]
\setbeamertemplate{caption label separator}{: }
\setbeamercolor{caption name}{fg=normal text.fg}
\beamertemplatenavigationsymbolsempty
% Prevent slide breaks in the middle of a paragraph
\widowpenalties 1 10000
\raggedbottom
\setbeamertemplate{part page}{
  \centering
  \begin{beamercolorbox}[sep=16pt,center]{part title}
    \usebeamerfont{part title}\insertpart\par
  \end{beamercolorbox}
}
\setbeamertemplate{section page}{
  \centering
  \begin{beamercolorbox}[sep=12pt,center]{part title}
    \usebeamerfont{section title}\insertsection\par
  \end{beamercolorbox}
}
\setbeamertemplate{subsection page}{
  \centering
  \begin{beamercolorbox}[sep=8pt,center]{part title}
    \usebeamerfont{subsection title}\insertsubsection\par
  \end{beamercolorbox}
}
\AtBeginPart{
  \frame{\partpage}
}
\AtBeginSection{
  \ifbibliography
  \else
    \frame{\sectionpage}
  \fi
}
\AtBeginSubsection{
  \frame{\subsectionpage}
}
\usepackage{amsmath,amssymb}
\usepackage{lmodern}
\usepackage{ifxetex,ifluatex}
\ifnum 0\ifxetex 1\fi\ifluatex 1\fi=0 % if pdftex
  \usepackage[T1]{fontenc}
  \usepackage[utf8]{inputenc}
  \usepackage{textcomp} % provide euro and other symbols
\else % if luatex or xetex
  \usepackage{unicode-math}
  \defaultfontfeatures{Scale=MatchLowercase}
  \defaultfontfeatures[\rmfamily]{Ligatures=TeX,Scale=1}
  \setmainfont[BoldFont = SF Pro Rounded Semibold]{SF Pro Rounded}
  \setmathfont[]{STIX Two Math}
\fi
\usefonttheme{serif} % use mainfont rather than sansfont for slide text
% Use upquote if available, for straight quotes in verbatim environments
\IfFileExists{upquote.sty}{\usepackage{upquote}}{}
\IfFileExists{microtype.sty}{% use microtype if available
  \usepackage[]{microtype}
  \UseMicrotypeSet[protrusion]{basicmath} % disable protrusion for tt fonts
}{}
\makeatletter
\@ifundefined{KOMAClassName}{% if non-KOMA class
  \IfFileExists{parskip.sty}{%
    \usepackage{parskip}
  }{% else
    \setlength{\parindent}{0pt}
    \setlength{\parskip}{6pt plus 2pt minus 1pt}}
}{% if KOMA class
  \KOMAoptions{parskip=half}}
\makeatother
\usepackage{xcolor}
\IfFileExists{xurl.sty}{\usepackage{xurl}}{} % add URL line breaks if available
\IfFileExists{bookmark.sty}{\usepackage{bookmark}}{\usepackage{hyperref}}
\hypersetup{
  pdftitle={305 Lecture 11.6 - Frames and Axioms},
  pdfauthor={Brian Weatherson},
  hidelinks,
  pdfcreator={LaTeX via pandoc}}
\urlstyle{same} % disable monospaced font for URLs
\newif\ifbibliography
\setlength{\emergencystretch}{3em} % prevent overfull lines
\providecommand{\tightlist}{%
  \setlength{\itemsep}{0pt}\setlength{\parskip}{0pt}}
\setcounter{secnumdepth}{-\maxdimen} % remove section numbering
\let\Tiny=\tiny

 \setbeamertemplate{navigation symbols}{} 

% \usetheme{Madrid}
 \usetheme[numbering=none, progressbar=foot]{metropolis}
 \usecolortheme{wolverine}
 \usepackage{color}
 \usepackage{MnSymbol}
% \usepackage{movie15}

\usepackage{amssymb}% http://ctan.org/pkg/amssymb
\usepackage{pifont}% http://ctan.org/pkg/pifont
\newcommand{\cmark}{\ding{51}}%
\newcommand{\xmark}{\ding{55}}%

\DeclareSymbolFont{symbolsC}{U}{txsyc}{m}{n}
\DeclareMathSymbol{\boxright}{\mathrel}{symbolsC}{128}
\DeclareMathAlphabet{\mathpzc}{OT1}{pzc}{m}{it}

\usepackage{tikz-qtree}
% \usepackage{markdown}
%\RequirePackage{bussproofs}
\usetikzlibrary{arrows.meta}
\RequirePackage[tableaux]{prooftrees}
\forestset{line numbering, close with = x}
% Allow for easy commas inside trees
\renewcommand{\,}{\text{, }}


\usepackage{tabulary}

\usepackage{open-logic-config}

\setlength{\parskip}{1ex plus 0.5ex minus 0.2ex}

\AtBeginSection[]
{
\begin{frame}
	\Huge{\color{darkblue} \insertsection}
\end{frame}
}

\renewenvironment*{quote}	
	{\list{}{\rightmargin   \leftmargin} \item } 	
	{\endlist }

\definecolor{darkgreen}{rgb}{0,0.7,0}
\definecolor{darkblue}{rgb}{0,0,0.8}

\newcommand{\starttab}{\begin{center}
\vspace{6pt}
\begin{tabular}}

\newcommand{\stoptab}{\end{tabular}
\vspace{6pt}
\end{center}
\noindent}


\newcommand{\sif}{\rightarrow}
\newcommand{\siff}{\leftrightarrow}
\newcommand{\EF}{\end{frame}}


\newcommand{\TreeStart}[1]{
%\end{frame}
\begin{frame}
\begin{center}
\begin{tikzpicture}[scale=#1]
\tikzset{every tree node/.style={align=center,anchor=north}}
%\Tree
}

\newcommand{\TreeEnd}{
\end{tikzpicture}
%\end{center}
}

\newcommand{\DisplayArg}[2]{
\begin{enumerate}
{#1}
\end{enumerate}
\vspace{-6pt}
\hrulefill

%\hspace{14pt} #2
%{\addtolength{\leftskip}{14pt} #2}
\begin{quote}
{\normalfont #2}
\end{quote}
\vspace{12pt}
}

\newenvironment{ProofTree}[1][1]{
\begin{center}
\begin{tikzpicture}[scale=#1]
\tikzset{every tree node/.style={align=center,anchor=south}}
}
{
\end{tikzpicture}
\end{center}
}

\newcommand{\TreeFrame}[2]{
\begin{columns}[c]
\column{0.5\textwidth}
\begin{center}
\begin{prooftree}{}
#1
\end{prooftree}
\end{center}
\column{0.45\textwidth}
%\begin{markdown}
#2
%\end{markdown}
\end{columns}
}

\newcommand{\ScaledTreeFrame}[3]{
\begin{columns}[c]
\column{0.5\textwidth}
\begin{center}
\scalebox{#1}{
\begin{prooftree}{}
#2
\end{prooftree}
}
\end{center}
\column{0.45\textwidth}
%\begin{markdown}
#3
%\end{markdown}
\end{columns}
}

\usepackage[bb=boondox]{mathalfa}
\DeclareMathAlphabet{\mathbx}{U}{BOONDOX-ds}{m}{n}
\SetMathAlphabet{\mathbx}{bold}{U}{BOONDOX-ds}{b}{n}
\DeclareMathAlphabet{\mathbbx} {U}{BOONDOX-ds}{b}{n}


\newenvironment{oltableau}{\center\tableau{}} %wff format={anchor = base west}}}
       {\endtableau\endcenter}
       
\newcommand{\formula}[1]{$#1$}

\usepackage{tabulary}
\usepackage{booktabs}

\def\begincols{\begin{columns}}
\def\begincol{\begin{column}}
\def\endcol{\end{column}}
\def\endcols{\end{columns}}

\usepackage[italic]{mathastext}
\usepackage{nicefrac}

\definecolor{mygreen}{RGB}{0, 100, 0}
\definecolor{mypink2}{RGB}{219, 48, 122}
\definecolor{dodgerblue}{RGB}{30,144,255}

%\def\True{\textcolor{dodgerblue}{\text{T}}}
%\def\False{\textcolor{red}{\text{F}}}

\def\True{\mathbb{T}}
\def\False{\mathbb{F}}

% This is because arguments didn't have enough space after them
\usepackage{etoolbox}
\AfterEndEnvironment{description}{\vspace{9pt}}
\AfterEndEnvironment{oltableau}{\vspace{9pt}}
\BeforeBeginEnvironment{oltableau}{\vspace{9pt}}
\AfterEndEnvironment{center}{\vspace{12pt}}
\BeforeBeginEnvironment{tabular}{\vspace{9pt}}

\setlength\heavyrulewidth{0pt}
\setlength\lightrulewidth{0pt}

%\def\toprule{}
%\def\bottomrule{}
%\def\midrule{}

\setbeamertemplate{caption}{\raggedright\insertcaption}

\ifluatex
  \usepackage{selnolig}  % disable illegal ligatures
\fi

\title{305 Lecture 11.6 - Frames and Axioms}
\author{Brian Weatherson}
\date{}

\begin{document}
\frame{\titlepage}

\begin{frame}{Plan}
\protect\hypertarget{plan}{}
\begin{itemize}
\tightlist
\item
  To introduce the relationship between frames and axioms.
\end{itemize}
\end{frame}

\begin{frame}{Associated Reading}
\protect\hypertarget{associated-reading}{}
\begin{itemize}
\tightlist
\item
  Boxes and Diamonds, section 4.1 and 4.2.
\end{itemize}
\end{frame}

\begin{frame}{Two Way Equivalence}
\protect\hypertarget{two-way-equivalence}{}
Some conditions on R are strongly tied to dedicated axioms:

\begin{itemize}
\tightlist
\item
  If the condition on R holds, then the axiom is guaranteed to hold.
\item
  If the condition on R does not hold, then there is guaranteed to be a
  way to make the axiom fail.
\end{itemize}

We'll illustrate this with the following pair:

\begin{itemize}
\tightlist
\item
  R is transitive
\item
  \(\Box A \rightarrow \Box \Box A\)
\end{itemize}
\end{frame}

\begin{frame}{What Are Frames}
\protect\hypertarget{what-are-frames}{}
\begin{itemize}
\tightlist
\item
  Frames are models without the valuations.
\item
  They are the \(\langle W, R \rangle\) part of
  \(\langle W, R, V \rangle\).
\end{itemize}
\end{frame}

\begin{frame}{From R to Axiom}
\protect\hypertarget{from-r-to-axiom}{}
In any frame where R is transitive, any instance of this schema is true
at any point in any model on the frame.

\begin{itemize}
\tightlist
\item
  \(\Box A \rightarrow \Box \Box A \pause\)
\end{itemize}

This is a really strong claim. It says you can pick any value you like
for the following three things, and \(\Box A \rightarrow \Box \Box A\)
will be true

\begin{enumerate}
\tightlist
\item
  Any valuation function \(V\) you like.
\item
  Any point \(w\) in \(W\) that you like.
\item
  Any substitution instance for \(A\) that you like.
\end{enumerate}
\end{frame}

\begin{frame}{From Axiom to R}
\protect\hypertarget{from-axiom-to-r}{}
The claim here is in a sense weaker, but still interesting. It says that
if the R in \(\langle W, R \rangle\) is not transitive, then there is
some way to make \(\Box A \rightarrow \Box \Box A\) false. \pause  That
is, if you pick the right values for the following three things:

\begin{enumerate}
\tightlist
\item
  The point \(w\) in \(W\);
\item
  The substitution instance for \(A\)
\item
  The valuation function \(V\);
\end{enumerate}

then \(\Box A \rightarrow \Box \Box A\) will be false.
\end{frame}

\begin{frame}{Picking the Substitution Instance}
\protect\hypertarget{picking-the-substitution-instance}{}
In this case, it isn't that hard to get all three.

\begin{enumerate}
\tightlist
\item
  Let \(w\) be a point in \(W\) such that for some \(x, y\), \(wRx\) and
  \(xRy\), but not \(wRy\). (Since \(R\) is not transitive, we know such
  a point exists.)
\item
  Let \(A\) be \(p\), i.e., an atomic sentence.
\item
  Let \(V\) make \(p\) true at \(z\) if and only if \(wRz \pause\).
\end{enumerate}

Then at \(w\), \(\Box p\) will be true (since \(p\) is true everywhere
that \(w\) can access), but \(\Box \Box p\) will be false. That's
because we know that \(p\) is false at the point \(y\) mentioned in 1.
\end{frame}

\begin{frame}{Reflexivity}
\protect\hypertarget{reflexivity}{}
The same equivalence holds between

\begin{itemize}
\tightlist
\item
  \(R\) is reflexive, i.e., for all \(x, xRx\).
\item
  The axiom \(\Box A \rightarrow A\) is valid on the frame.
\end{itemize}

\begin{description}
\tightlist
\item[Left-to-right]
If \(R\) is reflexive, then \(x\) itself is one of the accessible
worlds, so if \(\Box A\) is true at \(x\), then \(A\) must also be true
at \(x\).
\item[Right-to-left]
If \(R\) is not reflexive, then there is a point \(x\) such that
\(\neg xRx\). Let \(A\) be true everywhere but \(x\). Then \(\Box A\)
will be true at \(x\), since \(A\) is true at all the points it can
`see', but \(A\) is false.
\end{description}
\end{frame}

\begin{frame}{Symmetry}
\protect\hypertarget{symmetry}{}
The same equivalence holds between

\begin{itemize}
\tightlist
\item
  \(R\) is symmetric, i.e., for all \(x, y, xRy \rightarrow yRx\).
\item
  The axiom \(A \rightarrow \Box \Diamond A\) is valid on the frame.
\end{itemize}

\begin{description}
\tightlist
\item[Left-to-right]
If \(R\) is symmetric, then everywhere \(x\) can see can also see \(x\).
So if \(A\) is true at \(x\), then \(\Diamond A\) is true at all those
worlds. So \(\Box \Diamond A\) is true at \(x\).
\item[Right-to-left]
If \(R\) is not symmetric, then there is a pair \(x, y\) such that
\(xRy \wedge \neg yRx\). Let \(A\) be true only at \(x\). Since
\(\neg yRx\), \(\Diamond A\) is false at \(y\). And since \(xRy\),
\(\Box \Diamond A\) is false at \(x\).
\end{description}
\end{frame}

\begin{frame}{For Next Time}
\protect\hypertarget{for-next-time}{}
We'll talk about the philosophical significance of these results.
\end{frame}

\end{document}
