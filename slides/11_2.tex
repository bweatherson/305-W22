% Options for packages loaded elsewhere
\PassOptionsToPackage{unicode}{hyperref}
\PassOptionsToPackage{hyphens}{url}
%
\documentclass[
  ignorenonframetext,
]{beamer}
\usepackage{pgfpages}
\setbeamertemplate{caption}[numbered]
\setbeamertemplate{caption label separator}{: }
\setbeamercolor{caption name}{fg=normal text.fg}
\beamertemplatenavigationsymbolsempty
% Prevent slide breaks in the middle of a paragraph
\widowpenalties 1 10000
\raggedbottom
\setbeamertemplate{part page}{
  \centering
  \begin{beamercolorbox}[sep=16pt,center]{part title}
    \usebeamerfont{part title}\insertpart\par
  \end{beamercolorbox}
}
\setbeamertemplate{section page}{
  \centering
  \begin{beamercolorbox}[sep=12pt,center]{part title}
    \usebeamerfont{section title}\insertsection\par
  \end{beamercolorbox}
}
\setbeamertemplate{subsection page}{
  \centering
  \begin{beamercolorbox}[sep=8pt,center]{part title}
    \usebeamerfont{subsection title}\insertsubsection\par
  \end{beamercolorbox}
}
\AtBeginPart{
  \frame{\partpage}
}
\AtBeginSection{
  \ifbibliography
  \else
    \frame{\sectionpage}
  \fi
}
\AtBeginSubsection{
  \frame{\subsectionpage}
}
\usepackage{amsmath,amssymb}
\usepackage{lmodern}
\usepackage{ifxetex,ifluatex}
\ifnum 0\ifxetex 1\fi\ifluatex 1\fi=0 % if pdftex
  \usepackage[T1]{fontenc}
  \usepackage[utf8]{inputenc}
  \usepackage{textcomp} % provide euro and other symbols
\else % if luatex or xetex
  \usepackage{unicode-math}
  \defaultfontfeatures{Scale=MatchLowercase}
  \defaultfontfeatures[\rmfamily]{Ligatures=TeX,Scale=1}
  \setmainfont[BoldFont = SF Pro Rounded Semibold]{SF Pro Rounded}
  \setmathfont[]{STIX Two Math}
\fi
\usefonttheme{serif} % use mainfont rather than sansfont for slide text
% Use upquote if available, for straight quotes in verbatim environments
\IfFileExists{upquote.sty}{\usepackage{upquote}}{}
\IfFileExists{microtype.sty}{% use microtype if available
  \usepackage[]{microtype}
  \UseMicrotypeSet[protrusion]{basicmath} % disable protrusion for tt fonts
}{}
\makeatletter
\@ifundefined{KOMAClassName}{% if non-KOMA class
  \IfFileExists{parskip.sty}{%
    \usepackage{parskip}
  }{% else
    \setlength{\parindent}{0pt}
    \setlength{\parskip}{6pt plus 2pt minus 1pt}}
}{% if KOMA class
  \KOMAoptions{parskip=half}}
\makeatother
\usepackage{xcolor}
\IfFileExists{xurl.sty}{\usepackage{xurl}}{} % add URL line breaks if available
\IfFileExists{bookmark.sty}{\usepackage{bookmark}}{\usepackage{hyperref}}
\hypersetup{
  pdftitle={305 Lecture 11.2 - Truth in Modal Logic},
  pdfauthor={Brian Weatherson},
  hidelinks,
  pdfcreator={LaTeX via pandoc}}
\urlstyle{same} % disable monospaced font for URLs
\newif\ifbibliography
\setlength{\emergencystretch}{3em} % prevent overfull lines
\providecommand{\tightlist}{%
  \setlength{\itemsep}{0pt}\setlength{\parskip}{0pt}}
\setcounter{secnumdepth}{-\maxdimen} % remove section numbering
\let\Tiny=\tiny

 \setbeamertemplate{navigation symbols}{} 

% \usetheme{Madrid}
 \usetheme[numbering=none, progressbar=foot]{metropolis}
 \usecolortheme{wolverine}
 \usepackage{color}
 \usepackage{MnSymbol}
% \usepackage{movie15}

\usepackage{amssymb}% http://ctan.org/pkg/amssymb
\usepackage{pifont}% http://ctan.org/pkg/pifont
\newcommand{\cmark}{\ding{51}}%
\newcommand{\xmark}{\ding{55}}%

\DeclareSymbolFont{symbolsC}{U}{txsyc}{m}{n}
\DeclareMathSymbol{\boxright}{\mathrel}{symbolsC}{128}
\DeclareMathAlphabet{\mathpzc}{OT1}{pzc}{m}{it}

 \usepackage{tikz-qtree}
% \usepackage{markdown}
%\RequirePackage{bussproofs}
\RequirePackage[tableaux]{prooftrees}
\usetikzlibrary{arrows.meta}
 \forestset{line numbering, close with = x}
% Allow for easy commas inside trees
\renewcommand{\,}{\text{, }}


\usepackage{tabulary}

\usepackage{open-logic-config}

\setlength{\parskip}{1ex plus 0.5ex minus 0.2ex}

\AtBeginSection[]
{
\begin{frame}
	\Huge{\color{darkblue} \insertsection}
\end{frame}
}

\renewenvironment*{quote}	
	{\list{}{\rightmargin   \leftmargin} \item } 	
	{\endlist }

\definecolor{darkgreen}{rgb}{0,0.7,0}
\definecolor{darkblue}{rgb}{0,0,0.8}

\newcommand{\starttab}{\begin{center}
\vspace{6pt}
\begin{tabular}}

\newcommand{\stoptab}{\end{tabular}
\vspace{6pt}
\end{center}
\noindent}


\newcommand{\sif}{\rightarrow}
\newcommand{\siff}{\leftrightarrow}
\newcommand{\EF}{\end{frame}}


\newcommand{\TreeStart}[1]{
%\end{frame}
\begin{frame}
\begin{center}
\begin{tikzpicture}[scale=#1]
\tikzset{every tree node/.style={align=center,anchor=north}}
%\Tree
}

\newcommand{\TreeEnd}{
\end{tikzpicture}
%\end{center}
}

\newcommand{\DisplayArg}[2]{
\begin{enumerate}
{#1}
\end{enumerate}
\vspace{-6pt}
\hrulefill

%\hspace{14pt} #2
%{\addtolength{\leftskip}{14pt} #2}
\begin{quote}
{\normalfont #2}
\end{quote}
\vspace{12pt}
}

\newenvironment{ProofTree}[1][1]{
\begin{center}
\begin{tikzpicture}[scale=#1]
\tikzset{every tree node/.style={align=center,anchor=south}}
}
{
\end{tikzpicture}
\end{center}
}

\newcommand{\TreeFrame}[2]{
\begin{columns}[c]
\column{0.5\textwidth}
\begin{center}
\begin{prooftree}{}
#1
\end{prooftree}
\end{center}
\column{0.45\textwidth}
%\begin{markdown}
#2
%\end{markdown}
\end{columns}
}

\newcommand{\ScaledTreeFrame}[3]{
\begin{columns}[c]
\column{0.5\textwidth}
\begin{center}
\scalebox{#1}{
\begin{prooftree}{}
#2
\end{prooftree}
}
\end{center}
\column{0.45\textwidth}
%\begin{markdown}
#3
%\end{markdown}
\end{columns}
}

\usepackage[bb=boondox]{mathalfa}
\DeclareMathAlphabet{\mathbx}{U}{BOONDOX-ds}{m}{n}
\SetMathAlphabet{\mathbx}{bold}{U}{BOONDOX-ds}{b}{n}
\DeclareMathAlphabet{\mathbbx} {U}{BOONDOX-ds}{b}{n}


\newenvironment{oltableau}{\center\tableau{}} %wff format={anchor = base west}}}
       {\endtableau\endcenter}
       
\newcommand{\formula}[1]{$#1$}

\usepackage{tabulary}
\usepackage{booktabs}

\def\begincols{\begin{columns}}
\def\begincol{\begin{column}}
\def\endcol{\end{column}}
\def\endcols{\end{columns}}

\usepackage[italic]{mathastext}
\usepackage{nicefrac}

\definecolor{mygreen}{RGB}{0, 100, 0}
\definecolor{mypink2}{RGB}{219, 48, 122}
\definecolor{dodgerblue}{RGB}{30,144,255}

%\def\True{\textcolor{dodgerblue}{\text{T}}}
%\def\False{\textcolor{red}{\text{F}}}

\def\True{\mathbb{T}}
\def\False{\mathbb{F}}

% This is because arguments didn't have enough space after them
\usepackage{etoolbox}
\AfterEndEnvironment{description}{\vspace{9pt}}
\AfterEndEnvironment{oltableau}{\vspace{9pt}}
\BeforeBeginEnvironment{oltableau}{\vspace{9pt}}
\AfterEndEnvironment{center}{\vspace{12pt}}
\BeforeBeginEnvironment{tabular}{\vspace{9pt}}

\setlength\heavyrulewidth{0pt}
\setlength\lightrulewidth{0pt}

%\def\toprule{}
%\def\bottomrule{}
%\def\midrule{}

\setbeamertemplate{caption}{\raggedright\insertcaption}

\ifluatex
  \usepackage{selnolig}  % disable illegal ligatures
\fi

\title{305 Lecture 11.2 - Truth in Modal Logic}
\author{Brian Weatherson}
\date{}

\begin{document}
\frame{\titlepage}

\begin{frame}{Plan}
\protect\hypertarget{plan}{}
\begin{itemize}
\tightlist
\item
  To talk about what a model for a modal logic is.
\end{itemize}
\end{frame}

\begin{frame}{Associated Reading}
\protect\hypertarget{associated-reading}{}
\begin{itemize}
\tightlist
\item
  Boxes and Diamonds, section 3.3 and 3.4.
\end{itemize}
\end{frame}

\begin{frame}{Worlds}
\protect\hypertarget{worlds}{}
We start with Leibniz's idea that necessity is truth in all possible
worlds.

\begin{itemize}
\tightlist
\item
  Leibniz was interested in metaphysical necessity, so we'll have to
  qualify this a little, but it's a good idea.
\item
  So instead of saying that each proposition simply has a truth value,
  we'll say that there are many \textbf{worlds}, and at each world each
  proposition has a truth value.
\item
  But don't assume that propositions have the same truth value at each
  world.
\item
  In one world I might be standing, and in another world I might be
  sitting.
\end{itemize}
\end{frame}

\begin{frame}{What Are Worlds}
\protect\hypertarget{what-are-worlds}{}
We are well and truly not going to get into the metaphysics of worlds
here.

\begin{itemize}
\tightlist
\item
  Indeed, they need not even be anything like possible worlds in the
  sense that metaphysicians usually care about.
\item
  They might, for instance, be different times.
\item
  All we care about is that they are things at which propositions can be
  true or false.
\end{itemize}
\end{frame}

\begin{frame}{Valuations}
\protect\hypertarget{valuations}{}
A valuation function tells us which worlds atomic sentences are true at.

\begin{itemize}
\tightlist
\item
  These can be completely arbitrary; we don't put any restrictions on
  them.
\end{itemize}
\end{frame}

\begin{frame}{Truth at a World}
\protect\hypertarget{truth-at-a-world}{}
We want more generally a function that tells us whether a sentence is
true at a particular world.

\begin{itemize}
\tightlist
\item
  For sentences built up using \(\wedge, \vee, \rightarrow, \neg\), this
  is relatively easy.
\item
  We just keep on using truth tables.
\item
  So if at world \(w\), \(A\) is true and \(B\) is false, then
  \(A \wedge B\) is false and \(A \vee B\) is true.
\end{itemize}
\end{frame}

\begin{frame}{Modal Values}
\protect\hypertarget{modal-values}{}
We also need values for these sentences:

\begin{itemize}
\tightlist
\item
  \(\Box A\)
\item
  \(\Diamond A\)
\end{itemize}

It turns out these are more complicated - but not much more complicated.
\end{frame}

\begin{frame}{Accessibility}
\protect\hypertarget{accessibility}{}
The last part of our model is an \textbf{accessibility} relation between
worlds.

\begin{itemize}
\tightlist
\item
  Again, this can be completely arbitrary.
\item
  We don't yet put any restrictions on it.
\item
  Notably, we don't assume that it is \textbf{reflexive},
  \textbf{symmetric} or \textbf{transitive}
\end{itemize}
\end{frame}

\begin{frame}{Properties of Relations}
\protect\hypertarget{properties-of-relations}{}
\begin{itemize}
\tightlist
\item
  \(R\) is reflexive iff for all \(x\), \(xRx\).\pause
\item
  \(R\) is symmetric iff for all \(x, y\), if \(xRy\) then
  \(yRx\).\pause
\item
  \(R\) is transitive iff for all \(x, y, z\) if \(xRy\) and \(yRz\)
  then \(xRz\).\pause
\end{itemize}

A lot of relations we care about have one or more of these properties,
but not all do. Consider, for example, \textbf{admires} as an example of
a relation with none of them.
\end{frame}

\begin{frame}{Truth of Modal Formulas}
\protect\hypertarget{truth-of-modal-formulas}{}
A sentence \(\Box A\) is true at a world \(x\) just in case the
following condition is met:

\begin{itemize}
\tightlist
\item
  For all worlds \(y\) such that \(xRy\), \(A\) is true at world
  \(y\).\pause
\end{itemize}

A sentence \(\Diamond A\) is true at a world \(x\) just in case the
following condition is met:

\begin{itemize}
\tightlist
\item
  For some world \(y\) such that \(xRy\), \(A\) is true at world \(y\).
\end{itemize}
\end{frame}

\begin{frame}{Modal Truth}
\protect\hypertarget{modal-truth}{}
\begin{itemize}
\tightlist
\item
  Something is necessarily true iff it is true everywhere that is
  accessible.
\item
  Something is possibly true iff it is true somewhere accessible. \pause
\end{itemize}

We get back the Leibnizian idea that necessity is truth in all possible
worlds if we assume the accessibility relation is the universal
relation, i.e., \(xRy\) for all \(x, y\).
\end{frame}

\begin{frame}{Metaphysical Necessity}
\protect\hypertarget{metaphysical-necessity}{}
On this Leibnizian model, where all worlds can access all worlds,
iterated modalities are rather uninteresting. These three sentences are
true in the same worlds/models.

\begin{enumerate}
\tightlist
\item
  \(\Box A\)
\item
  \(\Box \Box A\)
\item
  \(\Diamond \Box A\)
\end{enumerate}

That's because if \(\Box A\) is true at any world, then it is true at
all worlds. In the general case, where we do not assume that \(R\) is
universal, these are not equivalent.
\end{frame}

\begin{frame}{For Next Time}
\protect\hypertarget{for-next-time}{}
We'll talk about the relationship between boxes and diamonds.
\end{frame}

\end{document}
