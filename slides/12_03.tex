% Options for packages loaded elsewhere
\PassOptionsToPackage{unicode}{hyperref}
\PassOptionsToPackage{hyphens}{url}
%
\documentclass[
  ignorenonframetext,
]{beamer}
\usepackage{pgfpages}
\setbeamertemplate{caption}[numbered]
\setbeamertemplate{caption label separator}{: }
\setbeamercolor{caption name}{fg=normal text.fg}
\beamertemplatenavigationsymbolsempty
% Prevent slide breaks in the middle of a paragraph
\widowpenalties 1 10000
\raggedbottom
\setbeamertemplate{part page}{
  \centering
  \begin{beamercolorbox}[sep=16pt,center]{part title}
    \usebeamerfont{part title}\insertpart\par
  \end{beamercolorbox}
}
\setbeamertemplate{section page}{
  \centering
  \begin{beamercolorbox}[sep=12pt,center]{part title}
    \usebeamerfont{section title}\insertsection\par
  \end{beamercolorbox}
}
\setbeamertemplate{subsection page}{
  \centering
  \begin{beamercolorbox}[sep=8pt,center]{part title}
    \usebeamerfont{subsection title}\insertsubsection\par
  \end{beamercolorbox}
}
\AtBeginPart{
  \frame{\partpage}
}
\AtBeginSection{
  \ifbibliography
  \else
    \frame{\sectionpage}
  \fi
}
\AtBeginSubsection{
  \frame{\subsectionpage}
}
\usepackage{amsmath,amssymb}
\usepackage{lmodern}
\usepackage{ifxetex,ifluatex}
\ifnum 0\ifxetex 1\fi\ifluatex 1\fi=0 % if pdftex
  \usepackage[T1]{fontenc}
  \usepackage[utf8]{inputenc}
  \usepackage{textcomp} % provide euro and other symbols
\else % if luatex or xetex
  \usepackage{unicode-math}
  \defaultfontfeatures{Scale=MatchLowercase}
  \defaultfontfeatures[\rmfamily]{Ligatures=TeX,Scale=1}
  \setmainfont[BoldFont = SF Pro Rounded Semibold]{SF Pro Rounded}
  \setmathfont[]{STIX Two Math}
\fi
\usefonttheme{serif} % use mainfont rather than sansfont for slide text
% Use upquote if available, for straight quotes in verbatim environments
\IfFileExists{upquote.sty}{\usepackage{upquote}}{}
\IfFileExists{microtype.sty}{% use microtype if available
  \usepackage[]{microtype}
  \UseMicrotypeSet[protrusion]{basicmath} % disable protrusion for tt fonts
}{}
\makeatletter
\@ifundefined{KOMAClassName}{% if non-KOMA class
  \IfFileExists{parskip.sty}{%
    \usepackage{parskip}
  }{% else
    \setlength{\parindent}{0pt}
    \setlength{\parskip}{6pt plus 2pt minus 1pt}}
}{% if KOMA class
  \KOMAoptions{parskip=half}}
\makeatother
\usepackage{xcolor}
\IfFileExists{xurl.sty}{\usepackage{xurl}}{} % add URL line breaks if available
\IfFileExists{bookmark.sty}{\usepackage{bookmark}}{\usepackage{hyperref}}
\hypersetup{
  pdftitle={305 Lecture 12.3 - Differences Between Modal Logics},
  pdfauthor={Brian Weatherson},
  hidelinks,
  pdfcreator={LaTeX via pandoc}}
\urlstyle{same} % disable monospaced font for URLs
\newif\ifbibliography
\setlength{\emergencystretch}{3em} % prevent overfull lines
\providecommand{\tightlist}{%
  \setlength{\itemsep}{0pt}\setlength{\parskip}{0pt}}
\setcounter{secnumdepth}{-\maxdimen} % remove section numbering
\let\Tiny=\tiny

 \setbeamertemplate{navigation symbols}{} 

% \usetheme{Madrid}
 \usetheme[numbering=none, progressbar=foot]{metropolis}
 \usecolortheme{wolverine}
 \usepackage{color}
 \usepackage{MnSymbol}
% \usepackage{movie15}

\usepackage{amssymb}% http://ctan.org/pkg/amssymb
\usepackage{pifont}% http://ctan.org/pkg/pifont
\newcommand{\cmark}{\ding{51}}%
\newcommand{\xmark}{\ding{55}}%

\DeclareSymbolFont{symbolsC}{U}{txsyc}{m}{n}
\DeclareMathSymbol{\boxright}{\mathrel}{symbolsC}{128}
\DeclareMathAlphabet{\mathpzc}{OT1}{pzc}{m}{it}

\usepackage{tikz-qtree}
% \usepackage{markdown}
%\RequirePackage{bussproofs}
\usetikzlibrary{arrows.meta}
\RequirePackage[tableaux]{prooftrees}
\forestset{line numbering, close with = x}
% Allow for easy commas inside trees
\renewcommand{\,}{\text{, }}


\usepackage{tabulary}

\usepackage{open-logic-config}

\setlength{\parskip}{1ex plus 0.5ex minus 0.2ex}

\AtBeginSection[]
{
\begin{frame}
	\Huge{\color{darkblue} \insertsection}
\end{frame}
}

\renewenvironment*{quote}	
	{\list{}{\rightmargin   \leftmargin} \item } 	
	{\endlist }

\definecolor{darkgreen}{rgb}{0,0.7,0}
\definecolor{darkblue}{rgb}{0,0,0.8}

\newcommand{\starttab}{\begin{center}
\vspace{6pt}
\begin{tabular}}

\newcommand{\stoptab}{\end{tabular}
\vspace{6pt}
\end{center}
\noindent}


\newcommand{\sif}{\rightarrow}
\newcommand{\siff}{\leftrightarrow}
\newcommand{\EF}{\end{frame}}


\newcommand{\TreeStart}[1]{
%\end{frame}
\begin{frame}
\begin{center}
\begin{tikzpicture}[scale=#1]
\tikzset{every tree node/.style={align=center,anchor=north}}
%\Tree
}

\newcommand{\TreeEnd}{
\end{tikzpicture}
%\end{center}
}

\newcommand{\DisplayArg}[2]{
\begin{enumerate}
{#1}
\end{enumerate}
\vspace{-6pt}
\hrulefill

%\hspace{14pt} #2
%{\addtolength{\leftskip}{14pt} #2}
\begin{quote}
{\normalfont #2}
\end{quote}
\vspace{12pt}
}

\newenvironment{ProofTree}[1][1]{
\begin{center}
\begin{tikzpicture}[scale=#1]
\tikzset{every tree node/.style={align=center,anchor=south}}
}
{
\end{tikzpicture}
\end{center}
}

\newcommand{\TreeFrame}[2]{
\begin{columns}[c]
\column{0.5\textwidth}
\begin{center}
\begin{prooftree}{}
#1
\end{prooftree}
\end{center}
\column{0.45\textwidth}
%\begin{markdown}
#2
%\end{markdown}
\end{columns}
}

\newcommand{\ScaledTreeFrame}[3]{
\begin{columns}[c]
\column{0.5\textwidth}
\begin{center}
\scalebox{#1}{
\begin{prooftree}{}
#2
\end{prooftree}
}
\end{center}
\column{0.45\textwidth}
%\begin{markdown}
#3
%\end{markdown}
\end{columns}
}

\usepackage[bb=boondox]{mathalfa}
\DeclareMathAlphabet{\mathbx}{U}{BOONDOX-ds}{m}{n}
\SetMathAlphabet{\mathbx}{bold}{U}{BOONDOX-ds}{b}{n}
\DeclareMathAlphabet{\mathbbx} {U}{BOONDOX-ds}{b}{n}


\newenvironment{oltableau}{\center\tableau{}} %wff format={anchor = base west}}}
       {\endtableau\endcenter}
       
\newcommand{\formula}[1]{$#1$}

\usepackage{tabulary}
\usepackage{booktabs}

\def\begincols{\begin{columns}}
\def\begincol{\begin{column}}
\def\endcol{\end{column}}
\def\endcols{\end{columns}}

\usepackage[italic]{mathastext}
\usepackage{nicefrac}

\definecolor{mygreen}{RGB}{0, 100, 0}
\definecolor{mypink2}{RGB}{219, 48, 122}
\definecolor{dodgerblue}{RGB}{30,144,255}

%\def\True{\textcolor{dodgerblue}{\text{T}}}
%\def\False{\textcolor{red}{\text{F}}}

\def\True{\mathbb{T}}
\def\False{\mathbb{F}}

% This is because arguments didn't have enough space after them
\usepackage{etoolbox}
\AfterEndEnvironment{description}{\vspace{9pt}}
\AfterEndEnvironment{oltableau}{\vspace{9pt}}
\BeforeBeginEnvironment{oltableau}{\vspace{9pt}}
\AfterEndEnvironment{center}{\vspace{12pt}}
\BeforeBeginEnvironment{tabular}{\vspace{9pt}}

\setlength\heavyrulewidth{0pt}
\setlength\lightrulewidth{0pt}

%\def\toprule{}
%\def\bottomrule{}
%\def\midrule{}

\setbeamertemplate{caption}{\raggedright\insertcaption}

\ifluatex
  \usepackage{selnolig}  % disable illegal ligatures
\fi

\title{305 Lecture 12.3 - Differences Between Modal Logics}
\author{Brian Weatherson}
\date{}

\begin{document}
\frame{\titlepage}

\begin{frame}{Plan}
\protect\hypertarget{plan}{}
\begin{itemize}
\tightlist
\item
  To look at the differences between different modal logics.
\end{itemize}
\end{frame}

\begin{frame}{Associated Reading}
\protect\hypertarget{associated-reading}{}
\begin{itemize}
\tightlist
\item
  Boxes and Diamonds, section 5.7.
\end{itemize}
\end{frame}

\begin{frame}{Many Logics}
\protect\hypertarget{many-logics}{}
\begin{itemize}
\tightlist
\item
  As we've said before, there is no such thing as modal logic
  (singular).
\item
  Rather, there are modal logics (plural).
\item
  How do we refer to them?
\end{itemize}
\end{frame}

\begin{frame}{Old Way - Axioms}
\protect\hypertarget{old-way---axioms}{}
Here's how we traditionally identified \textbf{K}, the basic modal
logic.

\begin{enumerate}
\tightlist
\item
  The logic just is the set of theorems (or logical truths).
\item
  Any theorem of propositional logic is a theorem.
\item
  If \(X\) and \(X \rightarrow Y\) are theorems, so is \(Y\).
\item
  If \(X\) is a theorem, so is \(\Box X\).
\item
  Any instance of
  \((\Box A \rightarrow \Box B) \rightarrow \Box (A \rightarrow B)\) is
  a theorem.
\end{enumerate}
\end{frame}

\begin{frame}{More Logics}
\protect\hypertarget{more-logics}{}
We extend that by adding new things after 5. Here are some things we
could add.

\begin{itemize}
\tightlist
\item
  Any instance of \(\Box A \rightarrow A\) is a theorem. (Call this
  \textbf{T}.)
\item
  Any instance of \(\Box A \rightarrow \Box \Box A\) is a theorem. (Call
  this \textbf{4}.)
\item
  Any instance of \(A \rightarrow \Box \Diamond A\) is a theorem. (Call
  this \textbf{B}.)
\end{itemize}

The logic KTB is what you get by adding the first and third of these to
K. The logic K4 is what you get by adding just the second, and so on.
\end{frame}

\begin{frame}{New Way - Models}
\protect\hypertarget{new-way---models}{}
But the axiomatic approach is often hard to work with.

\begin{itemize}
\tightlist
\item
  Proving even quite simple things this way can be a pain.
\item
  And figuring out which instances of these axioms to appeal to is hard
  to automate.
\item
  So instead we focus on the classes of models.
\end{itemize}
\end{frame}

\begin{frame}{Back to K}
\protect\hypertarget{back-to-k}{}
The basic logic \textbf{K} is defined as follows.

\begin{itemize}
\tightlist
\item
  A logic is still a set of theorems.
\item
  Something is a theorem of \textbf{K} just in case it is true at all
  points in all models \(\langle W, R, V \rangle\).
\end{itemize}
\end{frame}

\begin{frame}{Adding Things}
\protect\hypertarget{adding-things}{}
We get other logics by putting \emph{restrictions} on R.

\begin{itemize}
\tightlist
\item
  We could put on the restriction that R is reflexive (everything points
  to itself). Call this \textbf{T}.
\item
  We could put on the restriction that R is transitive (if you can get
  somewhere in two steps, you can get there in one). Call this
  \textbf{4}.
\item
  We could put on the restriction that R is symmetric (if you can get
  somewhere, you can get back.) Call this \textbf{B}.
\end{itemize}
\end{frame}

\begin{frame}{Naming Logics Again}
\protect\hypertarget{naming-logics-again}{}
And we can follow the same naming convention.

\begin{itemize}
\tightlist
\item
  KT4 is the logic of frames that are both reflexive (T) and transitive
  (4).
\item
  KB is the logic of frames that are symmetric.
\end{itemize}
\end{frame}

\begin{frame}{Two Special Names}
\protect\hypertarget{two-special-names}{}
There are two names that don't fit this convention, because they come
from a different historical tradition. This is the tradition that goes
back to Lewis \& Langford's 1932 discussion of modal logic, where they
introduced 5 logics called S1 through S5.

\begin{itemize}
\tightlist
\item
  S4 = KT4.
\item
  S5 = KT4B.
\end{itemize}
\end{frame}

\begin{frame}{Restrictions Imply More Rules}
\protect\hypertarget{restrictions-imply-more-rules}{}
When you put restrictions on the class of frames, more things are
guaranteed to be true.

\begin{itemize}
\tightlist
\item
  Generally, it's easier for something to be true throughout a smaller
  class of models than throughout a larger class.
\item
  So we need new rules for the restrictions.
\item
  These supplement, not replace, the old rules.
\item
  If you have two restrictions, you get two new sets of rules.
\end{itemize}
\end{frame}

\begin{frame}{Rules for T}
\protect\hypertarget{rules-for-t}{}
\begin{itemize}
\tightlist
\item
  If \(\Box A\) is true at \(x\), then \(A\) is true at \(x\).
\item
  If \(\Diamond A\) is false at \(x\), then \(A\) is false at \(x\).
\end{itemize}
\end{frame}

\begin{frame}{Rules for 4}
\protect\hypertarget{rules-for-4}{}
\begin{itemize}
\tightlist
\item
  If \(\Box A\) is true at \(x\), and \(x.y\) appears on the branch,
  then \(\Box A\) is true at \(x.y\).
\item
  If \(\Diamond A\) is false at \(x\), and \(x.y\) appears on the
  branch, then \(\Diamond A\) is false at \(x.y\)
\end{itemize}
\end{frame}

\begin{frame}{Rules for B}
\protect\hypertarget{rules-for-b}{}
\begin{itemize}
\tightlist
\item
  If \(\Box A\) is true at \(x.y\), then \(A\) is true at \(x\).
\item
  If \(\Diamond A\) is false at \(x.y\) then \(A\) is false at \(x\).
\end{itemize}
\end{frame}

\begin{frame}{For Next Time}
\protect\hypertarget{for-next-time}{}
We'll look at the special rules for the special logic S5.
\end{frame}

\end{document}
