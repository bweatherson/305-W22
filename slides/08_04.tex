% Options for packages loaded elsewhere
\PassOptionsToPackage{unicode}{hyperref}
\PassOptionsToPackage{hyphens}{url}
%
\documentclass[
  ignorenonframetext,
]{beamer}
\usepackage{pgfpages}
\setbeamertemplate{caption}[numbered]
\setbeamertemplate{caption label separator}{: }
\setbeamercolor{caption name}{fg=normal text.fg}
\beamertemplatenavigationsymbolsempty
% Prevent slide breaks in the middle of a paragraph
\widowpenalties 1 10000
\raggedbottom
\setbeamertemplate{part page}{
  \centering
  \begin{beamercolorbox}[sep=16pt,center]{part title}
    \usebeamerfont{part title}\insertpart\par
  \end{beamercolorbox}
}
\setbeamertemplate{section page}{
  \centering
  \begin{beamercolorbox}[sep=12pt,center]{part title}
    \usebeamerfont{section title}\insertsection\par
  \end{beamercolorbox}
}
\setbeamertemplate{subsection page}{
  \centering
  \begin{beamercolorbox}[sep=8pt,center]{part title}
    \usebeamerfont{subsection title}\insertsubsection\par
  \end{beamercolorbox}
}
\AtBeginPart{
  \frame{\partpage}
}
\AtBeginSection{
  \ifbibliography
  \else
    \frame{\sectionpage}
  \fi
}
\AtBeginSubsection{
  \frame{\subsectionpage}
}
\usepackage{amsmath,amssymb}
\usepackage{lmodern}
\usepackage{ifxetex,ifluatex}
\ifnum 0\ifxetex 1\fi\ifluatex 1\fi=0 % if pdftex
  \usepackage[T1]{fontenc}
  \usepackage[utf8]{inputenc}
  \usepackage{textcomp} % provide euro and other symbols
\else % if luatex or xetex
  \usepackage{unicode-math}
  \defaultfontfeatures{Scale=MatchLowercase}
  \defaultfontfeatures[\rmfamily]{Ligatures=TeX,Scale=1}
  \setmainfont[BoldFont = SF Pro Rounded Semibold]{SF Pro Rounded}
  \setmathfont[]{STIX Two Math}
\fi
\usefonttheme{serif} % use mainfont rather than sansfont for slide text
% Use upquote if available, for straight quotes in verbatim environments
\IfFileExists{upquote.sty}{\usepackage{upquote}}{}
\IfFileExists{microtype.sty}{% use microtype if available
  \usepackage[]{microtype}
  \UseMicrotypeSet[protrusion]{basicmath} % disable protrusion for tt fonts
}{}
\makeatletter
\@ifundefined{KOMAClassName}{% if non-KOMA class
  \IfFileExists{parskip.sty}{%
    \usepackage{parskip}
  }{% else
    \setlength{\parindent}{0pt}
    \setlength{\parskip}{6pt plus 2pt minus 1pt}}
}{% if KOMA class
  \KOMAoptions{parskip=half}}
\makeatother
\usepackage{xcolor}
\IfFileExists{xurl.sty}{\usepackage{xurl}}{} % add URL line breaks if available
\IfFileExists{bookmark.sty}{\usepackage{bookmark}}{\usepackage{hyperref}}
\hypersetup{
  pdftitle={305 Lecture 8.4 - Sampling without Replacement},
  pdfauthor={Brian Weatherson},
  hidelinks,
  pdfcreator={LaTeX via pandoc}}
\urlstyle{same} % disable monospaced font for URLs
\newif\ifbibliography
\usepackage{longtable,booktabs,array}
\usepackage{calc} % for calculating minipage widths
\usepackage{caption}
% Make caption package work with longtable
\makeatletter
\def\fnum@table{\tablename~\thetable}
\makeatother
\setlength{\emergencystretch}{3em} % prevent overfull lines
\providecommand{\tightlist}{%
  \setlength{\itemsep}{0pt}\setlength{\parskip}{0pt}}
\setcounter{secnumdepth}{-\maxdimen} % remove section numbering
\let\Tiny=\tiny

 \setbeamertemplate{navigation symbols}{} 

% \usetheme{Madrid}
 \usetheme[numbering=none, progressbar=foot]{metropolis}
 \usecolortheme{wolverine}
 \usepackage{color}
 \usepackage{MnSymbol}
% \usepackage{movie15}

\usepackage{amssymb}% http://ctan.org/pkg/amssymb
\usepackage{pifont}% http://ctan.org/pkg/pifont
\newcommand{\cmark}{\ding{51}}%
\newcommand{\xmark}{\ding{55}}%

\DeclareSymbolFont{symbolsC}{U}{txsyc}{m}{n}
\DeclareMathSymbol{\boxright}{\mathrel}{symbolsC}{128}
\DeclareMathAlphabet{\mathpzc}{OT1}{pzc}{m}{it}

\usepackage{tikz-qtree}
% \usepackage{markdown}
%\RequirePackage{bussproofs}
\usetikzlibrary{arrows.meta}
\RequirePackage[tableaux]{prooftrees}
\forestset{line numbering, close with = x}
% Allow for easy commas inside trees
\renewcommand{\,}{\text{, }}


\usepackage{tabulary}

\usepackage{open-logic-config}

\setlength{\parskip}{1ex plus 0.5ex minus 0.2ex}

\AtBeginSection[]
{
\begin{frame}
	\Huge{\color{darkblue} \insertsection}
\end{frame}
}

\renewenvironment*{quote}	
	{\list{}{\rightmargin   \leftmargin} \item } 	
	{\endlist }

\definecolor{darkgreen}{rgb}{0,0.7,0}
\definecolor{darkblue}{rgb}{0,0,0.8}

\newcommand{\starttab}{\begin{center}
\vspace{6pt}
\begin{tabular}}

\newcommand{\stoptab}{\end{tabular}
\vspace{6pt}
\end{center}
\noindent}


\newcommand{\sif}{\rightarrow}
\newcommand{\siff}{\leftrightarrow}
\newcommand{\EF}{\end{frame}}


\newcommand{\TreeStart}[1]{
%\end{frame}
\begin{frame}
\begin{center}
\begin{tikzpicture}[scale=#1]
\tikzset{every tree node/.style={align=center,anchor=north}}
%\Tree
}

\newcommand{\TreeEnd}{
\end{tikzpicture}
%\end{center}
}

\newcommand{\DisplayArg}[2]{
\begin{enumerate}
{#1}
\end{enumerate}
\vspace{-6pt}
\hrulefill

%\hspace{14pt} #2
%{\addtolength{\leftskip}{14pt} #2}
\begin{quote}
{\normalfont #2}
\end{quote}
\vspace{12pt}
}

\newenvironment{ProofTree}[1][1]{
\begin{center}
\begin{tikzpicture}[scale=#1]
\tikzset{every tree node/.style={align=center,anchor=south}}
}
{
\end{tikzpicture}
\end{center}
}

\newcommand{\TreeFrame}[2]{
\begin{columns}[c]
\column{0.5\textwidth}
\begin{center}
\begin{prooftree}{}
#1
\end{prooftree}
\end{center}
\column{0.45\textwidth}
%\begin{markdown}
#2
%\end{markdown}
\end{columns}
}

\newcommand{\ScaledTreeFrame}[3]{
\begin{columns}[c]
\column{0.5\textwidth}
\begin{center}
\scalebox{#1}{
\begin{prooftree}{}
#2
\end{prooftree}
}
\end{center}
\column{0.45\textwidth}
%\begin{markdown}
#3
%\end{markdown}
\end{columns}
}

\usepackage[bb=boondox]{mathalfa}
\DeclareMathAlphabet{\mathbx}{U}{BOONDOX-ds}{m}{n}
\SetMathAlphabet{\mathbx}{bold}{U}{BOONDOX-ds}{b}{n}
\DeclareMathAlphabet{\mathbbx} {U}{BOONDOX-ds}{b}{n}


\newenvironment{oltableau}{\center\tableau{}} %wff format={anchor = base west}}}
       {\endtableau\endcenter}
       
\newcommand{\formula}[1]{$#1$}

\usepackage{tabulary}
\usepackage{booktabs}

\def\begincols{\begin{columns}}
\def\begincol{\begin{column}}
\def\endcol{\end{column}}
\def\endcols{\end{columns}}

\usepackage[italic]{mathastext}
\usepackage{nicefrac}

\definecolor{mygreen}{RGB}{0, 100, 0}
\definecolor{mypink2}{RGB}{219, 48, 122}
\definecolor{dodgerblue}{RGB}{30,144,255}

%\def\True{\textcolor{dodgerblue}{\text{T}}}
%\def\False{\textcolor{red}{\text{F}}}

\def\True{\mathbb{T}}
\def\False{\mathbb{F}}

% This is because arguments didn't have enough space after them
\usepackage{etoolbox}
\AfterEndEnvironment{description}{\vspace{9pt}}
\AfterEndEnvironment{oltableau}{\vspace{9pt}}
\BeforeBeginEnvironment{oltableau}{\vspace{9pt}}
\AfterEndEnvironment{center}{\vspace{12pt}}
\BeforeBeginEnvironment{tabular}{\vspace{9pt}}

\setlength\heavyrulewidth{0pt}
\setlength\lightrulewidth{0pt}

%\def\toprule{}
%\def\bottomrule{}
%\def\midrule{}

\setbeamertemplate{caption}{\raggedright\insertcaption}

\setlength\lightrulewidth{0.3pt}
\AfterEndEnvironment{center}{\vspace{-3pt}}
\AfterEndEnvironment{longtable}{\vspace{-6pt}}
\ifluatex
  \usepackage{selnolig}  % disable illegal ligatures
\fi

\title{305 Lecture 8.4 - Sampling without Replacement}
\author{Brian Weatherson}
\date{}

\begin{document}
\frame{\titlepage}

\begin{frame}{Plan}
\protect\hypertarget{plan}{}
\begin{itemize}
\tightlist
\item
  To illustrate another special kind of updating on two data points:
  Sampling without Replacement
\end{itemize}
\end{frame}

\begin{frame}{Associated Reading}
\protect\hypertarget{associated-reading}{}
Odds and Ends, Chapter 9
\end{frame}

\begin{frame}{Dependence}
\protect\hypertarget{dependence}{}
What happens if the events \(B_1\) and \(B_2\) are dependent on one or
other of the hypotheses?

\begin{itemize}
\tightlist
\item
  The typical case is that they will be dependent on none or all of the
  hypotheses.
\item
  But it's possible in principle to have independence on some and
  dependence on others.
\item
  And in that case we have to use the more complicated procedure I'm
  about to describe.
\end{itemize}
\end{frame}

\begin{frame}{Sampling Without Replacement}
\protect\hypertarget{sampling-without-replacement}{}
The paradigm example of conditional dependence is sampling
\textbf{without replacement}.

\begin{itemize}
\tightlist
\item
  Assume you know which urn I'm using.
\item
  Then the draws without replacement won't be independent because every
  time you draw a marble, there are fewer marbles of that color to draw
  the next time.
\end{itemize}
\end{frame}

\begin{frame}{Example}
\protect\hypertarget{example}{}
Assume that I am using urn A. (Or assume that we are working out
conditional probabilities conditional on urn A.)

\begin{itemize}
\tightlist
\item
  For the first draw, the probability of red is 4 in 10, or 0.4.
\item
  Conditional on the first draw being red, the probability of the second
  draw being red is 3 in 9, or \(\frac{1}{3}\).
\item
  That's because there are now 9 marbles left, and 3 of them are red.
\end{itemize}
\end{frame}

\begin{frame}{Continuing the Example}
\protect\hypertarget{continuing-the-example}{}
So to work out the probability of some sequence of draws \(D_1, D_2\)
given a hypothesis \(X\) about the urn, we need to use the more
complicated rule.

\[
\Pr(D_1 \wedge D_2 | X) = \Pr(D_1 | X) \Pr(D_2 | X \wedge D_1)
\]

\pause

For example \[
\Pr(Red_1 \wedge Red_2 | A) = \Pr(Red_1 | A)\Pr(Red_2 | A \wedge Red_1) = \frac{4}{10} \times \frac{3}{9} = \frac{2}{15}
\]
\end{frame}

\begin{frame}{Continuing the Example}
\protect\hypertarget{continuing-the-example-1}{}
So to work out the probability of some sequence of draws \(D_1, D_2\)
given a hypothesis \(X\) about the urn, we need to use the more
complicated rule.

\[
\Pr(D_1 \wedge D_2 | X) = \Pr(D_1 | X) \Pr(D_2 | X \wedge D_1)
\]

For example \[
\Pr(Red_1 \wedge Red_2 | B) = \Pr(Red_1 | B)\Pr(Red_2 | B \wedge Red_1) = \frac{8}{10} \times \frac{7}{9} = \frac{28}{45}
\]
\end{frame}

\begin{frame}{Another Example}
\protect\hypertarget{another-example}{}
There are two urns in front of us.

\begin{itemize}
\tightlist
\item
  One of them - urn A - has 4 red marbles, 3 green marbles, and 3 blue
  marbles.
\item
  The other - urn B- has 8 red marbles, 1 green marbles and 1 blue
  marbles. \pause
\end{itemize}

One of the urns will be selected at random, and then two marbles drawn
from it \textbf{without replacement}.

\begin{itemize}
\tightlist
\item
  If both draws are red, what is the probability that Urn A was
  selected?
\end{itemize}
\end{frame}

\begin{frame}
\begin{longtable}[]{@{}cc@{}}
\toprule
& Red-Red \\
\midrule
\endhead
Urn A & \(0.5 \times \frac{4}{10} \times \frac{3}{9} = \frac{1}{15}\) \\
& \\
Urn B &
\(0.5 \times \frac{8}{10} \times \frac{7}{9} = \frac{14}{45}\) \\
& \\
\textbf{Total} & \(\frac{1}{15} + \frac{14}{45} = \frac{17}{45}\) \\
\bottomrule
\end{longtable}

\pause

\[
\Pr(A | Red-Red) = \frac{\Pr(A \wedge Red-Red)}{\Pr(Red-Red)} = \frac{\frac{1}{15}}{\frac{17}{45}} = \frac{3}{17}
\]

\bigskip

The probability of Urn A fell by a bit more.
\end{frame}

\begin{frame}{Yet Another Example}
\protect\hypertarget{yet-another-example}{}
There are two urns in front of us.

\begin{itemize}
\tightlist
\item
  One of them - urn A - has 4 red marbles, 3 green marbles, and 3 blue
  marbles.
\item
  The other - urn B- has 8 red marbles, 1 green marbles and 1 blue
  marbles. \pause
\end{itemize}

One of the urns will be selected at random, and then two marbles drawn
from it \textbf{with replacement}.

\begin{itemize}
\tightlist
\item
  If the first draw is red and the second green, what is the probability
  that Urn A was selected?
\end{itemize}
\end{frame}

\begin{frame}{The General Conjunction Rule}
\protect\hypertarget{the-general-conjunction-rule}{}
To work out the probability of some sequence of draws \(D_1, D_2\) given
a hypothesis \(X\) about the urn, we need to use the more complicated
rule. \[
\Pr(D_1 \wedge D_2 | X) = \Pr(D_1 | X) \Pr(D_2 | X \wedge D_1)
\]

\pause

So in this case we get \[
\Pr(Red_1 \wedge Green_2 | A) = \Pr(Red_1 | A)\Pr(Green_2 | A \wedge Red_1) = \frac{4}{10} \times \frac{3}{9} = \frac{2}{15}
\]
\end{frame}

\begin{frame}{The General Conjunction Rule}
\protect\hypertarget{the-general-conjunction-rule-1}{}
To work out the probability of some sequence of draws \(D_1, D_2\) given
a hypothesis \(X\) about the urn, we need to use the more complicated
rule. \[
\Pr(D_1 \wedge D_2 | X) = \Pr(D_1 | X) \Pr(D_2 | X \wedge D_1)
\]

And for Urn B we get \[
\Pr(Red_1 \wedge Green_2 | B) = \Pr(Red_1 | B)\Pr(Green_2 | B \wedge Red_1) = \frac{8}{10} \times \frac{1}{9} = \frac{4}{45}
\]
\end{frame}

\begin{frame}
\begin{longtable}[]{@{}cc@{}}
\toprule
& Red-Green \\
\midrule
\endhead
Urn A & \(0.5 \times \frac{4}{10} \times \frac{3}{9} = \frac{1}{15}\) \\
& \\
Urn B & \(0.5 \times \frac{8}{10} \times \frac{1}{9} = \frac{2}{45}\) \\
& \\
\textbf{Total} & \(\frac{1}{15} + \frac{2}{45} = \frac{5}{45}\) \\
\bottomrule
\end{longtable}

\pause

\[
\Pr(A | Red-Green) = \frac{\Pr(A \wedge Red-Green)}{\Pr(Red-Green)} = \frac{\frac{1}{15}}{\frac{5}{45}} = \frac{3}{5}
\]

\bigskip

Which, interestingly, is exactly the same as in the with replacement
case.
\end{frame}

\begin{frame}{For Next Time}
\protect\hypertarget{for-next-time}{}
We'll end the week with one last example.
\end{frame}

\end{document}
