% Options for packages loaded elsewhere
\PassOptionsToPackage{unicode}{hyperref}
\PassOptionsToPackage{hyphens}{url}
%
\documentclass[
  ignorenonframetext,
]{beamer}
\usepackage{pgfpages}
\setbeamertemplate{caption}[numbered]
\setbeamertemplate{caption label separator}{: }
\setbeamercolor{caption name}{fg=normal text.fg}
\beamertemplatenavigationsymbolsempty
% Prevent slide breaks in the middle of a paragraph
\widowpenalties 1 10000
\raggedbottom
\setbeamertemplate{part page}{
  \centering
  \begin{beamercolorbox}[sep=16pt,center]{part title}
    \usebeamerfont{part title}\insertpart\par
  \end{beamercolorbox}
}
\setbeamertemplate{section page}{
  \centering
  \begin{beamercolorbox}[sep=12pt,center]{part title}
    \usebeamerfont{section title}\insertsection\par
  \end{beamercolorbox}
}
\setbeamertemplate{subsection page}{
  \centering
  \begin{beamercolorbox}[sep=8pt,center]{part title}
    \usebeamerfont{subsection title}\insertsubsection\par
  \end{beamercolorbox}
}
\AtBeginPart{
  \frame{\partpage}
}
\AtBeginSection{
  \ifbibliography
  \else
    \frame{\sectionpage}
  \fi
}
\AtBeginSubsection{
  \frame{\subsectionpage}
}
\usepackage{amsmath,amssymb}
\usepackage{lmodern}
\usepackage{ifxetex,ifluatex}
\ifnum 0\ifxetex 1\fi\ifluatex 1\fi=0 % if pdftex
  \usepackage[T1]{fontenc}
  \usepackage[utf8]{inputenc}
  \usepackage{textcomp} % provide euro and other symbols
\else % if luatex or xetex
  \usepackage{unicode-math}
  \defaultfontfeatures{Scale=MatchLowercase}
  \defaultfontfeatures[\rmfamily]{Ligatures=TeX,Scale=1}
  \setmainfont[BoldFont = SF Pro Rounded Semibold]{SF Pro Rounded}
  \setmathfont[]{STIX Two Math}
\fi
\usefonttheme{serif} % use mainfont rather than sansfont for slide text
% Use upquote if available, for straight quotes in verbatim environments
\IfFileExists{upquote.sty}{\usepackage{upquote}}{}
\IfFileExists{microtype.sty}{% use microtype if available
  \usepackage[]{microtype}
  \UseMicrotypeSet[protrusion]{basicmath} % disable protrusion for tt fonts
}{}
\makeatletter
\@ifundefined{KOMAClassName}{% if non-KOMA class
  \IfFileExists{parskip.sty}{%
    \usepackage{parskip}
  }{% else
    \setlength{\parindent}{0pt}
    \setlength{\parskip}{6pt plus 2pt minus 1pt}}
}{% if KOMA class
  \KOMAoptions{parskip=half}}
\makeatother
\usepackage{xcolor}
\IfFileExists{xurl.sty}{\usepackage{xurl}}{} % add URL line breaks if available
\IfFileExists{bookmark.sty}{\usepackage{bookmark}}{\usepackage{hyperref}}
\hypersetup{
  pdftitle={305 Lecture 14.1 - Counterfactuals and Similarity},
  pdfauthor={Brian Weatherson},
  hidelinks,
  pdfcreator={LaTeX via pandoc}}
\urlstyle{same} % disable monospaced font for URLs
\newif\ifbibliography
\setlength{\emergencystretch}{3em} % prevent overfull lines
\providecommand{\tightlist}{%
  \setlength{\itemsep}{0pt}\setlength{\parskip}{0pt}}
\setcounter{secnumdepth}{-\maxdimen} % remove section numbering
\let\Tiny=\tiny

 \setbeamertemplate{navigation symbols}{} 

% \usetheme{Madrid}
 \usetheme[numbering=none, progressbar=foot]{metropolis}
 \usecolortheme{wolverine}
 \usepackage{color}
 \usepackage{MnSymbol}
% \usepackage{movie15}

\usepackage{amssymb}% http://ctan.org/pkg/amssymb
\usepackage{pifont}% http://ctan.org/pkg/pifont
\newcommand{\cmark}{\ding{51}}%
\newcommand{\xmark}{\ding{55}}%

\DeclareSymbolFont{symbolsC}{U}{txsyc}{m}{n}
\DeclareMathSymbol{\boxright}{\mathrel}{symbolsC}{128}
\DeclareMathAlphabet{\mathpzc}{OT1}{pzc}{m}{it}

\usepackage{tikz-qtree}
% \usepackage{markdown}
%\RequirePackage{bussproofs}
\usetikzlibrary{arrows.meta}
\RequirePackage[tableaux]{prooftrees}
\forestset{line numbering, close with = x}
% Allow for easy commas inside trees
\renewcommand{\,}{\text{, }}


\usepackage{tabulary}

\usepackage{open-logic-config}

\setlength{\parskip}{1ex plus 0.5ex minus 0.2ex}

\AtBeginSection[]
{
\begin{frame}
	\Huge{\color{darkblue} \insertsection}
\end{frame}
}

\renewenvironment*{quote}	
	{\list{}{\rightmargin   \leftmargin} \item } 	
	{\endlist }

\definecolor{darkgreen}{rgb}{0,0.7,0}
\definecolor{darkblue}{rgb}{0,0,0.8}

\newcommand{\starttab}{\begin{center}
\vspace{6pt}
\begin{tabular}}

\newcommand{\stoptab}{\end{tabular}
\vspace{6pt}
\end{center}
\noindent}


\newcommand{\sif}{\rightarrow}
\newcommand{\siff}{\leftrightarrow}
\newcommand{\EF}{\end{frame}}


\newcommand{\TreeStart}[1]{
%\end{frame}
\begin{frame}
\begin{center}
\begin{tikzpicture}[scale=#1]
\tikzset{every tree node/.style={align=center,anchor=north}}
%\Tree
}

\newcommand{\TreeEnd}{
\end{tikzpicture}
%\end{center}
}

\newcommand{\DisplayArg}[2]{
\begin{enumerate}
{#1}
\end{enumerate}
\vspace{-6pt}
\hrulefill

%\hspace{14pt} #2
%{\addtolength{\leftskip}{14pt} #2}
\begin{quote}
{\normalfont #2}
\end{quote}
\vspace{12pt}
}

\newenvironment{ProofTree}[1][1]{
\begin{center}
\begin{tikzpicture}[scale=#1]
\tikzset{every tree node/.style={align=center,anchor=south}}
}
{
\end{tikzpicture}
\end{center}
}

\newcommand{\TreeFrame}[2]{
\begin{columns}[c]
\column{0.5\textwidth}
\begin{center}
\begin{prooftree}{}
#1
\end{prooftree}
\end{center}
\column{0.45\textwidth}
%\begin{markdown}
#2
%\end{markdown}
\end{columns}
}

\newcommand{\ScaledTreeFrame}[3]{
\begin{columns}[c]
\column{0.5\textwidth}
\begin{center}
\scalebox{#1}{
\begin{prooftree}{}
#2
\end{prooftree}
}
\end{center}
\column{0.45\textwidth}
%\begin{markdown}
#3
%\end{markdown}
\end{columns}
}

\usepackage[bb=boondox]{mathalfa}
\DeclareMathAlphabet{\mathbx}{U}{BOONDOX-ds}{m}{n}
\SetMathAlphabet{\mathbx}{bold}{U}{BOONDOX-ds}{b}{n}
\DeclareMathAlphabet{\mathbbx} {U}{BOONDOX-ds}{b}{n}


\newenvironment{oltableau}{\center\tableau{}} %wff format={anchor = base west}}}
       {\endtableau\endcenter}
       
\newcommand{\formula}[1]{$#1$}

\usepackage{tabulary}
\usepackage{booktabs}

\def\begincols{\begin{columns}}
\def\begincol{\begin{column}}
\def\endcol{\end{column}}
\def\endcols{\end{columns}}

\usepackage[italic]{mathastext}
\usepackage{nicefrac}

\definecolor{mygreen}{RGB}{0, 100, 0}
\definecolor{mypink2}{RGB}{219, 48, 122}
\definecolor{dodgerblue}{RGB}{30,144,255}

%\def\True{\textcolor{dodgerblue}{\text{T}}}
%\def\False{\textcolor{red}{\text{F}}}

\def\True{\mathbb{T}}
\def\False{\mathbb{F}}

% This is because arguments didn't have enough space after them
\usepackage{etoolbox}
\AfterEndEnvironment{description}{\vspace{9pt}}
\AfterEndEnvironment{oltableau}{\vspace{9pt}}
\BeforeBeginEnvironment{oltableau}{\vspace{9pt}}
\AfterEndEnvironment{center}{\vspace{12pt}}
\BeforeBeginEnvironment{tabular}{\vspace{9pt}}

\setlength\heavyrulewidth{0pt}
\setlength\lightrulewidth{0pt}

%\def\toprule{}
%\def\bottomrule{}
%\def\midrule{}

\setbeamertemplate{caption}{\raggedright\insertcaption}

\ifluatex
  \usepackage{selnolig}  % disable illegal ligatures
\fi

\title{305 Lecture 14.1 - Counterfactuals and Similarity}
\author{Brian Weatherson}
\date{}

\begin{document}
\frame{\titlepage}

\begin{frame}{Plan}
\protect\hypertarget{plan}{}
To discuss the notion of similarity at the heart of Lewis's theory.
\end{frame}

\begin{frame}{Reading}
\protect\hypertarget{reading}{}
This isn't covered so much in the book.
\end{frame}

\begin{frame}{Julius Caeser in Command}
\protect\hypertarget{julius-caeser-in-command}{}
Which of these (if either) sounds true?

\begin{enumerate}
\tightlist
\item
  If Julius Caeser had commanded the US forces in Vietnam, he would have
  used nuclear weapons.
\item
  If Julius Caeser had commanded the US forces in Vietnam, he would have
  used catapults.
\end{enumerate}
\end{frame}

\begin{frame}{Respects of Similarity}
\protect\hypertarget{respects-of-similarity}{}
Lewis argued that either could be true, in different contexts, though
they couldn't both be true at once.

\begin{itemize}
\tightlist
\item
  A world where Caeser is in command in Vietnam is going to be very
  different to reality in several ways.
\item
  Making it similar requires highlighting some aspects of similarity,
  and, by necessity, downplaying others. \pause 
\item
  Hold fixed how belligerent Caeser is, and you get him using nuclear
  weapons. \pause 
\item
  Hold fixed his knowledge of weaponry, and you get him using catapults.
\end{itemize}
\end{frame}

\begin{frame}{Conjoining}
\protect\hypertarget{conjoining}{}
Could this be true?

\begin{itemize}
\tightlist
\item
  If Julius Caeser had commanded the US forces in Vietnam, he would have
  used catapults to fire nuclear weapons.
\end{itemize}
\end{frame}

\begin{frame}{Feature, Not Bug}
\protect\hypertarget{feature-not-bug}{}
It feels like in some sense the world in which Caeser uses nuclear
weapons is more similar to actuality, and in some other sense the world
in which he uses catapults is more similar. Lewis argued that this was a
good feature of his theory.

\begin{itemize}
\tightlist
\item
  These conditionals just are vague, and context-sensitive.
\item
  It's a good thing that we use a vague, content-sensitive notion like
  \textbf{similarity} to analyse them.
\item
  That doesn't make them too vague, or context-sensitive, it
  \textbf{explains} why they are so vague and context-sensitive. \pause 
\end{itemize}

I'm not sure whether that argument works, but I'm not going to
investigate it further.
\end{frame}

\begin{frame}{A Different Example}
\protect\hypertarget{a-different-example}{}
This seems like a tricky to figure out, historical question.

\begin{enumerate}
\tightlist
\item
  If Richard Nixon had ordered the use of nuclear weapons in Vietnam,
  his generals would have refused the order. \pause 
\end{enumerate}

You can imagine trying to figure that out by careful archival research,
and guesswork about the parts of the archives that are off-limits to
researchers. \pause 

\begin{itemize}
\tightlist
\item
  But there's an argument that philosophy shows 1 is true.
\end{itemize}
\end{frame}

\begin{frame}{Two Worlds}
\protect\hypertarget{two-worlds}{}
\begin{itemize}
\tightlist
\item
  Let \(w\) be the actual world where (I'll assume), Nixon never gave
  such an order.
\item
  Let \(w_1\) be the world where the order is given, and refused, and we
  never hear about it, and life goes on more or less as in \(w\).
\item
  Let \(w_2\) be the world where the order is given, carried out, and
  other relevant parties (especially Russia and China) react in ways
  you'd expect to nuclear weapons being launched against their
  ally/neighbor.
\end{itemize}
\end{frame}

\begin{frame}{Two Worlds}
\protect\hypertarget{two-worlds-1}{}
Which of these worlds is more similar to the actual world? \pause 

\begin{itemize}
\tightlist
\item
  There is a good case for \(w_1\).
\item
  It just requires a few generals to act a little out of character for a
  few minutes.
\item
  In \(w_2\), everyday life is changed in unrecognisable ways for
  millions of people; perhaps billions of people if it triggers a larger
  nuclear war.
\item
  So philosophy tells us that if Nixon had ordered nuclear weapons to be
  used, his generals would have refused.
\end{itemize}
\end{frame}

\begin{frame}{History and Philosophy}
\protect\hypertarget{history-and-philosophy}{}
This is an absurd result.

\begin{itemize}
\tightlist
\item
  Maybe Nixon's generals would have refused.
\item
  But you can only tell that by careful historical research, not by
  thinking about similarity of worlds.
\item
  Something must have gone wrong.
\end{itemize}
\end{frame}

\begin{frame}{Fixing The Problem}
\protect\hypertarget{fixing-the-problem}{}
Insist that similarity of patterns or regularities is more important
than similarity of particular facts.

\begin{itemize}
\tightlist
\item
  In \(w_1\), some generals (arguably) do something very out of
  character.
\item
  That's the kind of violation of a pattern or regularity that makes a
  big difference in similarity. (In the special sense of similarity that
  we care about.)
\item
  In \(w_2\) there are lots of particular facts that are different.
  There is a lot more radiation poisoning in Vietnam, and possibly in
  the United States (depending on the retaliation), but the patterns or
  regularities are not violated.
\end{itemize}
\end{frame}

\begin{frame}{A New Puzzle}
\protect\hypertarget{a-new-puzzle}{}
Which of these is true?

\begin{enumerate}
\tightlist
\item
  If I'd jumped out my office window, I would have been seriously
  injured (I'm 1.5 floors above some concrete steps).
\item
  If I'd jumped out my office window, I would only have done that if
  there were a net to catch me, so I wouldn't have been seriously
  injured.
\end{enumerate}
\end{frame}

\begin{frame}{Puzzle}
\protect\hypertarget{puzzle}{}
\begin{itemize}
\tightlist
\item
  At least some of the time, we want to say that 1 is true.
\item
  That's the `non-backtracking' reading of the conditional that's
  relevant for thinking about decision-making, moral responsibility,
  causation, etc.
\item
  But it's not clear Lewis can get that result.
\end{itemize}
\end{frame}

\begin{frame}{Three Worlds}
\protect\hypertarget{three-worlds}{}
\begin{itemize}
\tightlist
\item
  World \(w\) is the actual world where I stay nice and safe in my
  office working on these slides.
\item
  World \(w_3\) is where there is no net, I jump and am seriously
  injured.
\item
  World \(w_4\) is where I check that there is a net, and then jump.
\item
  World \(w_4\) differs from \(w\) in two respects - there is a net that
  doesn't really exist, and I jump out the window.
\item
  But world \(w_3\) differs in two ways as well - I jump out the window,
  and this is \textbf{really} out of character
\end{itemize}
\end{frame}

\begin{frame}{The New Puzzle}
\protect\hypertarget{the-new-puzzle}{}
The `solution' to the previous case seems to make this worse.

\begin{itemize}
\tightlist
\item
  If we prioritise patterns and regularities, then it looks like \(w_3\)
  is really dissimilar to the actual world.
\item
  But we want there to be an ordinary sense in which it's just true that
  if I had jumped out the window, I would have been seriously injured.
\end{itemize}
\end{frame}

\begin{frame}{Recap}
\protect\hypertarget{recap}{}
\begin{itemize}
\tightlist
\item
  If you use an overall, intuitive, measure of similarity, you get the
  wrong results in the nuclear weapons example.
\item
  If you use a measure that focuses on patterns and regularities, you
  get the wrong result (or at least rule out a good result) in the jump
  out window example.
\item
  This is starting to feel like a trap.
\end{itemize}
\end{frame}

\begin{frame}{A Way Through}
\protect\hypertarget{a-way-through}{}
Here's how Lewis suggested getting out of the trap. (I'm simplifying a
bit here; if you're interested you should read his paper
``Counterfactual Dependence and Time's Arrow''.)

\begin{itemize}
\tightlist
\item
  Treat the past and the future differently.
\item
  In terms of the past, what matters for similarity is keeping
  \textbf{everything}, patterns, regularities, particular facts,
  everything, \textbf{exactly the same}. The nearest worlds are an exact
  duplicate of ours from the Big Bang to as close as possible to the
  present.
\item
  In terms of the future, patterns and regularities are \textbf{much}
  more important.
\end{itemize}
\end{frame}

\begin{frame}{Intuitive Picture}
\protect\hypertarget{intuitive-picture}{}
How to find the nearest world where \(A\) is true.

\begin{itemize}
\tightlist
\item
  Start at the present time and work backwards.
\item
  What's the first time where you can make \(A\) true with only a single
  violation of patterns or regularities in the world. It might be a big
  violation - like gravity stopping for a second, but ideally it is just
  one violation.
\item
  Roll the world forward from that time according to the laws of the
  world - the physical laws, the biological laws, the psychological
  laws, the economic laws and so on.
\item
  If those laws leave lots of things open, then there are lots of worlds
  that are equally close to actuality where \(A\) is true.
\end{itemize}
\end{frame}

\begin{frame}{Two Puzzles}
\protect\hypertarget{two-puzzles}{}
I like this intuitive picture a lot - I think it captures a lot of what
we're trying to do with counterfactuals. But there are still puzzles
remaining. I'll end this lecture with two of them.

\begin{enumerate}
\tightlist
\item
  The car theft puzzle.
\item
  The baseball jinx puzzle.
\end{enumerate}
\end{frame}

\begin{frame}{The Car Theft}
\protect\hypertarget{the-car-theft}{}
\begin{itemize}
\tightlist
\item
  Imagine that yesterday I parked my car at work in a UM garage, then
  drove home at the end of the day.
\item
  My car wasn't stolen yesterday (like every other day).
\item
  But car thefts happen - it being stolen isn't like aliens invading.
  \pause 
\item
  Think about this counterfactual:
\end{itemize}

\begin{quote}
If my car had been stolen yesterday, it would have been stolen just
before midnight.
\end{quote}
\end{frame}

\begin{frame}{The Car Theft}
\protect\hypertarget{the-car-theft-1}{}
\begin{itemize}
\tightlist
\item
  Intuitively, that's not true.
\item
  Just before midnight, my car was locked in my garage.
\item
  During the day, it was in a public carpark at UM.
\item
  If it was going to get stolen yesterday, it's at least as likely that
  it would have been stolen during the day as late at night.
\end{itemize}
\end{frame}

\begin{frame}{Three Worlds (Again)}
\protect\hypertarget{three-worlds-again}{}
\begin{itemize}
\tightlist
\item
  Again, let \(w\) be the actual, no car theft, world.
\item
  Let \(w_5\) be the world where my car is stolen from a UM parking lot
  at midday.
\item
  Let \(w_6\) be the world where my car is stolen from my garage just
  before midnight.
\item
  In \(w_6\), things stay exactly the same as in \(w\) for much longer -
  for nearly 12 hours longer.
\item
  So by the metric we're using, it's more similar to actuality than
  \(w_5\).
\item
  So it will be true that had my car been stolen yesterday, it would
  have been stolen just before midnight.
\item
  This isn't very plausible, so we still need to tinker with the
  similarity metric.
\end{itemize}
\end{frame}

\begin{frame}{The Baseball Jinx}
\protect\hypertarget{the-baseball-jinx}{}
Imagine that I watch a baseball game, and my team loses. My friends
accuse me of jinxing the team. I defend myself with 1.

\begin{enumerate}
\tightlist
\item
  If I hadn't watched, they would still have lost. \pause 
\end{enumerate}

Here's an argument that 1 is, for all we know, false.
\end{frame}

\begin{frame}{Physics and Baseball}
\protect\hypertarget{physics-and-baseball}{}
\begin{itemize}
\tightlist
\item
  Possibly baseball is seriously indeterministic.
\item
  Possibly quantum indeterminacy in the player's brains causes
  differences in outcomes.
\item
  Possibly quantum indeterminacy in the interaction of the ball with the
  air as it travels to the plate causes just enough variation in the
  position of the ball when it's hit to make the difference between a
  home run and a fly ball at the wall.
\item
  Maybe physics will tell us otherwise, but it seems to me we can't rule
  out this kind of indeterminacy.
\end{itemize}
\end{frame}

\begin{frame}{Baseball and Uncertainty}
\protect\hypertarget{baseball-and-uncertainty}{}
Now compare these two worlds (to \(w\), the actual world).

\begin{itemize}
\tightlist
\item
  In \(w_7\), I don't watch, but the game goes \textbf{exactly} as it
  goes in \(w\), and my team loses.
\item
  In \(w_8\), things go \textbf{exactly} as they do in \(w\) up until
  the start of the game, when I don't sit down to watch it. Then they
  follow the laws of nature. But the laws are chancy, and in \(w_8\) the
  variations favor my team just enough that they narrowly win rather
  than narrowly lose. \pause 
\end{itemize}
\end{frame}

\begin{frame}{The Puzzle}
\protect\hypertarget{the-puzzle}{}
\begin{itemize}
\tightlist
\item
  If all we care about is exact match before the `break point' (when the
  first thing changes), then conformity to laws, patterns and
  regularities after the `break point', then \(w_7\) and \(w_8\) are
  equally close to actuality.
\item
  They both are exactly alike until I don't watch (rather than watch)
  and both conform to laws afterwards.
\item
  So the conditional, \emph{If I hadn't have watched, we still would
  have lost}, is false.
\item
  In one of the nearest worlds where I don't watch, we lose.
\end{itemize}
\end{frame}

\begin{frame}{Is This a Problem}
\protect\hypertarget{is-this-a-problem}{}
This doesn't sound great.

\begin{itemize}
\tightlist
\item
  We know that we don't make a difference to baseball games by watching
  or not watching them.
\item
  But this theory, which otherwise looks promising, seems to say that
  maybe we do.
\item
  This is just an open puzzle.
\end{itemize}
\end{frame}

\begin{frame}{Next Time}
\protect\hypertarget{next-time}{}
We'll talk more about the logic of these counterfactuals.
\end{frame}

\end{document}
