% Options for packages loaded elsewhere
\PassOptionsToPackage{unicode}{hyperref}
\PassOptionsToPackage{hyphens}{url}
%
\documentclass[
  ignorenonframetext,
]{beamer}
\usepackage{pgfpages}
\setbeamertemplate{caption}[numbered]
\setbeamertemplate{caption label separator}{: }
\setbeamercolor{caption name}{fg=normal text.fg}
\beamertemplatenavigationsymbolsempty
% Prevent slide breaks in the middle of a paragraph
\widowpenalties 1 10000
\raggedbottom
\setbeamertemplate{part page}{
  \centering
  \begin{beamercolorbox}[sep=16pt,center]{part title}
    \usebeamerfont{part title}\insertpart\par
  \end{beamercolorbox}
}
\setbeamertemplate{section page}{
  \centering
  \begin{beamercolorbox}[sep=12pt,center]{part title}
    \usebeamerfont{section title}\insertsection\par
  \end{beamercolorbox}
}
\setbeamertemplate{subsection page}{
  \centering
  \begin{beamercolorbox}[sep=8pt,center]{part title}
    \usebeamerfont{subsection title}\insertsubsection\par
  \end{beamercolorbox}
}
\AtBeginPart{
  \frame{\partpage}
}
\AtBeginSection{
  \ifbibliography
  \else
    \frame{\sectionpage}
  \fi
}
\AtBeginSubsection{
  \frame{\subsectionpage}
}
\usepackage{amsmath,amssymb}
\usepackage{lmodern}
\usepackage{ifxetex,ifluatex}
\ifnum 0\ifxetex 1\fi\ifluatex 1\fi=0 % if pdftex
  \usepackage[T1]{fontenc}
  \usepackage[utf8]{inputenc}
  \usepackage{textcomp} % provide euro and other symbols
\else % if luatex or xetex
  \usepackage{unicode-math}
  \defaultfontfeatures{Scale=MatchLowercase}
  \defaultfontfeatures[\rmfamily]{Ligatures=TeX,Scale=1}
  \setmainfont[BoldFont = SF Pro Rounded Semibold]{SF Pro Rounded}
  \setmathfont[]{STIX Two Math}
\fi
\usefonttheme{serif} % use mainfont rather than sansfont for slide text
% Use upquote if available, for straight quotes in verbatim environments
\IfFileExists{upquote.sty}{\usepackage{upquote}}{}
\IfFileExists{microtype.sty}{% use microtype if available
  \usepackage[]{microtype}
  \UseMicrotypeSet[protrusion]{basicmath} % disable protrusion for tt fonts
}{}
\makeatletter
\@ifundefined{KOMAClassName}{% if non-KOMA class
  \IfFileExists{parskip.sty}{%
    \usepackage{parskip}
  }{% else
    \setlength{\parindent}{0pt}
    \setlength{\parskip}{6pt plus 2pt minus 1pt}}
}{% if KOMA class
  \KOMAoptions{parskip=half}}
\makeatother
\usepackage{xcolor}
\IfFileExists{xurl.sty}{\usepackage{xurl}}{} % add URL line breaks if available
\IfFileExists{bookmark.sty}{\usepackage{bookmark}}{\usepackage{hyperref}}
\hypersetup{
  pdftitle={305 Lecture 8.1 - Base Rates},
  pdfauthor={Brian Weatherson},
  hidelinks,
  pdfcreator={LaTeX via pandoc}}
\urlstyle{same} % disable monospaced font for URLs
\newif\ifbibliography
\usepackage{longtable,booktabs,array}
\usepackage{calc} % for calculating minipage widths
\usepackage{caption}
% Make caption package work with longtable
\makeatletter
\def\fnum@table{\tablename~\thetable}
\makeatother
\setlength{\emergencystretch}{3em} % prevent overfull lines
\providecommand{\tightlist}{%
  \setlength{\itemsep}{0pt}\setlength{\parskip}{0pt}}
\setcounter{secnumdepth}{-\maxdimen} % remove section numbering
\let\Tiny=\tiny

 \setbeamertemplate{navigation symbols}{} 

% \usetheme{Madrid}
 \usetheme[numbering=none, progressbar=foot]{metropolis}
 \usecolortheme{wolverine}
 \usepackage{color}
 \usepackage{MnSymbol}
% \usepackage{movie15}

\usepackage{amssymb}% http://ctan.org/pkg/amssymb
\usepackage{pifont}% http://ctan.org/pkg/pifont
\newcommand{\cmark}{\ding{51}}%
\newcommand{\xmark}{\ding{55}}%

\DeclareSymbolFont{symbolsC}{U}{txsyc}{m}{n}
\DeclareMathSymbol{\boxright}{\mathrel}{symbolsC}{128}
\DeclareMathAlphabet{\mathpzc}{OT1}{pzc}{m}{it}

\usepackage{tikz-qtree}
% \usepackage{markdown}
%\RequirePackage{bussproofs}
\usetikzlibrary{arrows.meta}
\RequirePackage[tableaux]{prooftrees}
\forestset{line numbering, close with = x}
% Allow for easy commas inside trees
\renewcommand{\,}{\text{, }}


\usepackage{tabulary}

\usepackage{open-logic-config}

\setlength{\parskip}{1ex plus 0.5ex minus 0.2ex}

\AtBeginSection[]
{
\begin{frame}
	\Huge{\color{darkblue} \insertsection}
\end{frame}
}

\renewenvironment*{quote}	
	{\list{}{\rightmargin   \leftmargin} \item } 	
	{\endlist }

\definecolor{darkgreen}{rgb}{0,0.7,0}
\definecolor{darkblue}{rgb}{0,0,0.8}

\newcommand{\starttab}{\begin{center}
\vspace{6pt}
\begin{tabular}}

\newcommand{\stoptab}{\end{tabular}
\vspace{6pt}
\end{center}
\noindent}


\newcommand{\sif}{\rightarrow}
\newcommand{\siff}{\leftrightarrow}
\newcommand{\EF}{\end{frame}}


\newcommand{\TreeStart}[1]{
%\end{frame}
\begin{frame}
\begin{center}
\begin{tikzpicture}[scale=#1]
\tikzset{every tree node/.style={align=center,anchor=north}}
%\Tree
}

\newcommand{\TreeEnd}{
\end{tikzpicture}
%\end{center}
}

\newcommand{\DisplayArg}[2]{
\begin{enumerate}
{#1}
\end{enumerate}
\vspace{-6pt}
\hrulefill

%\hspace{14pt} #2
%{\addtolength{\leftskip}{14pt} #2}
\begin{quote}
{\normalfont #2}
\end{quote}
\vspace{12pt}
}

\newenvironment{ProofTree}[1][1]{
\begin{center}
\begin{tikzpicture}[scale=#1]
\tikzset{every tree node/.style={align=center,anchor=south}}
}
{
\end{tikzpicture}
\end{center}
}

\newcommand{\TreeFrame}[2]{
\begin{columns}[c]
\column{0.5\textwidth}
\begin{center}
\begin{prooftree}{}
#1
\end{prooftree}
\end{center}
\column{0.45\textwidth}
%\begin{markdown}
#2
%\end{markdown}
\end{columns}
}

\newcommand{\ScaledTreeFrame}[3]{
\begin{columns}[c]
\column{0.5\textwidth}
\begin{center}
\scalebox{#1}{
\begin{prooftree}{}
#2
\end{prooftree}
}
\end{center}
\column{0.45\textwidth}
%\begin{markdown}
#3
%\end{markdown}
\end{columns}
}

\usepackage[bb=boondox]{mathalfa}
\DeclareMathAlphabet{\mathbx}{U}{BOONDOX-ds}{m}{n}
\SetMathAlphabet{\mathbx}{bold}{U}{BOONDOX-ds}{b}{n}
\DeclareMathAlphabet{\mathbbx} {U}{BOONDOX-ds}{b}{n}


\newenvironment{oltableau}{\center\tableau{}} %wff format={anchor = base west}}}
       {\endtableau\endcenter}
       
\newcommand{\formula}[1]{$#1$}

\usepackage{tabulary}
\usepackage{booktabs}

\def\begincols{\begin{columns}}
\def\begincol{\begin{column}}
\def\endcol{\end{column}}
\def\endcols{\end{columns}}

\usepackage[italic]{mathastext}
\usepackage{nicefrac}

\definecolor{mygreen}{RGB}{0, 100, 0}
\definecolor{mypink2}{RGB}{219, 48, 122}
\definecolor{dodgerblue}{RGB}{30,144,255}

%\def\True{\textcolor{dodgerblue}{\text{T}}}
%\def\False{\textcolor{red}{\text{F}}}

\def\True{\mathbb{T}}
\def\False{\mathbb{F}}

% This is because arguments didn't have enough space after them
\usepackage{etoolbox}
\AfterEndEnvironment{description}{\vspace{9pt}}
\AfterEndEnvironment{oltableau}{\vspace{9pt}}
\BeforeBeginEnvironment{oltableau}{\vspace{9pt}}
\AfterEndEnvironment{center}{\vspace{12pt}}
\BeforeBeginEnvironment{tabular}{\vspace{9pt}}

\setlength\heavyrulewidth{0pt}
\setlength\lightrulewidth{0pt}

%\def\toprule{}
%\def\bottomrule{}
%\def\midrule{}

\setbeamertemplate{caption}{\raggedright\insertcaption}

\setlength\lightrulewidth{0.3pt}
\ifluatex
  \usepackage{selnolig}  % disable illegal ligatures
\fi

\title{305 Lecture 8.1 - Base Rates}
\author{Brian Weatherson}
\date{}

\begin{document}
\frame{\titlepage}

\begin{frame}{Plan}
\protect\hypertarget{plan}{}
\begin{itemize}
\tightlist
\item
  This lecture will talk about the importance of taking base rates into
  account.
\item
  To do that, I need to explain what a base rate is.
\item
  And to do that, it helps to start with a famous example.
\end{itemize}
\end{frame}

\begin{frame}{Associated Reading}
\protect\hypertarget{associated-reading}{}
Odds and Ends, section 8.1
\end{frame}

\begin{frame}{The Blue and Green Example}
\protect\hypertarget{the-blue-and-green-example}{}
Plan

\begin{itemize}
\tightlist
\item
  Go over the example.
\item
  Explain what the book says about it.
\item
  Say why I think it's a bit more complicated than calling one view a
  mistake. (The short version is that English is ambiguous around here,
  and we shouldn't assume one particular disambiguation.)
\end{itemize}
\end{frame}

\begin{frame}{Setup (Part One)}
\protect\hypertarget{setup-part-one}{}
A cab was involved in a hit and run accident at night. Two cab
companies, the Green and the Blue, operate in the city. You are given
the following data:

\begin{enumerate}
\tightlist
\item
  85\% of the cabs in the city are Green and 15\% are Blue.
\item
  A witness identified the cab as Blue.
\end{enumerate}
\end{frame}

\begin{frame}{Setup (Part Two)}
\protect\hypertarget{setup-part-two}{}
The court tested the reliability of the witness under the same
circumstances that existed on the night of the accident and concluded
that the witness correctly identified each one of the two colors 80\% of
the time and failed 20\% of the time.

\begin{itemize}
\tightlist
\item
  What is the probability that the cab involved in the accident was blue
  rather green?
\end{itemize}
\end{frame}

\begin{frame}{Reliability}
\protect\hypertarget{reliability}{}
Let's understand that last clause as meaning that the following two
claims are true.

\begin{itemize}
\tightlist
\item
  Pr(Correct Identification \(|\) Is-Green) = \(0.8\); and
\item
  Pr(Correct Identification \(|\) Is-Blue) = \(0.8\).
\end{itemize}

Let's also note that this isn't obvious that this is the right
translation. For one thing, it translates a single data point as a
conjunction.
\end{frame}

\begin{frame}{Restating Reliability}
\protect\hypertarget{restating-reliability}{}
\begin{itemize}[<+->]
\tightlist
\item
  Pr(Looks-Green \(|\) Is-Green) = \(0.8\).
\item
  So Pr(Looks-Blue \(|\) Is-Green) = \(0.2\).
\end{itemize}

\begin{itemize}[<+->]
\tightlist
\item
  Pr(Looks-Blue \(|\) Is-Blue) = \(0.8\).
\item
  So Pr(Looks-Green \(|\) Is-Blue) = \(0.2\).
\end{itemize}

\begin{itemize}[<+->]
\tightlist
\item
  And we also know Pr(Is-Green) = \(0.85\)
\end{itemize}
\end{frame}

\begin{frame}{Another Table}
\protect\hypertarget{another-table}{}
\begin{longtable}[]{@{}rcc@{}}
\toprule
& Looks-Green & Looks-Blue \\
\midrule
\endhead
Is-Green & \(0.85 \times 0.8 = 0.68\) & \(0.85 \times 0.2 = 0.17\) \\
Is-Blue & \(0.15 \times 0.2 = 0.03\) & \(0.15 \times 0.8 = 0.12\) \\
\bottomrule
\end{longtable}

Think about why each of those numbers are there.
\end{frame}

\begin{frame}{Conditional Probablility}
\protect\hypertarget{conditional-probablility}{}
Now focus on the right-hand column.

\begin{longtable}[]{@{}rcc@{}}
\toprule
& Looks-Green & Looks-Blue \\
\midrule
\endhead
Is-Green & \(0.85 \times 0.8 = 0.68\) & \(0.85 \times 0.2 = 0.17\) \\
Is-Blue & \(0.15 \times 0.2 = 0.03\) & \(0.15 \times 0.8 = 0.12\) \\
\bottomrule
\end{longtable}

\pause

\begin{itemize}
\tightlist
\item
  The probability of Looks Blue is \(0.17 + 0.12 = 0.29\) \pause
\item
  The probability of Is-Blue-and-Looks-Blue is \(0.12\). \pause
\item
  So the probability of Is-Blue given Looks-Blue is
  \(\frac{0.12}{0.29} \approx 0.41\).
\end{itemize}
\end{frame}

\begin{frame}{The Standard Story}
\protect\hypertarget{the-standard-story}{}
\begin{itemize}[<+->]
\tightlist
\item
  Even though the witness is pretty reliable, it is still more likely a
  green cab than a blue cab given their evidence.
\item
  When something is really probable to start with, it takes a lot of
  evidence to shift the probabilities.
\item
  We often tend to conflate the \textbf{direction} of the recent
  evidence with the \textbf{result} of that getting that evidence.
\item
  Even if Blue is better supported by the recent evidence than Green,
  the overall evidence might still better support Green.
\item
  And a special case of that is where all the indivudalised evidence
  supports Blue, but the generic evidence (the base rates) supports
  Green.
\end{itemize}
\end{frame}

\begin{frame}{Worry 1 of 3}
\protect\hypertarget{worry-1-of-3}{}
This is a really bad witness.

\begin{itemize}
\tightlist
\item
  If they'd just said green every time, they'd have been more reliable.
\item
  If you're less reliable than a tape recorder saying the same thing
  over and over again, that's bad.
\item
  So bad that we might worry about what's going on.
\end{itemize}
\end{frame}

\begin{frame}{Worry 2 of 3}
\protect\hypertarget{worry-2-of-3}{}
Why make the inference from Pr(Looks-Blue \(|\) Is-Blue) = \(0.8\) to
Pr(Looks-Green \(|\) Is-Blue) = \(0.2\)? \pause

\begin{itemize}
\tightlist
\item
  The text only says that (a) all the taxis are Blue or Green, and (b)
  the witness is 80\% accurate.
\item
  It doesn't follow that every time that they misidentify a Blue cab,
  they misidentify it as Green. \pause
\item
  Even if there aren't yellow cabs, they might see it as yellow. (Maybe
  they just moved from NYC). \pause
\item
  Or maybe a ``don't know'' verdict count as not a correct
  identification. So we need a column for ``don't know''.
\end{itemize}
\end{frame}

\begin{frame}{Worry 3 of 3}
\protect\hypertarget{worry-3-of-3}{}
Why interpret the accuracy claim as that conjunction rather than the
conjunction:

\begin{itemize}
\tightlist
\item
  Pr(Correct Identification \(|\) Looks-Green) = \(0.8\); and
\item
  Pr(Correct Identification \(|\) Looks-Blue) = \(0.8\).
\end{itemize}

Then the answer is 0.8. And I guess I don't see what in the English
makes it more likely to be read one way rather than another.
\end{frame}

\begin{frame}{Worry 3 of 3 rephrased}
\protect\hypertarget{worry-3-of-3-rephrased}{}
Put another way, if this conjunction was true, how would you describe
it?

\begin{itemize}
\tightlist
\item
  Pr(Correct Identification \(|\) Looks-Green) = \(0.8\); and
\item
  Pr(Correct Identification \(|\) Looks-Blue) = \(0.8\). \pause
\end{itemize}

I think I'd say the witness correctly identified the color 80\% of the
time.
\end{frame}

\begin{frame}{General Lesson}
\protect\hypertarget{general-lesson}{}
\begin{itemize}
\tightlist
\item
  When you get this kind of accuracy data in words - check what you are
  being told!
\item
  Data with very different implications might be described in similar
  words.
\item
  This is just as important a lesson as the lesson ``Attend to base
  rates''.
\end{itemize}
\end{frame}

\begin{frame}{For Next Time}
\protect\hypertarget{for-next-time}{}
We will end the week looking at some more complicated examples of
inverting conditional probability.
\end{frame}

\end{document}
