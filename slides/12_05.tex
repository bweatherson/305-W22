% Options for packages loaded elsewhere
\PassOptionsToPackage{unicode}{hyperref}
\PassOptionsToPackage{hyphens}{url}
%
\documentclass[
  ignorenonframetext,
]{beamer}
\usepackage{pgfpages}
\setbeamertemplate{caption}[numbered]
\setbeamertemplate{caption label separator}{: }
\setbeamercolor{caption name}{fg=normal text.fg}
\beamertemplatenavigationsymbolsempty
% Prevent slide breaks in the middle of a paragraph
\widowpenalties 1 10000
\raggedbottom
\setbeamertemplate{part page}{
  \centering
  \begin{beamercolorbox}[sep=16pt,center]{part title}
    \usebeamerfont{part title}\insertpart\par
  \end{beamercolorbox}
}
\setbeamertemplate{section page}{
  \centering
  \begin{beamercolorbox}[sep=12pt,center]{part title}
    \usebeamerfont{section title}\insertsection\par
  \end{beamercolorbox}
}
\setbeamertemplate{subsection page}{
  \centering
  \begin{beamercolorbox}[sep=8pt,center]{part title}
    \usebeamerfont{subsection title}\insertsubsection\par
  \end{beamercolorbox}
}
\AtBeginPart{
  \frame{\partpage}
}
\AtBeginSection{
  \ifbibliography
  \else
    \frame{\sectionpage}
  \fi
}
\AtBeginSubsection{
  \frame{\subsectionpage}
}
\usepackage{amsmath,amssymb}
\usepackage{lmodern}
\usepackage{ifxetex,ifluatex}
\ifnum 0\ifxetex 1\fi\ifluatex 1\fi=0 % if pdftex
  \usepackage[T1]{fontenc}
  \usepackage[utf8]{inputenc}
  \usepackage{textcomp} % provide euro and other symbols
\else % if luatex or xetex
  \usepackage{unicode-math}
  \defaultfontfeatures{Scale=MatchLowercase}
  \defaultfontfeatures[\rmfamily]{Ligatures=TeX,Scale=1}
  \setmainfont[BoldFont = SF Pro Rounded Semibold]{SF Pro Rounded}
  \setmathfont[]{STIX Two Math}
\fi
\usefonttheme{serif} % use mainfont rather than sansfont for slide text
% Use upquote if available, for straight quotes in verbatim environments
\IfFileExists{upquote.sty}{\usepackage{upquote}}{}
\IfFileExists{microtype.sty}{% use microtype if available
  \usepackage[]{microtype}
  \UseMicrotypeSet[protrusion]{basicmath} % disable protrusion for tt fonts
}{}
\makeatletter
\@ifundefined{KOMAClassName}{% if non-KOMA class
  \IfFileExists{parskip.sty}{%
    \usepackage{parskip}
  }{% else
    \setlength{\parindent}{0pt}
    \setlength{\parskip}{6pt plus 2pt minus 1pt}}
}{% if KOMA class
  \KOMAoptions{parskip=half}}
\makeatother
\usepackage{xcolor}
\IfFileExists{xurl.sty}{\usepackage{xurl}}{} % add URL line breaks if available
\IfFileExists{bookmark.sty}{\usepackage{bookmark}}{\usepackage{hyperref}}
\hypersetup{
  pdftitle={305 Lecture 12.5 - Proving Invalidity},
  pdfauthor={Brian Weatherson},
  hidelinks,
  pdfcreator={LaTeX via pandoc}}
\urlstyle{same} % disable monospaced font for URLs
\newif\ifbibliography
\setlength{\emergencystretch}{3em} % prevent overfull lines
\providecommand{\tightlist}{%
  \setlength{\itemsep}{0pt}\setlength{\parskip}{0pt}}
\setcounter{secnumdepth}{-\maxdimen} % remove section numbering
\let\Tiny=\tiny

 \setbeamertemplate{navigation symbols}{} 

% \usetheme{Madrid}
 \usetheme[numbering=none, progressbar=foot]{metropolis}
 \usecolortheme{wolverine}
 \usepackage{color}
 \usepackage{MnSymbol}
% \usepackage{movie15}

\usepackage{amssymb}% http://ctan.org/pkg/amssymb
\usepackage{pifont}% http://ctan.org/pkg/pifont
\newcommand{\cmark}{\ding{51}}%
\newcommand{\xmark}{\ding{55}}%

\DeclareSymbolFont{symbolsC}{U}{txsyc}{m}{n}
\DeclareMathSymbol{\boxright}{\mathrel}{symbolsC}{128}
\DeclareMathAlphabet{\mathpzc}{OT1}{pzc}{m}{it}

\usepackage{tikz-qtree}
% \usepackage{markdown}
%\RequirePackage{bussproofs}
\usetikzlibrary{arrows.meta}
\RequirePackage[tableaux]{prooftrees}
\forestset{line numbering, close with = x}
% Allow for easy commas inside trees
\renewcommand{\,}{\text{, }}


\usepackage{tabulary}

\usepackage{open-logic-config}

\setlength{\parskip}{1ex plus 0.5ex minus 0.2ex}

\AtBeginSection[]
{
\begin{frame}
	\Huge{\color{darkblue} \insertsection}
\end{frame}
}

\renewenvironment*{quote}	
	{\list{}{\rightmargin   \leftmargin} \item } 	
	{\endlist }

\definecolor{darkgreen}{rgb}{0,0.7,0}
\definecolor{darkblue}{rgb}{0,0,0.8}

\newcommand{\starttab}{\begin{center}
\vspace{6pt}
\begin{tabular}}

\newcommand{\stoptab}{\end{tabular}
\vspace{6pt}
\end{center}
\noindent}


\newcommand{\sif}{\rightarrow}
\newcommand{\siff}{\leftrightarrow}
\newcommand{\EF}{\end{frame}}


\newcommand{\TreeStart}[1]{
%\end{frame}
\begin{frame}
\begin{center}
\begin{tikzpicture}[scale=#1]
\tikzset{every tree node/.style={align=center,anchor=north}}
%\Tree
}

\newcommand{\TreeEnd}{
\end{tikzpicture}
%\end{center}
}

\newcommand{\DisplayArg}[2]{
\begin{enumerate}
{#1}
\end{enumerate}
\vspace{-6pt}
\hrulefill

%\hspace{14pt} #2
%{\addtolength{\leftskip}{14pt} #2}
\begin{quote}
{\normalfont #2}
\end{quote}
\vspace{12pt}
}

\newenvironment{ProofTree}[1][1]{
\begin{center}
\begin{tikzpicture}[scale=#1]
\tikzset{every tree node/.style={align=center,anchor=south}}
}
{
\end{tikzpicture}
\end{center}
}

\newcommand{\TreeFrame}[2]{
\begin{columns}[c]
\column{0.5\textwidth}
\begin{center}
\begin{prooftree}{}
#1
\end{prooftree}
\end{center}
\column{0.45\textwidth}
%\begin{markdown}
#2
%\end{markdown}
\end{columns}
}

\newcommand{\ScaledTreeFrame}[3]{
\begin{columns}[c]
\column{0.5\textwidth}
\begin{center}
\scalebox{#1}{
\begin{prooftree}{}
#2
\end{prooftree}
}
\end{center}
\column{0.45\textwidth}
%\begin{markdown}
#3
%\end{markdown}
\end{columns}
}

\usepackage[bb=boondox]{mathalfa}
\DeclareMathAlphabet{\mathbx}{U}{BOONDOX-ds}{m}{n}
\SetMathAlphabet{\mathbx}{bold}{U}{BOONDOX-ds}{b}{n}
\DeclareMathAlphabet{\mathbbx} {U}{BOONDOX-ds}{b}{n}


\newenvironment{oltableau}{\center\tableau{}} %wff format={anchor = base west}}}
       {\endtableau\endcenter}
       
\newcommand{\formula}[1]{$#1$}

\usepackage{tabulary}
\usepackage{booktabs}

\def\begincols{\begin{columns}}
\def\begincol{\begin{column}}
\def\endcol{\end{column}}
\def\endcols{\end{columns}}

\usepackage[italic]{mathastext}
\usepackage{nicefrac}

\definecolor{mygreen}{RGB}{0, 100, 0}
\definecolor{mypink2}{RGB}{219, 48, 122}
\definecolor{dodgerblue}{RGB}{30,144,255}

%\def\True{\textcolor{dodgerblue}{\text{T}}}
%\def\False{\textcolor{red}{\text{F}}}

\def\True{\mathbb{T}}
\def\False{\mathbb{F}}

% This is because arguments didn't have enough space after them
\usepackage{etoolbox}
\AfterEndEnvironment{description}{\vspace{9pt}}
\AfterEndEnvironment{oltableau}{\vspace{9pt}}
\BeforeBeginEnvironment{oltableau}{\vspace{9pt}}
\AfterEndEnvironment{center}{\vspace{12pt}}
\BeforeBeginEnvironment{tabular}{\vspace{9pt}}

\setlength\heavyrulewidth{0pt}
\setlength\lightrulewidth{0pt}

%\def\toprule{}
%\def\bottomrule{}
%\def\midrule{}

\setbeamertemplate{caption}{\raggedright\insertcaption}

\ifluatex
  \usepackage{selnolig}  % disable illegal ligatures
\fi

\title{305 Lecture 12.5 - Proving Invalidity}
\author{Brian Weatherson}
\date{}

\begin{document}
\frame{\titlepage}

\begin{frame}{Plan}
\protect\hypertarget{plan}{}
\begin{itemize}
\tightlist
\item
  Discuss how to use tableau to show invalidity.
\end{itemize}
\end{frame}

\begin{frame}{Associated Reading}
\protect\hypertarget{associated-reading}{}
\begin{itemize}
\tightlist
\item
  Not in book.
\end{itemize}
\end{frame}

\begin{frame}{Two Uses of Trees}
\protect\hypertarget{two-uses-of-trees}{}
We tell that an argument is valid or that a sentence is a theorem by
drawing a closed tree.

\begin{itemize}
\tightlist
\item
  In principle, we can also use trees to show that an argument is
  invalid, or that something is not a theorem.
\item
  In practice, it's a little tricky.
\end{itemize}
\end{frame}

\begin{frame}{Case One: Draw an Open Tree}
\protect\hypertarget{case-one-draw-an-open-tree}{}
Imagine we're working in KT, and we want to show that
\(\Box A \rightarrow \Box \Box A\) is not a theorem. Then we really can
draw a tree.
\end{frame}

\begin{frame}{\(\Box A \rightarrow \Box \Box A\)}
\protect\hypertarget{box-a-rightarrow-box-box-a}{}
\begin{oltableau}
[\pFmla{\False}{\Box A \rightarrow \Box \Box A}{1}, checked, just = \TAss
  [\pFmla{\True}{\Box A}{1}, just = {\TRule{\False}{\rightarrow}[1]}
    [\pFmla{\False}{\Box \Box A}{1}, checked, just = {\TRule{\False}{\rightarrow}[1]}
      [\pFmla{\True}{A}{1}, just = {T $\Box$ 2}
        [\pFmla{\False}{\Box A}{1.1}, checked, just = {\TRule{\False}{\Box}[3]}
          [\pFmla{\True}{A}{1.1}, just = {\TRule{\True}{\Box}[2]}        
            [\pFmla{\False}{A}{1.1.1}, just = {\TRule{\False}{\Box}[5]}
            ]
          ]
        ]
      ]
    ]
  ]
]
\end{oltableau}
\end{frame}

\begin{frame}{A Model}
\protect\hypertarget{a-model}{}
\begin{itemize}
\tightlist
\item
  Three worlds, \(w_1, w_{1.1}, w_{1.1.1}\).
\item
  The accessibility relations are
  \(w_1Rw_{1.1}, w_{1.1}Rw_{1.1.1}, w_1Rw_1, w_{1.1}Rw_{1.1}\) and
  \(w_{1.1.1}Rw_{1.1.1}\).
\item
  The first two are from the tree, the next three from the restriction.
\item
  \(A\) is true at \(w_1\) and \(w_{1.1}\) and false at \(w_{1.1.1}\).
\item
  So \(\Box A\) will be true only at \(w_1\).
\item
  So \(\Box \Box A\) will be false at \(w_1\), as required.
\end{itemize}
\end{frame}

\begin{frame}{First Problem}
\protect\hypertarget{first-problem}{}
We typically can't just tick off the sentences as we apply the rules for
them.

\begin{itemize}
\tightlist
\item
  Lots of the rules, especially for true \(\Box\) sentences and false
  \(\Diamond\) sentences, are open.
\item
  So to check the tree is finished, you have to go back and look at each
  of these sentences, and be sure that you really really have applied
  all the rules.
\end{itemize}
\end{frame}

\begin{frame}{Second Problem}
\protect\hypertarget{second-problem}{}
Sometimes the tree never ends.

\begin{itemize}
\tightlist
\item
  Imagine we're working in KT4.
\item
  And we're trying to work out whether this is a theorem
\item
  \(\Box \Diamond A \rightarrow \Diamond \Box B\)
\item
  At one level, it's obvious that it isn't a theorem.
\item
  But the tree is a mess.
\end{itemize}
\end{frame}

\begin{frame}
\begin{oltableau}
[\pFmla{\False}{\Box \Diamond A \rightarrow \Diamond \Box B}{1}, checked, just = \TAss,
  [\pFmla{\True}{\Box \Diamond A}{1}, just = {\TRule{\False}{\rightarrow}[1]},
    [\pFmla{\False}{\Diamond \Box B}{1}, just = {\TRule{\False}{\rightarrow}[1]},
      [\pFmla{\True}{\Diamond A}{1}, checked, just={T $\Box$, 2},
        [\pFmla{\False}{\Box B}{1}, checked, just={T $\Diamond$, 2},
          [\pFmla{\True}{A}{1.1}, just={\TRule{\True}{\Diamond}[4]},
            [\pFmla{\True}{\Diamond A}{1.1}, just={\TRule{\Box}{\True}[2]},
              [\pFmla{\False}{\Box B}{1.1}, just={\TRule{\False}{\Diamond}[3]},
                [\pFmla{\True}{\Box \Diamond A}{1.1}, just = {4 $\Box$, 2},
                ]
              ]
            ]
          ]
        ]
      ]
    ]
  ]
]
\end{oltableau}
\end{frame}

\begin{frame}{What Went Wrong}
\protect\hypertarget{what-went-wrong}{}
We can sort of kind of see the problem.

\begin{itemize}
\tightlist
\item
  The tree just repeats.
\item
  Maybe we can turn that into a model.
\end{itemize}
\end{frame}

\begin{frame}{The General Recipe}
\protect\hypertarget{the-general-recipe}{}
What shows something is not a theorem is a model where it is false at a
world.

\begin{itemize}
\tightlist
\item
  Take the open tree.
\item
  Each number on the tree is a world.
\item
  World \(x\) is always related to world \(x.y\).
\item
  Add R-relations that are required by the relevant restrictions.
\item
  Read truth for atomic sentences off what the tree says. It should be
  reasonably specific, though often it will leave gaps.
\end{itemize}
\end{frame}

\begin{frame}{The Harder Case}
\protect\hypertarget{the-harder-case}{}
What to do if the tree doesn't close.

\begin{enumerate}
\tightlist
\item
  Describe (but obviously don't draw) the infinite model.
\item
  Draw a model where one world stands in for many different strings of
  numbers.
\end{enumerate}
\end{frame}

\begin{frame}{For Up/Down Verdicts}
\protect\hypertarget{for-updown-verdicts}{}
If you can see that the tree will cycle and never complete, that's sort
of good enough.
\end{frame}

\begin{frame}{For Next Time}
\protect\hypertarget{for-next-time}{}
We'll go on to examples.
\end{frame}

\end{document}
