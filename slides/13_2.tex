% Options for packages loaded elsewhere
\PassOptionsToPackage{unicode}{hyperref}
\PassOptionsToPackage{hyphens}{url}
%
\documentclass[
  ignorenonframetext,
]{beamer}
\usepackage{pgfpages}
\setbeamertemplate{caption}[numbered]
\setbeamertemplate{caption label separator}{: }
\setbeamercolor{caption name}{fg=normal text.fg}
\beamertemplatenavigationsymbolsempty
% Prevent slide breaks in the middle of a paragraph
\widowpenalties 1 10000
\raggedbottom
\setbeamertemplate{part page}{
  \centering
  \begin{beamercolorbox}[sep=16pt,center]{part title}
    \usebeamerfont{part title}\insertpart\par
  \end{beamercolorbox}
}
\setbeamertemplate{section page}{
  \centering
  \begin{beamercolorbox}[sep=12pt,center]{part title}
    \usebeamerfont{section title}\insertsection\par
  \end{beamercolorbox}
}
\setbeamertemplate{subsection page}{
  \centering
  \begin{beamercolorbox}[sep=8pt,center]{part title}
    \usebeamerfont{subsection title}\insertsubsection\par
  \end{beamercolorbox}
}
\AtBeginPart{
  \frame{\partpage}
}
\AtBeginSection{
  \ifbibliography
  \else
    \frame{\sectionpage}
  \fi
}
\AtBeginSubsection{
  \frame{\subsectionpage}
}
\usepackage{amsmath,amssymb}
\usepackage{lmodern}
\usepackage{ifxetex,ifluatex}
\ifnum 0\ifxetex 1\fi\ifluatex 1\fi=0 % if pdftex
  \usepackage[T1]{fontenc}
  \usepackage[utf8]{inputenc}
  \usepackage{textcomp} % provide euro and other symbols
\else % if luatex or xetex
  \usepackage{unicode-math}
  \defaultfontfeatures{Scale=MatchLowercase}
  \defaultfontfeatures[\rmfamily]{Ligatures=TeX,Scale=1}
  \setmainfont[BoldFont = SF Pro Rounded Semibold]{SF Pro Rounded}
  \setmathfont[]{STIX Two Math}
\fi
\usefonttheme{serif} % use mainfont rather than sansfont for slide text
% Use upquote if available, for straight quotes in verbatim environments
\IfFileExists{upquote.sty}{\usepackage{upquote}}{}
\IfFileExists{microtype.sty}{% use microtype if available
  \usepackage[]{microtype}
  \UseMicrotypeSet[protrusion]{basicmath} % disable protrusion for tt fonts
}{}
\makeatletter
\@ifundefined{KOMAClassName}{% if non-KOMA class
  \IfFileExists{parskip.sty}{%
    \usepackage{parskip}
  }{% else
    \setlength{\parindent}{0pt}
    \setlength{\parskip}{6pt plus 2pt minus 1pt}}
}{% if KOMA class
  \KOMAoptions{parskip=half}}
\makeatother
\usepackage{xcolor}
\IfFileExists{xurl.sty}{\usepackage{xurl}}{} % add URL line breaks if available
\IfFileExists{bookmark.sty}{\usepackage{bookmark}}{\usepackage{hyperref}}
\hypersetup{
  pdftitle={305 Lecture 13.2 - Material Conditionals},
  pdfauthor={Brian Weatherson},
  hidelinks,
  pdfcreator={LaTeX via pandoc}}
\urlstyle{same} % disable monospaced font for URLs
\newif\ifbibliography
\setlength{\emergencystretch}{3em} % prevent overfull lines
\providecommand{\tightlist}{%
  \setlength{\itemsep}{0pt}\setlength{\parskip}{0pt}}
\setcounter{secnumdepth}{-\maxdimen} % remove section numbering
\let\Tiny=\tiny

 \setbeamertemplate{navigation symbols}{} 

% \usetheme{Madrid}
 \usetheme[numbering=none, progressbar=foot]{metropolis}
 \usecolortheme{wolverine}
 \usepackage{color}
 \usepackage{MnSymbol}
% \usepackage{movie15}

\usepackage{amssymb}% http://ctan.org/pkg/amssymb
\usepackage{pifont}% http://ctan.org/pkg/pifont
\newcommand{\cmark}{\ding{51}}%
\newcommand{\xmark}{\ding{55}}%

\DeclareSymbolFont{symbolsC}{U}{txsyc}{m}{n}
\DeclareMathSymbol{\boxright}{\mathrel}{symbolsC}{128}
\DeclareMathAlphabet{\mathpzc}{OT1}{pzc}{m}{it}

\usepackage{tikz-qtree}
% \usepackage{markdown}
%\RequirePackage{bussproofs}
\usetikzlibrary{arrows.meta}
\RequirePackage[tableaux]{prooftrees}
\forestset{line numbering, close with = x}
% Allow for easy commas inside trees
\renewcommand{\,}{\text{, }}


\usepackage{tabulary}

\usepackage{open-logic-config}

\setlength{\parskip}{1ex plus 0.5ex minus 0.2ex}

\AtBeginSection[]
{
\begin{frame}
	\Huge{\color{darkblue} \insertsection}
\end{frame}
}

\renewenvironment*{quote}	
	{\list{}{\rightmargin   \leftmargin} \item } 	
	{\endlist }

\definecolor{darkgreen}{rgb}{0,0.7,0}
\definecolor{darkblue}{rgb}{0,0,0.8}

\newcommand{\starttab}{\begin{center}
\vspace{6pt}
\begin{tabular}}

\newcommand{\stoptab}{\end{tabular}
\vspace{6pt}
\end{center}
\noindent}


\newcommand{\sif}{\rightarrow}
\newcommand{\siff}{\leftrightarrow}
\newcommand{\EF}{\end{frame}}


\newcommand{\TreeStart}[1]{
%\end{frame}
\begin{frame}
\begin{center}
\begin{tikzpicture}[scale=#1]
\tikzset{every tree node/.style={align=center,anchor=north}}
%\Tree
}

\newcommand{\TreeEnd}{
\end{tikzpicture}
%\end{center}
}

\newcommand{\DisplayArg}[2]{
\begin{enumerate}
{#1}
\end{enumerate}
\vspace{-6pt}
\hrulefill

%\hspace{14pt} #2
%{\addtolength{\leftskip}{14pt} #2}
\begin{quote}
{\normalfont #2}
\end{quote}
\vspace{12pt}
}

\newenvironment{ProofTree}[1][1]{
\begin{center}
\begin{tikzpicture}[scale=#1]
\tikzset{every tree node/.style={align=center,anchor=south}}
}
{
\end{tikzpicture}
\end{center}
}

\newcommand{\TreeFrame}[2]{
\begin{columns}[c]
\column{0.5\textwidth}
\begin{center}
\begin{prooftree}{}
#1
\end{prooftree}
\end{center}
\column{0.45\textwidth}
%\begin{markdown}
#2
%\end{markdown}
\end{columns}
}

\newcommand{\ScaledTreeFrame}[3]{
\begin{columns}[c]
\column{0.5\textwidth}
\begin{center}
\scalebox{#1}{
\begin{prooftree}{}
#2
\end{prooftree}
}
\end{center}
\column{0.45\textwidth}
%\begin{markdown}
#3
%\end{markdown}
\end{columns}
}

\usepackage[bb=boondox]{mathalfa}
\DeclareMathAlphabet{\mathbx}{U}{BOONDOX-ds}{m}{n}
\SetMathAlphabet{\mathbx}{bold}{U}{BOONDOX-ds}{b}{n}
\DeclareMathAlphabet{\mathbbx} {U}{BOONDOX-ds}{b}{n}


\newenvironment{oltableau}{\center\tableau{}} %wff format={anchor = base west}}}
       {\endtableau\endcenter}
       
\newcommand{\formula}[1]{$#1$}

\usepackage{tabulary}
\usepackage{booktabs}

\def\begincols{\begin{columns}}
\def\begincol{\begin{column}}
\def\endcol{\end{column}}
\def\endcols{\end{columns}}

\usepackage[italic]{mathastext}
\usepackage{nicefrac}

\definecolor{mygreen}{RGB}{0, 100, 0}
\definecolor{mypink2}{RGB}{219, 48, 122}
\definecolor{dodgerblue}{RGB}{30,144,255}

%\def\True{\textcolor{dodgerblue}{\text{T}}}
%\def\False{\textcolor{red}{\text{F}}}

\def\True{\mathbb{T}}
\def\False{\mathbb{F}}

% This is because arguments didn't have enough space after them
\usepackage{etoolbox}
\AfterEndEnvironment{description}{\vspace{9pt}}
\AfterEndEnvironment{oltableau}{\vspace{9pt}}
\BeforeBeginEnvironment{oltableau}{\vspace{9pt}}
\AfterEndEnvironment{center}{\vspace{12pt}}
\BeforeBeginEnvironment{tabular}{\vspace{9pt}}

\setlength\heavyrulewidth{0pt}
\setlength\lightrulewidth{0pt}

%\def\toprule{}
%\def\bottomrule{}
%\def\midrule{}

\setbeamertemplate{caption}{\raggedright\insertcaption}

\ifluatex
  \usepackage{selnolig}  % disable illegal ligatures
\fi

\title{305 Lecture 13.2 - Material Conditionals}
\author{Brian Weatherson}
\date{}

\begin{document}
\frame{\titlepage}

\begin{frame}{Plan}
\protect\hypertarget{plan}{}
\begin{itemize}
\tightlist
\item
  To discuss the conditional from propositional logic, the material
  conditional.
\end{itemize}
\end{frame}

\begin{frame}{Associated Reading}
\protect\hypertarget{associated-reading}{}
\begin{itemize}
\tightlist
\item
  Boxes and Diamonds, section 6.1-6.2
\end{itemize}
\end{frame}

\begin{frame}{What is It}
\protect\hypertarget{what-is-it}{}
The material conditional is what the truth tables say is the
conditional.

\begin{itemize}
\tightlist
\item
  We will write it as \(A \supset B\)
\item
  This just means \(\neg (A \wedge \neg B)\), i.e., \(\neg A \vee B\).
\end{itemize}

One interesting hypothesis is that this is the right way to interpret
English language conditionals.
\end{frame}

\begin{frame}{What About Hamlet}
\protect\hypertarget{what-about-hamlet}{}
Perhaps we should say that two interesting hypotheses are that the two
kinds of English language conditionals are best represented as material
conditionals.

\begin{itemize}
\tightlist
\item
  Hypothesis 1: If it were the case that \(A\) it would be the case that
  \(B\) = \(A \supset B\).
\item
  Hypothesis 2: If it is the case that \(A\) then it is the case that
  \(B\) = \(A \supset B\).
\end{itemize}

The first of these is wildly implausible; the second is a bit more
defensible.
\end{frame}

\begin{frame}{Material Subjunctives?}
\protect\hypertarget{material-subjunctives}{}
How can you complete this sentence so that it is true?

\begin{itemize}
\tightlist
\item
  If the United States had entered World War II in 1939, then \ldots{}
  \pause 
\end{itemize}

On the material conditional theory, any sentence whatsoever you put in
place of the dots will make this true. \pause  That's not very
plausible.
\end{frame}

\begin{frame}{Material Indicatives}
\protect\hypertarget{material-indicatives}{}
\begin{itemize}
\tightlist
\item
  But the material conditional theory is at least a little plausible for
  indicative conditionals.
\item
  At least, some smart people have defended it
\end{itemize}
\end{frame}

\begin{frame}{Nine Formal Properties}
\protect\hypertarget{nine-formal-properties}{}
\begin{enumerate}
\tightlist
\item
  Modus Ponens - \(A, A \rightarrow B \vDash B \pause\)
\item
  Agglomeration -
  \(A \rightarrow B, A \rightarrow C \vDash A \rightarrow (B \wedge C)\pause\)
\item
  Transitive -
  \(A \rightarrow B, B \rightarrow C \vDash A \rightarrow C\pause\)
\item
  Contraposition - \(A \rightarrow B \vDash \neg B \rightarrow \neg A\)
\item
  Antecedent Strengthening -
  \(A \rightarrow B \vDash (A \wedge C) \rightarrow B\)
\item
  Paradox 1 - \(B \vDash A \rightarrow B\)
\item
  Paradox 2 - \(\neg A \vDash A \rightarrow B\)
\item
  Strict Paradox - \(\Box B \vDash A \rightarrow B\)
\item
  Disjunction Paradox - \((A \rightarrow B) \vee (B \rightarrow A)\)
\end{enumerate}
\end{frame}

\begin{frame}{Nursery Rhyme}
\protect\hypertarget{nursery-rhyme}{}
For want of a nail the shoe was lost.

For want of a shoe the horse was lost.

For want of a horse the rider was lost.

For want of a rider the message was lost.

For want of a message the battle was lost.

For want of a battle the kingdom was lost.

And all for the want of a horseshoe nail.
\end{frame}

\begin{frame}{As An Argument}
\protect\hypertarget{as-an-argument}{}
\begin{enumerate}
\tightlist
\item
  If we hadn't lost the nail, we wouldn't have lost the shoe.
\item
  If we hadn't lost the shoe, we wouldn't have lost the horse.
\item
  If we hadn't lost the horse, we wouldn't have lost the rider
\item
  If we hadn't lost the rider, we wouldn't have lost the message
\item
  If we hadn't lost the message, we wouldn't have lost the battle
\item
  If we hadn't lost the battle, we wouldn't have lost the kingdom.
  \vspace{18pt}
\item
  So, if we hadn't lost the nail, we wouldn't have lost the kingdom.
\end{enumerate}
\end{frame}

\begin{frame}{Nine Formal Properties}
\protect\hypertarget{nine-formal-properties-1}{}
\begin{enumerate}
\tightlist
\item
  Modus Ponens - \(A, A \rightarrow B \vDash B\)
\item
  Agglomeration -
  \(A \rightarrow B, A \rightarrow C \vDash A \rightarrow (B \wedge C)\)
\item
  Transitive -
  \(A \rightarrow B, B \rightarrow C \vDash A \rightarrow C\)
\item
  Contraposition -
  \(A \rightarrow B \vDash \neg B \rightarrow \neg A \pause\)
\item
  Antecedent Strengthening -
  \(A \rightarrow B \vDash (A \wedge C) \rightarrow B \pause\)
\item
  Paradox 1 - \(B \vDash A \rightarrow B\pause\)
\item
  Paradox 2 - \(\neg A \vDash A \rightarrow B\pause\)
\item
  Strict Paradox - \(\Box B \vDash A \rightarrow B\pause\)
\item
  Disjunction Paradox - \((A \rightarrow B) \vee (B \rightarrow A)\)
\end{enumerate}
\end{frame}

\begin{frame}{Modus Ponens}
\protect\hypertarget{modus-ponens}{}
\begin{itemize}
\tightlist
\item
  Imagine that in some NCAA tournament, the only Big 10 teams left in
  the last 16 are Michigan and Ohio State.
\item
  And in this sport, Michigan is pretty good, while Ohio State is super
  lucky to make the last 16. \pause 
\end{itemize}

Now consider the following two conditionals.

\begin{enumerate}
\tightlist
\item
  If a Big Ten team wins, then if it isn't Michigan, it will be Ohio
  State.
\item
  If Michigan doesn't win, Ohio State will win.
\end{enumerate}
\end{frame}

\begin{frame}{Modus Ponens Counterexample?}
\protect\hypertarget{modus-ponens-counterexample}{}
Vann McGee made the following argument.

\begin{itemize}
\tightlist
\item
  Conditional 1 is true, almost by definition. \pause 
\item
  Conditional 2 is false. \pause 
\item
  And those judgments stay correct even if it turns out that a Big Ten
  team does in fact win (especially if it is Michigan). \pause 
\item
  In general, \(A \rightarrow (B \rightarrow C)\) and \(A\) seem like
  they can be true but (especially when \(A\) is not known)
  \(B \rightarrow C\) can seem false.
\end{itemize}
\end{frame}

\begin{frame}{Agglomeration}
\protect\hypertarget{agglomeration}{}
\begin{itemize}
\tightlist
\item
  This is very hard to find intuitive counterexamples to.
\item
  Take this to be a challenge.
\end{itemize}
\end{frame}

\begin{frame}{Transitivity}
\protect\hypertarget{transitivity}{}
\begin{itemize}
\tightlist
\item
  This often sound right if you run the examples in that order.
\item
  But it can sound very bad if you don't.
\end{itemize}

For example:

\begin{enumerate}
\tightlist
\item
  If I win the lottery tonight, I'll be very happy.
\item
  If I get hit by a car on the way home and win the lottery tonight,
  then I'll win the lottery tonight.
\item
  So if I get hit by a car on the way home and win the lottery tonight,
  I'll be very happy.
\end{enumerate}
\end{frame}

\begin{frame}{Contraposition}
\protect\hypertarget{contraposition}{}
Some of the examples to this can sound a little silly, but still they
arguably do sound like English.

\begin{enumerate}
\tightlist
\item
  If we win this election, we won't win it by a lot.
\item
  So if we win by a lot, we won't win.
\end{enumerate}
\end{frame}

\begin{frame}{Antecedent Strengthening}
\protect\hypertarget{antecedent-strengthening}{}
The same cases that are problems for transitivity are problems here.

\begin{enumerate}
\tightlist
\item
  If I win the lottery tonight, I'll be very happy.
\item
  So if I get hit by a car on the way home and win the lottery tonight,
  I'll be vey happy
\end{enumerate}
\end{frame}

\begin{frame}{The Paradoxes}
\protect\hypertarget{the-paradoxes}{}
\begin{itemize}
\tightlist
\item
  These are fairly easy to generate intuitive counterexamples for.
\item
  Note in particular that the disjunction one allows for \(B = \neg A\).
\item
  So we get the following: Either if we win, we'll lose, or if we lose,
  we'll win.
\end{itemize}
\end{frame}

\begin{frame}{For Next Time}
\protect\hypertarget{for-next-time}{}
We'll discuss the so called strict conditional.
\end{frame}

\end{document}
