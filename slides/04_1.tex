% Options for packages loaded elsewhere
\PassOptionsToPackage{unicode}{hyperref}
\PassOptionsToPackage{hyphens}{url}
%
\documentclass[
  ignorenonframetext,
]{beamer}
\usepackage{pgfpages}
\setbeamertemplate{caption}[numbered]
\setbeamertemplate{caption label separator}{: }
\setbeamercolor{caption name}{fg=normal text.fg}
\beamertemplatenavigationsymbolsempty
% Prevent slide breaks in the middle of a paragraph
\widowpenalties 1 10000
\raggedbottom
\setbeamertemplate{part page}{
  \centering
  \begin{beamercolorbox}[sep=16pt,center]{part title}
    \usebeamerfont{part title}\insertpart\par
  \end{beamercolorbox}
}
\setbeamertemplate{section page}{
  \centering
  \begin{beamercolorbox}[sep=12pt,center]{part title}
    \usebeamerfont{section title}\insertsection\par
  \end{beamercolorbox}
}
\setbeamertemplate{subsection page}{
  \centering
  \begin{beamercolorbox}[sep=8pt,center]{part title}
    \usebeamerfont{subsection title}\insertsubsection\par
  \end{beamercolorbox}
}
\AtBeginPart{
  \frame{\partpage}
}
\AtBeginSection{
  \ifbibliography
  \else
    \frame{\sectionpage}
  \fi
}
\AtBeginSubsection{
  \frame{\subsectionpage}
}
\usepackage{amsmath,amssymb}
\usepackage{lmodern}
\usepackage{ifxetex,ifluatex}
\ifnum 0\ifxetex 1\fi\ifluatex 1\fi=0 % if pdftex
  \usepackage[T1]{fontenc}
  \usepackage[utf8]{inputenc}
  \usepackage{textcomp} % provide euro and other symbols
\else % if luatex or xetex
  \usepackage{unicode-math}
  \defaultfontfeatures{Scale=MatchLowercase}
  \defaultfontfeatures[\rmfamily]{Ligatures=TeX,Scale=1}
  \setmainfont[BoldFont = SF Pro Rounded Semibold]{SF Pro Rounded}
  \setmathfont[]{STIX Two Math}
\fi
\usefonttheme{serif} % use mainfont rather than sansfont for slide text
% Use upquote if available, for straight quotes in verbatim environments
\IfFileExists{upquote.sty}{\usepackage{upquote}}{}
\IfFileExists{microtype.sty}{% use microtype if available
  \usepackage[]{microtype}
  \UseMicrotypeSet[protrusion]{basicmath} % disable protrusion for tt fonts
}{}
\makeatletter
\@ifundefined{KOMAClassName}{% if non-KOMA class
  \IfFileExists{parskip.sty}{%
    \usepackage{parskip}
  }{% else
    \setlength{\parindent}{0pt}
    \setlength{\parskip}{6pt plus 2pt minus 1pt}}
}{% if KOMA class
  \KOMAoptions{parskip=half}}
\makeatother
\usepackage{xcolor}
\IfFileExists{xurl.sty}{\usepackage{xurl}}{} % add URL line breaks if available
\IfFileExists{bookmark.sty}{\usepackage{bookmark}}{\usepackage{hyperref}}
\hypersetup{
  pdftitle={305 Lecture 4.1 - Using Truth Trees},
  pdfauthor={Brian Weatherson},
  hidelinks,
  pdfcreator={LaTeX via pandoc}}
\urlstyle{same} % disable monospaced font for URLs
\newif\ifbibliography
\setlength{\emergencystretch}{3em} % prevent overfull lines
\providecommand{\tightlist}{%
  \setlength{\itemsep}{0pt}\setlength{\parskip}{0pt}}
\setcounter{secnumdepth}{-\maxdimen} % remove section numbering
\let\Tiny=\tiny

 \setbeamertemplate{navigation symbols}{} 

% \usetheme{Madrid}
 \usetheme[numbering=none, progressbar=foot]{metropolis}
 \usecolortheme{wolverine}
 \usepackage{color}
 \usepackage{MnSymbol}
% \usepackage{movie15}

\usepackage{amssymb}% http://ctan.org/pkg/amssymb
\usepackage{pifont}% http://ctan.org/pkg/pifont
\newcommand{\cmark}{\ding{51}}%
\newcommand{\xmark}{\ding{55}}%

\DeclareSymbolFont{symbolsC}{U}{txsyc}{m}{n}
\DeclareMathSymbol{\boxright}{\mathrel}{symbolsC}{128}
\DeclareMathAlphabet{\mathpzc}{OT1}{pzc}{m}{it}

 \usepackage{tikz-qtree}
% \usepackage{markdown}
%\RequirePackage{bussproofs}
\RequirePackage[tableaux]{prooftrees}
\usetikzlibrary{arrows.meta}
 \forestset{line numbering, close with = x}
% Allow for easy commas inside trees
\renewcommand{\,}{\text{, }}


\usepackage{tabulary}

\usepackage{open-logic-config}

\setlength{\parskip}{1ex plus 0.5ex minus 0.2ex}

\AtBeginSection[]
{
\begin{frame}
	\Huge{\color{darkblue} \insertsection}
\end{frame}
}

\renewenvironment*{quote}	
	{\list{}{\rightmargin   \leftmargin} \item } 	
	{\endlist }

\definecolor{darkgreen}{rgb}{0,0.7,0}
\definecolor{darkblue}{rgb}{0,0,0.8}

\newcommand{\starttab}{\begin{center}
\vspace{6pt}
\begin{tabular}}

\newcommand{\stoptab}{\end{tabular}
\vspace{6pt}
\end{center}
\noindent}


\newcommand{\sif}{\rightarrow}
\newcommand{\siff}{\leftrightarrow}
\newcommand{\EF}{\end{frame}}


\newcommand{\TreeStart}[1]{
%\end{frame}
\begin{frame}
\begin{center}
\begin{tikzpicture}[scale=#1]
\tikzset{every tree node/.style={align=center,anchor=north}}
%\Tree
}

\newcommand{\TreeEnd}{
\end{tikzpicture}
%\end{center}
}

\newcommand{\DisplayArg}[2]{
\begin{enumerate}
{#1}
\end{enumerate}
\vspace{-6pt}
\hrulefill

%\hspace{14pt} #2
%{\addtolength{\leftskip}{14pt} #2}
\begin{quote}
{\normalfont #2}
\end{quote}
\vspace{12pt}
}

\newenvironment{ProofTree}[1][1]{
\begin{center}
\begin{tikzpicture}[scale=#1]
\tikzset{every tree node/.style={align=center,anchor=south}}
}
{
\end{tikzpicture}
\end{center}
}

\newcommand{\TreeFrame}[2]{
\begin{columns}[c]
\column{0.5\textwidth}
\begin{center}
\begin{prooftree}{}
#1
\end{prooftree}
\end{center}
\column{0.45\textwidth}
%\begin{markdown}
#2
%\end{markdown}
\end{columns}
}

\newcommand{\ScaledTreeFrame}[3]{
\begin{columns}[c]
\column{0.5\textwidth}
\begin{center}
\scalebox{#1}{
\begin{prooftree}{}
#2
\end{prooftree}
}
\end{center}
\column{0.45\textwidth}
%\begin{markdown}
#3
%\end{markdown}
\end{columns}
}

\usepackage[bb=boondox]{mathalfa}
\DeclareMathAlphabet{\mathbx}{U}{BOONDOX-ds}{m}{n}
\SetMathAlphabet{\mathbx}{bold}{U}{BOONDOX-ds}{b}{n}
\DeclareMathAlphabet{\mathbbx} {U}{BOONDOX-ds}{b}{n}


\newenvironment{oltableau}{\center\tableau{}} %wff format={anchor = base west}}}
       {\endtableau\endcenter}
       
\newcommand{\formula}[1]{$#1$}

\usepackage{tabulary}
\usepackage{booktabs}

\def\begincols{\begin{columns}}
\def\begincol{\begin{column}}
\def\endcol{\end{column}}
\def\endcols{\end{columns}}

\usepackage[italic]{mathastext}
\usepackage{nicefrac}

\definecolor{mygreen}{RGB}{0, 100, 0}
\definecolor{mypink2}{RGB}{219, 48, 122}
\definecolor{dodgerblue}{RGB}{30,144,255}

%\def\True{\textcolor{dodgerblue}{\text{T}}}
%\def\False{\textcolor{red}{\text{F}}}

\def\True{\mathbb{T}}
\def\False{\mathbb{F}}

% This is because arguments didn't have enough space after them
\usepackage{etoolbox}
\AfterEndEnvironment{description}{\vspace{9pt}}
\AfterEndEnvironment{oltableau}{\vspace{9pt}}
\BeforeBeginEnvironment{oltableau}{\vspace{9pt}}
\AfterEndEnvironment{center}{\vspace{12pt}}
\BeforeBeginEnvironment{tabular}{\vspace{9pt}}

\setlength\heavyrulewidth{0pt}
\setlength\lightrulewidth{0pt}

%\def\toprule{}
%\def\bottomrule{}
%\def\midrule{}

\setbeamertemplate{caption}{\raggedright\insertcaption}

\ifluatex
  \usepackage{selnolig}  % disable illegal ligatures
\fi

\title{305 Lecture 4.1 - Using Truth Trees}
\author{Brian Weatherson}
\date{}

\begin{document}
\frame{\titlepage}

\begin{frame}{Plan}
\protect\hypertarget{plan}{}
Showing how we use truth trees to check for logical truth and validity.
\end{frame}

\begin{frame}{Reading}
\protect\hypertarget{reading}{}
Boxes and Diamonds, section 2.4.
\end{frame}

\begin{frame}
\begin{columns}[T]
\begin{column}{0.48\textwidth}
\begin{oltableau}
[\sFmla{\True}{\neg A},
    [\sFmla{\False}{A}, just = {\TRule{\True}{\neg}[1]}]
]
\end{oltableau}
\end{column}

\begin{column}{0.48\textwidth}
\begin{oltableau}
[\sFmla{\False}{\neg A},
    [\sFmla{\True}{A}, just = {\TRule{\False}{\neg}[1]}]
]
\end{oltableau}
\end{column}
\end{columns}

\vspace{-3pt}

\begin{columns}[T]
\begin{column}{0.48\textwidth}
\begin{oltableau}
[\sFmla{\True}{A \wedge B},
    [\sFmla{\True}{A}, just = {\TRule{\True}{\wedge}[1]}
      [\sFmla{\True}{B}, just = {\TRule{\True}{\wedge}[1]}]
    ]
]
\end{oltableau}
\end{column}

\begin{column}{0.48\textwidth}
\begin{oltableau}
[\sFmla{\False}{A \wedge B},
    [\sFmla{\False}{A}, just = {\TRule{\False}{\wedge}[1]}]
    [\sFmla{\False}{B}, just = {\TRule{\False}{\wedge}[1]}]
]
\end{oltableau}
\end{column}
\end{columns}

\vspace{-3pt}

\begin{columns}[T]
\begin{column}{0.48\textwidth}
\begin{oltableau}
[\sFmla{\True}{A \vee B},
    [\sFmla{\True}{A}, just = {\TRule{\True}{\vee}[1]}]
    [\sFmla{\True}{B}, just = {\TRule{\True}{\vee}[1]}]
]
\end{oltableau}
\end{column}

\begin{column}{0.48\textwidth}
\begin{oltableau}
[\sFmla{\False}{A \vee B},
    [\sFmla{\False}{A}, just = {\TRule{\False}{\vee}[1]}
      [\sFmla{\False}{B}, just = {\TRule{\False}{\vee}[1]}]
    ]
]
\end{oltableau}
\end{column}
\end{columns}

\vspace{-3pt}

\begin{columns}[T]
\begin{column}{0.48\textwidth}
\begin{oltableau}
[\sFmla{\True}{A \rightarrow B},
    [\sFmla{\False}{A}, just = {\TRule{\True}{\rightarrow}[1]}]
    [\sFmla{\True}{B}, just = {\TRule{\True}{\rightarrow}[1]}]
]
\end{oltableau}
\end{column}

\begin{column}{0.48\textwidth}
\begin{oltableau}
[\sFmla{\False}{A \rightarrow B},
    [\sFmla{\True}{A}, just = {\TRule{\False}{\rightarrow}[1]}
      [\sFmla{\False}{B}, just = {\TRule{\False}{\rightarrow}[1]}]
    ]
]
\end{oltableau}
\end{column}
\end{columns}
\end{frame}

\begin{frame}{Closure}
\protect\hypertarget{closure}{}
\begin{itemize}
\tightlist
\item
  A tree without branches closes if there is a sentence on the tree that
  is marked as both true and false.
\item
  Here is an example. The first line is stipulated, the rest are
  derived.
\end{itemize}

\begin{oltableau}
[\sFmla{\True}{A \wedge \neg A},
  [\sFmla{\True}{A}, just = {\TRule{\True}{\wedge}[1]}
    [\sFmla{\True}{\neg A}, just = {\TRule{\True}{\wedge}[1]}
      [\sFmla{\False}{A}, just = {\TRule{\True}{\neg}[3]}, close
      ]
    ]
  ]
]
\end{oltableau}
\end{frame}

\begin{frame}{Interpreting Closure}
\protect\hypertarget{interpreting-closure}{}
\begin{itemize}
\tightlist
\item
  If a tree is closed, it means the initial assumptions can't be true.
\item
  So this tree means that the initial assumption \(A \wedge \neg A\)
  can't be true.
\end{itemize}

\begin{oltableau}
[\sFmla{\True}{A \wedge \neg A},
  [\sFmla{\True}{A}, just = {\TRule{\True}{\wedge}[1]}
    [\sFmla{\True}{\neg A}, just = {\TRule{\True}{\wedge}[1]}
      [\sFmla{\False}{A}, just = {\TRule{\True}{\neg}[3]}, close
      ]
    ]
  ]
]
\end{oltableau}
\end{frame}

\begin{frame}{Closure with Branches}
\protect\hypertarget{closure-with-branches}{}
\begin{itemize}
\tightlist
\item
  A branching tree closes if every branch closes.
\item
  The next slide has an example, with in this case the top two lines
  stipulated.
\end{itemize}
\end{frame}

\begin{frame}
\begin{oltableau}
[\sFmla{\True}{\neg A \wedge \neg B},
  [\sFmla{\True}{A \vee B},
    [\sFmla{\True}{\neg A}, just = {\TRule{\True}{\wedge}[1]}
      [\sFmla{\True}{\neg B}, just = {\TRule{\True}{\wedge}[1]}
        [\sFmla{\False}{A}, just = {\TRule{\True}{\neg}[3]}
          [\sFmla{\False}{B}, just = {\TRule{\True}{\neg}[4]}
            [\sFmla{\True}{A}, just = {\TRule{\True}{\vee}[2]}, close]
            [\sFmla{\True}{B}, just = {\TRule{\True}{\vee}[2]}, close]
          ]
        ]
      ]
    ]
  ]
]
\end{oltableau}
\end{frame}

\begin{frame}{Closure with Branches}
\protect\hypertarget{closure-with-branches-1}{}
\begin{itemize}
\tightlist
\item
  A tree with any open branches is open, i.e., not closed.
\item
  If a tree has some open branches and some closed branches, it is
  \textbf{open}.
\item
  All that matters is if all branches are closed.
\item
  The next slide is an example of an open tree.
\end{itemize}
\end{frame}

\begin{frame}
\begin{oltableau}
[\sFmla{\True}{\neg A \wedge B},
  [\sFmla{\True}{A \vee B},
    [\sFmla{\True}{\neg A}, just = {\TRule{\True}{\wedge}[1]}
      [\sFmla{\True}{B}, just = {\TRule{\True}{\wedge}[1]}
        [\sFmla{\False}{A}, just = {\TRule{\True}{\neg}[3]}
            [\sFmla{\True}{A}, just = {\TRule{\True}{\vee}[2]}, close]
            [\sFmla{\True}{B}, just = {\TRule{\True}{\vee}[2]}]
        ]
      ]
    ]
  ]
]
\end{oltableau}
\end{frame}

\begin{frame}{Logical Truth}
\protect\hypertarget{logical-truth}{}
Here is the algorithm for seeing whether a sentence is a logical truth.

\begin{enumerate}[<+->]
\tightlist
\item
  Start a tree by saying at line 1 that the sentence is \textbf{False}.
\item
  If the tree closes, it is a logical truth.
\item
  If the tree does not close, it is not a logical truth (of
  propositional logic).
\end{enumerate}
\end{frame}

\begin{frame}{Looking for Counterexamples (Tables)}
\protect\hypertarget{looking-for-counterexamples-tables}{}
\begin{itemize}
\tightlist
\item
  Truth tables didn't just tell us that something failed to be a logical
  truth (of propositional logic).
\item
  It told us where the failure was.
\item
  You didn't just know that there was an \textcolor{red}{$\False$} in
  the relevant column, you knew which row it was on.
\item
  And that told you where to look for counterexamples.
\end{itemize}
\end{frame}

\begin{frame}{Looking for Counterexamples (Trees)}
\protect\hypertarget{looking-for-counterexamples-trees}{}
\begin{itemize}
\tightlist
\item
  The same thing happens with trees.
\item
  By reading off the open branch, you can see where the sentence fails.
\item
  The trick is to focus on the \textbf{atomic} sentences on the branch.
\item
  These are the ones with no connectives at all.
\end{itemize}
\end{frame}

\begin{frame}
\begin{oltableau}
[\sFmla{\False}{A \rightarrow (A \wedge B)},
  [\sFmla{\True}{A}, just = {\TRule{\True}{\rightarrow}[1]}
    [\sFmla{\False}{A \wedge B}, just = {\TRule{\True}{\rightarrow}[1]}
      [\sFmla{\False}{A}, just = {\TRule{\False}{\wedge}[3]}, close]
      [\sFmla{\False}{B}, just = {\TRule{\False}{\wedge}[3]}]
    ]
  ]
]
\end{oltableau}

\begin{itemize}
\tightlist
\item
  The right hand branch doesn't close.
\item
  The atomics on that branch are that \(A\) is \textcolor{red}{$\True$}
  and \(B\) is \textcolor{red}{$\False$}.
\item
  So that's the line on the truth table where
  \(A \rightarrow (A \wedge B)\) is \textcolor{red}{$\False$}.
\end{itemize}
\end{frame}

\begin{frame}{Validity}
\protect\hypertarget{validity}{}
Here is the algorithm for seeing whether an argument is valid (in
propositional logic).

\begin{enumerate}[<+->]
\tightlist
\item
  Start a tree with one line for each premise, saying that the premise
  is \textbf{True}.
\item
  Then have a line that says the conclusion is \textbf{False}.
\item
  If the tree closes, the argument is valid.
\item
  If the tree does not close, the argumnt is not valid (in propositional
  logic).
\end{enumerate}
\end{frame}

\begin{frame}{For Next Time}
\protect\hypertarget{for-next-time}{}
We will illustrate this algorithm.
\end{frame}

\end{document}
