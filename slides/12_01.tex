% Options for packages loaded elsewhere
\PassOptionsToPackage{unicode}{hyperref}
\PassOptionsToPackage{hyphens}{url}
%
\documentclass[
  ignorenonframetext,
]{beamer}
\usepackage{pgfpages}
\setbeamertemplate{caption}[numbered]
\setbeamertemplate{caption label separator}{: }
\setbeamercolor{caption name}{fg=normal text.fg}
\beamertemplatenavigationsymbolsempty
% Prevent slide breaks in the middle of a paragraph
\widowpenalties 1 10000
\raggedbottom
\setbeamertemplate{part page}{
  \centering
  \begin{beamercolorbox}[sep=16pt,center]{part title}
    \usebeamerfont{part title}\insertpart\par
  \end{beamercolorbox}
}
\setbeamertemplate{section page}{
  \centering
  \begin{beamercolorbox}[sep=12pt,center]{part title}
    \usebeamerfont{section title}\insertsection\par
  \end{beamercolorbox}
}
\setbeamertemplate{subsection page}{
  \centering
  \begin{beamercolorbox}[sep=8pt,center]{part title}
    \usebeamerfont{subsection title}\insertsubsection\par
  \end{beamercolorbox}
}
\AtBeginPart{
  \frame{\partpage}
}
\AtBeginSection{
  \ifbibliography
  \else
    \frame{\sectionpage}
  \fi
}
\AtBeginSubsection{
  \frame{\subsectionpage}
}
\usepackage{amsmath,amssymb}
\usepackage{lmodern}
\usepackage{ifxetex,ifluatex}
\ifnum 0\ifxetex 1\fi\ifluatex 1\fi=0 % if pdftex
  \usepackage[T1]{fontenc}
  \usepackage[utf8]{inputenc}
  \usepackage{textcomp} % provide euro and other symbols
\else % if luatex or xetex
  \usepackage{unicode-math}
  \defaultfontfeatures{Scale=MatchLowercase}
  \defaultfontfeatures[\rmfamily]{Ligatures=TeX,Scale=1}
  \setmainfont[BoldFont = SF Pro Rounded Semibold]{SF Pro Rounded}
  \setmathfont[]{STIX Two Math}
\fi
\usefonttheme{serif} % use mainfont rather than sansfont for slide text
% Use upquote if available, for straight quotes in verbatim environments
\IfFileExists{upquote.sty}{\usepackage{upquote}}{}
\IfFileExists{microtype.sty}{% use microtype if available
  \usepackage[]{microtype}
  \UseMicrotypeSet[protrusion]{basicmath} % disable protrusion for tt fonts
}{}
\makeatletter
\@ifundefined{KOMAClassName}{% if non-KOMA class
  \IfFileExists{parskip.sty}{%
    \usepackage{parskip}
  }{% else
    \setlength{\parindent}{0pt}
    \setlength{\parskip}{6pt plus 2pt minus 1pt}}
}{% if KOMA class
  \KOMAoptions{parskip=half}}
\makeatother
\usepackage{xcolor}
\IfFileExists{xurl.sty}{\usepackage{xurl}}{} % add URL line breaks if available
\IfFileExists{bookmark.sty}{\usepackage{bookmark}}{\usepackage{hyperref}}
\hypersetup{
  pdftitle={305 Lecture 12.1 - Modal Tableau},
  pdfauthor={Brian Weatherson},
  hidelinks,
  pdfcreator={LaTeX via pandoc}}
\urlstyle{same} % disable monospaced font for URLs
\newif\ifbibliography
\setlength{\emergencystretch}{3em} % prevent overfull lines
\providecommand{\tightlist}{%
  \setlength{\itemsep}{0pt}\setlength{\parskip}{0pt}}
\setcounter{secnumdepth}{-\maxdimen} % remove section numbering
\let\Tiny=\tiny

 \setbeamertemplate{navigation symbols}{} 

% \usetheme{Madrid}
 \usetheme[numbering=none, progressbar=foot]{metropolis}
 \usecolortheme{wolverine}
 \usepackage{color}
 \usepackage{MnSymbol}
% \usepackage{movie15}

\usepackage{amssymb}% http://ctan.org/pkg/amssymb
\usepackage{pifont}% http://ctan.org/pkg/pifont
\newcommand{\cmark}{\ding{51}}%
\newcommand{\xmark}{\ding{55}}%

\DeclareSymbolFont{symbolsC}{U}{txsyc}{m}{n}
\DeclareMathSymbol{\boxright}{\mathrel}{symbolsC}{128}
\DeclareMathAlphabet{\mathpzc}{OT1}{pzc}{m}{it}

\usepackage{tikz-qtree}
% \usepackage{markdown}
%\RequirePackage{bussproofs}
\usetikzlibrary{arrows.meta}
\RequirePackage[tableaux]{prooftrees}
\forestset{line numbering, close with = x}
% Allow for easy commas inside trees
\renewcommand{\,}{\text{, }}


\usepackage{tabulary}

\usepackage{open-logic-config}

\setlength{\parskip}{1ex plus 0.5ex minus 0.2ex}

\AtBeginSection[]
{
\begin{frame}
	\Huge{\color{darkblue} \insertsection}
\end{frame}
}

\renewenvironment*{quote}	
	{\list{}{\rightmargin   \leftmargin} \item } 	
	{\endlist }

\definecolor{darkgreen}{rgb}{0,0.7,0}
\definecolor{darkblue}{rgb}{0,0,0.8}

\newcommand{\starttab}{\begin{center}
\vspace{6pt}
\begin{tabular}}

\newcommand{\stoptab}{\end{tabular}
\vspace{6pt}
\end{center}
\noindent}


\newcommand{\sif}{\rightarrow}
\newcommand{\siff}{\leftrightarrow}
\newcommand{\EF}{\end{frame}}


\newcommand{\TreeStart}[1]{
%\end{frame}
\begin{frame}
\begin{center}
\begin{tikzpicture}[scale=#1]
\tikzset{every tree node/.style={align=center,anchor=north}}
%\Tree
}

\newcommand{\TreeEnd}{
\end{tikzpicture}
%\end{center}
}

\newcommand{\DisplayArg}[2]{
\begin{enumerate}
{#1}
\end{enumerate}
\vspace{-6pt}
\hrulefill

%\hspace{14pt} #2
%{\addtolength{\leftskip}{14pt} #2}
\begin{quote}
{\normalfont #2}
\end{quote}
\vspace{12pt}
}

\newenvironment{ProofTree}[1][1]{
\begin{center}
\begin{tikzpicture}[scale=#1]
\tikzset{every tree node/.style={align=center,anchor=south}}
}
{
\end{tikzpicture}
\end{center}
}

\newcommand{\TreeFrame}[2]{
\begin{columns}[c]
\column{0.5\textwidth}
\begin{center}
\begin{prooftree}{}
#1
\end{prooftree}
\end{center}
\column{0.45\textwidth}
%\begin{markdown}
#2
%\end{markdown}
\end{columns}
}

\newcommand{\ScaledTreeFrame}[3]{
\begin{columns}[c]
\column{0.5\textwidth}
\begin{center}
\scalebox{#1}{
\begin{prooftree}{}
#2
\end{prooftree}
}
\end{center}
\column{0.45\textwidth}
%\begin{markdown}
#3
%\end{markdown}
\end{columns}
}

\usepackage[bb=boondox]{mathalfa}
\DeclareMathAlphabet{\mathbx}{U}{BOONDOX-ds}{m}{n}
\SetMathAlphabet{\mathbx}{bold}{U}{BOONDOX-ds}{b}{n}
\DeclareMathAlphabet{\mathbbx} {U}{BOONDOX-ds}{b}{n}


\newenvironment{oltableau}{\center\tableau{}} %wff format={anchor = base west}}}
       {\endtableau\endcenter}
       
\newcommand{\formula}[1]{$#1$}

\usepackage{tabulary}
\usepackage{booktabs}

\def\begincols{\begin{columns}}
\def\begincol{\begin{column}}
\def\endcol{\end{column}}
\def\endcols{\end{columns}}

\usepackage[italic]{mathastext}
\usepackage{nicefrac}

\definecolor{mygreen}{RGB}{0, 100, 0}
\definecolor{mypink2}{RGB}{219, 48, 122}
\definecolor{dodgerblue}{RGB}{30,144,255}

%\def\True{\textcolor{dodgerblue}{\text{T}}}
%\def\False{\textcolor{red}{\text{F}}}

\def\True{\mathbb{T}}
\def\False{\mathbb{F}}

% This is because arguments didn't have enough space after them
\usepackage{etoolbox}
\AfterEndEnvironment{description}{\vspace{9pt}}
\AfterEndEnvironment{oltableau}{\vspace{9pt}}
\BeforeBeginEnvironment{oltableau}{\vspace{9pt}}
\AfterEndEnvironment{center}{\vspace{12pt}}
\BeforeBeginEnvironment{tabular}{\vspace{9pt}}

\setlength\heavyrulewidth{0pt}
\setlength\lightrulewidth{0pt}

%\def\toprule{}
%\def\bottomrule{}
%\def\midrule{}

\setbeamertemplate{caption}{\raggedright\insertcaption}

\ifluatex
  \usepackage{selnolig}  % disable illegal ligatures
\fi

\title{305 Lecture 12.1 - Modal Tableau}
\author{Brian Weatherson}
\date{}

\begin{document}
\frame{\titlepage}

\begin{frame}{Plan}
\protect\hypertarget{plan}{}
\begin{itemize}
\tightlist
\item
  To introduce tableau for proving things in modal logic.
\end{itemize}
\end{frame}

\begin{frame}{Associated Reading}
\protect\hypertarget{associated-reading}{}
\begin{itemize}
\tightlist
\item
  Boxes and Diamonds, section 5.1
\end{itemize}
\end{frame}

\begin{frame}{Modal Tableau}
\protect\hypertarget{modal-tableau}{}
One big difference:

\begin{itemize}
\tightlist
\item
  On each line, we include reference to a world.
\item
  The line says that a particular proposition is true or false \emph{at
  a world}.
\item
  The tableau only close if the tableau says the same proposition is
  both true and false \emph{at the same world}.
\end{itemize}
\end{frame}

\begin{frame}{Referring to Worlds}
\protect\hypertarget{referring-to-worlds}{}
We refer to a world with a string of numbers, such as 1.2.1.3.

\begin{itemize}
\tightlist
\item
  The string tells you something (but not everything) about R relations.
\item
  World \(x\) can always access world \(x.y\).
\item
  So there is an R-relation from 1.2.1 to 1.2.1.3, and indeed to
  1.2.1.\(x\) for any \(x\).
\item
  These don't exhaust the R-relations; perhaps there is also an
  R-relation from 1.2.1.3 back to 1.2.1.
\item
  But the relation from \(x\) to \(x.y\) is guaranteed.
\item
  Note that worlds can be picked out by multiple strings - we do not
  assume that 1.1 and 1.2 are different, though we don't assume they are
  the same either.
\end{itemize}
\end{frame}

\begin{frame}{Rules}
\protect\hypertarget{rules}{}
The rules for the old connectives stay as they are.

\begin{itemize}
\tightlist
\item
  The only difference is that you have to note which world you are in.
\item
  So if you have that \(A \wedge B\) is true in 1.4.7, then you have to
  write down that \(A\) is true in 1.4.7, and that \(B\) is true in
  1.4.7.
\item
  And if \(A \vee B\) is true in 1.6.8 you have to have two branches,
  one where \(A\) is true in 1.6.8, and the other where \(B\) is true in
  1.6.8.
\item
  But otherwise things are as they were before.
\end{itemize}
\end{frame}

\begin{frame}{Rules for \(\Box\)}
\protect\hypertarget{rules-for-box}{}
If \(\Box A\) is true at \(x\), then for any \(x.y\) that is already on
the tree, we can infer that \(A\) is true at \(x.y\).

\begin{itemize}
\tightlist
\item
  Note: If there is no \(x.y\) on the tree, we can't assume that there
  is one.
\item
  \(\Box A\) can be true at a world because that world can't access any
  other world.
\end{itemize}
\end{frame}

\begin{frame}{Rules for \(\Box\) (cont)}
\protect\hypertarget{rules-for-box-cont}{}
If \(\Box A\) is false at \(x\), then we have to add a \textbf{new}
\(x.y\), and make \(A\) false at \(x.y\).

\begin{itemize}
\tightlist
\item
  It is very important that \(x.y\) be new.
\item
  We know that \(A\) is false at some accessible world, but we don't
  know which one.
\item
  For any given world, \(A\) might be true there, as long as it is false
  somewhere.
\item
  That's why we use a new number.
\item
  Remember that it might be that multiple strings refer to the same
  world.
\end{itemize}
\end{frame}

\begin{frame}{Rules for \(\Diamond\)}
\protect\hypertarget{rules-for-diamond}{}
If \(\Diamond A\) is true at \(x\), then we have to add a \textbf{new}
\(x.y\), and make \(A\) true at \(x.y\).

\begin{itemize}
\tightlist
\item
  It is very important that \(x.y\) be new.
\item
  We know that \(A\) is true at some accessible world, but we don't know
  which one.
\item
  For any given world, \(A\) might be false there, as long as it is
  false somewhere.
\item
  That's why we use a new number.
\item
  Remember that it might be that multiple strings refer to the same
  world.
\end{itemize}
\end{frame}

\begin{frame}{Rules for \(\Diamond\) (cont)}
\protect\hypertarget{rules-for-diamond-cont}{}
If \(\Diamond A\) is false at \(x\), then for any \(x.y\) that is
already on the tree, we can infer that \(A\) is false at \(x.y\).

\begin{itemize}
\tightlist
\item
  Note: If there is no \(x.y\) on the tree, we can't assume that there
  is one.
\item
  \(\Diamond A\) can be false at a world because that world can't access
  any other world.
\end{itemize}
\end{frame}

\begin{frame}{\((\Box A \wedge \Box B) \rightarrow \Box (A \wedge B)\)}
\protect\hypertarget{box-a-wedge-box-b-rightarrow-box-a-wedge-b}{}
\begin{oltableau}
    [\pFmla{\False}{(\Box\formula{A} \land \Box\formula{B}) \lif
        \Box (\formula{A} \land \formula{B})}{1},
      just =\TAss
      [\pFmla{\True}{\Box\formula{A} \land \Box\formula{B}}{1},
        just = {\TRule{\False}{\lif}[1]}
        [\pFmla{\False}{\Box(\formula{A} \land \formula{B})}{1},
          just = {\TRule{\False}{\lif}[1]}
          [\pFmla{\True}{\Box\formula{A}}{1},
            just = {\TRule{\True}{\land}[2]}
            [\pFmla{\True}{\Box\formula{B}}{1},
              just = {\TRule{\True}{\land}[2]}
              [\pFmla{\False}{\formula{A} \land \formula{B}}{1.1},
                just = {\TRule{\False}{\Box}[3]}
                [\pFmla{\False}{\formula{A}}{1.1}, 
                  just = {\TRule{\False}{\land}[6]}
                  [\pFmla{\True}{\formula{A}}{1.1},
                    just= {\TRule{\True}{\Box}[4]}, close]]
                [\pFmla{\False}{\formula{B}}{1.1}, 
                  just = {\TRule{\False}{\land}[6]}
                  [\pFmla{\True}{\formula{B}}{1.1},
                    just= {\TRule{\True}{\Box}[5]}, close]]
              ]
            ]
          ]
        ]
      ]
    ]
  \end{oltableau}
\end{frame}

\begin{frame}{\(\Diamond (A \vee B) \rightarrow (\Diamond A \vee \Diamond B)\)}
\protect\hypertarget{diamond-a-vee-b-rightarrow-diamond-a-vee-diamond-b}{}
\begin{oltableau}
    [\pFmla{\False}{\Diamond(\formula{A} \lor \formula{B}) \lif
        (\Diamond \formula{A} \lor \Diamond \formula{B})}{1},
      just =\TAss
      [\pFmla{\True}{\Diamond(\formula{A} \lor \formula{B})}{1},
        just = {\TRule{\False}{\lif}[1]}
        [\pFmla{\False}{\Diamond\formula{A} \lor \Diamond\formula{B}}{1},
          just = {\TRule{\False}{\lif}[1]}
          [\pFmla{\False}{\Diamond\formula{A}}{1},
            just = {\TRule{\False}{\lor}[3]}
            [\pFmla{\False}{\Diamond\formula{B}}{1},
              just = {\TRule{\False}{\lor}[3]}
              [\pFmla{\True}{\formula{A} \lor \formula{B}}{1.1},
                just = {\TRule{\True}{\Diamond}[2]},
                [\pFmla{\True}{\formula{A}}{1.1}, 
                  just = {\TRule{\True}{\lor}[6]}
                  [\pFmla{\False}{\formula{A}}{1.1},
                    just= {\TRule{\False}{\Diamond}[4]}, close]]
                [\pFmla{\True}{\formula{B}}{1.1}, 
                  just = {\TRule{\True}{\lor}[6]}
                  [\pFmla{\False}{\formula{B}}{1.1},
                    just= {\TRule{\False}{\Diamond}[5]}, close]]
              ]
            ]
          ]
        ]
      ]
    ]
  \end{oltableau}
\end{frame}

\begin{frame}{For Next Time}
\protect\hypertarget{for-next-time}{}
We'll look at how to extend this approach to other logics.
\end{frame}

\end{document}
