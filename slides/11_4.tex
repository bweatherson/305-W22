% Options for packages loaded elsewhere
\PassOptionsToPackage{unicode}{hyperref}
\PassOptionsToPackage{hyphens}{url}
%
\documentclass[
  ignorenonframetext,
]{beamer}
\usepackage{pgfpages}
\setbeamertemplate{caption}[numbered]
\setbeamertemplate{caption label separator}{: }
\setbeamercolor{caption name}{fg=normal text.fg}
\beamertemplatenavigationsymbolsempty
% Prevent slide breaks in the middle of a paragraph
\widowpenalties 1 10000
\raggedbottom
\setbeamertemplate{part page}{
  \centering
  \begin{beamercolorbox}[sep=16pt,center]{part title}
    \usebeamerfont{part title}\insertpart\par
  \end{beamercolorbox}
}
\setbeamertemplate{section page}{
  \centering
  \begin{beamercolorbox}[sep=12pt,center]{part title}
    \usebeamerfont{section title}\insertsection\par
  \end{beamercolorbox}
}
\setbeamertemplate{subsection page}{
  \centering
  \begin{beamercolorbox}[sep=8pt,center]{part title}
    \usebeamerfont{subsection title}\insertsubsection\par
  \end{beamercolorbox}
}
\AtBeginPart{
  \frame{\partpage}
}
\AtBeginSection{
  \ifbibliography
  \else
    \frame{\sectionpage}
  \fi
}
\AtBeginSubsection{
  \frame{\subsectionpage}
}
\usepackage{amsmath,amssymb}
\usepackage{lmodern}
\usepackage{ifxetex,ifluatex}
\ifnum 0\ifxetex 1\fi\ifluatex 1\fi=0 % if pdftex
  \usepackage[T1]{fontenc}
  \usepackage[utf8]{inputenc}
  \usepackage{textcomp} % provide euro and other symbols
\else % if luatex or xetex
  \usepackage{unicode-math}
  \defaultfontfeatures{Scale=MatchLowercase}
  \defaultfontfeatures[\rmfamily]{Ligatures=TeX,Scale=1}
  \setmainfont[BoldFont = SF Pro Rounded Semibold]{SF Pro Rounded}
  \setmathfont[]{STIX Two Math}
\fi
\usefonttheme{serif} % use mainfont rather than sansfont for slide text
% Use upquote if available, for straight quotes in verbatim environments
\IfFileExists{upquote.sty}{\usepackage{upquote}}{}
\IfFileExists{microtype.sty}{% use microtype if available
  \usepackage[]{microtype}
  \UseMicrotypeSet[protrusion]{basicmath} % disable protrusion for tt fonts
}{}
\makeatletter
\@ifundefined{KOMAClassName}{% if non-KOMA class
  \IfFileExists{parskip.sty}{%
    \usepackage{parskip}
  }{% else
    \setlength{\parindent}{0pt}
    \setlength{\parskip}{6pt plus 2pt minus 1pt}}
}{% if KOMA class
  \KOMAoptions{parskip=half}}
\makeatother
\usepackage{xcolor}
\IfFileExists{xurl.sty}{\usepackage{xurl}}{} % add URL line breaks if available
\IfFileExists{bookmark.sty}{\usepackage{bookmark}}{\usepackage{hyperref}}
\hypersetup{
  pdftitle={305 Lecture 11.4 - Truth in a Model},
  pdfauthor={Brian Weatherson},
  hidelinks,
  pdfcreator={LaTeX via pandoc}}
\urlstyle{same} % disable monospaced font for URLs
\newif\ifbibliography
\setlength{\emergencystretch}{3em} % prevent overfull lines
\providecommand{\tightlist}{%
  \setlength{\itemsep}{0pt}\setlength{\parskip}{0pt}}
\setcounter{secnumdepth}{-\maxdimen} % remove section numbering
\let\Tiny=\tiny

 \setbeamertemplate{navigation symbols}{} 

% \usetheme{Madrid}
 \usetheme[numbering=none, progressbar=foot]{metropolis}
 \usecolortheme{wolverine}
 \usepackage{color}
 \usepackage{MnSymbol}
% \usepackage{movie15}

\usepackage{amssymb}% http://ctan.org/pkg/amssymb
\usepackage{pifont}% http://ctan.org/pkg/pifont
\newcommand{\cmark}{\ding{51}}%
\newcommand{\xmark}{\ding{55}}%

\DeclareSymbolFont{symbolsC}{U}{txsyc}{m}{n}
\DeclareMathSymbol{\boxright}{\mathrel}{symbolsC}{128}
\DeclareMathAlphabet{\mathpzc}{OT1}{pzc}{m}{it}

 \usepackage{tikz-qtree}
% \usepackage{markdown}
%\RequirePackage{bussproofs}
\RequirePackage[tableaux]{prooftrees}
\usetikzlibrary{arrows.meta}
 \forestset{line numbering, close with = x}
% Allow for easy commas inside trees
\renewcommand{\,}{\text{, }}


\usepackage{tabulary}

\usepackage{open-logic-config}

\setlength{\parskip}{1ex plus 0.5ex minus 0.2ex}

\AtBeginSection[]
{
\begin{frame}
	\Huge{\color{darkblue} \insertsection}
\end{frame}
}

\renewenvironment*{quote}	
	{\list{}{\rightmargin   \leftmargin} \item } 	
	{\endlist }

\definecolor{darkgreen}{rgb}{0,0.7,0}
\definecolor{darkblue}{rgb}{0,0,0.8}

\newcommand{\starttab}{\begin{center}
\vspace{6pt}
\begin{tabular}}

\newcommand{\stoptab}{\end{tabular}
\vspace{6pt}
\end{center}
\noindent}


\newcommand{\sif}{\rightarrow}
\newcommand{\siff}{\leftrightarrow}
\newcommand{\EF}{\end{frame}}


\newcommand{\TreeStart}[1]{
%\end{frame}
\begin{frame}
\begin{center}
\begin{tikzpicture}[scale=#1]
\tikzset{every tree node/.style={align=center,anchor=north}}
%\Tree
}

\newcommand{\TreeEnd}{
\end{tikzpicture}
%\end{center}
}

\newcommand{\DisplayArg}[2]{
\begin{enumerate}
{#1}
\end{enumerate}
\vspace{-6pt}
\hrulefill

%\hspace{14pt} #2
%{\addtolength{\leftskip}{14pt} #2}
\begin{quote}
{\normalfont #2}
\end{quote}
\vspace{12pt}
}

\newenvironment{ProofTree}[1][1]{
\begin{center}
\begin{tikzpicture}[scale=#1]
\tikzset{every tree node/.style={align=center,anchor=south}}
}
{
\end{tikzpicture}
\end{center}
}

\newcommand{\TreeFrame}[2]{
\begin{columns}[c]
\column{0.5\textwidth}
\begin{center}
\begin{prooftree}{}
#1
\end{prooftree}
\end{center}
\column{0.45\textwidth}
%\begin{markdown}
#2
%\end{markdown}
\end{columns}
}

\newcommand{\ScaledTreeFrame}[3]{
\begin{columns}[c]
\column{0.5\textwidth}
\begin{center}
\scalebox{#1}{
\begin{prooftree}{}
#2
\end{prooftree}
}
\end{center}
\column{0.45\textwidth}
%\begin{markdown}
#3
%\end{markdown}
\end{columns}
}

\usepackage[bb=boondox]{mathalfa}
\DeclareMathAlphabet{\mathbx}{U}{BOONDOX-ds}{m}{n}
\SetMathAlphabet{\mathbx}{bold}{U}{BOONDOX-ds}{b}{n}
\DeclareMathAlphabet{\mathbbx} {U}{BOONDOX-ds}{b}{n}


\newenvironment{oltableau}{\center\tableau{}} %wff format={anchor = base west}}}
       {\endtableau\endcenter}
       
\newcommand{\formula}[1]{$#1$}

\usepackage{tabulary}
\usepackage{booktabs}

\def\begincols{\begin{columns}}
\def\begincol{\begin{column}}
\def\endcol{\end{column}}
\def\endcols{\end{columns}}

\usepackage[italic]{mathastext}
\usepackage{nicefrac}

\definecolor{mygreen}{RGB}{0, 100, 0}
\definecolor{mypink2}{RGB}{219, 48, 122}
\definecolor{dodgerblue}{RGB}{30,144,255}

%\def\True{\textcolor{dodgerblue}{\text{T}}}
%\def\False{\textcolor{red}{\text{F}}}

\def\True{\mathbb{T}}
\def\False{\mathbb{F}}

% This is because arguments didn't have enough space after them
\usepackage{etoolbox}
\AfterEndEnvironment{description}{\vspace{9pt}}
\AfterEndEnvironment{oltableau}{\vspace{9pt}}
\BeforeBeginEnvironment{oltableau}{\vspace{9pt}}
\AfterEndEnvironment{center}{\vspace{12pt}}
\BeforeBeginEnvironment{tabular}{\vspace{9pt}}

\setlength\heavyrulewidth{0pt}
\setlength\lightrulewidth{0pt}

%\def\toprule{}
%\def\bottomrule{}
%\def\midrule{}

\setbeamertemplate{caption}{\raggedright\insertcaption}

\ifluatex
  \usepackage{selnolig}  % disable illegal ligatures
\fi

\title{305 Lecture 11.4 - Truth in a Model}
\author{Brian Weatherson}
\date{}

\begin{document}
\frame{\titlepage}

\begin{frame}{Plan}
\protect\hypertarget{plan}{}
\begin{itemize}
\tightlist
\item
  To extend our discussion of truth at a world, to discussion of truth
  in a whole model.
\end{itemize}
\end{frame}

\begin{frame}{Associated Reading}
\protect\hypertarget{associated-reading}{}
\begin{itemize}
\tightlist
\item
  Boxes and Diamonds, section 3.5.
\end{itemize}
\end{frame}

\begin{frame}{Models}
\protect\hypertarget{models}{}
Models have three parts:

\begin{enumerate}
\tightlist
\item
  A set of worlds, typically called \(W\). \pause
\item
  A binary accessibility relation on those worlds, typically called
  \(R\). \pause
\item
  A valuation function on those worlds, typically called \(V\). \pause
\end{enumerate}

We'll write the models as \(\langle W, R, V\rangle\).
\end{frame}

\begin{frame}{Valuations}
\protect\hypertarget{valuations}{}
\(V\) is a function from atomic sentence letters to subsets of \(W\).

\begin{itemize}
\tightlist
\item
  It tells you when the atomic sentences are true.
\item
  When an atomic sentence is not true, it is false.
\end{itemize}
\end{frame}

\begin{frame}{Truth at a Point}
\protect\hypertarget{truth-at-a-point}{}
The general theory of truth is built up in stages from the basic theory.
Assume we have a model \(\langle W, R, V\rangle\), and a point
\(w \in W\), and are asking whether an arbitrary sentence is true at
\(w\) in \(\langle W, R, V\rangle\).

\begin{itemize}
\tightlist
\item
  \(p\) is true at \(w\) iff \(w \in V(p)\).\pause
\item
  \(\neg A\) is true at \(w\) iff \(A\) is not true at \(w\).
\item
  \(A \wedge B\) is true at \(w\) iff \(A\) is true and \(w\) and \(B\)
  is true at \(w\).
\item
  \(A \vee B\) is true at \(w\) iff \(A\) is true and \(w\) or \(B\) is
  true at \(w\).
\item
  \(A \rightarrow B\) is true at \(w\) iff \(A\) is false at \(w\) or
  \(B\) is true at \(w\).\pause
\end{itemize}

This just leaves the modal formulae. I'll set out the rules, then do
some worked examples.
\end{frame}

\begin{frame}{Necessary Truth at a Point}
\protect\hypertarget{necessary-truth-at-a-point}{}
First we'll do \(\Box A\).

\begin{itemize}
\tightlist
\item
  I'll read this as `Box \(A\)'.\pause
\item
  Intuitively, it means \textbf{It must be that A}, where \textbf{must}
  could be a metaphysical necessity, or an epistemic necessity, or a
  moral necessity, or anything else.\pause
\item
  And it is true at \(w\) just in case \(A\) is true at every world
  \(y\) such that \(wRy\).
\item
  Necessary truth is truth at all accessible worlds.
\end{itemize}
\end{frame}

\begin{frame}{Possible Truth at a Point}
\protect\hypertarget{possible-truth-at-a-point}{}
Now we'll do \(\Diamond A\).

\begin{itemize}
\tightlist
\item
  I'll read this as `Diamond \(A\)'.\pause
\item
  Intuitively, it means \textbf{It might be that A}, where
  \textbf{might} could be a metaphysical necessity, or an epistemic
  necessity, or a moral necessity, or anything else.\pause
\item
  And it is true at \(w\) just in case \(A\) is true at some world \(y\)
  such that \(wRy\).
\item
  Possible truth is truth at some accessible world.
\end{itemize}
\end{frame}

\begin{frame}{Iterated Modalities}
\protect\hypertarget{iterated-modalities}{}
We can run these rules in sequence.\pause

\begin{itemize}
\tightlist
\item
  What does it take for \(\Box \Box A\) to be true at \(w\)? \pause
\item
  It is for \(\Box A\) to be true at every \(y\) such that
  \(wRy\).\pause
\item
  And that means that \(A\) has to be true at every world \(z\) such
  that \(yRz\) (for any \(y\) such that \(wRy)\).
\end{itemize}
\end{frame}

\begin{frame}{Access}
\protect\hypertarget{access}{}
We can think, a little picturesquely, as the accessibility relation
being a `step' between worlds.

\begin{itemize}
\tightlist
\item
  If \(wRy\), then you can `step' from \(w\) to \(y\).\pause
\item
  \(\Box A\) means that anywhere you can step to from \(w\) is a world
  where \(A\) is true.\pause
\item
  And \(\Box \Box A\) means that anywhere you can get to in two steps
  from \(w\) is a world where \(A\) is true.
\end{itemize}
\end{frame}

\begin{frame}{Iterated Modalities}
\protect\hypertarget{iterated-modalities-1}{}
We can run the rules in sequence.\pause

\begin{itemize}
\tightlist
\item
  What does it take for \(\Diamond \Diamond A\) to be true at \(w\)?
  \pause
\item
  It is for \(\Diamond A\) to be true at some \(y\) such that
  \(wRy\).\pause
\item
  And that means that \(A\) has to be true at some world \(z\) such that
  \(yRz\) (for some \(y\) such that \(wRy)\).\pause
\item
  In the picturesque terms, you can get from \(w\) to an \(A\)-world in
  two steps.
\end{itemize}
\end{frame}

\begin{frame}{Mixed Modalities}
\protect\hypertarget{mixed-modalities}{}
What does it mean for \(\Diamond \Box A\) to be true at \(w\)?\pause

\begin{itemize}
\tightlist
\item
  There is some accessible world where \(\Box A\) is true.\pause
\item
  That is, there is some accessible world such that everywhere you can
  go from there, \(A\) is true.
\end{itemize}
\end{frame}

\begin{frame}{Mixed Modalities}
\protect\hypertarget{mixed-modalities-1}{}
What does it mean for \(\Box \Diamond A\) to be true at \(w\)?\pause

\begin{itemize}
\tightlist
\item
  At all accessible worlds, \(\Diamond A\) is true.\pause
\item
  That is, wherever you go, you can get to there is some accessible
  world such that everywhere you can go from there, \(A\) is true.
\end{itemize}
\end{frame}

\begin{frame}{Longer Sentences}
\protect\hypertarget{longer-sentences}{}
What does it mean for \(\Box(p \vee (q \rightarrow \Diamond r))\) to be
true at \(w\)? \pause

\begin{itemize}
\tightlist
\item
  It's for \(p \vee (q \rightarrow \Diamond r)\) to be true everywhere
  you can get to (in one step) from \(w\).
\item
  That is, at every one of those worlds, either \(p\) is true or
  \(q \rightarrow \Diamond r\) is true.\pause
\item
  That is, at every one of those worlds, either \(p\) is true, or \(q\)
  is false, or \(\Diamond r\) is true. \pause
\item
  That is, at every one of those worlds, either \(p\) is true, or \(q\)
  is false, or there is some world you can get to where \(r\) is true.
\end{itemize}
\end{frame}

\begin{frame}{Box and connectives}
\protect\hypertarget{box-and-connectives}{}
The general rule is just to apply the rules for sentences inside the
brackets at each world in \(W\), and then apply the rule for \(\Box\) or
\(\Diamond\). But there are three special cases worth thinking about.

\begin{itemize}
\tightlist
\item
  \(\Box(A \wedge B)\) means that all accessible worlds are \(A\) and
  \(B\) worlds. \pause
\item
  \(\Box(A \vee B)\) means that all accessible worlds make at least one
  of \(A\) and \(B\) true.\pause
\item
  \(\Box(A \rightarrow B)\) means that all accessible \(A\)-worlds are
  \(B\)-worlds.
\end{itemize}

We'll use that last one a lot.
\end{frame}

\begin{frame}{For Next Time}
\protect\hypertarget{for-next-time}{}
We'll discuss of examples of truth (and non-truth) in models to explain
this material.
\end{frame}

\end{document}
