% Options for packages loaded elsewhere
\PassOptionsToPackage{unicode}{hyperref}
\PassOptionsToPackage{hyphens}{url}
%
\documentclass[
  ignorenonframetext,
]{beamer}
\usepackage{pgfpages}
\setbeamertemplate{caption}[numbered]
\setbeamertemplate{caption label separator}{: }
\setbeamercolor{caption name}{fg=normal text.fg}
\beamertemplatenavigationsymbolsempty
% Prevent slide breaks in the middle of a paragraph
\widowpenalties 1 10000
\raggedbottom
\setbeamertemplate{part page}{
  \centering
  \begin{beamercolorbox}[sep=16pt,center]{part title}
    \usebeamerfont{part title}\insertpart\par
  \end{beamercolorbox}
}
\setbeamertemplate{section page}{
  \centering
  \begin{beamercolorbox}[sep=12pt,center]{part title}
    \usebeamerfont{section title}\insertsection\par
  \end{beamercolorbox}
}
\setbeamertemplate{subsection page}{
  \centering
  \begin{beamercolorbox}[sep=8pt,center]{part title}
    \usebeamerfont{subsection title}\insertsubsection\par
  \end{beamercolorbox}
}
\AtBeginPart{
  \frame{\partpage}
}
\AtBeginSection{
  \ifbibliography
  \else
    \frame{\sectionpage}
  \fi
}
\AtBeginSubsection{
  \frame{\subsectionpage}
}
\usepackage{amsmath,amssymb}
\usepackage{lmodern}
\usepackage{ifxetex,ifluatex}
\ifnum 0\ifxetex 1\fi\ifluatex 1\fi=0 % if pdftex
  \usepackage[T1]{fontenc}
  \usepackage[utf8]{inputenc}
  \usepackage{textcomp} % provide euro and other symbols
\else % if luatex or xetex
  \usepackage{unicode-math}
  \defaultfontfeatures{Scale=MatchLowercase}
  \defaultfontfeatures[\rmfamily]{Ligatures=TeX,Scale=1}
  \setmainfont[BoldFont = SF Pro Rounded Semibold]{SF Pro Rounded}
  \setmathfont[]{STIX Two Math}
\fi
\usefonttheme{serif} % use mainfont rather than sansfont for slide text
% Use upquote if available, for straight quotes in verbatim environments
\IfFileExists{upquote.sty}{\usepackage{upquote}}{}
\IfFileExists{microtype.sty}{% use microtype if available
  \usepackage[]{microtype}
  \UseMicrotypeSet[protrusion]{basicmath} % disable protrusion for tt fonts
}{}
\makeatletter
\@ifundefined{KOMAClassName}{% if non-KOMA class
  \IfFileExists{parskip.sty}{%
    \usepackage{parskip}
  }{% else
    \setlength{\parindent}{0pt}
    \setlength{\parskip}{6pt plus 2pt minus 1pt}}
}{% if KOMA class
  \KOMAoptions{parskip=half}}
\makeatother
\usepackage{xcolor}
\IfFileExists{xurl.sty}{\usepackage{xurl}}{} % add URL line breaks if available
\IfFileExists{bookmark.sty}{\usepackage{bookmark}}{\usepackage{hyperref}}
\hypersetup{
  pdftitle={305 Lecture 3.6 - Rules for Truth Trees},
  pdfauthor={Brian Weatherson},
  hidelinks,
  pdfcreator={LaTeX via pandoc}}
\urlstyle{same} % disable monospaced font for URLs
\newif\ifbibliography
\setlength{\emergencystretch}{3em} % prevent overfull lines
\providecommand{\tightlist}{%
  \setlength{\itemsep}{0pt}\setlength{\parskip}{0pt}}
\setcounter{secnumdepth}{-\maxdimen} % remove section numbering
\let\Tiny=\tiny

 \setbeamertemplate{navigation symbols}{} 

% \usetheme{Madrid}
 \usetheme[numbering=none, progressbar=foot]{metropolis}
 \usecolortheme{wolverine}
 \usepackage{color}
 \usepackage{MnSymbol}
% \usepackage{movie15}

\usepackage{amssymb}% http://ctan.org/pkg/amssymb
\usepackage{pifont}% http://ctan.org/pkg/pifont
\newcommand{\cmark}{\ding{51}}%
\newcommand{\xmark}{\ding{55}}%

\DeclareSymbolFont{symbolsC}{U}{txsyc}{m}{n}
\DeclareMathSymbol{\boxright}{\mathrel}{symbolsC}{128}
\DeclareMathAlphabet{\mathpzc}{OT1}{pzc}{m}{it}

 \usepackage{tikz-qtree}
% \usepackage{markdown}
%\RequirePackage{bussproofs}
\RequirePackage[tableaux]{prooftrees}
\usetikzlibrary{arrows.meta}
 \forestset{line numbering, close with = x}
% Allow for easy commas inside trees
\renewcommand{\,}{\text{, }}


\usepackage{tabulary}

\usepackage{open-logic-config}

\setlength{\parskip}{1ex plus 0.5ex minus 0.2ex}

\AtBeginSection[]
{
\begin{frame}
	\Huge{\color{darkblue} \insertsection}
\end{frame}
}

\renewenvironment*{quote}	
	{\list{}{\rightmargin   \leftmargin} \item } 	
	{\endlist }

\definecolor{darkgreen}{rgb}{0,0.7,0}
\definecolor{darkblue}{rgb}{0,0,0.8}

\newcommand{\starttab}{\begin{center}
\vspace{6pt}
\begin{tabular}}

\newcommand{\stoptab}{\end{tabular}
\vspace{6pt}
\end{center}
\noindent}


\newcommand{\sif}{\rightarrow}
\newcommand{\siff}{\leftrightarrow}
\newcommand{\EF}{\end{frame}}


\newcommand{\TreeStart}[1]{
%\end{frame}
\begin{frame}
\begin{center}
\begin{tikzpicture}[scale=#1]
\tikzset{every tree node/.style={align=center,anchor=north}}
%\Tree
}

\newcommand{\TreeEnd}{
\end{tikzpicture}
%\end{center}
}

\newcommand{\DisplayArg}[2]{
\begin{enumerate}
{#1}
\end{enumerate}
\vspace{-6pt}
\hrulefill

%\hspace{14pt} #2
%{\addtolength{\leftskip}{14pt} #2}
\begin{quote}
{\normalfont #2}
\end{quote}
\vspace{12pt}
}

\newenvironment{ProofTree}[1][1]{
\begin{center}
\begin{tikzpicture}[scale=#1]
\tikzset{every tree node/.style={align=center,anchor=south}}
}
{
\end{tikzpicture}
\end{center}
}

\newcommand{\TreeFrame}[2]{
\begin{columns}[c]
\column{0.5\textwidth}
\begin{center}
\begin{prooftree}{}
#1
\end{prooftree}
\end{center}
\column{0.45\textwidth}
%\begin{markdown}
#2
%\end{markdown}
\end{columns}
}

\newcommand{\ScaledTreeFrame}[3]{
\begin{columns}[c]
\column{0.5\textwidth}
\begin{center}
\scalebox{#1}{
\begin{prooftree}{}
#2
\end{prooftree}
}
\end{center}
\column{0.45\textwidth}
%\begin{markdown}
#3
%\end{markdown}
\end{columns}
}

\usepackage[bb=boondox]{mathalfa}
\DeclareMathAlphabet{\mathbx}{U}{BOONDOX-ds}{m}{n}
\SetMathAlphabet{\mathbx}{bold}{U}{BOONDOX-ds}{b}{n}
\DeclareMathAlphabet{\mathbbx} {U}{BOONDOX-ds}{b}{n}


\newenvironment{oltableau}{\center\tableau{}} %wff format={anchor = base west}}}
       {\endtableau\endcenter}
       
\newcommand{\formula}[1]{$#1$}

\usepackage{tabulary}
\usepackage{booktabs}

\def\begincols{\begin{columns}}
\def\begincol{\begin{column}}
\def\endcol{\end{column}}
\def\endcols{\end{columns}}

\usepackage[italic]{mathastext}
\usepackage{nicefrac}

\definecolor{mygreen}{RGB}{0, 100, 0}
\definecolor{mypink2}{RGB}{219, 48, 122}
\definecolor{dodgerblue}{RGB}{30,144,255}

%\def\True{\textcolor{dodgerblue}{\text{T}}}
%\def\False{\textcolor{red}{\text{F}}}

\def\True{\mathbb{T}}
\def\False{\mathbb{F}}

% This is because arguments didn't have enough space after them
\usepackage{etoolbox}
\AfterEndEnvironment{description}{\vspace{9pt}}
\AfterEndEnvironment{oltableau}{\vspace{9pt}}
\BeforeBeginEnvironment{oltableau}{\vspace{9pt}}
\AfterEndEnvironment{center}{\vspace{12pt}}
\BeforeBeginEnvironment{tabular}{\vspace{9pt}}

\setlength\heavyrulewidth{0pt}
\setlength\lightrulewidth{0pt}

%\def\toprule{}
%\def\bottomrule{}
%\def\midrule{}

\setbeamertemplate{caption}{\raggedright\insertcaption}

\ifluatex
  \usepackage{selnolig}  % disable illegal ligatures
\fi

\title{305 Lecture 3.6 - Rules for Truth Trees}
\author{Brian Weatherson}
\date{}

\begin{document}
\frame{\titlepage}

\begin{frame}{Plan}
\protect\hypertarget{plan}{}
This lecture introduces the rules we use for building up truth trees.
\end{frame}

\begin{frame}{Associated Reading}
\protect\hypertarget{associated-reading}{}
Boxes and Diamonds, sections 2.2-2.3.
\end{frame}

\begin{frame}{What Rules Do}
\protect\hypertarget{what-rules-do}{}
The rules tell you what new lines to write down given the lines you've
already got.

\begin{itemize}
\tightlist
\item
  To some extent they simply have to be memorised.
\item
  But hopefully they are all (except for the rules about
  \(\rightarrow\)) fairly intuitive.
\end{itemize}
\end{frame}

\begin{frame}{Rules for \(\neg\)}
\protect\hypertarget{rules-for-neg}{}
\begin{columns}[T]
\begin{column}{0.48\textwidth}
\begin{oltableau}
[\sFmla{\True}{\neg A}, 
    [\sFmla{\False}{A}, just = {\TRule{\True}{\neg}[1]}]
]
\end{oltableau}
\end{column}

\begin{column}{0.48\textwidth}
\begin{oltableau}
[\sFmla{\False}{\neg A}, 
    [\sFmla{\True}{A}, just = {\TRule{\False}{\neg}[1]}]
]
\end{oltableau}
\end{column}
\end{columns}

\pause

\begin{itemize}
\tightlist
\item
  Note that the line numbers are just for illustration, and are
  arbitrary in two senses.
\item
  First, you apply the rule wherever a sentence like \(\neg A\) appears,
  not just at line 1.
\item
  Second, you don't need to apply the rules immediately, so the
  successor line could come later than 2.
\end{itemize}
\end{frame}

\begin{frame}{Rule for true \(\wedge\) sentence}
\protect\hypertarget{rule-for-true-wedge-sentence}{}
\begin{oltableau}
[\sFmla{\True}{A \wedge B}, 
    [\sFmla{\True}{A}, just = {\TRule{\True}{\wedge}[1]}
      [\sFmla{\True}{B}, just = {\TRule{\True}{\wedge}[1]}]
    ]
]
\end{oltableau}

When you have a true \(\wedge\) sentence, you can write down that the
sentences either side of it are true.
\end{frame}

\begin{frame}{Rule for true \(\vee\) sentence}
\protect\hypertarget{rule-for-true-vee-sentence}{}
\begin{oltableau}
[\sFmla{\True}{A \vee B}, 
    [\sFmla{\True}{A}, just = {\TRule{\True}{\vee}[1]}]
    [\sFmla{\True}{B}, just = {\TRule{\True}{\vee}[1]}]
]
\end{oltableau}

\begin{itemize}
\tightlist
\item
  When you have a true \(\vee\) sentence, you create two
  \textbf{branches}.
\item
  The way to read the tree is that at least one of the branches must be
  all true. \pause
\item
  The `trunk' above the branching (in this case just line 1), is part of
  both branches.
\item
  Branches are inclusive; you are saying that at least one branch is
  true, not that precisely one is.
\end{itemize}
\end{frame}

\begin{frame}{Rule for false \(\wedge\) sentence}
\protect\hypertarget{rule-for-false-wedge-sentence}{}
\begin{oltableau}
[\sFmla{\False}{A \wedge B}, 
    [\sFmla{\False}{A}, just = {\TRule{\False}{\wedge}[1]}]
    [\sFmla{\False}{B}, just = {\TRule{\False}{\wedge}[1]}]
]
\end{oltableau}

\begin{itemize}
\tightlist
\item
  If an \(\wedge\) sentence is false, then we know that one or other (or
  both) of the sides are false.
\item
  So we create two branches, one where each side is false.
\end{itemize}
\end{frame}

\begin{frame}{Rule for false \(\vee\) sentence}
\protect\hypertarget{rule-for-false-vee-sentence}{}
\begin{oltableau}
[\sFmla{\False}{A \vee B}, 
    [\sFmla{\False}{A}, just = {\TRule{\False}{\vee}[1]}
      [\sFmla{\False}{B}, just = {\TRule{\False}{\vee}[1]}]
    ]
]
\end{oltableau}

When you have a false \(\vee\) sentence, you know that each side is
false, so you write down that they are both false.
\end{frame}

\begin{frame}{Justifying the rule for false \(\vee\) sentences}
\protect\hypertarget{justifying-the-rule-for-false-vee-sentences}{}
Recall the truth table for \(\vee\)

\begin{center}

\begin{tabular}{@{ }c@{ }@{ }c | c@{ }@{ }c@{ }@{ }c@{ }@{ }c@{ }@{ }c}
A & B &  & A & $\vee$ & B & \\
\hline 
$\True$ & $\True$ &  & $\True$ & \textcolor{red}{$\True$} & $\True$ & \\
$\True$ & $\False$ &  & $\True$ & \textcolor{red}{$\True$} & $\False$ & \\
$\False$ & $\True$ &  & $\False$ & \textcolor{red}{$\True$} & $\True$ & \\
$\False$ & $\False$ &  & $\False$ & \textcolor{red}{$\False$} & $\False$ & \\
\end{tabular}

\end{center}

\begin{itemize}
\tightlist
\item
  The only line where the whole sentence is \textcolor{red}{$\False$} is
  line 4.
\item
  So if a \(\vee\) sentence is \textcolor{red}{$\False$}, we know that
  we're on line 4.
\item
  And on line 4, both \(A\) and \(B\) are false.
\end{itemize}
\end{frame}

\begin{frame}{Rule for false \(\rightarrow\) sentence}
\protect\hypertarget{rule-for-false-rightarrow-sentence}{}
\begin{oltableau}
[\sFmla{\False}{A \rightarrow B}, 
    [\sFmla{\True}{A}, just = {\TRule{\False}{\rightarrow}[1]}
      [\sFmla{\False}{B}, just = {\TRule{\False}{\rightarrow}[1]}]
    ]
]
\end{oltableau}

When you have a false \(\rightarrow\) sentence, you know that the left
side is true and the right side is false, so you write those things
down.
\end{frame}

\begin{frame}{Justifying the rule for false \(\rightarrow\) sentences}
\protect\hypertarget{justifying-the-rule-for-false-rightarrow-sentences}{}
Recall the truth table for \(\rightarrow\)

\begin{center}

\begin{tabular}{@{ }c@{ }@{ }c | c@{ }@{ }c@{ }@{ }c@{ }@{ }c@{ }@{ }c}
A & B &  & A & $\rightarrow$ & B & \\
\hline 
$\True$ & $\True$ &  & $\True$ & \textcolor{red}{$\True$} & $\True$ & \\
$\True$ & $\False$ &  & $\True$ & \textcolor{red}{$\False$} & $\False$ & \\
$\False$ & $\True$ &  & $\False$ & \textcolor{red}{$\True$} & $\True$ & \\
$\False$ & $\False$ &  & $\False$ & \textcolor{red}{$\True$} & $\False$ & \\
\end{tabular}

\end{center}

\begin{itemize}
\tightlist
\item
  The only line where the whole sentence is \textcolor{red}{$\False$} is
  line 2.
\item
  So if a \(\rightarrow\) sentence is \textcolor{red}{$\False$}, we know
  that we're on line 2.
\item
  And on line 4,\(A\) is true and \(B\) are false.
\end{itemize}
\end{frame}

\begin{frame}{Rule for true \(\rightarrow\) sentence}
\protect\hypertarget{rule-for-true-rightarrow-sentence}{}
\begin{oltableau}
[\sFmla{\True}{A \rightarrow B}, 
    [\sFmla{\False}{A}, just = {\TRule{\True}{\rightarrow}[1]}]
    [\sFmla{\True}{B}, just = {\TRule{\True}{\rightarrow}[1]}]
]
\end{oltableau}

\begin{itemize}
\tightlist
\item
  When you have a true \(\rightarrow\) sentence, you create two
  \textbf{branches}.
\item
  On the first, \(A\) is false. That covers lines 3 and 4 of the truth
  table.
\item
  On the second, \(B\) is true. That covers lines 1 and 3 of the truth
  table.
\item
  Between them, they cover lines 1, 3 and 4 of the truth table.
\item
  And those are the lines where \(A \rightarrow B\) is true.
\end{itemize}
\end{frame}

\begin{frame}{Next Week}
\protect\hypertarget{next-week}{}
\begin{itemize}
\tightlist
\item
  We will look at some examples of truth trees.
\item
  I find truth trees are much easier in practice than in theory, so if
  this was all a bit abstract, hopefully it will be more intelligible
  once we work through some examples.
\end{itemize}
\end{frame}

\end{document}
