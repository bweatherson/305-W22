% Options for packages loaded elsewhere
\PassOptionsToPackage{unicode}{hyperref}
\PassOptionsToPackage{hyphens}{url}
%
\documentclass[
  ignorenonframetext,
]{beamer}
\usepackage{pgfpages}
\setbeamertemplate{caption}[numbered]
\setbeamertemplate{caption label separator}{: }
\setbeamercolor{caption name}{fg=normal text.fg}
\beamertemplatenavigationsymbolsempty
% Prevent slide breaks in the middle of a paragraph
\widowpenalties 1 10000
\raggedbottom
\setbeamertemplate{part page}{
  \centering
  \begin{beamercolorbox}[sep=16pt,center]{part title}
    \usebeamerfont{part title}\insertpart\par
  \end{beamercolorbox}
}
\setbeamertemplate{section page}{
  \centering
  \begin{beamercolorbox}[sep=12pt,center]{part title}
    \usebeamerfont{section title}\insertsection\par
  \end{beamercolorbox}
}
\setbeamertemplate{subsection page}{
  \centering
  \begin{beamercolorbox}[sep=8pt,center]{part title}
    \usebeamerfont{subsection title}\insertsubsection\par
  \end{beamercolorbox}
}
\AtBeginPart{
  \frame{\partpage}
}
\AtBeginSection{
  \ifbibliography
  \else
    \frame{\sectionpage}
  \fi
}
\AtBeginSubsection{
  \frame{\subsectionpage}
}
\usepackage{lmodern}
\usepackage{amssymb,amsmath}
\usepackage{ifxetex,ifluatex}
\ifnum 0\ifxetex 1\fi\ifluatex 1\fi=0 % if pdftex
  \usepackage[T1]{fontenc}
  \usepackage[utf8]{inputenc}
  \usepackage{textcomp} % provide euro and other symbols
\else % if luatex or xetex
  \usepackage{unicode-math}
  \defaultfontfeatures{Scale=MatchLowercase}
  \defaultfontfeatures[\rmfamily]{Ligatures=TeX,Scale=1}
  \setmainfont[BoldFont = SF Pro Rounded Semibold]{SF Pro Rounded}
  \setmathfont[]{STIX Two Math}
\fi
\usefonttheme{serif} % use mainfont rather than sansfont for slide text
% Use upquote if available, for straight quotes in verbatim environments
\IfFileExists{upquote.sty}{\usepackage{upquote}}{}
\IfFileExists{microtype.sty}{% use microtype if available
  \usepackage[]{microtype}
  \UseMicrotypeSet[protrusion]{basicmath} % disable protrusion for tt fonts
}{}
\makeatletter
\@ifundefined{KOMAClassName}{% if non-KOMA class
  \IfFileExists{parskip.sty}{%
    \usepackage{parskip}
  }{% else
    \setlength{\parindent}{0pt}
    \setlength{\parskip}{6pt plus 2pt minus 1pt}}
}{% if KOMA class
  \KOMAoptions{parskip=half}}
\makeatother
\usepackage{xcolor}
\IfFileExists{xurl.sty}{\usepackage{xurl}}{} % add URL line breaks if available
\IfFileExists{bookmark.sty}{\usepackage{bookmark}}{\usepackage{hyperref}}
\hypersetup{
  pdftitle={(\textbackslash Diamond A \textbackslash vee \textbackslash Diamond B) \textbackslash rightarrow \textbackslash Diamond (A \textbackslash vee B) (in K)},
  pdfauthor={Build a Tableau},
  hidelinks,
  pdfcreator={LaTeX via pandoc}}
\urlstyle{same} % disable monospaced font for URLs
\newif\ifbibliography
\setlength{\emergencystretch}{3em} % prevent overfull lines
\providecommand{\tightlist}{%
  \setlength{\itemsep}{0pt}\setlength{\parskip}{0pt}}
\setcounter{secnumdepth}{-\maxdimen} % remove section numbering
\let\Tiny=\tiny

 \setbeamertemplate{navigation symbols}{} 

% \usetheme{Madrid}
 \usetheme[numbering=none, progressbar=foot]{metropolis}
 \usecolortheme{wolverine}
 \usepackage{color}
 \usepackage{MnSymbol}
% \usepackage{movie15}

\usepackage{amssymb}% http://ctan.org/pkg/amssymb
\usepackage{pifont}% http://ctan.org/pkg/pifont
\newcommand{\cmark}{\ding{51}}%
\newcommand{\xmark}{\ding{55}}%

\DeclareSymbolFont{symbolsC}{U}{txsyc}{m}{n}
\DeclareMathSymbol{\boxright}{\mathrel}{symbolsC}{128}
\DeclareMathAlphabet{\mathpzc}{OT1}{pzc}{m}{it}

 \usepackage{tikz-qtree}
% \usepackage{markdown}
%\RequirePackage{bussproofs}
\RequirePackage[tableaux]{prooftrees}
\usetikzlibrary{arrows.meta}
 \forestset{line numbering, close with = x}
% Allow for easy commas inside trees
\renewcommand{\,}{\text{, }}


\usepackage{tabulary}

\usepackage{open-logic-config}

\setlength{\parskip}{1ex plus 0.5ex minus 0.2ex}

\AtBeginSection[]
{
\begin{frame}
	\Huge{\color{darkblue} \insertsection}
\end{frame}
}

\renewenvironment*{quote}	
	{\list{}{\rightmargin   \leftmargin} \item } 	
	{\endlist }

\definecolor{darkgreen}{rgb}{0,0.7,0}
\definecolor{darkblue}{rgb}{0,0,0.8}

\newcommand{\starttab}{\begin{center}
\vspace{6pt}
\begin{tabular}}

\newcommand{\stoptab}{\end{tabular}
\vspace{6pt}
\end{center}
\noindent}


\newcommand{\sif}{\rightarrow}
\newcommand{\siff}{\leftrightarrow}
\newcommand{\EF}{\end{frame}}


\newcommand{\TreeStart}[1]{
%\end{frame}
\begin{frame}
\begin{center}
\begin{tikzpicture}[scale=#1]
\tikzset{every tree node/.style={align=center,anchor=north}}
%\Tree
}

\newcommand{\TreeEnd}{
\end{tikzpicture}
%\end{center}
}

\newcommand{\DisplayArg}[2]{
\begin{enumerate}
{#1}
\end{enumerate}
\vspace{-6pt}
\hrulefill

%\hspace{14pt} #2
%{\addtolength{\leftskip}{14pt} #2}
\begin{quote}
{\normalfont #2}
\end{quote}
\vspace{12pt}
}

\newenvironment{ProofTree}[1][1]{
\begin{center}
\begin{tikzpicture}[scale=#1]
\tikzset{every tree node/.style={align=center,anchor=south}}
}
{
\end{tikzpicture}
\end{center}
}

\newcommand{\TreeFrame}[2]{
\begin{columns}[c]
\column{0.5\textwidth}
\begin{center}
\begin{prooftree}{}
#1
\end{prooftree}
\end{center}
\column{0.45\textwidth}
%\begin{markdown}
#2
%\end{markdown}
\end{columns}
}

\newcommand{\ScaledTreeFrame}[3]{
\begin{columns}[c]
\column{0.5\textwidth}
\begin{center}
\scalebox{#1}{
\begin{prooftree}{}
#2
\end{prooftree}
}
\end{center}
\column{0.45\textwidth}
%\begin{markdown}
#3
%\end{markdown}
\end{columns}
}

\usepackage[bb=boondox]{mathalfa}
\DeclareMathAlphabet{\mathbx}{U}{BOONDOX-ds}{m}{n}
\SetMathAlphabet{\mathbx}{bold}{U}{BOONDOX-ds}{b}{n}
\DeclareMathAlphabet{\mathbbx} {U}{BOONDOX-ds}{b}{n}


\newenvironment{oltableau}{\center\tableau{}} %wff format={anchor = base west}}}
       {\endtableau\endcenter}
       
\newcommand{\formula}[1]{$#1$}

\usepackage{tabulary}
\usepackage{booktabs}

\def\begincols{\begin{columns}}
\def\begincol{\begin{column}}
\def\endcol{\end{column}}
\def\endcols{\end{columns}}

\usepackage[italic]{mathastext}
\usepackage{nicefrac}

\definecolor{mygreen}{RGB}{0, 100, 0}
\definecolor{mypink2}{RGB}{219, 48, 122}
\definecolor{dodgerblue}{RGB}{30,144,255}

%\def\True{\textcolor{dodgerblue}{\text{T}}}
%\def\False{\textcolor{red}{\text{F}}}

\def\True{\mathbb{T}}
\def\False{\mathbb{F}}

\title{\((\Diamond A \vee \Diamond B) \rightarrow \Diamond (A \vee B)\) (in K)}
\author{Build a Tableau}
\date{To Check Whether it is Valid}

\begin{document}
\frame{\titlepage}

\begin{frame}{\((\Diamond A \vee \Diamond B) \rightarrow \Diamond (A \vee B)\)}
\protect\hypertarget{diamond-a-vee-diamond-b-rightarrow-diamond-a-vee-b}{}

\begin{oltableau}
[\pFmla{\False}{(\Diamond A \vee \Diamond B) \rightarrow \Diamond (A \vee B)}{1}, just = \TAss,
]
\end{oltableau}

\bigskip

Start with it being false at 1.

\end{frame}

\begin{frame}{\((\Diamond A \vee \Diamond B) \rightarrow \Diamond (A \vee B)\)}
\protect\hypertarget{diamond-a-vee-diamond-b-rightarrow-diamond-a-vee-b-1}{}

\begin{oltableau}
[\pFmla{\False}{(\Diamond A \vee \Diamond B) \rightarrow \Diamond (A \vee B)}{1}, checked, just = \TAss,
  [\pFmla{\True}{\Diamond A \vee \Diamond B}{1}, just = {\TRule{\False}{\rightarrow}[1]},
    [\pFmla{\False}{\Diamond (A \vee B)}{1}, just = {\TRule{\False}{\rightarrow}[1]}
    ]
  ]
]
\end{oltableau}

\bigskip

True antecedent, false consequent.

\end{frame}

\begin{frame}{\((\Diamond A \vee \Diamond B) \rightarrow \Diamond (A \vee B)\)}
\protect\hypertarget{diamond-a-vee-diamond-b-rightarrow-diamond-a-vee-b-2}{}

\begin{oltableau}
[\pFmla{\False}{(\Diamond A \vee \Diamond B) \rightarrow \Diamond (A \vee B)}{1}, checked, just = \TAss,
  [\pFmla{\True}{\Diamond A \vee \Diamond B}{1}, checked, just = {\TRule{\False}{\rightarrow}[1]},
    [\pFmla{\False}{\Diamond (A \vee B)}{1}, just = {\TRule{\False}{\rightarrow}[1]},
      [\pFmla{\True}{\Diamond A}{1}, just = {\TRule{\True}{\vee}[2]}]
      [\pFmla{\True}{\Diamond B}{1}, just = {\TRule{\True}{\vee}[2]}]
    ]
  ]
]
\end{oltableau}

\bigskip

Nothing to do but branch.

\end{frame}

\begin{frame}{\((\Diamond A \vee \Diamond B) \rightarrow \Diamond (A \vee B)\)}
\protect\hypertarget{diamond-a-vee-diamond-b-rightarrow-diamond-a-vee-b-3}{}

\begin{oltableau}
[\pFmla{\False}{(\Diamond A \vee \Diamond B) \rightarrow \Diamond (A \vee B)}{1}, checked, just = \TAss,
  [\pFmla{\True}{\Diamond A \vee \Diamond B}{1}, checked, just = {\TRule{\False}{\rightarrow}[1]},
    [\pFmla{\False}{\Diamond (A \vee B)}{1}, just = {\TRule{\False}{\rightarrow}[1]},
      [\pFmla{\True}{\Diamond A}{1}, checked, just = {\TRule{\True}{\vee}[2]}
        [\pFmla{\True}{A}{1.1}, just = {\TRule{\True}{\Diamond}[4]}
        ]
      ]
      [\pFmla{\True}{\Diamond B}{1}, checked, just = {\TRule{\True}{\vee}[2]}
        [\pFmla{\True}{B}{1.2}, just = {\TRule{\True}{\Diamond}[4]}
        ]
      ]
    ]
  ]
]
\end{oltableau}

\bigskip

Diamond sentences have to be made true somehow.

\end{frame}

\begin{frame}{\((\Diamond A \vee \Diamond B) \rightarrow \Diamond (A \vee B)\)}
\protect\hypertarget{diamond-a-vee-diamond-b-rightarrow-diamond-a-vee-b-4}{}

\begin{oltableau}
[\pFmla{\False}{(\Diamond A \vee \Diamond B) \rightarrow \Diamond (A \vee B)}{1}, checked, just = \TAss,
  [\pFmla{\True}{\Diamond A \vee \Diamond B}{1}, checked, just = {\TRule{\False}{\rightarrow}[1]},
    [\pFmla{\False}{\Diamond (A \vee B)}{1}, just = {\TRule{\False}{\rightarrow}[1]},
      [\pFmla{\True}{\Diamond A}{1}, checked, just = {\TRule{\True}{\vee}[2]}
        [\pFmla{\True}{A}{1.1}, just = {\TRule{\True}{\Diamond}[4]}
          [\pFmla{\False}{A \vee B}{1.1},  just = {\TRule{\False}{\Diamond}[3]}
          ]
        ]
      ]
      [\pFmla{\True}{\Diamond B}{1}, checked,  just = {\TRule{\True}{\vee}[2]}
        [\pFmla{\True}{B}{1.2}, just = {\TRule{\True}{\Diamond}[4]}
          [\pFmla{\False}{A \vee B}{1.2},  just = {\TRule{\False}{\Diamond}[3]}
          ]
        ]
      ]
    ]
  ]
]
\end{oltableau}

\bigskip

False diamond sentences are false at all accessible worlds.

\end{frame}

\begin{frame}{\((\Diamond A \vee \Diamond B) \rightarrow \Diamond (A \vee B)\)}
\protect\hypertarget{diamond-a-vee-diamond-b-rightarrow-diamond-a-vee-b-5}{}

\begin{oltableau}
[\pFmla{\False}{(\Diamond A \vee \Diamond B) \rightarrow \Diamond (A \vee B)}{1}, checked, just = \TAss,
  [\pFmla{\True}{\Diamond A \vee \Diamond B}{1}, checked, just = {\TRule{\False}{\rightarrow}[1]},
    [\pFmla{\False}{\Diamond (A \vee B)}{1}, just = {\TRule{\False}{\rightarrow}[1]},
      [\pFmla{\True}{\Diamond A}{1}, checked, just = {\TRule{\True}{\vee}[2]}
        [\pFmla{\True}{A}{1.1}, just = {\TRule{\True}{\Diamond}[4]}
          [\pFmla{\False}{A \vee B}{1.1},  checked, just = {\TRule{\False}{\Diamond}[3]}
            [\pFmla{\False}{A}{1.1}, just = {\TRule{\False}{\vee}[6]}, close]
          ]
        ]
      ]
      [\pFmla{\True}{\Diamond B}{1}, checked, just = {\TRule{\True}{\vee}[2]}
        [\pFmla{\True}{B}{1.2}, just = {\TRule{\True}{\Diamond}[4]}
          [\pFmla{\False}{A \vee B}{1.2},  checked, just = {\TRule{\False}{\Diamond}[3]}
            [\pFmla{\False}{B}{1.2}, just = {\TRule{\False}{\vee}[6]}, close]
          ]
        ]
      ]
    ]
  ]
]
\end{oltableau}

\bigskip

And false or sentences have each side of the or false.

\end{frame}

\begin{frame}{\((\Diamond A \vee \Diamond B) \rightarrow \Diamond (A \vee B)\)}
\protect\hypertarget{diamond-a-vee-diamond-b-rightarrow-diamond-a-vee-b-6}{}

\begin{oltableau}
[\pFmla{\False}{(\Diamond A \vee \Diamond B) \rightarrow \Diamond (A \vee B)}{1}, checked, just = \TAss,
  [\pFmla{\True}{\Diamond A \vee \Diamond B}{1}, checked, just = {\TRule{\False}{\rightarrow}[1]},
    [\pFmla{\False}{\Diamond (A \vee B)}{1}, just = {\TRule{\False}{\rightarrow}[1]},
      [\pFmla{\True}{\Diamond A}{1}, checked, just = {\TRule{\True}{\vee}[2]}
        [\pFmla{\True}{A}{1.1}, just = {\TRule{\True}{\Diamond}[4]}
          [\pFmla{\False}{A \vee B}{1.1},  checked, just = {\TRule{\False}{\Diamond}[3]}
            [\pFmla{\False}{A}{1.1}, just = {\TRule{\False}{\vee}[6]}, close]
          ]
        ]
      ]
      [\pFmla{\True}{\Diamond B}{1}, checked, just = {\TRule{\True}{\vee}[2]}
        [\pFmla{\True}{B}{1.2}, just = {\TRule{\True}{\Diamond}[4]}
          [\pFmla{\False}{A \vee B}{1.2}, checked,  just = {\TRule{\False}{\Diamond}[3]}
            [\pFmla{\False}{B}{1.2}, just = {\TRule{\False}{\vee}[6]}, close]
          ]
        ]
      ]
    ]
  ]
]
\end{oltableau}

I've cheated a bit here by just listing one justification for the lines
after the branch. It's ok because the tree is completely symmetric; the
same thing happens on each branch.

\end{frame}

\end{document}
