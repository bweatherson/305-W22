% Options for packages loaded elsewhere
\PassOptionsToPackage{unicode}{hyperref}
\PassOptionsToPackage{hyphens}{url}
%
\documentclass[
  ignorenonframetext,
]{beamer}
\usepackage{pgfpages}
\setbeamertemplate{caption}[numbered]
\setbeamertemplate{caption label separator}{: }
\setbeamercolor{caption name}{fg=normal text.fg}
\beamertemplatenavigationsymbolsempty
% Prevent slide breaks in the middle of a paragraph
\widowpenalties 1 10000
\raggedbottom
\setbeamertemplate{part page}{
  \centering
  \begin{beamercolorbox}[sep=16pt,center]{part title}
    \usebeamerfont{part title}\insertpart\par
  \end{beamercolorbox}
}
\setbeamertemplate{section page}{
  \centering
  \begin{beamercolorbox}[sep=12pt,center]{part title}
    \usebeamerfont{section title}\insertsection\par
  \end{beamercolorbox}
}
\setbeamertemplate{subsection page}{
  \centering
  \begin{beamercolorbox}[sep=8pt,center]{part title}
    \usebeamerfont{subsection title}\insertsubsection\par
  \end{beamercolorbox}
}
\AtBeginPart{
  \frame{\partpage}
}
\AtBeginSection{
  \ifbibliography
  \else
    \frame{\sectionpage}
  \fi
}
\AtBeginSubsection{
  \frame{\subsectionpage}
}
\usepackage{amsmath,amssymb}
\usepackage{lmodern}
\usepackage{ifxetex,ifluatex}
\ifnum 0\ifxetex 1\fi\ifluatex 1\fi=0 % if pdftex
  \usepackage[T1]{fontenc}
  \usepackage[utf8]{inputenc}
  \usepackage{textcomp} % provide euro and other symbols
\else % if luatex or xetex
  \usepackage{unicode-math}
  \defaultfontfeatures{Scale=MatchLowercase}
  \defaultfontfeatures[\rmfamily]{Ligatures=TeX,Scale=1}
  \setmainfont[BoldFont = SF Pro Rounded Semibold]{SF Pro Rounded}
  \setmathfont[]{STIX Two Math}
\fi
\usefonttheme{serif} % use mainfont rather than sansfont for slide text
% Use upquote if available, for straight quotes in verbatim environments
\IfFileExists{upquote.sty}{\usepackage{upquote}}{}
\IfFileExists{microtype.sty}{% use microtype if available
  \usepackage[]{microtype}
  \UseMicrotypeSet[protrusion]{basicmath} % disable protrusion for tt fonts
}{}
\makeatletter
\@ifundefined{KOMAClassName}{% if non-KOMA class
  \IfFileExists{parskip.sty}{%
    \usepackage{parskip}
  }{% else
    \setlength{\parindent}{0pt}
    \setlength{\parskip}{6pt plus 2pt minus 1pt}}
}{% if KOMA class
  \KOMAoptions{parskip=half}}
\makeatother
\usepackage{xcolor}
\IfFileExists{xurl.sty}{\usepackage{xurl}}{} % add URL line breaks if available
\IfFileExists{bookmark.sty}{\usepackage{bookmark}}{\usepackage{hyperref}}
\hypersetup{
  pdftitle={305 Lecture 3.4 - Features of Validity},
  pdfauthor={Brian Weatherson},
  hidelinks,
  pdfcreator={LaTeX via pandoc}}
\urlstyle{same} % disable monospaced font for URLs
\newif\ifbibliography
\setlength{\emergencystretch}{3em} % prevent overfull lines
\providecommand{\tightlist}{%
  \setlength{\itemsep}{0pt}\setlength{\parskip}{0pt}}
\setcounter{secnumdepth}{-\maxdimen} % remove section numbering
\let\Tiny=\tiny

 \setbeamertemplate{navigation symbols}{} 

% \usetheme{Madrid}
 \usetheme[numbering=none, progressbar=foot]{metropolis}
 \usecolortheme{wolverine}
 \usepackage{color}
 \usepackage{MnSymbol}
% \usepackage{movie15}

\usepackage{amssymb}% http://ctan.org/pkg/amssymb
\usepackage{pifont}% http://ctan.org/pkg/pifont
\newcommand{\cmark}{\ding{51}}%
\newcommand{\xmark}{\ding{55}}%

\DeclareSymbolFont{symbolsC}{U}{txsyc}{m}{n}
\DeclareMathSymbol{\boxright}{\mathrel}{symbolsC}{128}
\DeclareMathAlphabet{\mathpzc}{OT1}{pzc}{m}{it}

\usepackage{tikz-qtree}
% \usepackage{markdown}
%\RequirePackage{bussproofs}
\usetikzlibrary{arrows.meta}
\RequirePackage[tableaux]{prooftrees}
\forestset{line numbering, close with = x}
% Allow for easy commas inside trees
\renewcommand{\,}{\text{, }}


\usepackage{tabulary}

\usepackage{open-logic-config}

\setlength{\parskip}{1ex plus 0.5ex minus 0.2ex}

\AtBeginSection[]
{
\begin{frame}
	\Huge{\color{darkblue} \insertsection}
\end{frame}
}

\renewenvironment*{quote}	
	{\list{}{\rightmargin   \leftmargin} \item } 	
	{\endlist }

\definecolor{darkgreen}{rgb}{0,0.7,0}
\definecolor{darkblue}{rgb}{0,0,0.8}

\newcommand{\starttab}{\begin{center}
\vspace{6pt}
\begin{tabular}}

\newcommand{\stoptab}{\end{tabular}
\vspace{6pt}
\end{center}
\noindent}


\newcommand{\sif}{\rightarrow}
\newcommand{\siff}{\leftrightarrow}
\newcommand{\EF}{\end{frame}}


\newcommand{\TreeStart}[1]{
%\end{frame}
\begin{frame}
\begin{center}
\begin{tikzpicture}[scale=#1]
\tikzset{every tree node/.style={align=center,anchor=north}}
%\Tree
}

\newcommand{\TreeEnd}{
\end{tikzpicture}
%\end{center}
}

\newcommand{\DisplayArg}[2]{
\begin{enumerate}
{#1}
\end{enumerate}
\vspace{-6pt}
\hrulefill

%\hspace{14pt} #2
%{\addtolength{\leftskip}{14pt} #2}
\begin{quote}
{\normalfont #2}
\end{quote}
\vspace{12pt}
}

\newenvironment{ProofTree}[1][1]{
\begin{center}
\begin{tikzpicture}[scale=#1]
\tikzset{every tree node/.style={align=center,anchor=south}}
}
{
\end{tikzpicture}
\end{center}
}

\newcommand{\TreeFrame}[2]{
\begin{columns}[c]
\column{0.5\textwidth}
\begin{center}
\begin{prooftree}{}
#1
\end{prooftree}
\end{center}
\column{0.45\textwidth}
%\begin{markdown}
#2
%\end{markdown}
\end{columns}
}

\newcommand{\ScaledTreeFrame}[3]{
\begin{columns}[c]
\column{0.5\textwidth}
\begin{center}
\scalebox{#1}{
\begin{prooftree}{}
#2
\end{prooftree}
}
\end{center}
\column{0.45\textwidth}
%\begin{markdown}
#3
%\end{markdown}
\end{columns}
}

\usepackage[bb=boondox]{mathalfa}
\DeclareMathAlphabet{\mathbx}{U}{BOONDOX-ds}{m}{n}
\SetMathAlphabet{\mathbx}{bold}{U}{BOONDOX-ds}{b}{n}
\DeclareMathAlphabet{\mathbbx} {U}{BOONDOX-ds}{b}{n}


\newenvironment{oltableau}{\center\tableau{}} %wff format={anchor = base west}}}
       {\endtableau\endcenter}
       
\newcommand{\formula}[1]{$#1$}

\usepackage{tabulary}
\usepackage{booktabs}

\def\begincols{\begin{columns}}
\def\begincol{\begin{column}}
\def\endcol{\end{column}}
\def\endcols{\end{columns}}

\usepackage[italic]{mathastext}
\usepackage{nicefrac}

\definecolor{mygreen}{RGB}{0, 100, 0}
\definecolor{mypink2}{RGB}{219, 48, 122}
\definecolor{dodgerblue}{RGB}{30,144,255}

%\def\True{\textcolor{dodgerblue}{\text{T}}}
%\def\False{\textcolor{red}{\text{F}}}

\def\True{\mathbb{T}}
\def\False{\mathbb{F}}

% This is because arguments didn't have enough space after them
\usepackage{etoolbox}
\AfterEndEnvironment{description}{\vspace{9pt}}
\AfterEndEnvironment{oltableau}{\vspace{9pt}}
\BeforeBeginEnvironment{oltableau}{\vspace{9pt}}
\AfterEndEnvironment{center}{\vspace{12pt}}
\BeforeBeginEnvironment{tabular}{\vspace{9pt}}

\setlength\heavyrulewidth{0pt}
\setlength\lightrulewidth{0pt}

%\def\toprule{}
%\def\bottomrule{}
%\def\midrule{}

\setbeamertemplate{caption}{\raggedright\insertcaption}

\ifluatex
  \usepackage{selnolig}  % disable illegal ligatures
\fi

\title{305 Lecture 3.4 - Features of Validity}
\author{Brian Weatherson}
\date{}

\begin{document}
\frame{\titlepage}

\begin{frame}{Plan}
\protect\hypertarget{plan}{}
This lecture finishes our discussion of truth tables by looking some
properties validity has in the truth table system.
\end{frame}

\begin{frame}{Associated Reading}
\protect\hypertarget{associated-reading}{}
forall x, chapter 12, sections 12.5-12.7.
\end{frame}

\begin{frame}{The Rules}
\protect\hypertarget{the-rules}{}
\begin{itemize}
\tightlist
\item
  An argument is \textbf{invalid} if there is a row on the truth table
  where all the premises are true and the conclusion is false.
  (Roughly!)
\item
  It is \textbf{valid} if all the rows where the premises are all true,
  the conclusion is true as well.
\end{itemize}
\end{frame}

\begin{frame}{A Relevance Failure}
\protect\hypertarget{a-relevance-failure}{}
Is this argument valid?

\begin{description}
\tightlist
\item[~]
\(A\)
\item[\(\therefore\)]
B \(\vee \neg B\)
\end{description}

\pause

Yes!

\begin{itemize}
\tightlist
\item
  There is no line where the conclusion is false.
\item
  So there are no lines where the premise is true and the conclusion
  false.
\item
  So it is not invalid, i.e., it is valid.
\end{itemize}
\end{frame}

\begin{frame}{Terminology}
\protect\hypertarget{terminology}{}
Say a \textbf{valuation} is a function \(v\) from sentences to
\(\{\True, \False\}\) satisfying these constraints.

\begin{enumerate}
\tightlist
\item
  \(v(\neg A) = \True\) if \(v(A) = \False\), and \(v(\neg A) = \False\)
  otherwise.
\item
  \(v(A \vee B) = \True\) if \(v(A) = \True\) or \(v(B) = \True\), and
  \(v(A \vee B) = \False\) otherwise.
\item
  \(v(A \wedge B) = \True\) if \(v(A) = \True\) and \(v(B) = \True\),
  and \(v(A \wedge B) = \False\) otherwise.
\item
  \(v(A \rightarrow B) = \True\) if \(v(A) = \False\) or
  \(v(B) = \True\), and \(v(A \rightarrow B) = \False\) otherwise.
\end{enumerate}
\end{frame}

\begin{frame}{Restating}
\protect\hypertarget{restating}{}
\begin{itemize}[<+->]
\tightlist
\item
  An argument is valid relative to a class of valuations \(V\) iff any
  valuation \(v \in V\) that makes all the premises \(\True\) also makes
  the conclusion \(\True\).
\item
  An argument is truth functionally valid when the class \(V\) is the
  class of valuations satisfying the constraints on the previous slide.
\end{itemize}
\end{frame}

\begin{frame}{Very Technical Terminology}
\protect\hypertarget{very-technical-terminology}{}
\begin{itemize}
\tightlist
\item
  I'll use \(\Gamma \vDash A\) to mean that the argument with premises
  \(\Gamma\) and conclusion \(A\) is valid in this sense - i.e., all
  valuations that make all of \(\Gamma\) come out \(\True\) also make
  \(A\) come out \(\True\).
\item
  The double bar in \(\vDash\) is to represent that this is a kind of
  validity defined in terms of valuations (or, as we'll start calling
  them, models), and not proofs.
\item
  For purposes of 305, the difference between \(\vdash\) and \(\vDash\)
  is not important, and if this is the last logic/mathematical
  philosophy course you plan to take, you don't have to worry about
  this.
\item
  But I like being pedantic even when it isn't relevant to the course.
\end{itemize}
\end{frame}

\begin{frame}{Closure}
\protect\hypertarget{closure}{}
\begin{quote}
If \(\Gamma \vDash A\) and \(\Gamma \vDash A \rightarrow B\) then
\(\Gamma \vDash B \pause\).
\end{quote}

Proof: Assume this is false. So assume that \(\Gamma \nvDash B\). So
there is a valuation function \(v\) that makes everything in \(\Gamma\)
come out \(\True\) and \(B\) come out \(\False \pause\). Either
\(v(A) = \True\) or \(v(A) = \False \pause\). If \(v(A) = \True\), then
\(v(A \rightarrow B) = \False\), contradicting
\(\Gamma \vDash A \rightarrow B \pause\). If \(v(A) = \False\), then
\(v\) is a counterexample to \(\Gamma \vDash A\), but we know
\(\Gamma \vDash A\) is true. Either way, such a \(v\) cannot exist, so
\(\Gamma \vDash B\) is true.
\end{frame}

\begin{frame}{Monotony}
\protect\hypertarget{monotony}{}
\begin{quote}
If \(\Gamma \vDash A\), and \(\Gamma \subset \Delta\), then
\(\Delta \vDash A\).
\end{quote}

That is, adding premises can't turn an argument from being valid to
invalid.
\end{frame}

\begin{frame}{Monotony Proof}
\protect\hypertarget{monotony-proof}{}
\begin{itemize}
\tightlist
\item
  Assume that for all \(B \in \Delta, v(B) = \True\).
\item
  We need to prove that \(v(A) = \True\).\pause
\item
  Assume \(C \in \Gamma\).
\item
  Then \(C \in \Delta\), since \(\Gamma \subset \Delta\).\pause
\item
  So by hypothesis, \(v(C) = \True\), since everything in \(\Delta\) is
  \(\True\).\pause
\item
  So \(v\) is such that everything in \(\Gamma\) is \(\True\).\pause
\item
  And since \(\Gamma \vDash A\), that implies \(v(A) = \True\), as
  required.
\end{itemize}
\end{frame}

\begin{frame}{Monotony Commentary}
\protect\hypertarget{monotony-commentary}{}
\begin{itemize}
\tightlist
\item
  This idea, that adding premises doesn't destroy validity, only works
  for logical arguments.
\item
  It isn't true for good arguments in general.
\end{itemize}
\end{frame}

\begin{frame}{Tweety the First}
\protect\hypertarget{tweety-the-first}{}
\begin{description}
\tightlist
\item[~]
Tweety is a bird.
\item[\(\therefore\)]
Tweety flies.
\end{description}

That's a perfectly good, though not logically valid, argument.
\end{frame}

\begin{frame}{Tweety the Second}
\protect\hypertarget{tweety-the-second}{}
\begin{description}
\tightlist
\item[~]
Tweety is a bird.
\item[~]
Tweety is black and white, lives in Antarctica, and lays large eggs.
\item[\(\therefore\)]
Tweety flies.
\end{description}

That's not a very good argument!
\end{frame}

\begin{frame}{Transitivity}
\protect\hypertarget{transitivity}{}
\begin{quote}
If \(\Gamma \vDash A\) and \(\Delta \cup A \vDash B\) then
\(\Gamma \cup \Delta \vDash B\)
\end{quote}

If some premises entail \(A\), and some other premises plus \(A\) entail
\(B\), then the two sets of premises between them entail \(B\).
\pause This is crucial for being able to chain together lines of
reasoning.
\end{frame}

\begin{frame}{Transitivity Proof}
\protect\hypertarget{transitivity-proof}{}
\begin{itemize}
\tightlist
\item
  Assume that for all \(C \in \Gamma \cup \Delta, v(C) = \True\).
\item
  We need to prove \(v(B) = \True\).\pause
\item
  Since everything in \(\Gamma\) is \(\True\) according to \(v\), and
  \(\Gamma \vDash A\), it follows that \(v(A) = \True\).\pause
\item
  Since everything in \(\Delta\) is \(\True\) according to \(v\), and
  \(A\) is \(\True\) according to \(v\), and \(\Delta \cup A \vDash B\),
  it follows that \(v(B) = \True\), as required.
\end{itemize}
\end{frame}

\begin{frame}{Deduction Theorem}
\protect\hypertarget{deduction-theorem}{}
This is why we define \(\rightarrow\) the way we do.

\begin{quote}
\(\Gamma \vDash A \rightarrow B\) if and only if
\(\Gamma \cup A \vDash B\).
\end{quote}

Note that there are two claims here - one each direction. We need to
prove each.
\end{frame}

\begin{frame}{Deduction Theorem Left-to-Right}
\protect\hypertarget{deduction-theorem-left-to-right}{}
\begin{itemize}
\tightlist
\item
  Assume \(\Gamma \vDash A \rightarrow B\), and prove
  \(\Gamma \cup A \vDash B\).
\item
  So assume \(v(C) = \True\) for all \(C \in \Gamma\), and
  \(v(A) = \True\), and aim to prove \(v(B) = \True\).\pause
\item
  Since \(\Gamma \vDash A \rightarrow B\) and \(v(C) = \True\) for all
  \(C \in \Gamma\), it follows that
  \(v(A \rightarrow B) = \True\).\pause
\item
  Since \(v(A \rightarrow B) = \True\) and \(v(A) = \True\), it must be
  that \(v(B) = \True\), since that's the only line on the truth table
  where \(A \rightarrow B\) and \(A\) are both \(\True\).
\end{itemize}
\end{frame}

\begin{frame}{Deduction Theorem Right-to-Left}
\protect\hypertarget{deduction-theorem-right-to-left}{}
\begin{itemize}
\tightlist
\item
  Assume that \(\Gamma \cup A \vDash B\), and prove
  \(\Gamma \vDash A \rightarrow B\).
\item
  So assume \(v(C) = \True\) for all \(C \in \Gamma\), and prove
  \(v(A \rightarrow B) = \True\). \pause
\item
  Either \(v(A) = \True\) or \(v(A) = \False\). Take each case in
  turn.\pause
\item
  If \(v(A) = \True\), then since \(v(C) = \True\) for all
  \(C \in \Gamma\), and \(\Gamma \cup A \vDash B\), it follows that
  \(v(B) = \True \pause\), so \(v(A \rightarrow B) = \True \pause\).
\item
  If \(v(A) = \False\), it follows directly that
  \(v(A \rightarrow B) = \True \pause\).
\item
  Either way, \(v(A \rightarrow B) = \True\) as required.
\end{itemize}
\end{frame}

\begin{frame}{Deduction Theorem Comments}
\protect\hypertarget{deduction-theorem-comments}{}
\begin{itemize}
\tightlist
\item
  This is a striking result.
\item
  It shows that proving \(A \rightarrow B\) is just the same as proving
  \(B\), assuming you're allowed to add \(A\) as an extra assumption.
\item
  And that's a good thing, intuitively. That is how we prove
  conditionals.
\item
  But it only works if you have the (very odd looking) truth table that
  we're using for \(\rightarrow\).
\item
  This is the main reason for thinking, despite it's odd appearance,
  that this truth table is the right one for \(\rightarrow\).
\end{itemize}
\end{frame}

\begin{frame}{For Next Time}
\protect\hypertarget{for-next-time}{}
We will start working on a different way to analyse arguments: truth
trees.
\end{frame}

\end{document}
