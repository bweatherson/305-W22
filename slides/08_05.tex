% Options for packages loaded elsewhere
\PassOptionsToPackage{unicode}{hyperref}
\PassOptionsToPackage{hyphens}{url}
%
\documentclass[
  ignorenonframetext,
]{beamer}
\usepackage{pgfpages}
\setbeamertemplate{caption}[numbered]
\setbeamertemplate{caption label separator}{: }
\setbeamercolor{caption name}{fg=normal text.fg}
\beamertemplatenavigationsymbolsempty
% Prevent slide breaks in the middle of a paragraph
\widowpenalties 1 10000
\raggedbottom
\setbeamertemplate{part page}{
  \centering
  \begin{beamercolorbox}[sep=16pt,center]{part title}
    \usebeamerfont{part title}\insertpart\par
  \end{beamercolorbox}
}
\setbeamertemplate{section page}{
  \centering
  \begin{beamercolorbox}[sep=12pt,center]{part title}
    \usebeamerfont{section title}\insertsection\par
  \end{beamercolorbox}
}
\setbeamertemplate{subsection page}{
  \centering
  \begin{beamercolorbox}[sep=8pt,center]{part title}
    \usebeamerfont{subsection title}\insertsubsection\par
  \end{beamercolorbox}
}
\AtBeginPart{
  \frame{\partpage}
}
\AtBeginSection{
  \ifbibliography
  \else
    \frame{\sectionpage}
  \fi
}
\AtBeginSubsection{
  \frame{\subsectionpage}
}
\usepackage{amsmath,amssymb}
\usepackage{lmodern}
\usepackage{ifxetex,ifluatex}
\ifnum 0\ifxetex 1\fi\ifluatex 1\fi=0 % if pdftex
  \usepackage[T1]{fontenc}
  \usepackage[utf8]{inputenc}
  \usepackage{textcomp} % provide euro and other symbols
\else % if luatex or xetex
  \usepackage{unicode-math}
  \defaultfontfeatures{Scale=MatchLowercase}
  \defaultfontfeatures[\rmfamily]{Ligatures=TeX,Scale=1}
  \setmainfont[BoldFont = SF Pro Rounded Semibold]{SF Pro Rounded}
  \setmathfont[]{STIX Two Math}
\fi
\usefonttheme{serif} % use mainfont rather than sansfont for slide text
% Use upquote if available, for straight quotes in verbatim environments
\IfFileExists{upquote.sty}{\usepackage{upquote}}{}
\IfFileExists{microtype.sty}{% use microtype if available
  \usepackage[]{microtype}
  \UseMicrotypeSet[protrusion]{basicmath} % disable protrusion for tt fonts
}{}
\makeatletter
\@ifundefined{KOMAClassName}{% if non-KOMA class
  \IfFileExists{parskip.sty}{%
    \usepackage{parskip}
  }{% else
    \setlength{\parindent}{0pt}
    \setlength{\parskip}{6pt plus 2pt minus 1pt}}
}{% if KOMA class
  \KOMAoptions{parskip=half}}
\makeatother
\usepackage{xcolor}
\IfFileExists{xurl.sty}{\usepackage{xurl}}{} % add URL line breaks if available
\IfFileExists{bookmark.sty}{\usepackage{bookmark}}{\usepackage{hyperref}}
\hypersetup{
  pdftitle={305 Lecture 8.5 - More Sampling without Replacement},
  pdfauthor={Brian Weatherson},
  hidelinks,
  pdfcreator={LaTeX via pandoc}}
\urlstyle{same} % disable monospaced font for URLs
\newif\ifbibliography
\usepackage{longtable,booktabs,array}
\usepackage{calc} % for calculating minipage widths
\usepackage{caption}
% Make caption package work with longtable
\makeatletter
\def\fnum@table{\tablename~\thetable}
\makeatother
\setlength{\emergencystretch}{3em} % prevent overfull lines
\providecommand{\tightlist}{%
  \setlength{\itemsep}{0pt}\setlength{\parskip}{0pt}}
\setcounter{secnumdepth}{-\maxdimen} % remove section numbering
\let\Tiny=\tiny

 \setbeamertemplate{navigation symbols}{} 

% \usetheme{Madrid}
 \usetheme[numbering=none, progressbar=foot]{metropolis}
 \usecolortheme{wolverine}
 \usepackage{color}
 \usepackage{MnSymbol}
% \usepackage{movie15}

\usepackage{amssymb}% http://ctan.org/pkg/amssymb
\usepackage{pifont}% http://ctan.org/pkg/pifont
\newcommand{\cmark}{\ding{51}}%
\newcommand{\xmark}{\ding{55}}%

\DeclareSymbolFont{symbolsC}{U}{txsyc}{m}{n}
\DeclareMathSymbol{\boxright}{\mathrel}{symbolsC}{128}
\DeclareMathAlphabet{\mathpzc}{OT1}{pzc}{m}{it}

\usepackage{tikz-qtree}
% \usepackage{markdown}
%\RequirePackage{bussproofs}
\usetikzlibrary{arrows.meta}
\RequirePackage[tableaux]{prooftrees}
\forestset{line numbering, close with = x}
% Allow for easy commas inside trees
\renewcommand{\,}{\text{, }}


\usepackage{tabulary}

\usepackage{open-logic-config}

\setlength{\parskip}{1ex plus 0.5ex minus 0.2ex}

\AtBeginSection[]
{
\begin{frame}
	\Huge{\color{darkblue} \insertsection}
\end{frame}
}

\renewenvironment*{quote}	
	{\list{}{\rightmargin   \leftmargin} \item } 	
	{\endlist }

\definecolor{darkgreen}{rgb}{0,0.7,0}
\definecolor{darkblue}{rgb}{0,0,0.8}

\newcommand{\starttab}{\begin{center}
\vspace{6pt}
\begin{tabular}}

\newcommand{\stoptab}{\end{tabular}
\vspace{6pt}
\end{center}
\noindent}


\newcommand{\sif}{\rightarrow}
\newcommand{\siff}{\leftrightarrow}
\newcommand{\EF}{\end{frame}}


\newcommand{\TreeStart}[1]{
%\end{frame}
\begin{frame}
\begin{center}
\begin{tikzpicture}[scale=#1]
\tikzset{every tree node/.style={align=center,anchor=north}}
%\Tree
}

\newcommand{\TreeEnd}{
\end{tikzpicture}
%\end{center}
}

\newcommand{\DisplayArg}[2]{
\begin{enumerate}
{#1}
\end{enumerate}
\vspace{-6pt}
\hrulefill

%\hspace{14pt} #2
%{\addtolength{\leftskip}{14pt} #2}
\begin{quote}
{\normalfont #2}
\end{quote}
\vspace{12pt}
}

\newenvironment{ProofTree}[1][1]{
\begin{center}
\begin{tikzpicture}[scale=#1]
\tikzset{every tree node/.style={align=center,anchor=south}}
}
{
\end{tikzpicture}
\end{center}
}

\newcommand{\TreeFrame}[2]{
\begin{columns}[c]
\column{0.5\textwidth}
\begin{center}
\begin{prooftree}{}
#1
\end{prooftree}
\end{center}
\column{0.45\textwidth}
%\begin{markdown}
#2
%\end{markdown}
\end{columns}
}

\newcommand{\ScaledTreeFrame}[3]{
\begin{columns}[c]
\column{0.5\textwidth}
\begin{center}
\scalebox{#1}{
\begin{prooftree}{}
#2
\end{prooftree}
}
\end{center}
\column{0.45\textwidth}
%\begin{markdown}
#3
%\end{markdown}
\end{columns}
}

\usepackage[bb=boondox]{mathalfa}
\DeclareMathAlphabet{\mathbx}{U}{BOONDOX-ds}{m}{n}
\SetMathAlphabet{\mathbx}{bold}{U}{BOONDOX-ds}{b}{n}
\DeclareMathAlphabet{\mathbbx} {U}{BOONDOX-ds}{b}{n}


\newenvironment{oltableau}{\center\tableau{}} %wff format={anchor = base west}}}
       {\endtableau\endcenter}
       
\newcommand{\formula}[1]{$#1$}

\usepackage{tabulary}
\usepackage{booktabs}

\def\begincols{\begin{columns}}
\def\begincol{\begin{column}}
\def\endcol{\end{column}}
\def\endcols{\end{columns}}

\usepackage[italic]{mathastext}
\usepackage{nicefrac}

\definecolor{mygreen}{RGB}{0, 100, 0}
\definecolor{mypink2}{RGB}{219, 48, 122}
\definecolor{dodgerblue}{RGB}{30,144,255}

%\def\True{\textcolor{dodgerblue}{\text{T}}}
%\def\False{\textcolor{red}{\text{F}}}

\def\True{\mathbb{T}}
\def\False{\mathbb{F}}

% This is because arguments didn't have enough space after them
\usepackage{etoolbox}
\AfterEndEnvironment{description}{\vspace{9pt}}
\AfterEndEnvironment{oltableau}{\vspace{9pt}}
\BeforeBeginEnvironment{oltableau}{\vspace{9pt}}
\AfterEndEnvironment{center}{\vspace{12pt}}
\BeforeBeginEnvironment{tabular}{\vspace{9pt}}

\setlength\heavyrulewidth{0pt}
\setlength\lightrulewidth{0pt}

%\def\toprule{}
%\def\bottomrule{}
%\def\midrule{}

\setbeamertemplate{caption}{\raggedright\insertcaption}

\setlength\lightrulewidth{0.3pt}
\AfterEndEnvironment{center}{\vspace{-3pt}}
\AfterEndEnvironment{longtable}{\vspace{-6pt}}
\ifluatex
  \usepackage{selnolig}  % disable illegal ligatures
\fi

\title{305 Lecture 8.5 - More Sampling without Replacement}
\author{Brian Weatherson}
\date{}

\begin{document}
\frame{\titlepage}

\begin{frame}{Plan}
\protect\hypertarget{plan}{}
\begin{itemize}
\tightlist
\item
  To go over a more complicated example of sampling without replacement.
\end{itemize}
\end{frame}

\begin{frame}{Associated Reading}
\protect\hypertarget{associated-reading}{}
Odds and Ends, Chapter 9
\end{frame}

\begin{frame}{Last (Difficult) Example}
\protect\hypertarget{last-difficult-example}{}
\begin{itemize}
\tightlist
\item
  There are four urns in the room, three of type X, one of type Y.
\item
  The type X urns have 4 blue marbles and 2 yellow marbles.
\item
  The type Y urn has 5 blue marbles and 3 yellow marbles.
\item
  One of the four urns was selected at random.
\item
  Then two marbles were selected \textbf{without replacement} from the
  randomly selected urn.
\item
  The first was blue, the second was yellow.
\item
  A third marble is about to be selected.
\item
  What is the probability that it is blue?
\end{itemize}
\end{frame}

\begin{frame}
\begin{longtable}[]{@{}cc@{}}
\toprule
Urn & Blue-then-Yellow \\
\midrule
\endhead
Type X &
\(\frac{3}{4} \times \frac{4}{6} \times \frac{2}{5} = \frac{1}{5}\) \\
& \\
Type Y & \\
& \\
\textbf{Total} & \\
\bottomrule
\end{longtable}

\[
\Pr(X \wedge Blue_1 \wedge Yellow_2) = \Pr(X) \times \Pr(Blue_1 | X) \times \Pr(Yellow_2 | X \wedge Blue_1)
\]
\end{frame}

\begin{frame}
\begin{longtable}[]{@{}cc@{}}
\toprule
Urn & Blue-then-Yellow \\
\midrule
\endhead
Type X &
\(\frac{3}{4} \times \frac{4}{6} \times \frac{2}{5} = \frac{1}{5}\) \\
& \\
Type Y &
\(\frac{1}{4} \times \frac{5}{8} \times \frac{3}{7} = \frac{15}{224}\) \\
& \\
\textbf{Total} & \\
\bottomrule
\end{longtable}

\[
\Pr(Y \wedge Blue_1 \wedge Yellow_2) = \Pr(Y) \times \Pr(Blue_1 | Y) \times \Pr(Yellow_2 | Y \wedge Blue_1)
\]
\end{frame}

\begin{frame}
\begin{longtable}[]{@{}cc@{}}
\toprule
Urn & Blue-then-Yellow \\
\midrule
\endhead
Type X & \(\nicefrac{1}{5}\) \\
Type Y & \(\nicefrac{15}{224}\) \\
\textbf{Total} & \(\nicefrac{299}{1120}\) \\
\bottomrule
\end{longtable}

You should double check this, but I think

\[
\frac{1}{5} + \frac{15}{224} = \frac{299}{1120}
\]

\bigskip

So that's \(\Pr(Blue_1 \wedge Yellow_2)\)
\end{frame}

\begin{frame}{Conditional Probabilities}
\protect\hypertarget{conditional-probabilities}{}
\[
\Pr(X | Blue_1 \wedge Yellow_2) = \frac{\Pr(X \wedge Blue_1 \wedge Yellow_2)}{\Pr(Blue_1 \wedge Yellow_2)} = \frac{\frac{1}{5}}{\frac{299}{1120}} = \frac{224}{299}
\]

\[
\Pr(Y | Blue_1 \wedge Yellow_2) = \frac{\Pr(Y \wedge Blue_1 \wedge Yellow_2)}{\Pr(Blue_1 \wedge Yellow_2)} = \frac{\frac{15}{224}}{\frac{299}{1120}} = \frac{75}{299}
\]

The probability of Y is ever so fractionally higher than when we
started.
\end{frame}

\begin{frame}{Next Marble}
\protect\hypertarget{next-marble}{}
\begin{itemize}
\tightlist
\item
  If X (and Blue-followed-by-Yellow), the probability of next marble
  being blue is \(\frac{3}{4}\).
\item
  If Y (and Blue-followed-by-Yellow), the probability of next marble
  being blue is \(\frac{2}{3}\). \pause
\item
  So overall probability of next marble being blue is
\end{itemize}

\[
 \frac{224}{299} \times \frac{3}{4} + \frac{75}{299} \times \frac{2}{3} = \frac{218}{299} \approx 0.729
\]
\end{frame}

\begin{frame}{General Strategy of Last Slide}
\protect\hypertarget{general-strategy-of-last-slide}{}
\begin{itemize}
\tightlist
\item
  If there are two hypotheses X and Y, and you want to know the
  probability of some event E, it will be given by
\end{itemize}

\[
\Pr(E) = \Pr(X)\Pr(E | X) + \Pr(Y)\Pr(E | Y)
\]

And that generalises to the case where there are multiple hypotheses
\(H_1, \dots H_n\)

\[
\Pr(E) = \Pr(H_1)\Pr(E | H_1) + \dots +  \Pr(H_n)\Pr(E | H_n)
\]
\end{frame}

\begin{frame}{For Next Time}
\protect\hypertarget{for-next-time}{}
Next week we will look at the use of probability in decision making, and
in science.
\end{frame}

\end{document}
