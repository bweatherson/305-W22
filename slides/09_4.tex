% Options for packages loaded elsewhere
\PassOptionsToPackage{unicode}{hyperref}
\PassOptionsToPackage{hyphens}{url}
%
\documentclass[
  ignorenonframetext,
]{beamer}
\usepackage{pgfpages}
\setbeamertemplate{caption}[numbered]
\setbeamertemplate{caption label separator}{: }
\setbeamercolor{caption name}{fg=normal text.fg}
\beamertemplatenavigationsymbolsempty
% Prevent slide breaks in the middle of a paragraph
\widowpenalties 1 10000
\raggedbottom
\setbeamertemplate{part page}{
  \centering
  \begin{beamercolorbox}[sep=16pt,center]{part title}
    \usebeamerfont{part title}\insertpart\par
  \end{beamercolorbox}
}
\setbeamertemplate{section page}{
  \centering
  \begin{beamercolorbox}[sep=12pt,center]{part title}
    \usebeamerfont{section title}\insertsection\par
  \end{beamercolorbox}
}
\setbeamertemplate{subsection page}{
  \centering
  \begin{beamercolorbox}[sep=8pt,center]{part title}
    \usebeamerfont{subsection title}\insertsubsection\par
  \end{beamercolorbox}
}
\AtBeginPart{
  \frame{\partpage}
}
\AtBeginSection{
  \ifbibliography
  \else
    \frame{\sectionpage}
  \fi
}
\AtBeginSubsection{
  \frame{\subsectionpage}
}
\usepackage{amsmath,amssymb}
\usepackage{lmodern}
\usepackage{ifxetex,ifluatex}
\ifnum 0\ifxetex 1\fi\ifluatex 1\fi=0 % if pdftex
  \usepackage[T1]{fontenc}
  \usepackage[utf8]{inputenc}
  \usepackage{textcomp} % provide euro and other symbols
\else % if luatex or xetex
  \usepackage{unicode-math}
  \defaultfontfeatures{Scale=MatchLowercase}
  \defaultfontfeatures[\rmfamily]{Ligatures=TeX,Scale=1}
  \setmainfont[BoldFont = SF Pro Rounded Semibold]{SF Pro Rounded}
  \setmathfont[]{STIX Two Math}
\fi
\usefonttheme{serif} % use mainfont rather than sansfont for slide text
% Use upquote if available, for straight quotes in verbatim environments
\IfFileExists{upquote.sty}{\usepackage{upquote}}{}
\IfFileExists{microtype.sty}{% use microtype if available
  \usepackage[]{microtype}
  \UseMicrotypeSet[protrusion]{basicmath} % disable protrusion for tt fonts
}{}
\makeatletter
\@ifundefined{KOMAClassName}{% if non-KOMA class
  \IfFileExists{parskip.sty}{%
    \usepackage{parskip}
  }{% else
    \setlength{\parindent}{0pt}
    \setlength{\parskip}{6pt plus 2pt minus 1pt}}
}{% if KOMA class
  \KOMAoptions{parskip=half}}
\makeatother
\usepackage{xcolor}
\IfFileExists{xurl.sty}{\usepackage{xurl}}{} % add URL line breaks if available
\IfFileExists{bookmark.sty}{\usepackage{bookmark}}{\usepackage{hyperref}}
\hypersetup{
  pdftitle={305 Lecture 9.4 - Maximise Expected Utility},
  pdfauthor={Brian Weatherson},
  hidelinks,
  pdfcreator={LaTeX via pandoc}}
\urlstyle{same} % disable monospaced font for URLs
\newif\ifbibliography
\usepackage{longtable,booktabs,array}
\usepackage{calc} % for calculating minipage widths
\usepackage{caption}
% Make caption package work with longtable
\makeatletter
\def\fnum@table{\tablename~\thetable}
\makeatother
\setlength{\emergencystretch}{3em} % prevent overfull lines
\providecommand{\tightlist}{%
  \setlength{\itemsep}{0pt}\setlength{\parskip}{0pt}}
\setcounter{secnumdepth}{-\maxdimen} % remove section numbering
\let\Tiny=\tiny

 \setbeamertemplate{navigation symbols}{} 

% \usetheme{Madrid}
 \usetheme[numbering=none, progressbar=foot]{metropolis}
 \usecolortheme{wolverine}
 \usepackage{color}
 \usepackage{MnSymbol}
% \usepackage{movie15}

\usepackage{amssymb}% http://ctan.org/pkg/amssymb
\usepackage{pifont}% http://ctan.org/pkg/pifont
\newcommand{\cmark}{\ding{51}}%
\newcommand{\xmark}{\ding{55}}%

\DeclareSymbolFont{symbolsC}{U}{txsyc}{m}{n}
\DeclareMathSymbol{\boxright}{\mathrel}{symbolsC}{128}
\DeclareMathAlphabet{\mathpzc}{OT1}{pzc}{m}{it}

 \usepackage{tikz-qtree}
% \usepackage{markdown}
%\RequirePackage{bussproofs}
\RequirePackage[tableaux]{prooftrees}
\usetikzlibrary{arrows.meta}
 \forestset{line numbering, close with = x}
% Allow for easy commas inside trees
\renewcommand{\,}{\text{, }}


\usepackage{tabulary}

\usepackage{open-logic-config}

\setlength{\parskip}{1ex plus 0.5ex minus 0.2ex}

\AtBeginSection[]
{
\begin{frame}
	\Huge{\color{darkblue} \insertsection}
\end{frame}
}

\renewenvironment*{quote}	
	{\list{}{\rightmargin   \leftmargin} \item } 	
	{\endlist }

\definecolor{darkgreen}{rgb}{0,0.7,0}
\definecolor{darkblue}{rgb}{0,0,0.8}

\newcommand{\starttab}{\begin{center}
\vspace{6pt}
\begin{tabular}}

\newcommand{\stoptab}{\end{tabular}
\vspace{6pt}
\end{center}
\noindent}


\newcommand{\sif}{\rightarrow}
\newcommand{\siff}{\leftrightarrow}
\newcommand{\EF}{\end{frame}}


\newcommand{\TreeStart}[1]{
%\end{frame}
\begin{frame}
\begin{center}
\begin{tikzpicture}[scale=#1]
\tikzset{every tree node/.style={align=center,anchor=north}}
%\Tree
}

\newcommand{\TreeEnd}{
\end{tikzpicture}
%\end{center}
}

\newcommand{\DisplayArg}[2]{
\begin{enumerate}
{#1}
\end{enumerate}
\vspace{-6pt}
\hrulefill

%\hspace{14pt} #2
%{\addtolength{\leftskip}{14pt} #2}
\begin{quote}
{\normalfont #2}
\end{quote}
\vspace{12pt}
}

\newenvironment{ProofTree}[1][1]{
\begin{center}
\begin{tikzpicture}[scale=#1]
\tikzset{every tree node/.style={align=center,anchor=south}}
}
{
\end{tikzpicture}
\end{center}
}

\newcommand{\TreeFrame}[2]{
\begin{columns}[c]
\column{0.5\textwidth}
\begin{center}
\begin{prooftree}{}
#1
\end{prooftree}
\end{center}
\column{0.45\textwidth}
%\begin{markdown}
#2
%\end{markdown}
\end{columns}
}

\newcommand{\ScaledTreeFrame}[3]{
\begin{columns}[c]
\column{0.5\textwidth}
\begin{center}
\scalebox{#1}{
\begin{prooftree}{}
#2
\end{prooftree}
}
\end{center}
\column{0.45\textwidth}
%\begin{markdown}
#3
%\end{markdown}
\end{columns}
}

\usepackage[bb=boondox]{mathalfa}
\DeclareMathAlphabet{\mathbx}{U}{BOONDOX-ds}{m}{n}
\SetMathAlphabet{\mathbx}{bold}{U}{BOONDOX-ds}{b}{n}
\DeclareMathAlphabet{\mathbbx} {U}{BOONDOX-ds}{b}{n}


\newenvironment{oltableau}{\center\tableau{}} %wff format={anchor = base west}}}
       {\endtableau\endcenter}
       
\newcommand{\formula}[1]{$#1$}

\usepackage{tabulary}
\usepackage{booktabs}

\def\begincols{\begin{columns}}
\def\begincol{\begin{column}}
\def\endcol{\end{column}}
\def\endcols{\end{columns}}

\usepackage[italic]{mathastext}
\usepackage{nicefrac}

\definecolor{mygreen}{RGB}{0, 100, 0}
\definecolor{mypink2}{RGB}{219, 48, 122}
\definecolor{dodgerblue}{RGB}{30,144,255}

%\def\True{\textcolor{dodgerblue}{\text{T}}}
%\def\False{\textcolor{red}{\text{F}}}

\def\True{\mathbb{T}}
\def\False{\mathbb{F}}

% This is because arguments didn't have enough space after them
\usepackage{etoolbox}
\AfterEndEnvironment{description}{\vspace{9pt}}
\AfterEndEnvironment{oltableau}{\vspace{9pt}}
\BeforeBeginEnvironment{oltableau}{\vspace{9pt}}
\AfterEndEnvironment{center}{\vspace{12pt}}
\BeforeBeginEnvironment{tabular}{\vspace{9pt}}

\setlength\heavyrulewidth{0pt}
\setlength\lightrulewidth{0pt}

%\def\toprule{}
%\def\bottomrule{}
%\def\midrule{}

\setbeamertemplate{caption}{\raggedright\insertcaption}

\ifluatex
  \usepackage{selnolig}  % disable illegal ligatures
\fi

\title{305 Lecture 9.4 - Maximise Expected Utility}
\author{Brian Weatherson}
\date{}

\begin{document}
\frame{\titlepage}

\begin{frame}{Plan}
\protect\hypertarget{plan}{}
\begin{itemize}
\tightlist
\item
  In this lecture we'll talk about the core rule of formal decision
  theory: Maximise Expected Utility.
\end{itemize}
\end{frame}

\begin{frame}{Associated Reading}
\protect\hypertarget{associated-reading}{}
Odds and Ends, Chapter 12
\end{frame}

\begin{frame}{The Rule}
\protect\hypertarget{the-rule}{}
\begin{itemize}
\tightlist
\item
  The orthodox view in modern decision theory is that the right decision
  is the one that maximises the expected utility of your choice.
\item
  The rational decision maximises not actual utility, but expected
  utility.
\end{itemize}
\end{frame}

\begin{frame}{Airline Example (Several Variants)}
\protect\hypertarget{airline-example-several-variants}{}
It turns out that what to do turns on three factors.

\begin{enumerate}
\tightlist
\item
  How likely bad weather is.
\item
  How much you have to gain by flying the cheaper airline in good
  weather.
\item
  How much you have to lose by flying the cheaper airline in bad
  weather.
\end{enumerate}

It is plausible, I think, that these three should matter.
\end{frame}

\begin{frame}{Version One}
\protect\hypertarget{version-one}{}
Lots to gain, relatively little to lose, high probability of good
weather.

\begin{longtable}[]{@{}
  >{\raggedright\arraybackslash}p{(\columnwidth - 4\tabcolsep) * \real{0.28}}
  >{\centering\arraybackslash}p{(\columnwidth - 4\tabcolsep) * \real{0.25}}
  >{\centering\arraybackslash}p{(\columnwidth - 4\tabcolsep) * \real{0.21}}@{}}
\toprule
& Good weather Pr=0.8 & Bad Weather Pr=0.2 \\ \addlinespace
\midrule
\endhead
Fly Cheap Airline & 10 & 0 \\ \addlinespace
Fly Good Airline & 6 & 5 \\ \addlinespace
\bottomrule
\end{longtable}
\end{frame}

\begin{frame}{Utility Calculation}
\protect\hypertarget{utility-calculation}{}
We can work out the expected utility of each action fairly easily.

\begin{align*}
Exp(\text{Cheap Airline}) &= 0.8 \times 10 + 0.2 \times 0 \\
 &= 8 + 0 \\
 &= 8 \\
Exp(\text{Reliable Airline}) &= 0.8 \times 6 + 0.2 \times 5 \\
 &= 4.8 + 1 \\
 &= 5.8 
\end{align*}

\begin{itemize}
\tightlist
\item
  So the cheap airline has an expected utility of 8, the reliable
  airline has an expected utility of 5.8.
\item
  The cheap airline has a higher expected utility, so it is what should
  be taken.
\end{itemize}
\end{frame}

\begin{frame}{Other versions}
\protect\hypertarget{other-versions}{}
\begin{itemize}
\tightlist
\item
  We'll now look at three changes to the example.
\item
  Each change should intuitively change the correct decision, and we'll
  see that the maximise expected utility rule does change in each case.
\end{itemize}
\end{frame}

\begin{frame}{More Downside if Bad Weather}
\protect\hypertarget{more-downside-if-bad-weather}{}
\begin{longtable}[]{@{}
  >{\raggedright\arraybackslash}p{(\columnwidth - 4\tabcolsep) * \real{0.28}}
  >{\centering\arraybackslash}p{(\columnwidth - 4\tabcolsep) * \real{0.25}}
  >{\centering\arraybackslash}p{(\columnwidth - 4\tabcolsep) * \real{0.21}}@{}}
\toprule
& Good weather Pr=0.8 & Bad Weather Pr=0.2 \\ \addlinespace
\midrule
\endhead
Fly Cheap Airline & 10 & -20 \\ \addlinespace
Fly Good Airline & 6 & 5 \\ \addlinespace
\bottomrule
\end{longtable}
\end{frame}

\begin{frame}{Utility Calculations}
\protect\hypertarget{utility-calculations}{}
Here are the new expected utility considerations.

\begin{align*}
Exp(\text{Cheap Airline}) &= 0.8 \times 10 + 0.2 \times -20 \\
 &= 8 + (-4) \\
 &= 4 \\
Exp(\text{Reliable Airline}) &= 0.8 \times 6 + 0.2 \times 5 \\
 &= 4.8 + 1 \\
 &= 5.8 
\end{align*}

\begin{itemize}
\tightlist
\item
  Now the expected utility of catching the reliable airline is higher
  than the expected utility of catching the cheap airline.
\item
  So it is better to catch the reliable airline.
\end{itemize}
\end{frame}

\begin{frame}{Less to Gain by Cheaper Airline}
\protect\hypertarget{less-to-gain-by-cheaper-airline}{}
\begin{longtable}[]{@{}
  >{\raggedright\arraybackslash}p{(\columnwidth - 4\tabcolsep) * \real{0.28}}
  >{\centering\arraybackslash}p{(\columnwidth - 4\tabcolsep) * \real{0.25}}
  >{\centering\arraybackslash}p{(\columnwidth - 4\tabcolsep) * \real{0.21}}@{}}
\toprule
& Good weather Pr=0.8 & Bad Weather Pr=0.2 \\ \addlinespace
\midrule
\endhead
Fly Cheap Airline & 10 & 0 \\ \addlinespace
Fly Good Airline & 9 & 8 \\ \addlinespace
\bottomrule
\end{longtable}
\end{frame}

\begin{frame}{Utility Calculations}
\protect\hypertarget{utility-calculations-1}{}
Here are the revised expected utility considerations.

\begin{align*}
Exp(\text{Cheap Airline}) &= 0.8 \times 10 + 0.2 \times 0 \\
 &= 8 + 0 \\
 &= 8 \\
Exp(\text{Reliable Airline}) &= 0.8 \times 9 + 0.2 \times 8 \\
 &= 7.2 + 1.6 \\
 &= 8.8 
\end{align*}

And again this is enough to make the reliable airline the better choice.
\end{frame}

\begin{frame}{Bad Weather More Likely}
\protect\hypertarget{bad-weather-more-likely}{}
\begin{longtable}[]{@{}
  >{\raggedright\arraybackslash}p{(\columnwidth - 4\tabcolsep) * \real{0.28}}
  >{\centering\arraybackslash}p{(\columnwidth - 4\tabcolsep) * \real{0.25}}
  >{\centering\arraybackslash}p{(\columnwidth - 4\tabcolsep) * \real{0.21}}@{}}
\toprule
& Good weather Pr=0.3 & Bad Weather Pr=0.7 \\ \addlinespace
\midrule
\endhead
Fly Cheap Airline & 10 & 0 \\ \addlinespace
Fly Good Airline & 6 & 5 \\ \addlinespace
\bottomrule
\end{longtable}
\end{frame}

\begin{frame}{Utility Calculations}
\protect\hypertarget{utility-calculations-2}{}
We can work out the expected utility of each action fairly easily.

\begin{align*}
Exp(\text{Cheap Airline}) &= 0.3 \times 10 + 0.7 \times 0 \\
 &= 3 + 0 \\
 &= 3 \\
Exp(\text{Reliable Airline}) &= 0.3 \times 6 + 0.7 \times 5 \\
 &= 1.8 + 3.5 \\
 &= 5.3 
\end{align*}
\end{frame}

\begin{frame}{Summarising the Cases}
\protect\hypertarget{summarising-the-cases}{}
We've looked at four versions of the same case. In each case the
ordering of the outcomes, from best to worst, was:

\begin{enumerate}
\tightlist
\item
  Cheap airline and good weather
\item
  Reliable airline and good weather
\item
  Reliable airline and bad weather
\item
  Cheap airline and bad weather
\end{enumerate}

But this doesn't settle what to do; these three factors all matter.
\end{frame}

\begin{frame}{For Next Time}
\protect\hypertarget{for-next-time}{}
\begin{itemize}
\tightlist
\item
  We will look at the relationship between utility and money.
\end{itemize}
\end{frame}

\end{document}
