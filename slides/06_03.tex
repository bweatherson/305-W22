% Options for packages loaded elsewhere
\PassOptionsToPackage{unicode}{hyperref}
\PassOptionsToPackage{hyphens}{url}
%
\documentclass[
  ignorenonframetext,
]{beamer}
\title{305 Lecture 6.3 - Probability Trees}
\author{Brian Weatherson}
\date{}

\usepackage{pgfpages}
\setbeamertemplate{caption}[numbered]
\setbeamertemplate{caption label separator}{: }
\setbeamercolor{caption name}{fg=normal text.fg}
\beamertemplatenavigationsymbolsempty
% Prevent slide breaks in the middle of a paragraph
\widowpenalties 1 10000
\raggedbottom
\setbeamertemplate{part page}{
  \centering
  \begin{beamercolorbox}[sep=16pt,center]{part title}
    \usebeamerfont{part title}\insertpart\par
  \end{beamercolorbox}
}
\setbeamertemplate{section page}{
  \centering
  \begin{beamercolorbox}[sep=12pt,center]{part title}
    \usebeamerfont{section title}\insertsection\par
  \end{beamercolorbox}
}
\setbeamertemplate{subsection page}{
  \centering
  \begin{beamercolorbox}[sep=8pt,center]{part title}
    \usebeamerfont{subsection title}\insertsubsection\par
  \end{beamercolorbox}
}
\AtBeginPart{
  \frame{\partpage}
}
\AtBeginSection{
  \ifbibliography
  \else
    \frame{\sectionpage}
  \fi
}
\AtBeginSubsection{
  \frame{\subsectionpage}
}
\usepackage{amsmath,amssymb}
\usepackage{lmodern}
\usepackage{iftex}
\ifPDFTeX
  \usepackage[T1]{fontenc}
  \usepackage[utf8]{inputenc}
  \usepackage{textcomp} % provide euro and other symbols
\else % if luatex or xetex
  \usepackage{unicode-math}
  \defaultfontfeatures{Scale=MatchLowercase}
  \defaultfontfeatures[\rmfamily]{Ligatures=TeX,Scale=1}
  \setmainfont[BoldFont = SF Pro Rounded Semibold]{SF Pro Rounded}
  \setmathfont[]{STIX Two Math}
\fi
\usefonttheme{serif} % use mainfont rather than sansfont for slide text
% Use upquote if available, for straight quotes in verbatim environments
\IfFileExists{upquote.sty}{\usepackage{upquote}}{}
\IfFileExists{microtype.sty}{% use microtype if available
  \usepackage[]{microtype}
  \UseMicrotypeSet[protrusion]{basicmath} % disable protrusion for tt fonts
}{}
\makeatletter
\@ifundefined{KOMAClassName}{% if non-KOMA class
  \IfFileExists{parskip.sty}{%
    \usepackage{parskip}
  }{% else
    \setlength{\parindent}{0pt}
    \setlength{\parskip}{6pt plus 2pt minus 1pt}}
}{% if KOMA class
  \KOMAoptions{parskip=half}}
\makeatother
\usepackage{xcolor}
\IfFileExists{xurl.sty}{\usepackage{xurl}}{} % add URL line breaks if available
\IfFileExists{bookmark.sty}{\usepackage{bookmark}}{\usepackage{hyperref}}
\hypersetup{
  pdftitle={305 Lecture 6.3 - Probability Trees},
  pdfauthor={Brian Weatherson},
  hidelinks,
  pdfcreator={LaTeX via pandoc}}
\urlstyle{same} % disable monospaced font for URLs
\newif\ifbibliography
\usepackage{longtable,booktabs,array}
\usepackage{calc} % for calculating minipage widths
\usepackage{caption}
% Make caption package work with longtable
\makeatletter
\def\fnum@table{\tablename~\thetable}
\makeatother
\setlength{\emergencystretch}{3em} % prevent overfull lines
\providecommand{\tightlist}{%
  \setlength{\itemsep}{0pt}\setlength{\parskip}{0pt}}
\setcounter{secnumdepth}{-\maxdimen} % remove section numbering
\let\Tiny=\tiny

 \setbeamertemplate{navigation symbols}{} 

% \usetheme{Madrid}
 \usetheme[numbering=none, progressbar=foot]{metropolis}
 \usecolortheme{wolverine}
 \usepackage{color}
 \usepackage{MnSymbol}
% \usepackage{movie15}

\usepackage{amssymb}% http://ctan.org/pkg/amssymb
\usepackage{pifont}% http://ctan.org/pkg/pifont
\newcommand{\cmark}{\ding{51}}%
\newcommand{\xmark}{\ding{55}}%

\DeclareSymbolFont{symbolsC}{U}{txsyc}{m}{n}
\DeclareMathSymbol{\boxright}{\mathrel}{symbolsC}{128}
\DeclareMathAlphabet{\mathpzc}{OT1}{pzc}{m}{it}

\usepackage{tikz-qtree}
% \usepackage{markdown}
%\RequirePackage{bussproofs}
\usetikzlibrary{arrows.meta}
\RequirePackage[tableaux]{prooftrees}
\forestset{line numbering, close with = x}
% Allow for easy commas inside trees
\renewcommand{\,}{\text{, }}


\usepackage{tabulary}

\usepackage{open-logic-config}

\setlength{\parskip}{1ex plus 0.5ex minus 0.2ex}

\AtBeginSection[]
{
\begin{frame}
	\Huge{\color{darkblue} \insertsection}
\end{frame}
}

\renewenvironment*{quote}	
	{\list{}{\rightmargin   \leftmargin} \item } 	
	{\endlist }

\definecolor{darkgreen}{rgb}{0,0.7,0}
\definecolor{darkblue}{rgb}{0,0,0.8}

\newcommand{\starttab}{\begin{center}
\vspace{6pt}
\begin{tabular}}

\newcommand{\stoptab}{\end{tabular}
\vspace{6pt}
\end{center}
\noindent}


\newcommand{\sif}{\rightarrow}
\newcommand{\siff}{\leftrightarrow}
\newcommand{\EF}{\end{frame}}


\newcommand{\TreeStart}[1]{
%\end{frame}
\begin{frame}
\begin{center}
\begin{tikzpicture}[scale=#1]
\tikzset{every tree node/.style={align=center,anchor=north}}
%\Tree
}

\newcommand{\TreeEnd}{
\end{tikzpicture}
%\end{center}
}

\newcommand{\DisplayArg}[2]{
\begin{enumerate}
{#1}
\end{enumerate}
\vspace{-6pt}
\hrulefill

%\hspace{14pt} #2
%{\addtolength{\leftskip}{14pt} #2}
\begin{quote}
{\normalfont #2}
\end{quote}
\vspace{12pt}
}

\newenvironment{ProofTree}[1][1]{
\begin{center}
\begin{tikzpicture}[scale=#1]
\tikzset{every tree node/.style={align=center,anchor=south}}
}
{
\end{tikzpicture}
\end{center}
}

\newcommand{\TreeFrame}[2]{
\begin{columns}[c]
\column{0.5\textwidth}
\begin{center}
\begin{prooftree}{}
#1
\end{prooftree}
\end{center}
\column{0.45\textwidth}
%\begin{markdown}
#2
%\end{markdown}
\end{columns}
}

\newcommand{\ScaledTreeFrame}[3]{
\begin{columns}[c]
\column{0.5\textwidth}
\begin{center}
\scalebox{#1}{
\begin{prooftree}{}
#2
\end{prooftree}
}
\end{center}
\column{0.45\textwidth}
%\begin{markdown}
#3
%\end{markdown}
\end{columns}
}

\usepackage[bb=boondox]{mathalfa}
\DeclareMathAlphabet{\mathbx}{U}{BOONDOX-ds}{m}{n}
\SetMathAlphabet{\mathbx}{bold}{U}{BOONDOX-ds}{b}{n}
\DeclareMathAlphabet{\mathbbx} {U}{BOONDOX-ds}{b}{n}


\newenvironment{oltableau}{\center\tableau{}} %wff format={anchor = base west}}}
       {\endtableau\endcenter}
       
\newcommand{\formula}[1]{$#1$}

\usepackage{tabulary}
\usepackage{booktabs}

\def\begincols{\begin{columns}}
\def\begincol{\begin{column}}
\def\endcol{\end{column}}
\def\endcols{\end{columns}}

\usepackage[italic]{mathastext}
\usepackage{nicefrac}

\definecolor{mygreen}{RGB}{0, 100, 0}
\definecolor{mypink2}{RGB}{219, 48, 122}
\definecolor{dodgerblue}{RGB}{30,144,255}

%\def\True{\textcolor{dodgerblue}{\text{T}}}
%\def\False{\textcolor{red}{\text{F}}}

\def\True{\mathbb{T}}
\def\False{\mathbb{F}}

% This is because arguments didn't have enough space after them
\usepackage{etoolbox}
\AfterEndEnvironment{description}{\vspace{9pt}}
\AfterEndEnvironment{oltableau}{\vspace{9pt}}
\BeforeBeginEnvironment{oltableau}{\vspace{9pt}}
\AfterEndEnvironment{center}{\vspace{12pt}}
\BeforeBeginEnvironment{tabular}{\vspace{9pt}}

\setlength\heavyrulewidth{0pt}
\setlength\lightrulewidth{0pt}

%\def\toprule{}
%\def\bottomrule{}
%\def\midrule{}

\setbeamertemplate{caption}{\raggedright\insertcaption}

\setlength\lightrulewidth{0.3pt}
\ifLuaTeX
  \usepackage{selnolig}  % disable illegal ligatures
\fi

\begin{document}
\frame{\titlepage}

\begin{frame}{Plan}
\protect\hypertarget{plan}{}
To introduce the notion of probability trees, and show how they can be
used for estimating probabilities.
\end{frame}

\begin{frame}{Reading}
\protect\hypertarget{reading}{}
\emph{Odds and Ends}, sections 1.1 and 6.2.
\end{frame}

\begin{frame}{Trees}
\protect\hypertarget{trees}{}
\begin{itemize}
\tightlist
\item
  Often, we can't just write down numbers for the possibilities.
\item
  But we can write down, or at least make reasonable guesses about,
  numbers for certain events.
\item
  And we can think about how things are likely to go given those events
  happen.
\item
  This is the tree structure that is used in \emph{Odds and Ends}.
\end{itemize}
\end{frame}

\begin{frame}{Flight Forecasting}
\protect\hypertarget{flight-forecasting}{}
\begin{itemize}
\tightlist
\item
  So let's say you're flying from Detroit to Salt Lake City via Denver
  (which apparently is a common route), next Monday.
\item
  And you want to know how likely it is you'll get there on time.
\item
  The weather looks fine here in Michigan, but changing planes in Denver
  during winter can be a bit of a lottery.
\item
  You've got a tight connection, so if the first flight is delayed,
  you'll probably be delayed.
\item
  And that's not the only thing that could go wrong with planes.
\item
  But it's hard to put into numbers how it affects things.
\end{itemize}
\end{frame}

\begin{frame}{One Method}
\protect\hypertarget{one-method}{}
\begin{itemize}
\tightlist
\item
  Divide up the state space.
\item
  Work out the probability of being in one or other part of the space.
\item
  Work out the probability of being delayed given you are in one or
  other part of the space.
\item
  Add things up.
\end{itemize}
\end{frame}

\begin{frame}{Nothing is Reliable}
\protect\hypertarget{nothing-is-reliable}{}
\begin{itemize}[<+->]
\tightlist
\item
  There are a lot of ways to do this.
\item
  You could divide up the space by how many flights are delayed next
  Monday.
\item
  Or you could divide up the space by how many airline employees show up
  for work on Monday.
\item
  Or, and let's work with this one, you could divide it up by what the
  weather is like in Denver. The advantage of this is that we can get an
  independent assessment of it without knowing a ton about the state of
  the aviation industry.
\end{itemize}
\end{frame}

\begin{frame}{Three States}
\protect\hypertarget{three-states}{}
\begin{enumerate}
\tightlist
\item
  Blizzard.
\item
  Bad weather, but not a blizzard.
\item
  Clear skies.
\end{enumerate}
\end{frame}

\begin{frame}{Two Step Process}
\protect\hypertarget{two-step-process}{}
\begin{enumerate}
\tightlist
\item
  Work out (or at least estimate) probability of each state.
\item
  Work out probability of being delayed within each of those states.
\end{enumerate}

This will involve a lot of guesswork - do not make travel planning
decisions on the basis of the numbers I'm about to use because they are
really pulled out of the air - but it's a very helpful heuristic to at
least approximate the reality.
\end{frame}

\begin{frame}{Probabilities of States}
\protect\hypertarget{probabilities-of-states}{}
Let's say the probabilities look like this.

\begin{itemize}
\tightlist
\item
  Blizzard - 10\%, or 0.1.
\item
  Other bad weather - 60\%, or 0.6.
\item
  Good weather - 30\%, or 0.3.
\end{itemize}
\end{frame}

\begin{frame}{Flight Probabilities}
\protect\hypertarget{flight-probabilities}{}
Then we want to work out how likely it is that you get there on time in
each of these three possibilities.
\end{frame}

\begin{frame}{Blizzard}
\protect\hypertarget{blizzard}{}
\begin{itemize}
\tightlist
\item
  This one's easy - you're going to be late.
\item
  Your plane won't land, or the plane you're connecting to won't land,
  or your plane won't take off.
\item
  Never say never, but the probability that you'll be late is 1.
\end{itemize}
\end{frame}

\begin{frame}{Bad Weather}
\protect\hypertarget{bad-weather}{}
\begin{itemize}
\tightlist
\item
  It snows a lot in Denver, and yet planes get through.
\item
  But still, things start going wrong when there's bad weather, and
  you've got two points where things can go wrong.
\item
  So let's say it's a 25\% chance you're delayed in this situation.
\item
  Across the year, just over 90\% of flights from Detroit to Denver are
  on time, and just under 90\% of flights from Denver to Salt Lake City
  are on time, but things are worse in winter - even without a blizzard
  - that's why the 25\%.
\end{itemize}
\end{frame}

\begin{frame}{Good Weather}
\protect\hypertarget{good-weather}{}
\begin{itemize}
\tightlist
\item
  Now it would have to be some other kind of bad luck to be late.
\item
  That happens, but let's put it at 10\% likelihood.
\item
  If you wake up on Monday and see it's sunny in Denver, it seems you
  should be at least 90\% confident you'll get to Salt Lake on time.
\end{itemize}
\end{frame}

\begin{frame}{The Giant Table}
\protect\hypertarget{the-giant-table}{}
\begin{longtable}[]{@{}lcc@{}}
\toprule
& On Time & Late \\
\midrule
\endhead
Blizzard & \(0.1 \times 0.00 = 0.00\) & \(0.1 \times 1.00 = 0.10\) \\
Bad Weather & \(0.6 \times 0.75 = 0.45\) & \(0.6 \times 0.25 = 0.15\) \\
Good weather & \(0.3 \times 0.90 = 0.27\) &
\(0.3 \times 0.10 = 0.03\) \\
\bottomrule
\end{longtable}

So the probability that you arrive on time (given these literally made
up assumptions) is \(0.00 + 0.45 + 0.27 = 0.72\), or 72\%, and the
probability that you're delayed is 28\%.
\end{frame}

\begin{frame}{Probabilities and Errors}
\protect\hypertarget{probabilities-and-errors}{}
\begin{itemize}
\tightlist
\item
  The error bars on that calculation are massive.
\item
  But it's a kind of sanity check on how you think things are going.
\item
  Especially in situations where only a handful of paths lead to a
  salient result (like in playoffs in sports, or when thinking about the
  likelihood of a particular challenger winning), doing the tree even
  with favorable numbers can give you a conservative estimate of some
  probability.
\end{itemize}
\end{frame}

\begin{frame}{Three Cases Where This is More Precise}
\protect\hypertarget{three-cases-where-this-is-more-precise}{}
\begin{enumerate}
\tightlist
\item
  Probabilities are stipulated
\item
  Probabilities are due to well understood chance processes (like
  gambling devices)
\item
  Probabilities are derived from very large data sets
\end{enumerate}
\end{frame}

\begin{frame}{For Next Time}
\protect\hypertarget{for-next-time}{}
\begin{itemize}
\tightlist
\item
  We will look at an example where the probabilities are indeed
  stipulated.
\end{itemize}
\end{frame}

\end{document}
