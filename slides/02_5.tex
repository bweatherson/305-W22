% Options for packages loaded elsewhere
\PassOptionsToPackage{unicode}{hyperref}
\PassOptionsToPackage{hyphens}{url}
%
\documentclass[
  ignorenonframetext,
]{beamer}
\usepackage{pgfpages}
\setbeamertemplate{caption}[numbered]
\setbeamertemplate{caption label separator}{: }
\setbeamercolor{caption name}{fg=normal text.fg}
\beamertemplatenavigationsymbolsempty
% Prevent slide breaks in the middle of a paragraph
\widowpenalties 1 10000
\raggedbottom
\setbeamertemplate{part page}{
  \centering
  \begin{beamercolorbox}[sep=16pt,center]{part title}
    \usebeamerfont{part title}\insertpart\par
  \end{beamercolorbox}
}
\setbeamertemplate{section page}{
  \centering
  \begin{beamercolorbox}[sep=12pt,center]{part title}
    \usebeamerfont{section title}\insertsection\par
  \end{beamercolorbox}
}
\setbeamertemplate{subsection page}{
  \centering
  \begin{beamercolorbox}[sep=8pt,center]{part title}
    \usebeamerfont{subsection title}\insertsubsection\par
  \end{beamercolorbox}
}
\AtBeginPart{
  \frame{\partpage}
}
\AtBeginSection{
  \ifbibliography
  \else
    \frame{\sectionpage}
  \fi
}
\AtBeginSubsection{
  \frame{\subsectionpage}
}
\usepackage{amsmath,amssymb}
\usepackage{lmodern}
\usepackage{ifxetex,ifluatex}
\ifnum 0\ifxetex 1\fi\ifluatex 1\fi=0 % if pdftex
  \usepackage[T1]{fontenc}
  \usepackage[utf8]{inputenc}
  \usepackage{textcomp} % provide euro and other symbols
\else % if luatex or xetex
  \usepackage{unicode-math}
  \defaultfontfeatures{Scale=MatchLowercase}
  \defaultfontfeatures[\rmfamily]{Ligatures=TeX,Scale=1}
  \setmainfont[BoldFont = SF Pro Rounded Semibold]{SF Pro Rounded}
  \setmathfont[]{STIX Two Math}
\fi
\usefonttheme{serif} % use mainfont rather than sansfont for slide text
% Use upquote if available, for straight quotes in verbatim environments
\IfFileExists{upquote.sty}{\usepackage{upquote}}{}
\IfFileExists{microtype.sty}{% use microtype if available
  \usepackage[]{microtype}
  \UseMicrotypeSet[protrusion]{basicmath} % disable protrusion for tt fonts
}{}
\makeatletter
\@ifundefined{KOMAClassName}{% if non-KOMA class
  \IfFileExists{parskip.sty}{%
    \usepackage{parskip}
  }{% else
    \setlength{\parindent}{0pt}
    \setlength{\parskip}{6pt plus 2pt minus 1pt}}
}{% if KOMA class
  \KOMAoptions{parskip=half}}
\makeatother
\usepackage{xcolor}
\IfFileExists{xurl.sty}{\usepackage{xurl}}{} % add URL line breaks if available
\IfFileExists{bookmark.sty}{\usepackage{bookmark}}{\usepackage{hyperref}}
\hypersetup{
  pdftitle={305 Lecture 2.5 - Basic Truth Tables},
  pdfauthor={Brian Weatherson},
  hidelinks,
  pdfcreator={LaTeX via pandoc}}
\urlstyle{same} % disable monospaced font for URLs
\newif\ifbibliography
\setlength{\emergencystretch}{3em} % prevent overfull lines
\providecommand{\tightlist}{%
  \setlength{\itemsep}{0pt}\setlength{\parskip}{0pt}}
\setcounter{secnumdepth}{-\maxdimen} % remove section numbering
\let\Tiny=\tiny

 \setbeamertemplate{navigation symbols}{} 

% \usetheme{Madrid}
 \usetheme[numbering=none, progressbar=foot]{metropolis}
 \usecolortheme{wolverine}
 \usepackage{color}
 \usepackage{MnSymbol}
% \usepackage{movie15}

\usepackage{amssymb}% http://ctan.org/pkg/amssymb
\usepackage{pifont}% http://ctan.org/pkg/pifont
\newcommand{\cmark}{\ding{51}}%
\newcommand{\xmark}{\ding{55}}%

\DeclareSymbolFont{symbolsC}{U}{txsyc}{m}{n}
\DeclareMathSymbol{\boxright}{\mathrel}{symbolsC}{128}
\DeclareMathAlphabet{\mathpzc}{OT1}{pzc}{m}{it}

 \usepackage{tikz-qtree}
% \usepackage{markdown}
%\RequirePackage{bussproofs}
\RequirePackage[tableaux]{prooftrees}
\usetikzlibrary{arrows.meta}
 \forestset{line numbering, close with = x}
% Allow for easy commas inside trees
\renewcommand{\,}{\text{, }}


\usepackage{tabulary}

\usepackage{open-logic-config}

\setlength{\parskip}{1ex plus 0.5ex minus 0.2ex}

\AtBeginSection[]
{
\begin{frame}
	\Huge{\color{darkblue} \insertsection}
\end{frame}
}

\renewenvironment*{quote}	
	{\list{}{\rightmargin   \leftmargin} \item } 	
	{\endlist }

\definecolor{darkgreen}{rgb}{0,0.7,0}
\definecolor{darkblue}{rgb}{0,0,0.8}

\newcommand{\starttab}{\begin{center}
\vspace{6pt}
\begin{tabular}}

\newcommand{\stoptab}{\end{tabular}
\vspace{6pt}
\end{center}
\noindent}


\newcommand{\sif}{\rightarrow}
\newcommand{\siff}{\leftrightarrow}
\newcommand{\EF}{\end{frame}}


\newcommand{\TreeStart}[1]{
%\end{frame}
\begin{frame}
\begin{center}
\begin{tikzpicture}[scale=#1]
\tikzset{every tree node/.style={align=center,anchor=north}}
%\Tree
}

\newcommand{\TreeEnd}{
\end{tikzpicture}
%\end{center}
}

\newcommand{\DisplayArg}[2]{
\begin{enumerate}
{#1}
\end{enumerate}
\vspace{-6pt}
\hrulefill

%\hspace{14pt} #2
%{\addtolength{\leftskip}{14pt} #2}
\begin{quote}
{\normalfont #2}
\end{quote}
\vspace{12pt}
}

\newenvironment{ProofTree}[1][1]{
\begin{center}
\begin{tikzpicture}[scale=#1]
\tikzset{every tree node/.style={align=center,anchor=south}}
}
{
\end{tikzpicture}
\end{center}
}

\newcommand{\TreeFrame}[2]{
\begin{columns}[c]
\column{0.5\textwidth}
\begin{center}
\begin{prooftree}{}
#1
\end{prooftree}
\end{center}
\column{0.45\textwidth}
%\begin{markdown}
#2
%\end{markdown}
\end{columns}
}

\newcommand{\ScaledTreeFrame}[3]{
\begin{columns}[c]
\column{0.5\textwidth}
\begin{center}
\scalebox{#1}{
\begin{prooftree}{}
#2
\end{prooftree}
}
\end{center}
\column{0.45\textwidth}
%\begin{markdown}
#3
%\end{markdown}
\end{columns}
}

\usepackage[bb=boondox]{mathalfa}
\DeclareMathAlphabet{\mathbx}{U}{BOONDOX-ds}{m}{n}
\SetMathAlphabet{\mathbx}{bold}{U}{BOONDOX-ds}{b}{n}
\DeclareMathAlphabet{\mathbbx} {U}{BOONDOX-ds}{b}{n}


\newenvironment{oltableau}{\center\tableau{}} %wff format={anchor = base west}}}
       {\endtableau\endcenter}
       
\newcommand{\formula}[1]{$#1$}

\usepackage{tabulary}
\usepackage{booktabs}

\def\begincols{\begin{columns}}
\def\begincol{\begin{column}}
\def\endcol{\end{column}}
\def\endcols{\end{columns}}

\usepackage[italic]{mathastext}
\usepackage{nicefrac}

\definecolor{mygreen}{RGB}{0, 100, 0}
\definecolor{mypink2}{RGB}{219, 48, 122}
\definecolor{dodgerblue}{RGB}{30,144,255}

%\def\True{\textcolor{dodgerblue}{\text{T}}}
%\def\False{\textcolor{red}{\text{F}}}

\def\True{\mathbb{T}}
\def\False{\mathbb{F}}

% This is because arguments didn't have enough space after them
\usepackage{etoolbox}
\AfterEndEnvironment{description}{\vspace{9pt}}
\AfterEndEnvironment{oltableau}{\vspace{9pt}}
\BeforeBeginEnvironment{oltableau}{\vspace{9pt}}
\AfterEndEnvironment{center}{\vspace{12pt}}
\BeforeBeginEnvironment{tabular}{\vspace{9pt}}

\setlength\heavyrulewidth{0pt}
\setlength\lightrulewidth{0pt}

%\def\toprule{}
%\def\bottomrule{}
%\def\midrule{}

\setbeamertemplate{caption}{\raggedright\insertcaption}

\ifluatex
  \usepackage{selnolig}  % disable illegal ligatures
\fi

\title{305 Lecture 2.5 - Basic Truth Tables}
\author{Brian Weatherson}
\date{}

\begin{document}
\frame{\titlepage}

\begin{frame}{Plan}
\protect\hypertarget{plan}{}
This lecture is the truth tables for the basic connectives.
\end{frame}

\begin{frame}{Associated Reading}
\protect\hypertarget{associated-reading}{}
\begin{itemize}
\tightlist
\item
  We're still working through forall x chapters 9-11.
\item
  This is primarily about chapter 9.
\item
  We're not going to cover biconditionals here (or elsewhere in this
  course).
\end{itemize}
\end{frame}

\begin{frame}{Four Main Connectives}
\protect\hypertarget{four-main-connectives}{}
\begin{itemize}
\tightlist
\item
  Building truth tables requires, unfortunately, a small amount of
  memorization.
\item
  In particular, you just have to memorize the truth tables for each of
  the connectives.
\item
  Equally unfortunately, justifying yourself using truth tables requires
  justifying these basic tables.
\item
  And as we'll see, that's not trivial.
\item
  But that's for much down the line - let's learn how to use these
  first, then we'll get to justifying them.
\end{itemize}
\end{frame}

\begin{frame}{Negation Table}
\protect\hypertarget{negation-table}{}
\begin{center}
\begin{tabular}{@{ }c | c@{ }@{ }c}
A & $\neg$ & A\\
\hline 
$\True$ & \textcolor{red}{$\False$} & $\True$\\
$\False$ & \textcolor{red}{$\True$} & $\False$\\
\end{tabular}

\end{center}

You should read it as saying that if \(A\) is \(\True\) then \(\neg A\)
is \(\False\), and if \(A\) is \(\False\), then \(\neg A\) is \(\True\).
\end{frame}

\begin{frame}{The Conjunction Table}
\protect\hypertarget{the-conjunction-table}{}
\begin{center}
\begin{tabular}{@{ }c@{ }@{ }c | c@{ }@{ }c@{ }@{ }c@{ }@{ }c@{ }@{ }c}
A & B &  & A & $\wedge$ & B & \\
\hline 
$\True$ & $\True$ &  & $\True$ & \textcolor{red}{$\True$} & $\True$ & \\
$\True$ & $\False$ &  & $\True$ & \textcolor{red}{$\False$} & $\False$ & \\
$\False$ & $\True$ &  & $\False$ & \textcolor{red}{$\False$} & $\True$ & \\
$\False$ & $\False$ &  & $\False$ & \textcolor{red}{$\False$} & $\False$ & \\
\end{tabular}
\end{center}
\end{frame}

\begin{frame}{Conjunction in Words}
\protect\hypertarget{conjunction-in-words}{}
\begin{itemize}
\tightlist
\item
  A conjunction is \(\True\) if both conjuncts are \(\True\), and is
  \(\False\) otherwise.
\end{itemize}
\end{frame}

\begin{frame}{The Disjunction Table}
\protect\hypertarget{the-disjunction-table}{}
\begin{center}
\begin{tabular}{@{ }c@{ }@{ }c | c@{ }@{ }c@{ }@{ }c@{ }@{ }c@{ }@{ }c}
A & B &  & A & $\vee$ & B & \\
\hline 
$\True$ & $\True$ &  & $\True$ & \textcolor{red}{$\True$} & $\True$ & \\
$\True$ & $\False$ &  & $\True$ & \textcolor{red}{$\True$} & $\False$ & \\
$\False$ & $\True$ &  & $\False$ & \textcolor{red}{$\True$} & $\True$ & \\
$\False$ & $\False$ &  & $\False$ & \textcolor{red}{$\False$} & $\False$ & \\
\end{tabular}
\end{center}
\end{frame}

\begin{frame}{Disjunction in Words}
\protect\hypertarget{disjunction-in-words}{}
\begin{itemize}
\tightlist
\item
  A disjunction is \(\True\) if either disjunct is \(\True\), and is
  \(\False\) otherwise.
\end{itemize}
\end{frame}

\begin{frame}{The Conditional Table}
\protect\hypertarget{the-conditional-table}{}
\begin{center}
\begin{tabular}{@{ }c@{ }@{ }c | c@{ }@{ }c@{ }@{ }c@{ }@{ }c@{ }@{ }c}
A & B &  & A & $\rightarrow$ & B & \\
\hline 
$\True$ & $\True$ &  & $\True$ & \textcolor{red}{$\True$} & $\True$ & \\
$\True$ & $\False$ &  & $\True$ & \textcolor{red}{$\False$} & $\False$ & \\
$\False$ & $\True$ &  & $\False$ & \textcolor{red}{$\True$} & $\True$ & \\
$\False$ & $\False$ &  & $\False$ & \textcolor{red}{$\True$} & $\False$ & \\
\end{tabular}
\end{center}
\end{frame}

\begin{frame}{Material Implication}
\protect\hypertarget{material-implication}{}
Note that these three sentences have exactly the same table.

\begin{center}

\begin{tabular}{@{ }c@{ }@{ }c | c@{ }@{ }c@{ }@{ }c@{ }@{ }c@{ }@{ }c | c@{ }@{ }c@{ }@{ }c@{ }@{ }c@{ }@{ }c@{ }@{ }c | c@{ }@{}c@{}@{ }c@{ }@{ }c@{ }@{ }c@{ }@{ }c@{ }@{}c@{ }}
A & B &  & A & $\rightarrow$ & B &  &  & $\neg$ & A & $\vee$ & B &  & $\neg$ & ( & A & $\wedge$ & $\neg$ & B & )\\
\hline 
$\True$ & $\True$ &  & $\True$ & \textcolor{red}{$\True$} & $\True$ &  &  & $\False$ & $\True$ & \textcolor{red}{$\True$} & $\True$ &  & \textcolor{red}{$\True$} &  & $\True$ & $\False$ & $\False$ & $\True$ & \\
$\True$ & $\False$ &  & $\True$ & \textcolor{red}{$\False$} & $\False$ &  &  & $\False$ & $\True$ & \textcolor{red}{$\False$} & $\False$ &  & \textcolor{red}{$\False$} &  & $\True$ & $\True$ & $\True$ & $\False$ & \\
$\False$ & $\True$ &  & $\False$ & \textcolor{red}{$\True$} & $\True$ &  &  & $\True$ & $\False$ & \textcolor{red}{$\True$} & $\True$ &  & \textcolor{red}{$\True$} &  & $\False$ & $\False$ & $\False$ & $\True$ & \\
$\False$ & $\False$ &  & $\False$ & \textcolor{red}{$\True$} & $\False$ &  &  & $\True$ & $\False$ & \textcolor{red}{$\True$} & $\False$ &  & \textcolor{red}{$\True$} &  & $\False$ & $\False$ & $\True$ & $\False$ & \\
\end{tabular}

\end{center}

This conditional is sometimes called \textbf{material implication}.
\end{frame}

\begin{frame}{Oddities}
\protect\hypertarget{oddities}{}
It is certainly an odd interpretation of `if' that makes these sentences
turn out true.

\begin{itemize}
\tightlist
\item
  If I am 200 years old, then Michigan is part of Canada.
\item
  If I am in Los Angeles, then I am in Ann Arbor.
\end{itemize}

But they are both true on this table.
\end{frame}

\begin{frame}{Arguments}
\protect\hypertarget{arguments}{}
\begin{itemize}
\tightlist
\item
  It turns out that interpreting the conditional this way makes the most
  sense of the role of conditionals in certain arguments, in particular
  to do with disjunctive syllogism.
\item
  There is an allusion to this at the end of chapter 1 of \emph{Boxes
  and Diamonds}.
\end{itemize}
\end{frame}

\begin{frame}{Arguments}
\protect\hypertarget{arguments-1}{}
The big advantage of thinking of `if' this way is that it guarantees
that for any value of \(A, B, C\), these two arguments agree on validity
- that is, they are either both valid or both invalid.

\begin{quote}
\(A, B \vdash C\)
\end{quote}

\begin{quote}
\(A \vdash B \rightarrow C\)
\end{quote}

And plausibly those should be the same. \(A\) suffices for
\(B \rightarrow C\) just in case \(A\) and \(B\) together suffice for
\(C\).
\end{frame}

\begin{frame}{For Next Time}
\protect\hypertarget{for-next-time}{}
We'll talk about how to use these basic truth tables to build larger
truth tables.
\end{frame}

\end{document}
