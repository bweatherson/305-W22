% Options for packages loaded elsewhere
\PassOptionsToPackage{unicode}{hyperref}
\PassOptionsToPackage{hyphens}{url}
%
\documentclass[
  ignorenonframetext,
]{beamer}
\usepackage{pgfpages}
\setbeamertemplate{caption}[numbered]
\setbeamertemplate{caption label separator}{: }
\setbeamercolor{caption name}{fg=normal text.fg}
\beamertemplatenavigationsymbolsempty
% Prevent slide breaks in the middle of a paragraph
\widowpenalties 1 10000
\raggedbottom
\setbeamertemplate{part page}{
  \centering
  \begin{beamercolorbox}[sep=16pt,center]{part title}
    \usebeamerfont{part title}\insertpart\par
  \end{beamercolorbox}
}
\setbeamertemplate{section page}{
  \centering
  \begin{beamercolorbox}[sep=12pt,center]{part title}
    \usebeamerfont{section title}\insertsection\par
  \end{beamercolorbox}
}
\setbeamertemplate{subsection page}{
  \centering
  \begin{beamercolorbox}[sep=8pt,center]{part title}
    \usebeamerfont{subsection title}\insertsubsection\par
  \end{beamercolorbox}
}
\AtBeginPart{
  \frame{\partpage}
}
\AtBeginSection{
  \ifbibliography
  \else
    \frame{\sectionpage}
  \fi
}
\AtBeginSubsection{
  \frame{\subsectionpage}
}
\usepackage{amsmath,amssymb}
\usepackage{lmodern}
\usepackage{ifxetex,ifluatex}
\ifnum 0\ifxetex 1\fi\ifluatex 1\fi=0 % if pdftex
  \usepackage[T1]{fontenc}
  \usepackage[utf8]{inputenc}
  \usepackage{textcomp} % provide euro and other symbols
\else % if luatex or xetex
  \usepackage{unicode-math}
  \defaultfontfeatures{Scale=MatchLowercase}
  \defaultfontfeatures[\rmfamily]{Ligatures=TeX,Scale=1}
  \setmainfont[BoldFont = SF Pro Rounded Semibold]{SF Pro Rounded}
  \setmathfont[]{STIX Two Math}
\fi
\usefonttheme{serif} % use mainfont rather than sansfont for slide text
% Use upquote if available, for straight quotes in verbatim environments
\IfFileExists{upquote.sty}{\usepackage{upquote}}{}
\IfFileExists{microtype.sty}{% use microtype if available
  \usepackage[]{microtype}
  \UseMicrotypeSet[protrusion]{basicmath} % disable protrusion for tt fonts
}{}
\makeatletter
\@ifundefined{KOMAClassName}{% if non-KOMA class
  \IfFileExists{parskip.sty}{%
    \usepackage{parskip}
  }{% else
    \setlength{\parindent}{0pt}
    \setlength{\parskip}{6pt plus 2pt minus 1pt}}
}{% if KOMA class
  \KOMAoptions{parskip=half}}
\makeatother
\usepackage{xcolor}
\IfFileExists{xurl.sty}{\usepackage{xurl}}{} % add URL line breaks if available
\IfFileExists{bookmark.sty}{\usepackage{bookmark}}{\usepackage{hyperref}}
\hypersetup{
  pdftitle={305 Lecture 13.6 - Minimal Change Semantics},
  pdfauthor={Brian Weatherson},
  hidelinks,
  pdfcreator={LaTeX via pandoc}}
\urlstyle{same} % disable monospaced font for URLs
\newif\ifbibliography
\setlength{\emergencystretch}{3em} % prevent overfull lines
\providecommand{\tightlist}{%
  \setlength{\itemsep}{0pt}\setlength{\parskip}{0pt}}
\setcounter{secnumdepth}{-\maxdimen} % remove section numbering
\let\Tiny=\tiny

 \setbeamertemplate{navigation symbols}{} 

% \usetheme{Madrid}
 \usetheme[numbering=none, progressbar=foot]{metropolis}
 \usecolortheme{wolverine}
 \usepackage{color}
 \usepackage{MnSymbol}
% \usepackage{movie15}

\usepackage{amssymb}% http://ctan.org/pkg/amssymb
\usepackage{pifont}% http://ctan.org/pkg/pifont
\newcommand{\cmark}{\ding{51}}%
\newcommand{\xmark}{\ding{55}}%

\DeclareSymbolFont{symbolsC}{U}{txsyc}{m}{n}
\DeclareMathSymbol{\boxright}{\mathrel}{symbolsC}{128}
\DeclareMathAlphabet{\mathpzc}{OT1}{pzc}{m}{it}

\usepackage{tikz-qtree}
% \usepackage{markdown}
%\RequirePackage{bussproofs}
\usetikzlibrary{arrows.meta}
\RequirePackage[tableaux]{prooftrees}
\forestset{line numbering, close with = x}
% Allow for easy commas inside trees
\renewcommand{\,}{\text{, }}


\usepackage{tabulary}

\usepackage{open-logic-config}

\setlength{\parskip}{1ex plus 0.5ex minus 0.2ex}

\AtBeginSection[]
{
\begin{frame}
	\Huge{\color{darkblue} \insertsection}
\end{frame}
}

\renewenvironment*{quote}	
	{\list{}{\rightmargin   \leftmargin} \item } 	
	{\endlist }

\definecolor{darkgreen}{rgb}{0,0.7,0}
\definecolor{darkblue}{rgb}{0,0,0.8}

\newcommand{\starttab}{\begin{center}
\vspace{6pt}
\begin{tabular}}

\newcommand{\stoptab}{\end{tabular}
\vspace{6pt}
\end{center}
\noindent}


\newcommand{\sif}{\rightarrow}
\newcommand{\siff}{\leftrightarrow}
\newcommand{\EF}{\end{frame}}


\newcommand{\TreeStart}[1]{
%\end{frame}
\begin{frame}
\begin{center}
\begin{tikzpicture}[scale=#1]
\tikzset{every tree node/.style={align=center,anchor=north}}
%\Tree
}

\newcommand{\TreeEnd}{
\end{tikzpicture}
%\end{center}
}

\newcommand{\DisplayArg}[2]{
\begin{enumerate}
{#1}
\end{enumerate}
\vspace{-6pt}
\hrulefill

%\hspace{14pt} #2
%{\addtolength{\leftskip}{14pt} #2}
\begin{quote}
{\normalfont #2}
\end{quote}
\vspace{12pt}
}

\newenvironment{ProofTree}[1][1]{
\begin{center}
\begin{tikzpicture}[scale=#1]
\tikzset{every tree node/.style={align=center,anchor=south}}
}
{
\end{tikzpicture}
\end{center}
}

\newcommand{\TreeFrame}[2]{
\begin{columns}[c]
\column{0.5\textwidth}
\begin{center}
\begin{prooftree}{}
#1
\end{prooftree}
\end{center}
\column{0.45\textwidth}
%\begin{markdown}
#2
%\end{markdown}
\end{columns}
}

\newcommand{\ScaledTreeFrame}[3]{
\begin{columns}[c]
\column{0.5\textwidth}
\begin{center}
\scalebox{#1}{
\begin{prooftree}{}
#2
\end{prooftree}
}
\end{center}
\column{0.45\textwidth}
%\begin{markdown}
#3
%\end{markdown}
\end{columns}
}

\usepackage[bb=boondox]{mathalfa}
\DeclareMathAlphabet{\mathbx}{U}{BOONDOX-ds}{m}{n}
\SetMathAlphabet{\mathbx}{bold}{U}{BOONDOX-ds}{b}{n}
\DeclareMathAlphabet{\mathbbx} {U}{BOONDOX-ds}{b}{n}


\newenvironment{oltableau}{\center\tableau{}} %wff format={anchor = base west}}}
       {\endtableau\endcenter}
       
\newcommand{\formula}[1]{$#1$}

\usepackage{tabulary}
\usepackage{booktabs}

\def\begincols{\begin{columns}}
\def\begincol{\begin{column}}
\def\endcol{\end{column}}
\def\endcols{\end{columns}}

\usepackage[italic]{mathastext}
\usepackage{nicefrac}

\definecolor{mygreen}{RGB}{0, 100, 0}
\definecolor{mypink2}{RGB}{219, 48, 122}
\definecolor{dodgerblue}{RGB}{30,144,255}

%\def\True{\textcolor{dodgerblue}{\text{T}}}
%\def\False{\textcolor{red}{\text{F}}}

\def\True{\mathbb{T}}
\def\False{\mathbb{F}}

% This is because arguments didn't have enough space after them
\usepackage{etoolbox}
\AfterEndEnvironment{description}{\vspace{9pt}}
\AfterEndEnvironment{oltableau}{\vspace{9pt}}
\BeforeBeginEnvironment{oltableau}{\vspace{9pt}}
\AfterEndEnvironment{center}{\vspace{12pt}}
\BeforeBeginEnvironment{tabular}{\vspace{9pt}}

\setlength\heavyrulewidth{0pt}
\setlength\lightrulewidth{0pt}

%\def\toprule{}
%\def\bottomrule{}
%\def\midrule{}

\setbeamertemplate{caption}{\raggedright\insertcaption}

\ifluatex
  \usepackage{selnolig}  % disable illegal ligatures
\fi

\title{305 Lecture 13.6 - Minimal Change Semantics}
\author{Brian Weatherson}
\date{}

\begin{document}
\frame{\titlepage}

\begin{frame}{Plan}
\protect\hypertarget{plan}{}
To discuss the `nearest possible world' approach to counterfactuals.
\end{frame}

\begin{frame}{Reading}
\protect\hypertarget{reading}{}
Still chapter 7 of Boxes and Diamonds
\end{frame}

\begin{frame}{Basic Idea}
\protect\hypertarget{basic-idea}{}
\begin{itemize}
\tightlist
\item
  Replace the on/off accessibility relation between worlds with a
  distance measure \(d\).
\item
  So \(d(x, y)\), where \(x, y\) are worlds, measures how similar \(x\)
  and \(y\) are.
\item
  But we'll sometimes talk as if it is a distance measure, that tracks
  `how far apart' the worlds are.
\item
  And for simplicity, we'll assume \(d(x, y)\) is always an integer -
  there are the worlds 1 unit apart, 2 units apart, etc.
\end{itemize}
\end{frame}

\begin{frame}{Nearest World}
\protect\hypertarget{nearest-world}{}
\begin{itemize}
\tightlist
\item
  So if we're evaluating ``If A were true, B would be true'', we do the
  following.
\item
  We find the nearest world, or worlds, where A is true.
\item
  We see A is actually true, if not we look one unit away and see if
  there are any A worlds there, if not we look two units away and see if
  there are A worlds there, and so on until we find some A worlds.
\item
  Say the distance they are separated from us is \(d\).
\item
  Then ``If A were true, B would be true'' is true just in case all the
  worlds distance \(d\) away where A is true also make B true.
\end{itemize}
\end{frame}

\begin{frame}{Short Version}
\protect\hypertarget{short-version}{}
\emph{If A were true, B would be true}, means

\begin{itemize}
\tightlist
\item
  All of the nearest A-worlds are B-worlds.
\end{itemize}

From now on, I'll sometimes write this as \(A \boxright B\).
\end{frame}

\begin{frame}{Variably Strict}
\protect\hypertarget{variably-strict}{}
\begin{itemize}
\tightlist
\item
  This makes the conditional a variably strict conditional.
\item
  It's strict, because it requires all A worlds to be B worlds.
\item
  But it's variably strict, because which worlds it ranges over varies
  with what the antecedent is.
\end{itemize}
\end{frame}

\begin{frame}{Intuition Check}
\protect\hypertarget{intuition-check}{}
Here's the thought behind Lewis's view.

\begin{itemize}
\tightlist
\item
  When a counterfactual has a normal antecedent, the only worlds that
  matter to its truth are fairly normal worlds.
\item
  To figure out what would have happened if I'd had bacon and eggs for
  breakfast, you only have to consider worlds that are really a lot like
  the actual world.
\item
  But to figure out what would have happened if Columbus's fleet had
  sunk, you have to think about worlds that are really different to
  reality.
\end{itemize}
\end{frame}

\begin{frame}{How Far Away}
\protect\hypertarget{how-far-away}{}
We can even try to think about really wild counterfactuals.

\begin{itemize}[<+->]
\tightlist
\item
  What would happen if a vampire was the starting QB for the University
  of Michigan?
\item
  Or a zombie?
\item
  Or the number 7? \pause
\end{itemize}

Maybe some of those don't make sense - there are no worlds, not even
distant ones, where they are actual. But we can go a fair way until we
get to that point.
\end{frame}

\begin{frame}{Recap}
\protect\hypertarget{recap}{}
The textbook calls this theory ``minimal change semantics''. Here's a
reminder of how it works.

\begin{itemize}
\tightlist
\item
  Let's say we want to find out whether \(A \boxright B\) is true at
  \(w\) (we'll use \(\boxright\) for the conditional we're about to
  define).
\item
  We find the world \(x\) such that \(A\) is true at \(x\), and \(x\) is
  closest to \(w\), i.e., is the least distance away.
\item
  Then we look to see whether \(B\) is true.
\item
  If so, \(A \boxright B\) is true.
\item
  If not, \(A \boxright B\) is false.
\end{itemize}
\end{frame}

\begin{frame}{Complication 1: No A-world}
\protect\hypertarget{complication-1-no-a-world}{}
We'll just stipulate that if there are no \(A\)-worlds at all, then
\(A \boxright B\) is true for all \(B\). These cases are weird, and I'll
mostly set them aside.
\end{frame}

\begin{frame}{Complication 2: No `closest' A-world}
\protect\hypertarget{complication-2-no-closest-a-world}{}
Here's something that can happen if you drop the assumption that the
distances are integers.

\begin{itemize}
\tightlist
\item
  For any distance \(d > 2\), there is a world \(x\) such that the
  distance between \(w\) and \(x\) is \(d\), and \(A\) is true at \(x\).
\item
  But for any distance \(d \leq 2\), there is no world \(x\) such that
  the distance between \(w\) and \(x\) is \(d\), and \(A\) is true at
  \(x \pause\).
\item
  So there isn't a \textbf{closest} \(A\)-world, because you can get
  closer and closer to 2, and find a yet closer \(A\)-world.
\item
  This is also a weird possibility, and I'll set it aside.
\end{itemize}
\end{frame}

\begin{frame}{Complication 3: Many equally closest A-worlds}
\protect\hypertarget{complication-3-many-equally-closest-a-worlds}{}
This is more philosophically substantial.

\begin{itemize}
\tightlist
\item
  Imagine that for \(d\) less than 2, there is no \(A\)-world distance
  \(d\) away.
\item
  But at distance 2, there are multiple worlds where \(A\) is true.
\item
  And at some of them \(B\) is true, and at others \(B\) is false.
\end{itemize}
\end{frame}

\begin{frame}{One solution: Require all of them to be \(B\) worlds}
\protect\hypertarget{one-solution-require-all-of-them-to-be-b-worlds}{}
As I noted above David Lewis said that in that case we should say:

\begin{itemize}
\tightlist
\item
  \(A \boxright B\) is false. \pause  But also
\item
  \(A \boxright \neg B\) is false.
\end{itemize}

In general \(A \boxright\) something requires that the something is true
at all the nearest \(A\)-worlds. And neither \(B\) nor \(\neg B\) is
true at all of them, so neither is true.
\end{frame}

\begin{frame}{Intuition Pump}
\protect\hypertarget{intuition-pump}{}
Could you have 1 and 2 false, but 3 true?

\begin{enumerate}
\tightlist
\item
  If a UM student had been elected mayor of Ann Arbor, it would have
  been an undergraduate.
\item
  If a UM student had been elected mayor of Ann Arbor, it would have
  been a postgraduate.
\item
  If a UM student had been elected mayor of Ann Arbor, it would have
  been either an undergraduate or a postgraduate.
\end{enumerate}

Lewis said that you could have 1 and 2 both false, but 3 true, and
that's why \(\boxright\) should work this way.
\end{frame}

\begin{frame}{Another solution: Deny this is possible}
\protect\hypertarget{another-solution-deny-this-is-possible}{}
But the other great founder of this tradition, Robert Stalnaker, argued
that we should want these things to be equivalent (at least if
\(\boxright\) was going to represent something in English).

\begin{enumerate}
\tightlist
\item
  \(A \boxright B \vee A \boxright C\)
\item
  \(A \boxright (B \vee C)\)
\end{enumerate}

And we couldn't have that on Lewis's picture.
\end{frame}

\begin{frame}{Stalnaker's Solution}
\protect\hypertarget{stalnakers-solution}{}
\begin{itemize}
\tightlist
\item
  It is a constraint on the distance metric that there are no ties.
\item
  The similarity measure is a strict ordering.
\item
  And it's still a matter of debate whether Lewis or Stalnaker is right
  about this.
\end{itemize}
\end{frame}

\begin{frame}{Next Time}
\protect\hypertarget{next-time}{}
\begin{itemize}
\tightlist
\item
  We'll talk more about this notion of similarity.
\end{itemize}
\end{frame}

\end{document}
