% Options for packages loaded elsewhere
\PassOptionsToPackage{unicode}{hyperref}
\PassOptionsToPackage{hyphens}{url}
%
\documentclass[
  ignorenonframetext,
]{beamer}
\usepackage{pgfpages}
\setbeamertemplate{caption}[numbered]
\setbeamertemplate{caption label separator}{: }
\setbeamercolor{caption name}{fg=normal text.fg}
\beamertemplatenavigationsymbolsempty
% Prevent slide breaks in the middle of a paragraph
\widowpenalties 1 10000
\raggedbottom
\setbeamertemplate{part page}{
  \centering
  \begin{beamercolorbox}[sep=16pt,center]{part title}
    \usebeamerfont{part title}\insertpart\par
  \end{beamercolorbox}
}
\setbeamertemplate{section page}{
  \centering
  \begin{beamercolorbox}[sep=12pt,center]{part title}
    \usebeamerfont{section title}\insertsection\par
  \end{beamercolorbox}
}
\setbeamertemplate{subsection page}{
  \centering
  \begin{beamercolorbox}[sep=8pt,center]{part title}
    \usebeamerfont{subsection title}\insertsubsection\par
  \end{beamercolorbox}
}
\AtBeginPart{
  \frame{\partpage}
}
\AtBeginSection{
  \ifbibliography
  \else
    \frame{\sectionpage}
  \fi
}
\AtBeginSubsection{
  \frame{\subsectionpage}
}
\usepackage{amsmath,amssymb}
\usepackage{lmodern}
\usepackage{ifxetex,ifluatex}
\ifnum 0\ifxetex 1\fi\ifluatex 1\fi=0 % if pdftex
  \usepackage[T1]{fontenc}
  \usepackage[utf8]{inputenc}
  \usepackage{textcomp} % provide euro and other symbols
\else % if luatex or xetex
  \usepackage{unicode-math}
  \defaultfontfeatures{Scale=MatchLowercase}
  \defaultfontfeatures[\rmfamily]{Ligatures=TeX,Scale=1}
  \setmainfont[BoldFont = SF Pro Rounded Semibold]{SF Pro Rounded}
  \setmathfont[]{STIX Two Math}
\fi
\usefonttheme{serif} % use mainfont rather than sansfont for slide text
% Use upquote if available, for straight quotes in verbatim environments
\IfFileExists{upquote.sty}{\usepackage{upquote}}{}
\IfFileExists{microtype.sty}{% use microtype if available
  \usepackage[]{microtype}
  \UseMicrotypeSet[protrusion]{basicmath} % disable protrusion for tt fonts
}{}
\makeatletter
\@ifundefined{KOMAClassName}{% if non-KOMA class
  \IfFileExists{parskip.sty}{%
    \usepackage{parskip}
  }{% else
    \setlength{\parindent}{0pt}
    \setlength{\parskip}{6pt plus 2pt minus 1pt}}
}{% if KOMA class
  \KOMAoptions{parskip=half}}
\makeatother
\usepackage{xcolor}
\IfFileExists{xurl.sty}{\usepackage{xurl}}{} % add URL line breaks if available
\IfFileExists{bookmark.sty}{\usepackage{bookmark}}{\usepackage{hyperref}}
\hypersetup{
  pdftitle={305 Lecture 9.6 - Allais Paradox},
  pdfauthor={Brian Weatherson},
  hidelinks,
  pdfcreator={LaTeX via pandoc}}
\urlstyle{same} % disable monospaced font for URLs
\newif\ifbibliography
\usepackage{longtable,booktabs,array}
\usepackage{calc} % for calculating minipage widths
\usepackage{caption}
% Make caption package work with longtable
\makeatletter
\def\fnum@table{\tablename~\thetable}
\makeatother
\setlength{\emergencystretch}{3em} % prevent overfull lines
\providecommand{\tightlist}{%
  \setlength{\itemsep}{0pt}\setlength{\parskip}{0pt}}
\setcounter{secnumdepth}{-\maxdimen} % remove section numbering
\let\Tiny=\tiny

 \setbeamertemplate{navigation symbols}{} 

% \usetheme{Madrid}
 \usetheme[numbering=none, progressbar=foot]{metropolis}
 \usecolortheme{wolverine}
 \usepackage{color}
 \usepackage{MnSymbol}
% \usepackage{movie15}

\usepackage{amssymb}% http://ctan.org/pkg/amssymb
\usepackage{pifont}% http://ctan.org/pkg/pifont
\newcommand{\cmark}{\ding{51}}%
\newcommand{\xmark}{\ding{55}}%

\DeclareSymbolFont{symbolsC}{U}{txsyc}{m}{n}
\DeclareMathSymbol{\boxright}{\mathrel}{symbolsC}{128}
\DeclareMathAlphabet{\mathpzc}{OT1}{pzc}{m}{it}

 \usepackage{tikz-qtree}
% \usepackage{markdown}
%\RequirePackage{bussproofs}
\RequirePackage[tableaux]{prooftrees}
\usetikzlibrary{arrows.meta}
 \forestset{line numbering, close with = x}
% Allow for easy commas inside trees
\renewcommand{\,}{\text{, }}


\usepackage{tabulary}

\usepackage{open-logic-config}

\setlength{\parskip}{1ex plus 0.5ex minus 0.2ex}

\AtBeginSection[]
{
\begin{frame}
	\Huge{\color{darkblue} \insertsection}
\end{frame}
}

\renewenvironment*{quote}	
	{\list{}{\rightmargin   \leftmargin} \item } 	
	{\endlist }

\definecolor{darkgreen}{rgb}{0,0.7,0}
\definecolor{darkblue}{rgb}{0,0,0.8}

\newcommand{\starttab}{\begin{center}
\vspace{6pt}
\begin{tabular}}

\newcommand{\stoptab}{\end{tabular}
\vspace{6pt}
\end{center}
\noindent}


\newcommand{\sif}{\rightarrow}
\newcommand{\siff}{\leftrightarrow}
\newcommand{\EF}{\end{frame}}


\newcommand{\TreeStart}[1]{
%\end{frame}
\begin{frame}
\begin{center}
\begin{tikzpicture}[scale=#1]
\tikzset{every tree node/.style={align=center,anchor=north}}
%\Tree
}

\newcommand{\TreeEnd}{
\end{tikzpicture}
%\end{center}
}

\newcommand{\DisplayArg}[2]{
\begin{enumerate}
{#1}
\end{enumerate}
\vspace{-6pt}
\hrulefill

%\hspace{14pt} #2
%{\addtolength{\leftskip}{14pt} #2}
\begin{quote}
{\normalfont #2}
\end{quote}
\vspace{12pt}
}

\newenvironment{ProofTree}[1][1]{
\begin{center}
\begin{tikzpicture}[scale=#1]
\tikzset{every tree node/.style={align=center,anchor=south}}
}
{
\end{tikzpicture}
\end{center}
}

\newcommand{\TreeFrame}[2]{
\begin{columns}[c]
\column{0.5\textwidth}
\begin{center}
\begin{prooftree}{}
#1
\end{prooftree}
\end{center}
\column{0.45\textwidth}
%\begin{markdown}
#2
%\end{markdown}
\end{columns}
}

\newcommand{\ScaledTreeFrame}[3]{
\begin{columns}[c]
\column{0.5\textwidth}
\begin{center}
\scalebox{#1}{
\begin{prooftree}{}
#2
\end{prooftree}
}
\end{center}
\column{0.45\textwidth}
%\begin{markdown}
#3
%\end{markdown}
\end{columns}
}

\usepackage[bb=boondox]{mathalfa}
\DeclareMathAlphabet{\mathbx}{U}{BOONDOX-ds}{m}{n}
\SetMathAlphabet{\mathbx}{bold}{U}{BOONDOX-ds}{b}{n}
\DeclareMathAlphabet{\mathbbx} {U}{BOONDOX-ds}{b}{n}


\newenvironment{oltableau}{\center\tableau{}} %wff format={anchor = base west}}}
       {\endtableau\endcenter}
       
\newcommand{\formula}[1]{$#1$}

\usepackage{tabulary}
\usepackage{booktabs}

\def\begincols{\begin{columns}}
\def\begincol{\begin{column}}
\def\endcol{\end{column}}
\def\endcols{\end{columns}}

\usepackage[italic]{mathastext}
\usepackage{nicefrac}

\definecolor{mygreen}{RGB}{0, 100, 0}
\definecolor{mypink2}{RGB}{219, 48, 122}
\definecolor{dodgerblue}{RGB}{30,144,255}

%\def\True{\textcolor{dodgerblue}{\text{T}}}
%\def\False{\textcolor{red}{\text{F}}}

\def\True{\mathbb{T}}
\def\False{\mathbb{F}}

% This is because arguments didn't have enough space after them
\usepackage{etoolbox}
\AfterEndEnvironment{description}{\vspace{9pt}}
\AfterEndEnvironment{oltableau}{\vspace{9pt}}
\BeforeBeginEnvironment{oltableau}{\vspace{9pt}}
\AfterEndEnvironment{center}{\vspace{12pt}}
\BeforeBeginEnvironment{tabular}{\vspace{9pt}}

\setlength\heavyrulewidth{0pt}
\setlength\lightrulewidth{0pt}

%\def\toprule{}
%\def\bottomrule{}
%\def\midrule{}

\setbeamertemplate{caption}{\raggedright\insertcaption}

\ifluatex
  \usepackage{selnolig}  % disable illegal ligatures
\fi

\title{305 Lecture 9.6 - Allais Paradox}
\author{Brian Weatherson}
\date{}

\begin{document}
\frame{\titlepage}

\begin{frame}{Plan}
\protect\hypertarget{plan}{}
\begin{itemize}
\tightlist
\item
  In this lecture we'll talk about a famous puzzle for the story I've
  been telling you about the relationship between utility and money: the
  Allais paradox.
\end{itemize}
\end{frame}

\begin{frame}{Associated Reading}
\protect\hypertarget{associated-reading}{}
Odds and Ends, section 13.1.
\end{frame}

\begin{frame}{Sure Thing Principle}
\protect\hypertarget{sure-thing-principle}{}
Assume two bets A and B have the following characteristic.

\begin{itemize}
\tightlist
\item
  For some proposition \(p\), A and B have the exact same return. \pause
\end{itemize}

Then the Sure Thing Principle says the following.

\begin{itemize}
\tightlist
\item
  Changing what that return is won't change your preference between A
  and B.
\end{itemize}
\end{frame}

\begin{frame}{Intuitive Example}
\protect\hypertarget{intuitive-example}{}
Assume that B is a \textbf{conditional bet} - a bet on \(q\) that only
takes place if something is true. So if you take the bet, the following
things happen.

\begin{itemize}
\tightlist
\item
  So if the condition, and \(q\) is true, you win, let's say, \$10.
\item
  If the condition, and \(q\) is false, you lose \$10.
\item
  But if the condition is not met, then the bet is called off.
\end{itemize}

E.g., you might bet a friend that if the Rose Bowl is played this year,
Michigan will win it. The bet is simply called off if the Rose Bowl
doesn't happen. Let A be the action of simply not taking this bet, and
staying at the status quo. And let \(p\) be that the Rose Bowl doesn't
happen. So A and B have the same payoff if \(p\), since then the bet is
called off.
\end{frame}

\begin{frame}{A Change}
\protect\hypertarget{a-change}{}
You find out, apparently because you've been doing more gambling in your
spare time than is good for you, that you have another bet that wins 5
dollars if \(p\), i.e., if the Rose Bowl doesn't happen, and returns
nothing otherwise.

\begin{itemize}
\tightlist
\item
  Could this change your preferences over A vs B?
\end{itemize}
\end{frame}

\begin{frame}{The Argument for No}
\protect\hypertarget{the-argument-for-no}{}
Either \(p\) is true or it isn't.

\begin{itemize}
\tightlist
\item
  If it is, then whether you would have got \$5 if it were not doesn't
  make a difference to whether you prefer A or B.
\item
  If it is not, then you should still be indifferent between A and B.
\item
  And this doesn't look like it just applies to this case.
\item
  It looks like a perfectly general weak dominance argument.
\end{itemize}
\end{frame}

\begin{frame}{Expected Utility Theory and Constraints on Choice}
\protect\hypertarget{expected-utility-theory-and-constraints-on-choice}{}
\begin{itemize}
\tightlist
\item
  Orthodox expected utility theory, the theory that says you should
  maximise expected utility, puts very few constraints on individual
  choices.
\item
  But it puts quite striking constraints on sets of choices.
\item
  It says you can't prefer A to B, and B to C, and C to A, for example.
\item
  And it says that the Sure Thing Principle, a principle about what
  changes in payouts licence a change of preferences, holds.
\end{itemize}
\end{frame}

\begin{frame}{Allais}
\protect\hypertarget{allais}{}
Maurice Allais (1911-2010) developed the most famous objection to the
Sure Thing Principle.

\begin{itemize}
\tightlist
\item
  It is a pair of two-way choices, and an intuitively rational pair of
  preferences among them.
\item
  Expected utility can make sense of either one of the pair, but not
  both.
\end{itemize}
\end{frame}

\begin{frame}{Allais - First Part}
\protect\hypertarget{allais---first-part}{}
You have a choice between:

\begin{enumerate}
[A.]
\tightlist
\item
  A 10\% chance of \$5,000,000.
\item
  An 11\% chance of \$1,000,000.
\end{enumerate}

What do you choose?
\end{frame}

\begin{frame}{Allais - Second Part}
\protect\hypertarget{allais---second-part}{}
That was a hypothetical. Now for real you have a choice between:

\begin{enumerate}
[A.]
\setcounter{enumi}{2}
\tightlist
\item
  A 10\% chance of \$5,000,000, an 89\% chance of \$1,000,000, and a 1\%
  chance of nothing.
\item
  \$1,000,000.
\end{enumerate}

What do you choose?
\end{frame}

\begin{frame}{Allais's Argument}
\protect\hypertarget{allaiss-argument}{}
\begin{itemize}
\tightlist
\item
  It is rational to prefer A to B, and D to C. \pause
\item
  Expected utility theory says that you can't prefer both A to B, and D
  to C. \pause
\item
  So expected utility theory is false.
\end{itemize}
\end{frame}

\begin{frame}{Allais Table}
\protect\hypertarget{allais-table}{}
Imagine you have 10 blue marbles in an urn, 89 maize marbles, and 1
scarlet marble.

\begin{longtable}[]{@{}cccc@{}}
\toprule
& Blue & Maize & Scarlet \\ \addlinespace
\midrule
\endhead
A & 5M & 0 & 0 \\ \addlinespace
B & 1M & 0 & 1M \\ \addlinespace
& & & \\ \addlinespace
C & 5M & 1M & 0 \\ \addlinespace
D & 1M & 1M & 1M \\ \addlinespace
\bottomrule
\end{longtable}

All that changed from AB to CD was that we changed how much the payout
was if Maize, without changing the fact that it was equal.
\end{frame}

\begin{frame}{Why Expected Utility Theory Can't Handle This}
\protect\hypertarget{why-expected-utility-theory-cant-handle-this}{}
Let \(u(5M) = x\) and \(u(1M) = y\).

\begin{itemize}
\tightlist
\item
  If A is preferred to B, then \(0.1x > 0.11y\), since those are the
  expected utilities of A and B. \pause
\item
  So adding \(0.89y\) to both sides, we get \(0.1x + 0.89y > y\). \pause
\item
  But those just are the expected utilities of C and D.
\item
  So if A is preferred to B, then C is preferred to D.
\end{itemize}
\end{frame}

\begin{frame}{An argument against Allais}
\protect\hypertarget{an-argument-against-allais}{}
Assume you'll find out, both in the AB case and the CD case, whether the
marble was maize, or not-maize before you are told its color.

\begin{itemize}
\tightlist
\item
  At that point, in the AB case, what will you wish you'd chosen? \pause
\item
  If B, or you are indifferent, then you shouldn't have preferred A in
  the first place. \pause
\item
  If A, then do the same thing in the CD case.
\item
  If you find out the marble is maize, you don't care.
\item
  If you find out it's not maize, then you're back in the exact same
  puzzle, so you should prefer C to D.
\item
  So by weak dominance, you should prefer C to D overall.
\end{itemize}
\end{frame}

\begin{frame}{Decision Theory for Allais Agents}
\protect\hypertarget{decision-theory-for-allais-agents}{}
\begin{itemize}
\tightlist
\item
  This was originally developed by the Australian economist John
  Quiggin, and recently expanded by the American philosopher Lara
  Buchak.
\item
  Very roughly, you replace the \(Pr\) in expected utility theory with
  some function \(f(Pr)\) where \(f\) measures the agent's attitude to
  risk.
\item
  If \(f(x) < x\), the agent is risk averse, if \(f(x) > x\) they are
  risk seeking.
\item
  If you let \(f(x) = x^2\), it's easy to model the Allais preferences.
\end{itemize}
\end{frame}

\begin{frame}{For Next Time}
\protect\hypertarget{for-next-time}{}
\begin{itemize}
\tightlist
\item
  We will move on to thinking about how to use probability in learning
  about the world.
\end{itemize}
\end{frame}

\end{document}
