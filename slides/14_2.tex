% Options for packages loaded elsewhere
\PassOptionsToPackage{unicode}{hyperref}
\PassOptionsToPackage{hyphens}{url}
%
\documentclass[
  ignorenonframetext,
]{beamer}
\usepackage{pgfpages}
\setbeamertemplate{caption}[numbered]
\setbeamertemplate{caption label separator}{: }
\setbeamercolor{caption name}{fg=normal text.fg}
\beamertemplatenavigationsymbolsempty
% Prevent slide breaks in the middle of a paragraph
\widowpenalties 1 10000
\raggedbottom
\setbeamertemplate{part page}{
  \centering
  \begin{beamercolorbox}[sep=16pt,center]{part title}
    \usebeamerfont{part title}\insertpart\par
  \end{beamercolorbox}
}
\setbeamertemplate{section page}{
  \centering
  \begin{beamercolorbox}[sep=12pt,center]{part title}
    \usebeamerfont{section title}\insertsection\par
  \end{beamercolorbox}
}
\setbeamertemplate{subsection page}{
  \centering
  \begin{beamercolorbox}[sep=8pt,center]{part title}
    \usebeamerfont{subsection title}\insertsubsection\par
  \end{beamercolorbox}
}
\AtBeginPart{
  \frame{\partpage}
}
\AtBeginSection{
  \ifbibliography
  \else
    \frame{\sectionpage}
  \fi
}
\AtBeginSubsection{
  \frame{\subsectionpage}
}
\usepackage{amsmath,amssymb}
\usepackage{lmodern}
\usepackage{ifxetex,ifluatex}
\ifnum 0\ifxetex 1\fi\ifluatex 1\fi=0 % if pdftex
  \usepackage[T1]{fontenc}
  \usepackage[utf8]{inputenc}
  \usepackage{textcomp} % provide euro and other symbols
\else % if luatex or xetex
  \usepackage{unicode-math}
  \defaultfontfeatures{Scale=MatchLowercase}
  \defaultfontfeatures[\rmfamily]{Ligatures=TeX,Scale=1}
  \setmainfont[BoldFont = SF Pro Rounded Semibold]{SF Pro Rounded}
  \setmathfont[]{STIX Two Math}
\fi
\usefonttheme{serif} % use mainfont rather than sansfont for slide text
% Use upquote if available, for straight quotes in verbatim environments
\IfFileExists{upquote.sty}{\usepackage{upquote}}{}
\IfFileExists{microtype.sty}{% use microtype if available
  \usepackage[]{microtype}
  \UseMicrotypeSet[protrusion]{basicmath} % disable protrusion for tt fonts
}{}
\makeatletter
\@ifundefined{KOMAClassName}{% if non-KOMA class
  \IfFileExists{parskip.sty}{%
    \usepackage{parskip}
  }{% else
    \setlength{\parindent}{0pt}
    \setlength{\parskip}{6pt plus 2pt minus 1pt}}
}{% if KOMA class
  \KOMAoptions{parskip=half}}
\makeatother
\usepackage{xcolor}
\IfFileExists{xurl.sty}{\usepackage{xurl}}{} % add URL line breaks if available
\IfFileExists{bookmark.sty}{\usepackage{bookmark}}{\usepackage{hyperref}}
\hypersetup{
  pdftitle={305 Lecture 14.2 - The Logic Of Counterfactuals},
  pdfauthor={Brian Weatherson},
  hidelinks,
  pdfcreator={LaTeX via pandoc}}
\urlstyle{same} % disable monospaced font for URLs
\newif\ifbibliography
\setlength{\emergencystretch}{3em} % prevent overfull lines
\providecommand{\tightlist}{%
  \setlength{\itemsep}{0pt}\setlength{\parskip}{0pt}}
\setcounter{secnumdepth}{-\maxdimen} % remove section numbering
\let\Tiny=\tiny

 \setbeamertemplate{navigation symbols}{} 

% \usetheme{Madrid}
 \usetheme[numbering=none, progressbar=foot]{metropolis}
 \usecolortheme{wolverine}
 \usepackage{color}
 \usepackage{MnSymbol}
% \usepackage{movie15}

\usepackage{amssymb}% http://ctan.org/pkg/amssymb
\usepackage{pifont}% http://ctan.org/pkg/pifont
\newcommand{\cmark}{\ding{51}}%
\newcommand{\xmark}{\ding{55}}%

\DeclareSymbolFont{symbolsC}{U}{txsyc}{m}{n}
\DeclareMathSymbol{\boxright}{\mathrel}{symbolsC}{128}
\DeclareMathAlphabet{\mathpzc}{OT1}{pzc}{m}{it}

\usepackage{tikz-qtree}
% \usepackage{markdown}
%\RequirePackage{bussproofs}
\usetikzlibrary{arrows.meta}
\RequirePackage[tableaux]{prooftrees}
\forestset{line numbering, close with = x}
% Allow for easy commas inside trees
\renewcommand{\,}{\text{, }}


\usepackage{tabulary}

\usepackage{open-logic-config}

\setlength{\parskip}{1ex plus 0.5ex minus 0.2ex}

\AtBeginSection[]
{
\begin{frame}
	\Huge{\color{darkblue} \insertsection}
\end{frame}
}

\renewenvironment*{quote}	
	{\list{}{\rightmargin   \leftmargin} \item } 	
	{\endlist }

\definecolor{darkgreen}{rgb}{0,0.7,0}
\definecolor{darkblue}{rgb}{0,0,0.8}

\newcommand{\starttab}{\begin{center}
\vspace{6pt}
\begin{tabular}}

\newcommand{\stoptab}{\end{tabular}
\vspace{6pt}
\end{center}
\noindent}


\newcommand{\sif}{\rightarrow}
\newcommand{\siff}{\leftrightarrow}
\newcommand{\EF}{\end{frame}}


\newcommand{\TreeStart}[1]{
%\end{frame}
\begin{frame}
\begin{center}
\begin{tikzpicture}[scale=#1]
\tikzset{every tree node/.style={align=center,anchor=north}}
%\Tree
}

\newcommand{\TreeEnd}{
\end{tikzpicture}
%\end{center}
}

\newcommand{\DisplayArg}[2]{
\begin{enumerate}
{#1}
\end{enumerate}
\vspace{-6pt}
\hrulefill

%\hspace{14pt} #2
%{\addtolength{\leftskip}{14pt} #2}
\begin{quote}
{\normalfont #2}
\end{quote}
\vspace{12pt}
}

\newenvironment{ProofTree}[1][1]{
\begin{center}
\begin{tikzpicture}[scale=#1]
\tikzset{every tree node/.style={align=center,anchor=south}}
}
{
\end{tikzpicture}
\end{center}
}

\newcommand{\TreeFrame}[2]{
\begin{columns}[c]
\column{0.5\textwidth}
\begin{center}
\begin{prooftree}{}
#1
\end{prooftree}
\end{center}
\column{0.45\textwidth}
%\begin{markdown}
#2
%\end{markdown}
\end{columns}
}

\newcommand{\ScaledTreeFrame}[3]{
\begin{columns}[c]
\column{0.5\textwidth}
\begin{center}
\scalebox{#1}{
\begin{prooftree}{}
#2
\end{prooftree}
}
\end{center}
\column{0.45\textwidth}
%\begin{markdown}
#3
%\end{markdown}
\end{columns}
}

\usepackage[bb=boondox]{mathalfa}
\DeclareMathAlphabet{\mathbx}{U}{BOONDOX-ds}{m}{n}
\SetMathAlphabet{\mathbx}{bold}{U}{BOONDOX-ds}{b}{n}
\DeclareMathAlphabet{\mathbbx} {U}{BOONDOX-ds}{b}{n}


\newenvironment{oltableau}{\center\tableau{}} %wff format={anchor = base west}}}
       {\endtableau\endcenter}
       
\newcommand{\formula}[1]{$#1$}

\usepackage{tabulary}
\usepackage{booktabs}

\def\begincols{\begin{columns}}
\def\begincol{\begin{column}}
\def\endcol{\end{column}}
\def\endcols{\end{columns}}

\usepackage[italic]{mathastext}
\usepackage{nicefrac}

\definecolor{mygreen}{RGB}{0, 100, 0}
\definecolor{mypink2}{RGB}{219, 48, 122}
\definecolor{dodgerblue}{RGB}{30,144,255}

%\def\True{\textcolor{dodgerblue}{\text{T}}}
%\def\False{\textcolor{red}{\text{F}}}

\def\True{\mathbb{T}}
\def\False{\mathbb{F}}

% This is because arguments didn't have enough space after them
\usepackage{etoolbox}
\AfterEndEnvironment{description}{\vspace{9pt}}
\AfterEndEnvironment{oltableau}{\vspace{9pt}}
\BeforeBeginEnvironment{oltableau}{\vspace{9pt}}
\AfterEndEnvironment{center}{\vspace{12pt}}
\BeforeBeginEnvironment{tabular}{\vspace{9pt}}

\setlength\heavyrulewidth{0pt}
\setlength\lightrulewidth{0pt}

%\def\toprule{}
%\def\bottomrule{}
%\def\midrule{}

\setbeamertemplate{caption}{\raggedright\insertcaption}

\ifluatex
  \usepackage{selnolig}  % disable illegal ligatures
\fi

\title{305 Lecture 14.2 - The Logic Of Counterfactuals}
\author{Brian Weatherson}
\date{}

\begin{document}
\frame{\titlepage}

\begin{frame}{Plan}
\protect\hypertarget{plan}{}
\begin{itemize}
\tightlist
\item
  To discuss some features of the logic of counterfactual conditionals.
\end{itemize}
\end{frame}

\begin{frame}{Associated Reading}
\protect\hypertarget{associated-reading}{}
\begin{itemize}
\tightlist
\item
  \emph{Boxes and Diamonds}, chapter 7.
\end{itemize}
\end{frame}

\begin{frame}{Antecedent Strengthening}
\protect\hypertarget{antecedent-strengthening}{}
This fails on the minimal change semantics.
\end{frame}

\begin{frame}{Antecedent Strengthening}
\protect\hypertarget{antecedent-strengthening-1}{}
What is it?

\begin{enumerate}
\tightlist
\item
  \(A \boxright B\).
\item
  so, \((A \wedge C) \boxright B\)
\end{enumerate}
\end{frame}

\begin{frame}{Antecedent Strengthening}
\protect\hypertarget{antecedent-strengthening-2}{}
Why does it fail? Because this is possible.

\begin{itemize}
\tightlist
\item
  The nearest \(A\)-world is 10 units away, and at it, \(B\) is true,
  but \(C\) is false.
\item
  The nearest \(A \wedge C\)-world is 100 units away, and at it, \(C\)
  is true (of course), but \(B\) is false.
\end{itemize}
\end{frame}

\begin{frame}{Antecedent Strengthening}
\protect\hypertarget{antecedent-strengthening-3}{}
Put in less technical terms, it fails when these things happen at once.

\begin{itemize}[<+->]
\tightlist
\item
  In all normal worlds, when \(A\) is true, \(B\) is true and \(C\) is
  false.
\item
  So there are no normal worlds where \(A \wedge C\).
\item
  But in the \(A \wedge C\)-worlds that are only a bit abnormal, \(B\)
  is false.
\end{itemize}
\end{frame}

\begin{frame}{Skiing Example}
\protect\hypertarget{skiing-example}{}
Here are some things we might imagine are true in all normal
possibilities.

\begin{itemize}
\tightlist
\item
  When it snows, it snows a normal amount - it isn't a blizzard.
\item
  When it snows, Jack goes skiing.
\item
  Jack has no disposition to want to ski in a blizzard. \pause
\end{itemize}

So in the most normal blizzard world, Jack does not go skiing. And we
have both of the following.

\begin{itemize}
\tightlist
\item
  If it had snowed, Jack would have gone skiing.
\item
  If there had been snow and a blizzard, Jack would not have gone
  skiing.
\end{itemize}
\end{frame}

\begin{frame}{Transitivity (or Conditional Syllogism)}
\protect\hypertarget{transitivity-or-conditional-syllogism}{}
This argument seems initially compelling for any interpretation of `if'.

\begin{enumerate}
\tightlist
\item
  If \(A, B\).
\item
  If \(B, C\).
\item
  So if \(A, C\).
\end{enumerate}
\end{frame}

\begin{frame}{Transitivity for Subjunctives}
\protect\hypertarget{transitivity-for-subjunctives}{}
On the nearest possible world, or minimal change, semantics, this will
fail for subjunctive conditionals.

\begin{itemize}
\tightlist
\item
  Imagine there is no nearby world where \(A\).
\item
  The nearest \(A\) worlds (which in these cases will be quite distant)
  are worlds where \(B\) is true and \(C\) is false.
\item
  There are nearby, normal \(B\) worlds, and they are all worlds where
  \(C\) is true.
\end{itemize}
\end{frame}

\begin{frame}{Real Life Example}
\protect\hypertarget{real-life-example}{}
\begin{enumerate}
\tightlist
\item
  If there had been a hurricane and a blizzard on Presidents' Day
  Weekend, it would have snowed on Presidents' Day Weekend. \pause 
\item
  If it had snowed on Presidents' Day Weekend, Jack would have gone
  skiing on Presidents' Day Weekend. \pause 
\item
  So if there had been a hurricane and a blizzard on Presidents' Day
  Weekend, Jack would have gone skiing on Presidents' Day Weekend.
\end{enumerate}
\end{frame}

\begin{frame}{Link To Antecedent Strengthening}
\protect\hypertarget{link-to-antecedent-strengthening}{}
\begin{enumerate}
\tightlist
\item
  If I had crashed my car last week and won the lottery on the weekend,
  I'd have crashed my car last week. \pause 
\item
  If I had crashed my car last week, I'd now have less money than
  actually do. \pause 
\item
  So if I had crashed my car last week and won the lottery on the
  weekend, I'd now have less money than actually do.
\end{enumerate}
\end{frame}

\begin{frame}{Recipe Here}
\protect\hypertarget{recipe-here}{}
Assume that you have \emph{If A, C}, but not \emph{If A and B, C}.

\begin{enumerate}
\tightlist
\item
  If \(A\) and \(B\), \(A\).
\item
  If \(A, C\).
\item
  So, If \(A\) and \(B\), \(C\).
\end{enumerate}

That should have true premises and a false conclusion.
\end{frame}

\begin{frame}{But Wait}
\protect\hypertarget{but-wait}{}
But don't we reason this way all the time.

\begin{enumerate}
\tightlist
\item
  If there had been a major earthquake in Ann Arbor last weekend,
  several buildings in Ann Arbor would have been destroyed. \pause 
\item
  If several buildings in Ann Arbor were destroyed, I would have been
  very upset. \pause 
\item
  So if there had been a major earthquake in Ann Arbor on the weekend, I
  would have been very upset.
\end{enumerate}

The point is not that 1-3 look true here, it's that reasoning from 1 and
2 to 3 looks like good, ordinary, everyday reasoning.
\end{frame}

\begin{frame}{Modified Transitivity}
\protect\hypertarget{modified-transitivity}{}
This argument form is valid on the minimal change semantics.

\begin{enumerate}
\tightlist
\item
  If \(A, B\)
\item
  If \(A \wedge B, C\)
\item
  So, if \(A, C\)
\end{enumerate}
\end{frame}

\begin{frame}{Proof Of Modified Transitivity}
\protect\hypertarget{proof-of-modified-transitivity}{}
\begin{itemize}
\tightlist
\item
  Let \(d\) be the distance of the nearest \(A\) world.
\item
  By premise 1, all of those nearest \(A\) worlds, which are distance
  \(d\) away, are \(B\) worlds. \pause 
\item
  So they are also the nearest \(A \wedge B\) worlds. (There can't be
  any closer, because then they would be closer \(A\) worlds.) \pause 
\item
  By premise 2 then, all these \(A \wedge B\) worlds distance \(d\) away
  are also \(C\) worlds.
\item
  So all the \(A\) worlds distance \(d\) away are \(C\) worlds. \pause 
\item
  So, if \(A, C\).
\end{itemize}
\end{frame}

\begin{frame}{Explanation}
\protect\hypertarget{explanation}{}
Lewis suggested that was what was really going on when we use
transitivity arguments in everyday life.

\begin{itemize}
\tightlist
\item
  The middle premise is really \emph{If A and B, C}, not just \emph{If
  B, C}.
\item
  Is this plausible? I don't know.
\end{itemize}
\end{frame}

\begin{frame}{That's All!}
\protect\hypertarget{thats-all}{}
Next class will just be about revision before the final. Thanks for
staying to the end!
\end{frame}

\end{document}
