% Options for packages loaded elsewhere
\PassOptionsToPackage{unicode}{hyperref}
\PassOptionsToPackage{hyphens}{url}
%
\documentclass[
  ignorenonframetext,
]{beamer}
\usepackage{pgfpages}
\setbeamertemplate{caption}[numbered]
\setbeamertemplate{caption label separator}{: }
\setbeamercolor{caption name}{fg=normal text.fg}
\beamertemplatenavigationsymbolsempty
% Prevent slide breaks in the middle of a paragraph
\widowpenalties 1 10000
\raggedbottom
\setbeamertemplate{part page}{
  \centering
  \begin{beamercolorbox}[sep=16pt,center]{part title}
    \usebeamerfont{part title}\insertpart\par
  \end{beamercolorbox}
}
\setbeamertemplate{section page}{
  \centering
  \begin{beamercolorbox}[sep=12pt,center]{part title}
    \usebeamerfont{section title}\insertsection\par
  \end{beamercolorbox}
}
\setbeamertemplate{subsection page}{
  \centering
  \begin{beamercolorbox}[sep=8pt,center]{part title}
    \usebeamerfont{subsection title}\insertsubsection\par
  \end{beamercolorbox}
}
\AtBeginPart{
  \frame{\partpage}
}
\AtBeginSection{
  \ifbibliography
  \else
    \frame{\sectionpage}
  \fi
}
\AtBeginSubsection{
  \frame{\subsectionpage}
}
\usepackage{amsmath,amssymb}
\usepackage{lmodern}
\usepackage{ifxetex,ifluatex}
\ifnum 0\ifxetex 1\fi\ifluatex 1\fi=0 % if pdftex
  \usepackage[T1]{fontenc}
  \usepackage[utf8]{inputenc}
  \usepackage{textcomp} % provide euro and other symbols
\else % if luatex or xetex
  \usepackage{unicode-math}
  \defaultfontfeatures{Scale=MatchLowercase}
  \defaultfontfeatures[\rmfamily]{Ligatures=TeX,Scale=1}
  \setmainfont[BoldFont = SF Pro Rounded Semibold]{SF Pro Rounded}
  \setmathfont[]{STIX Two Math}
\fi
\usefonttheme{serif} % use mainfont rather than sansfont for slide text
% Use upquote if available, for straight quotes in verbatim environments
\IfFileExists{upquote.sty}{\usepackage{upquote}}{}
\IfFileExists{microtype.sty}{% use microtype if available
  \usepackage[]{microtype}
  \UseMicrotypeSet[protrusion]{basicmath} % disable protrusion for tt fonts
}{}
\makeatletter
\@ifundefined{KOMAClassName}{% if non-KOMA class
  \IfFileExists{parskip.sty}{%
    \usepackage{parskip}
  }{% else
    \setlength{\parindent}{0pt}
    \setlength{\parskip}{6pt plus 2pt minus 1pt}}
}{% if KOMA class
  \KOMAoptions{parskip=half}}
\makeatother
\usepackage{xcolor}
\IfFileExists{xurl.sty}{\usepackage{xurl}}{} % add URL line breaks if available
\IfFileExists{bookmark.sty}{\usepackage{bookmark}}{\usepackage{hyperref}}
\hypersetup{
  pdftitle={305 Lecture 13.3 - Strict Conditionals},
  pdfauthor={Brian Weatherson},
  hidelinks,
  pdfcreator={LaTeX via pandoc}}
\urlstyle{same} % disable monospaced font for URLs
\newif\ifbibliography
\setlength{\emergencystretch}{3em} % prevent overfull lines
\providecommand{\tightlist}{%
  \setlength{\itemsep}{0pt}\setlength{\parskip}{0pt}}
\setcounter{secnumdepth}{-\maxdimen} % remove section numbering
\let\Tiny=\tiny

 \setbeamertemplate{navigation symbols}{} 

% \usetheme{Madrid}
 \usetheme[numbering=none, progressbar=foot]{metropolis}
 \usecolortheme{wolverine}
 \usepackage{color}
 \usepackage{MnSymbol}
% \usepackage{movie15}

\usepackage{amssymb}% http://ctan.org/pkg/amssymb
\usepackage{pifont}% http://ctan.org/pkg/pifont
\newcommand{\cmark}{\ding{51}}%
\newcommand{\xmark}{\ding{55}}%

\DeclareSymbolFont{symbolsC}{U}{txsyc}{m}{n}
\DeclareMathSymbol{\boxright}{\mathrel}{symbolsC}{128}
\DeclareMathAlphabet{\mathpzc}{OT1}{pzc}{m}{it}

\usepackage{tikz-qtree}
% \usepackage{markdown}
%\RequirePackage{bussproofs}
\usetikzlibrary{arrows.meta}
\RequirePackage[tableaux]{prooftrees}
\forestset{line numbering, close with = x}
% Allow for easy commas inside trees
\renewcommand{\,}{\text{, }}


\usepackage{tabulary}

\usepackage{open-logic-config}

\setlength{\parskip}{1ex plus 0.5ex minus 0.2ex}

\AtBeginSection[]
{
\begin{frame}
	\Huge{\color{darkblue} \insertsection}
\end{frame}
}

\renewenvironment*{quote}	
	{\list{}{\rightmargin   \leftmargin} \item } 	
	{\endlist }

\definecolor{darkgreen}{rgb}{0,0.7,0}
\definecolor{darkblue}{rgb}{0,0,0.8}

\newcommand{\starttab}{\begin{center}
\vspace{6pt}
\begin{tabular}}

\newcommand{\stoptab}{\end{tabular}
\vspace{6pt}
\end{center}
\noindent}


\newcommand{\sif}{\rightarrow}
\newcommand{\siff}{\leftrightarrow}
\newcommand{\EF}{\end{frame}}


\newcommand{\TreeStart}[1]{
%\end{frame}
\begin{frame}
\begin{center}
\begin{tikzpicture}[scale=#1]
\tikzset{every tree node/.style={align=center,anchor=north}}
%\Tree
}

\newcommand{\TreeEnd}{
\end{tikzpicture}
%\end{center}
}

\newcommand{\DisplayArg}[2]{
\begin{enumerate}
{#1}
\end{enumerate}
\vspace{-6pt}
\hrulefill

%\hspace{14pt} #2
%{\addtolength{\leftskip}{14pt} #2}
\begin{quote}
{\normalfont #2}
\end{quote}
\vspace{12pt}
}

\newenvironment{ProofTree}[1][1]{
\begin{center}
\begin{tikzpicture}[scale=#1]
\tikzset{every tree node/.style={align=center,anchor=south}}
}
{
\end{tikzpicture}
\end{center}
}

\newcommand{\TreeFrame}[2]{
\begin{columns}[c]
\column{0.5\textwidth}
\begin{center}
\begin{prooftree}{}
#1
\end{prooftree}
\end{center}
\column{0.45\textwidth}
%\begin{markdown}
#2
%\end{markdown}
\end{columns}
}

\newcommand{\ScaledTreeFrame}[3]{
\begin{columns}[c]
\column{0.5\textwidth}
\begin{center}
\scalebox{#1}{
\begin{prooftree}{}
#2
\end{prooftree}
}
\end{center}
\column{0.45\textwidth}
%\begin{markdown}
#3
%\end{markdown}
\end{columns}
}

\usepackage[bb=boondox]{mathalfa}
\DeclareMathAlphabet{\mathbx}{U}{BOONDOX-ds}{m}{n}
\SetMathAlphabet{\mathbx}{bold}{U}{BOONDOX-ds}{b}{n}
\DeclareMathAlphabet{\mathbbx} {U}{BOONDOX-ds}{b}{n}


\newenvironment{oltableau}{\center\tableau{}} %wff format={anchor = base west}}}
       {\endtableau\endcenter}
       
\newcommand{\formula}[1]{$#1$}

\usepackage{tabulary}
\usepackage{booktabs}

\def\begincols{\begin{columns}}
\def\begincol{\begin{column}}
\def\endcol{\end{column}}
\def\endcols{\end{columns}}

\usepackage[italic]{mathastext}
\usepackage{nicefrac}

\definecolor{mygreen}{RGB}{0, 100, 0}
\definecolor{mypink2}{RGB}{219, 48, 122}
\definecolor{dodgerblue}{RGB}{30,144,255}

%\def\True{\textcolor{dodgerblue}{\text{T}}}
%\def\False{\textcolor{red}{\text{F}}}

\def\True{\mathbb{T}}
\def\False{\mathbb{F}}

% This is because arguments didn't have enough space after them
\usepackage{etoolbox}
\AfterEndEnvironment{description}{\vspace{9pt}}
\AfterEndEnvironment{oltableau}{\vspace{9pt}}
\BeforeBeginEnvironment{oltableau}{\vspace{9pt}}
\AfterEndEnvironment{center}{\vspace{12pt}}
\BeforeBeginEnvironment{tabular}{\vspace{9pt}}

\setlength\heavyrulewidth{0pt}
\setlength\lightrulewidth{0pt}

%\def\toprule{}
%\def\bottomrule{}
%\def\midrule{}

\setbeamertemplate{caption}{\raggedright\insertcaption}

\ifluatex
  \usepackage{selnolig}  % disable illegal ligatures
\fi

\title{305 Lecture 13.3 - Strict Conditionals}
\author{Brian Weatherson}
\date{}

\begin{document}
\frame{\titlepage}

\begin{frame}{Plan}
\protect\hypertarget{plan}{}
\begin{itemize}
\tightlist
\item
  To discuss the strict conditional.
\end{itemize}
\end{frame}

\begin{frame}{Associated Reading}
\protect\hypertarget{associated-reading}{}
\begin{itemize}
\tightlist
\item
  Boxes and Diamonds, section 6.3
\end{itemize}
\end{frame}

\begin{frame}{Definition of Strict Conditional}
\protect\hypertarget{definition-of-strict-conditional}{}
\begin{itemize}
\tightlist
\item
  \emph{If A, B} means \emph{Necessarily, A materially implies B}.
\item
  In symbols, \(A \rightarrow B = \Box (A \supset B)\)
\end{itemize}
\end{frame}

\begin{frame}{What Box}
\protect\hypertarget{what-box}{}
Two main options

\begin{itemize}
\tightlist
\item
  Universal - \(\Box (A \supset B)\) means that every \(A\) world is a
  \(B\) world, i.e., \(A \wedge \neg B\) is impossible.
\item
  Epistemic \(\Box (A \supset B)\) means that every \(A\) world
  \emph{that could be actual for all we know} is a \(B\) world, i.e.,
  \(A \wedge \neg B\) is known to be false.
\end{itemize}

And there are other complications you could try. We'll \emph{mostly}
stay out of these.
\end{frame}

\begin{frame}{The Nine Puzzle Cases}
\protect\hypertarget{the-nine-puzzle-cases}{}
\begin{enumerate}
\tightlist
\item
  Modus Ponens - \(A, A \rightarrow B \vDash B\)
\item
  Agglomeration -
  \(A \rightarrow B, A \rightarrow C \vDash A \rightarrow (B \wedge C)\)
\item
  Transitive -
  \(A \rightarrow B, B \rightarrow C \vDash A \rightarrow C\)
\item
  Contraposition - \(A \rightarrow B \vDash \neg B \rightarrow \neg A\)
\item
  Antecedent Strengthening -
  \(A \rightarrow B \vDash (A \wedge C) \rightarrow B\)
\item
  Paradox 1 - \(B \vDash A \rightarrow B\)
\item
  Paradox 2 - \(\neg A \vDash A \rightarrow B\)
\item
  Strict Paradox - \(\Box B \vDash A \rightarrow B\)
\item
  Disjunction Paradox - \((A \rightarrow B) \vee (B \rightarrow A)\)
\end{enumerate}
\end{frame}

\begin{frame}{Modus Ponens}
\protect\hypertarget{modus-ponens}{}
Still valid.

\begin{itemize}
\tightlist
\item
  Assume \(A\), and that every \(A\) world is a \(B\) world.
\item
  Then this is an \(A\) world, so it is a \(B\) world.
\end{itemize}
\end{frame}

\begin{frame}{Agglomeration.}
\protect\hypertarget{agglomeration.}{}
Still valid.

\begin{itemize}
\tightlist
\item
  Assume every \(A\) world is a \(B\) world, and that every \(A\) world
  is a \(C\) world.
\item
  Then every \(A\) world will be a \(B \wedge C\) world.
\item
  So \(A \rightarrow (B \wedge C)\)
\end{itemize}
\end{frame}

\begin{frame}{Transitivity}
\protect\hypertarget{transitivity}{}
Still valid.

\begin{itemize}
\tightlist
\item
  Assume every \(A\) world is a \(B\) world, and every \(B\) world is a
  \(C\) world.
\item
  Then every \(A\) world is a \(C\) world.
\item
  So \(A \rightarrow C\)
\end{itemize}
\end{frame}

\begin{frame}{Contraposition}
\protect\hypertarget{contraposition}{}
Still valid.

\begin{itemize}
\tightlist
\item
  Assume every \(A\) world is a \(B\) world.
\item
  Consider an arbitrary \(\neg B\) world.
\item
  It can't be an \(A\) world, because then some \(A\) world would not be
  a \(B\) world.
\item
  So it's a \(\neg A\) world.
\item
  So all \(\neg B\) worlds are \(\neg A\) worlds
\item
  So \(\neg B \rightarrow \neg A\).
\end{itemize}
\end{frame}

\begin{frame}{Antecedent Strengthening}
\protect\hypertarget{antecedent-strengthening}{}
Still valid.

\begin{itemize}
\tightlist
\item
  Assume every \(A\) world is a \(B\) world.
\item
  Consider an \(A \wedge C\) world.
\item
  It's an \(A\) world, by the rule for true \(\wedge\).
\item
  So it's a \(B\) world.
\item
  So \((A \wedge C) \rightarrow B\).
\end{itemize}
\end{frame}

\begin{frame}{Paradox 1}
\protect\hypertarget{paradox-1}{}
Not valid!

\begin{itemize}
\tightlist
\item
  Even if \(B\) is true, as long as \(A \wedge \neg B\) is possible,
  \(A \rightarrow B\) will be false.
\end{itemize}
\end{frame}

\begin{frame}{Paradox 2}
\protect\hypertarget{paradox-2}{}
Not valid!

\begin{itemize}
\tightlist
\item
  Even if \(A\) is not actually true, as long as \emph{some} world make
  \(A \wedge \neg B\) true, then \(A \rightarrow B\) will be false.
\end{itemize}
\end{frame}

\begin{frame}{Strict Paradox}
\protect\hypertarget{strict-paradox}{}
Still valid.

\begin{itemize}
\tightlist
\item
  Assume every world is a \(B\) world.
\item
  Then every \(A\) world is a world, and hence a \(B\) world.
\item
  So \(A \rightarrow B\).
\end{itemize}
\end{frame}

\begin{frame}{Disjunction Paradox}
\protect\hypertarget{disjunction-paradox}{}
Not valid.

\begin{itemize}
\tightlist
\item
  Just let \(A, B\) be two independent propositions, so all four binary
  possibilities are realised.
\item
  Then neither disjunction will be true.
\end{itemize}
\end{frame}

\begin{frame}{Summary}
\protect\hypertarget{summary}{}
This is better than the material conditional theory, but not perfect.
And we haven't said anything about the subjunctive conditionals.
\end{frame}

\begin{frame}{For Next Time}
\protect\hypertarget{for-next-time}{}
We'll talk a bit more about one of these rules, antecedent
strengthening.
\end{frame}

\end{document}
